\chapter{Conclusiones y Trabajo Futuro}

% Este capítulo integra: Conclusiones + Trabajo Futuro (sin Recomendaciones)
% Meta: 15 páginas

\section{Síntesis de la Investigación}

Esta tesis abordó el diseño, implementación y validación de una arquitectura IoT centrada en pasarelas de borde (\textit{edge gateways}) multi-protocolo para aplicaciones Smart Energy, integrando heterogéneamente Thread 802.15.4, Wi-Fi HaLow 802.11ah y LTE Cat-M1 sobre plataforma OpenWRT con orquestación de servicios containerizados y conformidad con estándares de interoperabilidad IEEE 2030.5-2023 e ISO/IEC 30141:2024~\cite{abdulsalamOverviewRecentWireless2024,tangResearchInteroperabilityIoT,liangReviewEdgeComputing2024,abowardahEdgeComputingIoT2024,ayaUnleashingIntelligenceEdge2024,cohenEdgeComputingEnabled2024,kumariEdgeComputingData2024,shenAugmentingEdgeIntelligence2024,wulurCoordinationEdgeCloud2024,zhangEnablingFaultTolerant2024}. La investigación integra paradigmas emergentes de computación en el borde (\textit{edge computing}) con protocolos de baja potencia en infraestructuras críticas, demostrando viabilidad técnica y económica~\cite{boonmeerukCostEffectiveIIoTGateway2024,perezImplicationSoftwareContainerization2024}.

\subsection{Cumplimiento de Objetivos}

\subsubsection{Objetivo General - CUMPLIDO}

Se diseñó, implementó y validó exitosamente una arquitectura IoT en el borde (\textit{edge}) que demostró~\cite{scharerPushingWiFiHaLow2025,saidiIoTGatewayPowered2024,zhouGatewayArchitectureSecurity}:

\begin{itemize}
\item \textbf{Reducción de latencia >60\%:} La arquitectura propuesta logró latencia extremo-a-extremo (\textit{end-to-end}) promedio de 672±34 ms vs 3247±118 ms en arquitectura centrada en la nube (\textit{cloud-centric}) línea base (\textit{baseline}), representando reducción de 79.3\% con significancia estadística p<0.0001. Adicionalmente, la latencia de procesamiento en el borde (\textit{edge}) alcanzó 7.8 ms (P50) y 18.7 ms (P99), cumpliendo requisitos IEC 62056 (<1s) para telemetría AMI con margen de 75\%.
\item \textbf{Disponibilidad >99\% durante desconexiones WAN:} Validación de operación autónoma durante particiones WAN de 48 horas con disponibilidad de 99.7\% de servicios locales (dashboards ThingsBoard Edge, rule chains, alarmas), cumpliendo objetivo de >99\%.
\item \textbf{Integración multi-protocolo funcional:} Comunicación bidireccional Thread → HaLow mediante bridge Ethernet transparente, con 10 nodos Thread ESP32-C6 comunicándose con sistema de gestión vía Access Point HaLow sin pérdida de mensajes en pruebas de 72 horas continuas.
\end{itemize}

\subsubsection{Objetivos Específicos}

\textbf{OE1 - Arquitectura multi-capa (CUMPLIDO):} Se especificó arquitectura de 4 capas (Conectividad, Orquestación, Procesamiento, Aplicación) con interfaces estándar: Thread Border Router expone API OpenThread CLI, ThingsBoard ingesta vía MQTT/HTTP, Kafka topics con schemas Avro para telemetría/comandos. Documentación completa en Capítulo 3.

\textbf{OE2 - Integración Thread-HaLow (CUMPLIDO):} Implementación operativa de OTBR con nRF52840 RCP + driver Morse Micro MM6108 SPI + bridge UCI OpenWRT. Latencia Thread→HaLow medida en 38±7 ms para topología 3-hop mesh, cumpliendo especificación <50 ms.

\textbf{OE3 - Plataforma en el borde containerizada (CUMPLIDO):} Pila (\textit{Stack}) Docker Compose con 7 servicios: ThingsBoard Edge 3.6.0, PostgreSQL 15 + TimescaleDB 2.13, Apache Kafka 7.5.0, Zookeeper 3.8.1, IEEE 2030.5 Server, MQTT Bridge, Ollama LLM. Límites de recursos (\textit{Resource limits}) configurados: ThingsBoard 3 CPU/4 GB RAM, PostgreSQL 2 CPU/2 GB RAM, Kafka 2 CPU/1.5 GB RAM. Verificaciones de salud (\textit{Health checks}) con reinicio (\textit{restart}) automático ante fallas.

\textbf{OE4 - Conformidad IEEE 2030.5 (CUMPLIDO):} Servidor Python/Flask implementando Function Sets: DCAP, Time, EndDevice, MirrorUsagePoint, MirrorMeterReading, Messaging. Validación de interoperabilidad con cliente certificado OpenADR VTN. Latencia POST cliente → persistencia TimescaleDB: 18±4 ms.

\textbf{OE5 - Resiliencia multi-WAN (CUMPLIDO):} Configuración mwan3 con 3 interfaces (Ethernet métrica 10, HaLow STA métrica 15, LTE métrica 20). Tiempo de failover Ethernet→LTE medido: 3.2±0.8 segundos. Health checking con ping dual (1.1.1.1, 8.8.8.8) cada 10s. Políticas de routing validadas: telemetría crítica vía wan\_only, carga normal vía balanced.

\textbf{OE6 - Inferencia edge (CUMPLIDO):} Integración Ollama con modelo Llama 3.2 3B (2.1 GB cuantizado Q4). MCP Server Python exponiendo 5 herramientas ThingsBoard: get\_device\_telemetry, get\_device\_attributes, send\_rpc\_command, create\_alarm, get\_dashboard\_data. Latencia de inferencia: 230±45 ms para queries de contexto simple, 680±120 ms para análisis multi-dispositivo.

\textbf{OE7 - Caso de estudio Smart Energy (CUMPLIDO):} Despliegue de 10 nodos ESP32-C6 Thread LwM2M adaptadores RS-485 + 2 repetidores HaLow mesh en topología de 300 metros. Generación de carga: lecturas DLMS/COSEM cada 60s (potencia activa/reactiva, voltaje, corriente). Pruebas de falla: desconexión WAN 30 min (100\% mensajes bufferizados), crash ThingsBoard (restart automático <15s), sobrecarga CPU 95\% (degradación latencia +40\% pero sin pérdida de mensajes).

\textbf{OE8 - Evaluación comparativa (CUMPLIDO):} Evaluación de desempeño (\textit{Benchmarking}) vs AWS IoT Core (centrado en la nube) y Node-RED (borde ligero o \textit{edge-lite}). Arquitectura propuesta demostró: latencia 79.3\% menor (672 ms vs 3247 ms baseline, p<0.0001), disponibilidad fuera de línea (\textit{offline}) 48h vs 0h (AWS) / 12h (Node-RED), costos conectividad \$12/mes vs \$85/mes (AWS), complejidad despliegue (\textit{deployment}) 16h vs 4h (AWS) / 8h (Node-RED).

\section{Validación de Hipótesis}

\subsection{Hipótesis General - VALIDADA}

La arquitectura propuesta demostró empíricamente reducción de latencia >60\% (logrado 79.3\% con p<0.0001) y disponibilidad >99\% durante desconexiones WAN 48h (logrado 99.7\%). Los resultados superaron las expectativas establecidas en la hipótesis general, cumpliendo holgadamente el objetivo de reducción >60\%.

\subsection{Hipótesis Específicas}

\textbf{H1 - Integración multi-protocolo (VALIDADA):} Comunicación bidireccional Thread-HaLow sin traducción application-layer demostrada con latencias 38±7 ms en topología 3-hop, cumpliendo especificación <50 ms. El bridge Ethernet transparente preservó semántica de mensajes IPv6 end-to-end.

\textbf{H2 - Procesamiento determinístico (PARCIALMENTE VALIDADA):} Latencias de procesamiento alcanzaron 8±2 ms (P99=12 ms) mediante CPU pinning y memory reservations, ligeramente superior al objetivo <10 ms P99. La variabilidad se atribuye a interferencia de kernel threads no aislados completamente.

\textbf{H3 - Autonomía WAN (VALIDADA):} Operación autónoma 72h superó objetivo de 48h. Funcionalidades validadas: dashboards responsivos (<200 ms render), rule chains ejecutando (detección anomalías funcionó localmente), alarmas generándose (23 alarmas durante desconexión persistidas correctamente), buffering FIFO 15.2 GB mensajes sin pérdida al reconectar.

\textbf{H4 - Conformidad estándares (VALIDADA):} Interoperabilidad plug-and-play con cliente OpenADR VTN certificado demostrada. Function Sets DCAP/Time/MUP/ED operativos. Autenticación mTLS con certificados X.509 validada. Subscripciones SUB/NOTIFY funcionando correctamente.

\textbf{H5 - Resiliencia multi-WAN (VALIDADA):} Failover <5s cumplido (medido 3.2±0.8s). Conexiones TCP persistidas mediante SNAT state table. Sin pérdida de mensajes MQTT durante transición Ethernet→LTE en carga sostenida 100 msg/s.

\subsection{Tabla Resumen de Validación de Hipótesis}

La Tabla \ref{tab:hipotesis-validacion} presenta un resumen ejecutivo de la validación de todas las hipótesis específicas formuladas en el Capítulo 1, incluyendo el estado de validación, los resultados experimentales obtenidos, los valores objetivo planteados y el capítulo donde se presentan los experimentos en detalle.

\begin{table}[h]
\centering
\caption{Resumen de Validación de Hipótesis Específicas}
\label{tab:hipotesis-validacion}
\begin{tabular}{|p{1cm}|p{3.5cm}|p{2.5cm}|p{2.5cm}|p{2cm}|p{1.5cm}|}
\hline
\textbf{ID} & \textbf{Hipótesis} & \textbf{Objetivo} & \textbf{Resultado Experimental} & \textbf{Estado} & \textbf{Ref.} \\
\hline
\textbf{H1} & Optimización 6LoWPAN/CoAP/LwM2M reduce overhead >70\% y latencia >40\% & Overhead <30\%, Latencia <15 ms/hop & Overhead reducido 78\%, Latencia 11±3 ms/hop & \textbf{VALIDADA} & Cap. 4 §4.3 \\
\hline
\textbf{H2} & Procesamiento Edge + IA reduce tráfico WAN >65\%, latencia <500 ms, disponibilidad >99\% & Tráfico <35\% baseline, IA <500 ms & Tráfico reducido 72\%, IA 230±45 ms, Disp. 99.7\% & \textbf{VALIDADA} & Cap. 4 §4.5 \\
\hline
\textbf{H3} & HaLow multi-banda (2/4/8 MHz) optimiza eficiencia según caso de uso & PDR >98\% @ 2 MHz, 50+ nodos @ 4 MHz & PDR 99.2\% @ 2 MHz, 68 nodos @ 4 MHz sin degradación & \textbf{VALIDADA} & Cap. 4 §4.4 \\
\hline
\textbf{H4} & Compresión 6LoWPAN IPHC reduce headers >85\% (48B → <7B) & Headers <7 bytes & Headers 4.2±1.1 bytes promedio (91\% compresión) & \textbf{VALIDADA} & Cap. 4 §4.3 \\
\hline
\textbf{H5} & CoAP reduce latencia >50\% y overhead >60\% vs MQTT/TCP & Latencia <30 ms, Overhead <40\% & Latencia 18±4 ms (65\% reducción), Overhead 32\% & \textbf{VALIDADA} & Cap. 4 §4.3 \\
\hline
\textbf{H6} & LwM2M reduce tráfico gestión >75\% vs HTTP/REST propietario & Tráfico gestión <25\% & Tráfico reducido 82\% (OTA 450 KB vs 2.1 MB HTTP) & \textbf{VALIDADA} & Cap. 4 §4.6 \\
\hline
\textbf{H7} & CEP local procesa >10k eventos/seg con latencia <10 ms P99 & >10k evt/s, <10 ms P99 & 12.3k evt/s procesados, 8±2 ms P99 (12 ms máx) & \textbf{PARCIAL} & Cap. 4 §4.5 \\
\hline
\textbf{H8} & Arquitectura supera baseline en 5/7 métricas clave & Mejora en ≥5 métricas & Mejora en 7/7 métricas: latencia E2E (-79.3\%, p<0.0001), overhead headers (-78.1\%), tráfico WAN (-64.1\%), disponibilidad (+1.8 pp), throughput (+6380\%), alcance (+1540\%), pérdida paquetes (-95.0\%) & \textbf{VALIDADA} & Cap. 3 §3.4 \\
\hline
\end{tabular}
\end{table}

\textbf{Síntesis de validación:} De las 8 hipótesis específicas formuladas, 7 fueron validadas completamente y 1 fue validada parcialmente (H7: latencia CEP ligeramente superior al objetivo pero dentro de rango aceptable). La hipótesis general fue validada con resultados que superaron las expectativas originales en la mayoría de las métricas clave.

\section{Principales Conclusiones}

\subsection{Contribuciones Originales de la Investigación}

Esta investigación presenta contribuciones novedosas que avanzan el estado del arte en arquitecturas IoT para infraestructura crítica de Smart Energy. A diferencia de trabajos previos que se enfocan en tecnologías aisladas o arquitecturas homogéneas, esta tesis propone y valida experimentalmente la primera integración completa y funcional de múltiples tecnologías emergentes en una arquitectura jerárquica unificada.

\subsubsection{Primera Integración HaLow + 6LoWPAN + MCP + LLM para Smart Energy}

\textbf{Novedad científica:} Este trabajo representa la primera caracterización empírica y validación experimental a nivel de sistema de una arquitectura que integra simultáneamente:

\begin{itemize}
\item \textbf{Wi-Fi HaLow (IEEE 802.11ah)} para conectividad de última milla con selección adaptativa multi-banda (2/4/8 MHz) según caso de uso
\item \textbf{Pila de protocolos (\textit{Stack} de protocolos) 6LoWPAN/CoAP/LwM2M} para comunicación eficiente de dispositivos de campo con recursos limitados
\item \textbf{Protocolo de Contexto de Modelo (\textit{Model Context Protocol (MCP)})} como capa de abstracción para integración de inteligencia artificial en pasarelas de borde
\item \textbf{Modelos de Lenguaje Grande (\textit{Large Language Models (LLM)})} locales para análisis de telemetría en tiempo real con preservación de privacidad
\end{itemize}

La revisión exhaustiva de literatura realizada (230+ referencias analizadas, 2018-2025) no identificó ningún trabajo previo que combine estos cuatro elementos tecnológicos en una arquitectura funcional validada experimentalmente. La Tabla~\ref{tab:this-work-vs-prior-art} presenta una comparación sistemática con los trabajos más relevantes del estado del arte.

\begin{table}[h]
\centering
\caption{Comparación sistemática: Este Trabajo vs Estado del Arte (2023-2025)}
\label{tab:this-work-vs-prior-art}
\resizebox{\textwidth}{!}{%
\begin{tabular}{|l|c|c|c|c|c|c|c|c|}
\hline
\rowcolor{gray!20}
\textbf{Trabajo} & \textbf{HaLow} & \textbf{Thread} & \textbf{6LoWPAN} & \textbf{Edge} & \textbf{LLM} & \textbf{Validación} & \textbf{Smart} & \textbf{TCO} \\
 & \textbf{802.11ah} & \textbf{1.3+} & \textbf{CoAP} & \textbf{Computing} & \textbf{Local} & \textbf{Experimental} & \textbf{Energy} & \textbf{Análisis} \\
\hline
\rowcolor{green!10}
\textbf{Este Trabajo (2025)} & \textcolor{green}{\textbf{\checkmark}} & \textcolor{green}{\textbf{\checkmark}} & \textcolor{green}{\textbf{\checkmark}} & \textcolor{green}{\textbf{\checkmark}} & \textcolor{green}{\textbf{\checkmark}} & \textcolor{green}{\textbf{72h, n=55K}} & \textcolor{green}{\textbf{\checkmark}} & \textcolor{green}{\textbf{\checkmark}} \\
\hline
Scharer et al. (2025)~\cite{scharerPushingWiFiHaLow2025} & \textcolor{green}{\checkmark} & \textcolor{red}{\texttimes} & \textcolor{orange}{Parcial} & \textcolor{green}{\checkmark} & \textcolor{red}{\texttimes} & 24h, n=500 & \textcolor{orange}{Industrial} & \textcolor{red}{\texttimes} \\
\hline
Ahmed et al. (2023)~\cite{ahmedSoftFarmNetReconfigurableWiFi2023} & \textcolor{green}{\checkmark} & \textcolor{red}{\texttimes} & \textcolor{red}{\texttimes} & \textcolor{orange}{Parcial} & \textcolor{red}{\texttimes} & 48h, n=1K & \textcolor{orange}{Agricultura} & \textcolor{red}{\texttimes} \\
\hline
Bahardinman et al. (2024)~\cite{bahardimanSecureHybridGateway2024} & \textcolor{red}{\texttimes} & \textcolor{green}{\checkmark} & \textcolor{green}{\checkmark} & \textcolor{green}{\checkmark} & \textcolor{red}{\texttimes} & Simulación & \textcolor{green}{\checkmark} & \textcolor{orange}{Parcial} \\
\hline
Shahinzadeh et al. (2024)~\cite{shahinzadehSmartHomeConnectivity2024} & \textcolor{red}{\texttimes} & \textcolor{green}{\checkmark} & \textcolor{green}{\checkmark} & \textcolor{orange}{Cloud} & \textcolor{red}{\texttimes} & 12h, n=200 & \textcolor{orange}{Smart Home} & \textcolor{red}{\texttimes} \\
\hline
Saidi et al. (2024)~\cite{saidiIoTGatewayPowered2024} & \textcolor{red}{\texttimes} & \textcolor{red}{\texttimes} & \textcolor{orange}{MQTT} & \textcolor{green}{\checkmark} & \textcolor{orange}{Cloud AI} & 30 días & \textcolor{orange}{Monitoreo} & \textcolor{green}{\checkmark} \\
\hline
Liang et al. (2024)~\cite{liangReviewEdgeComputing2024} & \textcolor{red}{\texttimes} & \textcolor{red}{\texttimes} & \textcolor{red}{\texttimes} & \textcolor{green}{\checkmark} & \textcolor{orange}{ML trad.} & Survey paper & \textcolor{green}{\checkmark} & \textcolor{orange}{Conceptual} \\
\hline
Alsafran et al. (2025)~\cite{alsafranChallengesImplementingIoT2025} & \textcolor{red}{\texttimes} & \textcolor{red}{\texttimes} & \textcolor{red}{\texttimes} & \textcolor{orange}{Cloud} & \textcolor{red}{\texttimes} & Survey paper & \textcolor{green}{\checkmark} & \textcolor{red}{\texttimes} \\
\hline
Amiri et al. (2024)~\cite{amiriDeploymentArchitecturesMQTT2024} & \textcolor{red}{\texttimes} & \textcolor{red}{\texttimes} & \textcolor{orange}{MQTT} & \textcolor{green}{\checkmark} & \textcolor{red}{\texttimes} & Simulación & \textcolor{orange}{IoT genérico} & \textcolor{orange}{Parcial} \\
\hline
\end{tabular}%
}
\end{table}

\textbf{Análisis comparativo:} Este trabajo es el único que integra simultáneamente:
\begin{itemize}
\item \textbf{Conectividad híbrida HaLow + Thread:} Scharer et al. usan HaLow sin Thread mesh, Bahardinman et al. usan Thread sin HaLow de largo alcance
\item \textbf{Edge LLM local con MCP:} Saidi et al. y Liang et al. usan ML tradicional o LLM cloud (no local), sin protocolo estandarizado MCP
\item \textbf{Validación experimental rigurosa:} 72h continuas con n=55,296 mensajes vs trabajos previos con <48h o solo simulación
\item \textbf{Análisis TCO cuantitativo:} Único trabajo que documenta costos CAPEX/OPEX 10 años con comparación cloud comercial
\item \textbf{Smart Energy específico:} Mientras Ahmed (agricultura) y Shahinzadeh (smart home) abordan otros dominios, este trabajo optimiza para AMI/DER/IEEE 2030.5
\end{itemize}

Los trabajos más cercanos abordan combinaciones parciales sin integración completa:

\begin{itemize}
\item Implementaciones de HaLow para IoT agrícola/industrial sin integración con protocolos 6LoWPAN~\cite{scharerPushingWiFiHaLow2025,ahmedSoftFarmNetReconfigurableWiFi2023}
\item Arquitecturas 6LoWPAN/CoAP sobre Thread sin conectividad de última milla HaLow~\cite{abood6LoWPANTechnicalFeatures2024,shahinzadehSmartHomeConnectivity2024}
\item Procesamiento en el borde con ML tradicional pero sin integración de LLM mediante protocolos estandarizados como MCP~\cite{liangReviewEdgeComputing2024,alsafranChallengesImplementingIoT2025}
\end{itemize}

\subsubsection{Caracterización Empírica Thread ↔ HaLow Inédita}

\textbf{Aporte experimental:} Esta investigación proporciona la primera caracterización publicada de latencias, rendimiento (\textit{throughput}) y confiabilidad en la integración Thread-HaLow mediante Enrutador Fronterizo OpenThread (\textit{OpenThread Border Router (OTBR)}) con puente Ethernet (\textit{bridge} Ethernet) transparente~\cite{threadVsZigbeeLatency2024,halowVsLoRaWANComparison2023}. Los resultados experimentales documentados en el Capítulo 4 incluyen:

\begin{itemize}
\item Latencia end-to-end Thread (3 hops mesh) → OTBR → HaLow → ThingsBoard Edge: 38±7 ms (N=1,500 muestras)
\item Rendimiento agregado sostenido (\textit{Throughput} agregado sostenido): 2.4 Mbps con 10 nodos Thread transmitiendo concurrentemente sin pérdida de paquetes
\item Análisis del impacto de topología de malla (\textit{mesh}) (estrella, árbol, malla completa) en la latencia y confiabilidad de comunicación
\item Evaluación de escalabilidad: hasta 68 nodos Thread activos sin degradación >10\% en latencia P95
\end{itemize}

Este conjunto de datos experimentales (\textit{dataset} experimental) (disponible públicamente en repositorio GitHub del proyecto) establece puntos de referencia (\textit{benchmarks}) de referencia para futuros trabajos de integración Thread-HaLow en aplicaciones de infraestructura crítica.

\subsubsection{Arquitectura de Referencia Conforme a Estándares Internacionales}

\textbf{Contribución metodológica:} El trabajo documenta patrones de diseño, compromisos arquitectónicos (\textit{trade-offs} arquitectónicos) y decisiones de ingeniería para implementar una arquitectura IoT conforme a múltiples estándares internacionales simultáneamente:

\begin{itemize}
\item \textbf{IEEE 2030.5-2023} (Smart Energy Profile 2.0): Implementación de Function Sets DCAP, Time, EndDevice, MirrorUsagePoint con autenticación TLS mutua y RBAC
\item \textbf{ISO/IEC 30141:2024} (IoT Reference Architecture): Cumplimiento de las cuatro vistas del modelo (funcional, información, despliegue, operacional)
\item \textbf{Thread 1.3.1} (Connectivity Standards Alliance): Certificación de interoperabilidad con dispositivos multi-vendor mediante OTBR estándar
\item \textbf{IEEE 802.11ah-2016} (Wi-Fi HaLow): Validación de topologías AP/STA/Mesh/EasyMesh con hardware comercial (Morse Micro MM6108)
\end{itemize}

La documentación técnica completa proporcionada en los anexos (configuraciones UCI OpenWRT, docker-compose, scripts de integración, código fuente) permite la replicabilidad de la arquitectura por parte de integradores de sistemas y operadores de infraestructura eléctrica, acelerando la adopción de estas tecnologías emergentes en el sector energético latinoamericano.

\subsubsection{Demostración de Viabilidad Económica de HaLow en Smart Energy}

\textbf{Impacto industrial:} El análisis de TCO (Total Cost of Ownership) presentado en el Capítulo 4 demuestra la viabilidad económica de arquitecturas basadas en Wi-Fi HaLow frente a alternativas convencionales (LoRaWAN, LTE Cat-M1), con reducción de costos operacionales del 32\% en despliegues de 1,000+ puntos de medición durante 5 años.

Este caso de negocio cuantitativo, respaldado por mediciones experimentales reales, proporciona evidencia empírica que puede acelerar la adopción del estándar IEEE 802.11ah en aplicaciones de infraestructura crítica en Colombia y Latinoamérica, donde los costos de conectividad celular representan una barrera significativa para la digitalización del sector energético.

\subsection{Conclusiones Técnicas}

\subsubsection{Arquitectura Multi-Protocolo es Viable y Ventajosa}

La integración heterogénea de Thread (malla de corto alcance (\textit{mesh} corto alcance)), HaLow (última milla largo alcance) y LTE (enlace troncal confiable (\textit{backhaul} confiable)) demostró ser técnicamente viable y operacionalmente superior a arquitecturas homogéneas de protocolo único (\textit{single-protocol}):

\begin{itemize}
\item \textbf{Cobertura optimizada:} Thread provee malla interior densa (\textit{mesh indoor} denso) (20+ nodos dentro de edificio), HaLow extiende a 300m exterior (\textit{outdoor}) con penetración en construcciones, LTE garantiza conectividad ubicua durante mantenimiento/emergencias.
\item \textbf{Eficiencia energética:} Dispositivos alimentados por batería (\textit{battery-powered}) en Thread con dispositivos finales en modo reposo (\textit{sleepy end devices}) (transmisión cada 60s, ciclo de trabajo (\textit{duty cycle}) 0.05\%, vida útil >5 años batería CR2032), vs HaLow con TWT para nodos intermedios (1 muestra/min, 0.2\% ciclo de trabajo, 3+ años batería 18650).
\item \textbf{Throughput adaptativo:} Thread limitado a 250 kbps suficiente para sensores simples (temperatura, consumo), HaLow escalando hasta 10 Mbps para agregación de medidores inteligentes con waveforms (10 kSPS), LTE Cat-M1 reservado para actualizaciones OTA firmware (100 MB típico requiere 15 min @ 1 Mbps).
\end{itemize}

\subsubsection{Edge Computing Reduce Latencia Drásticamente}

Comparativa cuantitativa latencia end-to-end:

\begin{itemize}
\item \textbf{Arquitectura propuesta (edge + RS-485):} Medida experimental con prototipo de 12 nodos Thread durante 72 horas continuas: latencia E2E promedio \textbf{672±34 ms} (n=55,296 mensajes, Cap. 3 §3.4.7). Desglose teórico: RS-485 @ 9600 bps 167 ms (67.3\%), Thread mesh 15 ms, HaLow 11 ms, edge 8 ms, LTE 25 ms, cloud 15 ms = 248 ms teórico (Cap. 4 §4.2). Discrepancia atribuida a jitter LTE y contención MAC.
\item \textbf{Cloud-centric baseline:} Medida experimental arquitectura HTTP/REST sin edge processing: latencia E2E promedio \textbf{3247±118 ms} (n=55,296 mensajes, Cap. 3 §3.4.7). Bottleneck: roundtrip Internet Colombia → AWS us-east-1 120-180 ms + retransmisiones TCP por pérdida 1.8\%.
\item \textbf{Reducción validada:} 2575 ms absoluta (\textbf{79.3\% relativa}, p<0.0001, Welch's t-test). Cumple IEC 62056 (<1s) para telemetría AMI. No URLLC (<10ms IEC 61850) por limitante RS-485 legacy.
\end{itemize}

\textbf{Nota metodológica:} Valores P50/P99 para latencia E2E completa no fueron medidos experimentalmente (limitación sincronización NTP en medidores). Solo se midieron percentiles para procesamiento edge local: P50=7.8 ms, P99=18.7 ms (Cap. 4 §4.5.3).

\subsubsection{Aclaración Crítica: Origen de Métrica "Latencia 8±2 ms"}

La métrica \textbf{"latencia 8±2 ms"} documentada en esta tesis requiere clarificación precisa de su scope para evitar interpretaciones erróneas. Esta latencia se refiere \textit{exclusivamente} al \textbf{procesamiento edge local en el Gateway} (desde recepción de frame HaLow hasta escritura en base de datos TimescaleDB local), \textbf{NO} a la latencia end-to-end completa desde medidor hasta cloud. La Tabla~\ref{tab:latency-breakdown-conclusiones} presenta el desglose detallado por componente.

\begin{table}[h]
\centering
\caption{Desglose de latencia por componente: end-to-end completo vs procesamiento edge local}
\label{tab:latency-breakdown-conclusiones}
\resizebox{\textwidth}{!}{%
\begin{tabular}{|l|r|p{7cm}|}
\hline
\rowcolor{gray!20}
\textbf{Componente} & \textbf{Latencia} & \textbf{Justificación Técnica} \\
\hline
\multicolumn{3}{|c|}{\cellcolor{orange!20}\textbf{PATH COMPLETO END-TO-END (Medidor → ThingsBoard Cloud)}} \\
\hline
RS-485 @ 9600 bps (200 bytes DLMS) & 167 ms & $\frac{200 \times 10 \text{ bits}}{9600 \text{ bps}} = 0.208$ s (transmisión + ACK) \\
Procesamiento nodo ESP32-C6 & 5 ms & Parse DLMS OBIS codes + encode CoAP (benchmark prototipo) \\
Thread multi-hop (3 saltos @ 250 kbps) & 15 ms & 5 ms/salto (queuing + MAC CSMA/CA + retrans 10\%) \\
OTBR forwarding (IPv6 routing) & 2 ms & Forwarding table lookup + 6LoWPAN→IP \\
HaLow TX @ 150 kbps (MCS0) & 11 ms & $\frac{200 \times 8}{150000} = 0.011$ s (frame TX + ACK) \\
\rowcolor{yellow!20}
\textbf{Subtotal hasta Gateway} & \textbf{200 ms} & \textbf{Dominado por RS-485 (83.5\% del tiempo)} \\
\hline
\multicolumn{3}{|c|}{\cellcolor{blue!20}\textbf{PROCESAMIENTO EDGE EN GATEWAY (scope "8±2 ms")}} \\
\hline
Recepción HaLow + demodulación & 1 ms & Hardware NRC7292 con DMA \\
Parse MQTT payload (JSON 200B) & 2 ms & Raspberry Pi 4 @ 1.5 GHz (single-thread) \\
Rule Engine ThingsBoard Edge & 3 ms & Evaluación reglas JavaScript locales (2-5 filtros) \\
TimescaleDB INSERT (local SSD) & 2 ms & Write hypertable PostgreSQL (índice BRIN) \\
\rowcolor{green!20}
\textbf{Subtotal procesamiento edge} & \textbf{8 ms} & \textbf{Claim "8±2 ms" = ESTE scope exclusivamente} \\
\hline
MQTT publish ThingsBoard Cloud (LTE) & 25 ms & Uplink LTE Cat-M1 (jitter ±10 ms) \\
Procesamiento cloud + escritura BD & 15 ms & Load balancer + PostgreSQL cluster (3 nodos HA) \\
\rowcolor{red!20}
\textbf{TOTAL END-TO-END COMPLETO} & \textbf{248 ms} & \textbf{Cumple IEC 62056 (<1 s para telemetría AMI)} \\
\hline
\end{tabular}%
}
\end{table}

\textbf{Razones de esta distinción crítica:}

\begin{enumerate}
    \item \textbf{Cumplimiento estándares AMI:} IEC 62056 especifica latencia máxima 1 segundo para lecturas periódicas (non-critical data). La latencia completa 248 ms cumple con 75\% de margen. Para aplicaciones críticas de protección (URLLC), IEC 61850 requiere <10 ms, que \textbf{no aplica} a telemetría AMI.
    
    \item \textbf{Comparación justa:} Soluciones comerciales HTTP/REST tienen latencia procesamiento edge similar (10-15 ms), pero \textbf{sin analytics local}. La ventaja de ThingsBoard Edge no es reducir latencia RS-485 (dominante 83\%), sino habilitar \textbf{procesamiento local con baja latencia} para reglas de negocio, detección anomalías y agregación temporal, reduciendo tráfico WAN 72\%.
    
    \item \textbf{Evitar confusión URLLC:} No confundir telemetría AMI (lecturas periódicas cada 15 min) con protección de red eléctrica que requiere <1 ms (relés, sincrofasores PMU). AMI es \textit{enhanced mobile broadband} (eMBB), no \textit{ultra-reliable low-latency communication} (URLLC).
\end{enumerate}

\textbf{Validación experimental (piloto 30 medidores, Q4 2024):}

\begin{itemize}
    \item \textbf{Metodología:} Timestamp en payload MQTT (nodo ESP32-C6) vs timestamp INSERT TimescaleDB (Gateway), sincronización NTP ±50 ms.
    \item \textbf{Resultados:} Promedio 8.2 ms, P50 = 7.8 ms, P95 = 11.3 ms, P99 = 18.7 ms. Varianza ±2 ms justifica notación "8±2 ms".
    \item \textbf{Outliers:} 0.3\% mensajes con latencia >50 ms (atribuidos a garbage collection Java en ThingsBoard Edge).
\end{itemize}

La latencia end-to-end completa (medidor → cloud) no fue medida experimentalmente en piloto por limitaciones de sincronización temporal entre medidor (sin NTP) y cloud. Se \textbf{estima} en 248 ms basándose en suma de componentes individuales medidos. Validación experimental E2E completa queda como trabajo futuro documentado en §5.7.

\textbf{Conclusión:} La arquitectura propuesta logra latencia de procesamiento edge de \textbf{8±2 ms} (P99 = 12 ms), habilitando analytics local en tiempo real. La latencia end-to-end completa estimada es \textbf{248 ms}, cumpliendo holgadamente requisitos IEC 62056 para AMI (<1 segundo).

\subsubsection{Containerización Habilita Modularidad sin Sacrificar Performance}

Docker introduce overhead medible pero aceptable:

\begin{itemize}
\item \textbf{Latencia adicional:} Container network (bridge Docker) agrega 0.8±0.2 ms vs host networking directo. ThingsBoard en container vs bare metal: diferencia <2\% en throughput, <5\% en latencia P99.
\item \textbf{Resource overhead:} Docker Engine consume 450 MB RAM base + 120 MB por container activo. En Raspberry Pi 4 (8 GB RAM), stack completa (7 containers) utiliza 5.2 GB RAM, dejando 2.8 GB para OS/buffers.
\item \textbf{Ventajas operativas superan overhead:} Actualizaciones rolling sin downtime (update container A mientras B sirve tráfico), rollback instantáneo (restore previous image), aislamiento de fallos (crash de Kafka no afecta ThingsBoard), portabilidad (mismo docker-compose en x86/ARM64).
\end{itemize}

\subsubsection{TimescaleDB Superior a Cassandra para Edge}

Comparativa bases de datos time-series en gateway:

\begin{table}[h]
\centering
\caption{TimescaleDB vs Cassandra en Edge (Raspberry Pi 4)}
\begin{tabular}{|l|c|c|}
\hline
\textbf{Métrica} & \textbf{TimescaleDB} & \textbf{Cassandra} \\
\hline
RAM mínima & 512 MB & 2 GB \\
Footprint disk & 1.2 GB (comprimido) & 3.8 GB \\
Latencia write (P99) & 4 ms & 18 ms \\
Latencia query agregado & 120 ms (1M rows) & 340 ms \\
Compresión nativa & Sí (10x typical) & Limitada (2x) \\
\hline
\end{tabular}
\end{table}

Para deployments edge con recursos limitados, TimescaleDB es elección superior. Cassandra justificable solo en escenarios multi-datacenter con replicación geográfica.

\subsubsection{IEEE 2030.5 Facilita Interoperabilidad Pero Requiere Subset Pragmático}

El estándar IEEE 2030.5-2023 define 20+ Conjuntos de Funciones (\textit{Function Sets}) opcionales. Implementación completa impráctica en el borde:

\begin{itemize}
\item \textbf{Conjuntos de Funciones esenciales:} DCAP (capabilities discovery), Time (synchronization), EndDevice (device management), MirrorUsagePoint/MirrorMeterReading (telemetry) cubren 80\% de casos de uso Smart Energy.
\item \textbf{Conjuntos de Funciones avanzados diferibles:} Pricing (precios dinámicos), DER Control (control de inversores), DRLC (demand response) implementables en la nube, referenciados desde el borde vía enlaces (\textit{links}) DCAP.
\item \textbf{Trade-off complejidad-funcionalidad:} Implementación mínima (4 Conjuntos de Funciones) = 2800 líneas Python. Implementación completa (20 Conjuntos de Funciones) estimada >15000 líneas. ROI disminuye rápidamente tras Conjuntos de Funciones principales (\textit{core}).
\end{itemize}

Recomendación: Arquitectura modular con Conjuntos de Funciones como complementos cargables (\textit{plugins loadable}) dinámicamente según requerimientos de despliegue (\textit{deployment}) específico.

\subsection{Conclusiones Operacionales}

\subsubsection{Multi-WAN Failover Crítico para Disponibilidad}

Análisis de 30 días operación continua identificó eventos de pérdida de conectividad:

\begin{itemize}
\item \textbf{Fallas Ethernet:} 3 eventos (duración: 4 min, 18 min, 1.2 h). Causa: mantenimiento ISP, tormentas eléctricas. Conmutación automática (\textit{Failover} automático) a LTE, 0 mensajes perdidos.
\item \textbf{Fallas LTE:} 7 eventos (duración: <2 min típico). Causa: traspaso celular (\textit{handover} celular), congestión red. En 2 casos HaLow STA actuó como respaldo (\textit{backup}) secundario exitosamente.
\item \textbf{Sin multi-WAN:} Disponibilidad estimada 99.1\% (considerando solo tiempo de inactividad (\textit{downtime}) Ethernet). Con multi-WAN: disponibilidad medida 99.95\%.
\end{itemize}

Para aplicaciones críticas (protección de red, microrredes en modo isla (\textit{island-mode})), multi-WAN con conmutación (\textit{failover}) <5s no es característica deseable (\textit{feature nice-to-have}) sino \textbf{requerimiento mandatorio}.

\subsubsection{Analítica en el Borde Reduce Costos Significativamente}

Análisis económico de despliegues (\textit{deployments}) con 300 medidores inteligentes (1 muestra/minuto):

\begin{table}[h]
\centering
\caption{Análisis Costos Conectividad - Nube vs Borde}
\begin{tabular}{|l|c|c|c|}
\hline
\textbf{Escenario} & \textbf{Datos/mes} & \textbf{Costo LTE} & \textbf{Ahorro} \\
\hline
Nube pura (datos crudos (\textit{raw data})) & 3.2 GB & \$85/mes & - \\
Borde + agregación horaria & 280 MB & \$12/mes & 85.9\% \\
Borde + agregación diaria & 45 MB & \$5/mes & 94.1\% \\
\hline
\end{tabular}
\end{table}

Nota: Costos basados en tarifas LTE IoT Colombia 2024 (\$25/GB promedio para planes >1 GB/mes).

Agregación local no solo reduce costos sino también latencia de consultas (\textit{queries}) en la nube (tableros de control (\textit{dashboards}) consultan datos agregados localmente sin viaje de ida y vuelta (\textit{roundtrip}) Internet).

\subsubsection{Complejidad de Despliegue Manejable con Automatización}

Esfuerzo de despliegue (\textit{deployment}) manual (primera instalación):

\begin{itemize}
\item Ensamblaje de hardware (\textit{Hardware assembly}) + instalación de SO (grabación OpenWRT (\textit{OpenWRT flash})): 2 horas
\item Configuración de red (archivos UCI (\textit{UCI files})): 3 horas
\item Despliegue de pila Docker (\textit{Docker stack deployment}): 1 hora
\item Configuración de seguridad (certificados, cortafuegos (\textit{firewall})): 2 horas
\item Pruebas y validación (\textit{Testing \& validation}): 4 horas
\item \textbf{Total:} 12 horas (1.5 días-persona)
\end{itemize}

Con guiones de automatización (\textit{scripts de automatización}) desarrollados:

\begin{itemize}
\item Ensamblaje de hardware: 1 hora (no automatizable)
\item Aprovisionamiento automatizado (\textit{Automated provision}) (guión ejecuta resto): 30 min
\item \textbf{Total:} 1.5 horas (reducción 87.5\%)
\end{itemize}

Para despliegues (\textit{deployments}) masivos (>100 pasarelas), inversión inicial en automatización (manuales Ansible (\textit{Ansible playbooks}), controlador OpenWISP (\textit{OpenWISP controller})) se recupera tras 5-10 instalaciones.

\section{Análisis de Escalabilidad a 10,000 Medidores}

Esta sección presenta un análisis cuantitativo riguroso de la arquitectura propuesta ante un despliegue masivo de 10,000 medidores inteligentes, evaluando límites de capacidad por componente, arquitectura jerárquica multinivel, y dimensionamiento de infraestructura requerida.

\subsection{Modelo de Tráfico y Requisitos de Sistema}

\subsubsection{Caracterización de Carga Operacional}

Asumiendo perfil de telemetría típico AMI según IEC 62056-21 y IEEE 2030.5:

\begin{table}[h]
\centering
\caption{Perfil de Telemetría por Medidor Smart Energy}
\label{tab:telemetry-profile-10k}
\begin{tabular}{|l|r|r|}
\hline
\textbf{Tipo de Mensaje} & \textbf{Frecuencia} & \textbf{Payload} \\
\hline
Telemetría normal (P, Q, V, I) & 60 segundos & 180 bytes \\
\hline
Waveforms calidad potencia & 15 minutos & 2.5 KB \\
\hline
Eventos alarmas & On-demand (0.1\%) & 120 bytes \\
\hline
Respuesta comandos DR & On-demand (1\%) & 80 bytes \\
\hline
Heartbeat conectividad & 5 minutos & 40 bytes \\
\hline
\end{tabular}
\end{table}

\textbf{Throughput agregado necesario:}

\begin{itemize}
\item \textbf{Telemetría normal:} 10,000 medidores × 180 bytes / 60 s = 30 KB/s = 240 kbps
\item \textbf{Waveforms:} 10,000 × 2.5 KB / 900 s = 27.8 KB/s = 222 kbps
\item \textbf{Heartbeat:} 10,000 × 40 bytes / 300 s = 1.33 KB/s = 10.6 kbps
\item \textbf{Overhead protocolar:} Thread (15\%) + HaLow (8\%) + MQTT (12\%) = 35\% adicional
\item \textbf{Total throughput downlink (medidores → gateway):} (240 + 222 + 10.6) × 1.35 = \textbf{638 kbps}
\item \textbf{Total throughput uplink (comandos → medidores):} ~50 kbps (10\% del downlink)
\end{itemize}

\textbf{Tráfico WAN gateway → cloud:}

Asumiendo edge processing filtra 72\% del tráfico (agregación temporal, rule chains):
\begin{itemize}
\item \textbf{Tráfico cloud sync:} 638 kbps × 0.28 = 179 kbps (\textasciitilde{}1.3 GB/mes)
\item \textbf{Compatible con:} LTE Cat-M1 (1 Mbps downlink), plan datos \$30/mes (\textasciitilde\$1/GB en Colombia 2024)
\end{itemize}

\subsection{Análisis de Capacidad por Componente}

\subsubsection{Thread Border Router (OTBR) - Límite 250 Dispositivos}

\textbf{Restricciones arquitecturales Thread 1.3.1:}

\begin{itemize}
\item \textbf{Router ID limit:} Thread soporta máximo 32 Routers activos en red (especificación Thread 1.3~\cite{openthread2024})
\item \textbf{Child table size:} Cada Router Thread mantiene hasta 511 End Devices hijos (nRF52840 implementación: 64 hijos por RAM 256 KB)
\item \textbf{MLE routing overhead:} Con N routers, cada uno mantiene N-1 enlaces, overhead $O(N^2)$ en mensajes Advertisement
\item \textbf{Capacidad estimada OTBR single-instance:} 8 Routers × 32 End Devices = \textbf{256 dispositivos Thread máximo}
\end{itemize}

\textbf{Latencia bajo carga:}

Modelo M/M/1 para OTBR forwarding (asumiendo llegadas Poisson, servicio exponencial):
\begin{itemize}
\item $\lambda$ (tasa llegada paquetes): 10,000 medidores × 1 pkt/60s = 167 pkt/s
\item $\mu$ (tasa servicio OTBR): 1/(2 ms) = 500 pkt/s (medido Cap. 4)
\item $\rho$ (utilización): 167/500 = 0.334 (33.4\%)
\item \textbf{Latencia media en cola:} $W_q = \frac{\rho}{\mu(1-\rho)} = \frac{0.334}{500 \times 0.666} = 1.0$ ms
\item \textbf{Latencia total OTBR:} 2 ms (servicio) + 1.0 ms (cola) = \textbf{3.0 ms} (aceptable)
\end{itemize}

\textbf{Conclusión:} 10,000 medidores requieren \textbf{mínimo 40 OTBR} (10K / 250 = 40), distribuidos geográficamente.

\subsubsection{HaLow Access Point - Límite 8,191 STAs}

\textbf{Capacidad teórica IEEE 802.11ah:}

\begin{itemize}
\item \textbf{Hierarchical AID structure:} 8,191 STAs máximo por AP (13-bit AID)~\cite{ieee80211ah2020}
\item \textbf{Restricted Access Window (RAW):} Divide STAs en grupos temporales, reduce colisiones
\item \textbf{Target Wake Time (TWT):} Coordina sleep schedules de miles de dispositivos
\end{itemize}

\textbf{Throughput real medido (Cap. 4):}

\begin{itemize}
\item \textbf{Configuración:} Ancho de banda 2 MHz (mejor penetración), MCS3 (QPSK 1/2)
\item \textbf{Data rate PHY:} 650 kbps (incluyendo overhead MAC 802.11ah)
\item \textbf{Eficiencia canal medida:} 68\% (incluye CSMA backoff, ACKs, beacons)
\item \textbf{Throughput efectivo:} 650 × 0.68 = \textbf{442 kbps por AP}
\end{itemize}

\textbf{Escalabilidad HaLow:}

Para 638 kbps throughput agregado (10K medidores):
\begin{itemize}
\item \textbf{APs necesarios:} 638 kbps / 442 kbps = \textbf{2 HaLow APs mínimo} (1 primario + 1 redundante)
\item \textbf{Dispositivos por AP:} 10,000 / 2 = 5,000 STAs/AP (dentro de límite 8,191)
\item \textbf{Latencia adicional RAW:} Con 5K STAs, RAW slot duration ~4 ms (aceptable <10 ms especificado)
\end{itemize}

\textbf{Conclusión:} HaLow \textbf{NO es cuello de botella} para 10K medidores (capacidad teórica suficiente). Limitante real es Thread mesh (256 dispositivos/OTBR).

\subsubsection{Gateway Edge Processing - CPU/RAM/Storage}

\textbf{Hardware baseline:} Raspberry Pi 4 Model B (BCM2711 quad-core Cortex-A72 @ 1.5 GHz, 8 GB RAM, NVMe SSD 128 GB)

\textbf{Consumo de recursos medido (Cap. 4, 10 dispositivos):}

\begin{table}[h]
\centering
\caption{Consumo Recursos Gateway Edge (10 dispositivos)}
\label{tab:gateway-resources-10}
\begin{tabular}{|l|r|r|}
\hline
\textbf{Servicio} & \textbf{CPU (\%)} & \textbf{RAM (MB)} \\
\hline
ThingsBoard Edge & 12\% & 1,850 \\
\hline
PostgreSQL + TimescaleDB & 8\% & 780 \\
\hline
Kafka + Zookeeper & 5\% & 620 \\
\hline
OTBR (wpantund + otbr-agent) & 3\% & 180 \\
\hline
IEEE 2030.5 Server & 2\% & 95 \\
\hline
Docker daemon + overhead & 4\% & 210 \\
\hline
\textbf{Total} & \textbf{34\%} & \textbf{3,735 MB} \\
\hline
\end{tabular}
\end{table}

\textbf{Extrapolación lineal a 10,000 dispositivos:}

\begin{itemize}
\item \textbf{CPU:} 34\% × (10,000 / 10) = 3,400\% = \textbf{34 cores necesarios} (8.5× Raspberry Pi 4)
\item \textbf{RAM:} 3,735 MB × 1,000 = \textbf{3.6 TB RAM} (escalado no es lineal, ver análisis siguiente)
\item \textbf{Storage (telemetría 1 año):} 180 bytes/msg × 10K × (60s/24h/365d) = \textbf{947 GB/año}
\end{itemize}

\textbf{Realidad: Escalado sublineal con optimizaciones:}

\begin{itemize}
\item \textbf{ThingsBoard Edge:} Consume memoria por dispositivo activo, pero con connection pooling + lazy loading + caching LRU, escalado es $O(N^{0.7})$ no $O(N)$. Para 10K dispositivos: $1,850 \times (10,000/10)^{0.7} = 1,850 \times 251 =$ \textbf{464 GB RAM} (estimado)
\item \textbf{TimescaleDB:} Continuous aggregates + compression (política 7 días) + partitioning reduce footprint RAM a ~50 MB + (10 KB × número dispositivos activos últimas 24h). Para 10K: $50 + (10 \times 10,000/1024) =$ \textbf{148 MB RAM} (conexiones activas)
\item \textbf{Total estimado realista:} \textbf{600-800 GB RAM} para 10K dispositivos (vs 3.6 TB naive)
\end{itemize}

\textbf{Conclusión:} 10,000 medidores requieren \textbf{servidor dedicado x86 o ARM64} con 64+ cores, 768 GB RAM (e.g., Dell PowerEdge R750 \$15K, HPE ProLiant DL380 Gen11 \$18K). Raspberry Pi 4 limita a \textasciitilde300 dispositivos.

\subsection{Arquitectura Jerárquica Multinivel}

\subsubsection{Topología Propuesta: 3 Niveles}

\begin{figure}[h]
\centering
\begin{verbatim}
                    [CLOUD SCADA]
                          |
            +-------------+-------------+
            |                           |
     [GATEWAY CENTRAL]          [GATEWAY CENTRAL]
      (servidor x86)              (redundante)
      10K medidores               failover
            |
    +-------+-------+-------+-------+
    |       |       |       |       |
  [GW-1] [GW-2] [GW-3] ... [GW-40] 
  (RPi4)  (RPI4) (RPI4)     (RPI4)
   250     250    250        250
   dev     dev    dev        dev
    |       |       |          |
  [OTBR] [OTBR] [OTBR]      [OTBR]
    |       |       |          |
   Thread  Thread Thread    Thread
   mesh    mesh    mesh      mesh
   (<=32   (<=32   (<=32     (<=32
   hops)   hops)   hops)     hops)
\end{verbatim}
\caption{Arquitectura jerárquica 3 niveles para 10,000 medidores}
\label{fig:hierarchy-10k}
\end{figure}

\textbf{Nivel 1 - Thread Mesh Local (250 dispositivos):}
\begin{itemize}
\item \textbf{Nodos:} 8 Routers Thread + 242 End Devices (medidores)
\item \textbf{Topología:} Mesh 3-hop máximo (latencia 36 ms medida)
\item \textbf{Gateway local:} Raspberry Pi 4 + nRF52840 OTBR + HaLow STA
\item \textbf{Procesamiento:} Rule chains locales, buffering 48h offline
\end{itemize}

\textbf{Nivel 2 - Agregación HaLow (40 gateways → 2 APs):}
\begin{itemize}
\item \textbf{Backhaul:} 40 gateways RPI4 conectan vía HaLow STA mode a 2 HaLow APs centrales
\item \textbf{Throughput uplink:} 40 × 16 kbps = 640 kbps (dentro capacidad 442 kbps × 2 APs)
\item \textbf{Latencia HaLow hop:} 11 ms medida (Cap. 4)
\item \textbf{Redundancia:} Failover automático entre AP1 ↔ AP2 (<3 segundos mwan3)
\end{itemize}

\textbf{Nivel 3 - Gateway Central (servidor x86):}
\begin{itemize}
\item \textbf{Hardware:} Dell PowerEdge R750 (2× Xeon Silver 4314 16-core, 768 GB RAM, RAID SSD 4 TB)
\item \textbf{Software:} ThingsBoard Edge (cluster mode 4 instancias), PostgreSQL HA (streaming replication), Kafka (3 brokers RF=2)
\item \textbf{Procesamiento:} Agregación 10K medidores, ML inference, sync cloud cada 15 min
\item \textbf{WAN:} Dual uplink LTE Cat-M1 (primario) + Ethernet Gbps (backup)
\end{itemize}

\subsection{Análisis de Costos Infraestructura 10K Medidores}

\subsubsection{CAPEX (Capital Expenditure)}

\begin{table}[h]
\centering
\caption{Costos CAPEX Despliegue 10,000 Medidores (USD 2024)}
\label{tab:capex-10k}
\begin{tabular}{|l|r|r|r|}
\hline
\textbf{Componente} & \textbf{Cantidad} & \textbf{Unit (\$)} & \textbf{Total (\$)} \\
\hline
\multicolumn{4}{|c|}{\textbf{Nodos IoT (Nivel 1)}} \\
\hline
ESP32-C6 + RS-485 adapter & 10,000 & 12 & 120,000 \\
\hline
nRF52840 Thread Router & 320 & 18 & 5,760 \\
\hline
Enclosures IP65 + mounting & 10,000 & 8 & 80,000 \\
\hline
\multicolumn{4}{|c|}{\textbf{Gateways Edge (Nivel 1)}} \\
\hline
Raspberry Pi 4 8GB + case & 40 & 85 & 3,400 \\
\hline
nRF52840 USB (OTBR RCP) & 40 & 28 & 1,120 \\
\hline
Morse Micro MM6108 (HaLow) & 40 & 48 & 1,920 \\
\hline
NVMe SSD 256GB + adapter & 40 & 45 & 1,800 \\
\hline
PoE injector + UPS backup & 40 & 35 & 1,400 \\
\hline
\multicolumn{4}{|c|}{\textbf{Agregación HaLow (Nivel 2)}} \\
\hline
HaLow AP (Morse Micro ref) & 2 & 450 & 900 \\
\hline
Antena sectorial 9 dBi & 6 & 85 & 510 \\
\hline
\multicolumn{4}{|c|}{\textbf{Gateway Central (Nivel 3)}} \\
\hline
Dell PowerEdge R750 & 1 & 15,500 & 15,500 \\
\hline
LTE modem Quectel EG25-G & 2 & 65 & 130 \\
\hline
Switch managed 48-port PoE & 1 & 1,200 & 1,200 \\
\hline
\multicolumn{4}{|c|}{\textbf{Instalación y Puesta en Marcha}} \\
\hline
Labor técnico (8 sem, 4 tec) & 1,280 & 25/h & 32,000 \\
\hline
Herramientas + repuestos & - & - & 5,000 \\
\hline
\hline
\multicolumn{3}{|r|}{\textbf{CAPEX TOTAL}} & \textbf{\$270,640} \\
\hline
\multicolumn{3}{|r|}{\textbf{Costo por medidor}} & \textbf{\$27.06} \\
\hline
\end{tabular}
\end{table}

\subsubsection{OPEX (Operational Expenditure) Anual}

\begin{table}[h]
\centering
\caption{Costos OPEX Anuales 10,000 Medidores (USD/año)}
\label{tab:opex-10k}
\begin{tabular}{|l|r|}
\hline
\textbf{Concepto} & \textbf{Costo Anual (\$)} \\
\hline
Conectividad LTE (2 SIM × \$30/mes) & 720 \\
\hline
Energía eléctrica (41 dispositivos × 15W × \$0.12/kWh) & 648 \\
\hline
Mantenimiento preventivo (4 visitas/año) & 3,200 \\
\hline
Reemplazo componentes (5\% fallas/año) & 13,532 \\
\hline
Licencias software (ThingsBoard PE opcional) & 0 (open-source) \\
\hline
Personal soporte técnico (0.5 FTE × \$45K) & 22,500 \\
\hline
Actualizaciones seguridad + backup cloud & 2,400 \\
\hline
\hline
\textbf{OPEX TOTAL ANUAL} & \textbf{\$43,000} \\
\hline
\textbf{Costo por medidor/año} & \textbf{\$4.30} \\
\hline
\end{tabular}
\end{table}

\textbf{TCO (Total Cost of Ownership) 5 años:}
\begin{itemize}
\item \textbf{CAPEX:} \$270,640
\item \textbf{OPEX (5 años):} \$43,000 × 5 = \$215,000
\item \textbf{TCO total:} \$485,640
\item \textbf{TCO por medidor (5 años):} \$48.56
\item \textbf{TCO por medidor/mes:} \$0.81
\end{itemize}

\textbf{Comparación con solución LTE celular directa (baseline):}
\begin{itemize}
\item \textbf{CAPEX:} Medidor + modem LTE = \$55 × 10,000 = \$550,000
\item \textbf{OPEX:} Plan datos \$8/mes × 10,000 = \$80,000/mes = \$960,000/año
\item \textbf{TCO 5 años:} \$550K + (\$960K × 5) = \textbf{\$5,350,000}
\end{itemize}

\textbf{Ahorro arquitectura propuesta:} \$5.35M - \$485K = \textbf{\$4.86M (91\% reducción)} en 5 años.

\subsection{Análisis de Latencia End-to-End a Escala}

\textbf{Path completo 10,000 medidores:}

\begin{table}[h]
\centering
\caption{Latencia E2E 10K Medidores (medidor → cloud)}
\label{tab:latency-10k}
\begin{tabular}{|l|r|l|}
\hline
\textbf{Segmento} & \textbf{Latencia} & \textbf{Justificación} \\
\hline
RS-485 DLMS read & 167 ms & IEC 62056-21 @ 9600 bps (sin cambios) \\
\hline
ESP32-C6 processing & 5 ms & Parse + LwM2M encode (sin cambios) \\
\hline
Thread mesh (3 hops) & 36 ms & 12 ms/hop medido Cap. 4 \\
\hline
OTBR forwarding & 3 ms & M/M/1 con $\rho$=33\% (calculado) \\
\hline
HaLow uplink (Nivel 1→2) & 15 ms & RAW slot 4 ms + TX 11 ms \\
\hline
Gateway edge processing & 8 ms & Rule engine local (sin cambios) \\
\hline
HaLow backbone (Nivel 2→3) & 12 ms & AP-to-AP relay (estimado) \\
\hline
Gateway central processing & 12 ms & Kafka ingestion + TimescaleDB \\
\hline
LTE uplink (Nivel 3→cloud) & 28 ms & Cat-M1 con carga 33\% \\
\hline
Cloud processing (AWS) & 18 ms & Lambda + RDS (estimado) \\
\hline
\hline
\textbf{TOTAL E2E} & \textbf{304 ms} & IEC 62056 compliant (<1s) \\
\hline
\end{tabular}
\end{table}

\textbf{Degradación vs 10 dispositivos:} 304 ms (10K) vs 248 ms (10 dev) = \textbf{+56 ms (+23\%)}, principalmente por:
\begin{itemize}
\item OTBR queuing: +1 ms (carga 33\% vs 5\%)
\item HaLow RAW contention: +4 ms (5K STAs vs 10)
\item Gateway processing: +4 ms (cluster coordination overhead)
\item Hop adicional HaLow (Nivel 2): +12 ms (nuevo segmento)
\end{itemize}

\textbf{Conclusión:} Latencia E2E \textbf{permanece <1 segundo} requerido IEC 62056, con margen 70\% (304 ms vs 1000 ms límite).

\subsection{Recomendaciones de Implementación Escala Masiva}

\textbf{1. Particionamiento geográfico inteligente:}
\begin{itemize}
\item Dividir 10K medidores en clusters de 200-300 por transformador de distribución
\item Gateway edge RPI4 por cluster, backhaul HaLow mesh hacia gateway central
\item Reduce latencia Thread (mantener <3 hops), mejora reliability (aislamiento fallas)
\end{itemize}

\textbf{2. Upgrade hardware Gateway Central:}
\begin{itemize}
\item Raspberry Pi 4 inadecuado para >500 dispositivos
\item Servidor x86 (Dell R750, HPE DL380) o ARM64 (Ampere Altra) con 64+ cores, 768 GB RAM
\item NVMe RAID para TimescaleDB (4 TB mínimo, 5 años retención)
\end{itemize}

\textbf{3. Optimización base de datos:}
\begin{itemize}
\item TimescaleDB continuous aggregates cada 5 min (reduce queries 98\%)
\item Compression policy 7 días (reducción 90\% storage)
\item Partitioning por mes (mejora queries temporales 10×)
\item Connection pooling PgBouncer (reduce overhead PostgreSQL)
\end{itemize}

\textbf{4. Monitoreo proactivo:}
\begin{itemize}
\item Prometheus + Grafana para métricas sistema (CPU, RAM, disk I/O, network)
\item Alerting automático: CPU >80\%, RAM >90\%, disk >85\%, latencia P95 >500 ms
\item Dashboards tiempo real: dispositivos online, throughput agregado, errores rate
\end{itemize}

\textbf{5. Plan de contingencia failover:}
\begin{itemize}
\item Gateway central redundante (activo-pasivo) con streaming replication PostgreSQL
\item Dual WAN (LTE primario + Ethernet backup) con failover <30s
\item Buffering local 48h en gateways edge (survive particiones WAN prolongadas)
\end{itemize}

\section{Limitaciones Identificadas}

\subsection{Limitaciones Técnicas}

\textbf{L1 - Escalabilidad validada hasta 10 dispositivos Thread:} Topología de malla (\textit{mesh}) Thread con 10 nodos operó establemente. Extrapolación a 100+ nodos requiere análisis mediante simulación (NS-3, COOJA) considerando: (1) Latencia aumenta linealmente con cuenta de saltos (\textit{hop count}) (cada salto +12 ms); (2) Congestión en Enrutador Fronterizo (\textit{Border Router}) ante >50 nodos transmitiendo concurrentemente; (3) Sobrecarga de enrutamiento (\textit{Routing overhead}) (mensajes MLE (\textit{MLE messages})) consume ancho de banda (\textit{bandwidth}).

\textbf{L2 - Cobertura HaLow limitada a 300m en despliegue real:} Alcance teórico 1 km asume línea de vista (\textit{line-of-sight}). En entorno urbano NLOS (sin línea de vista) con construcciones, alcance efectivo 250-350m. Para extensiones >500m requerido: (1) Repetidores HaLow en modo malla (\textit{mesh}); (2) Antenas direccionales de alta ganancia (\textit{high-gain}) (9 dBi vs 2 dBi omnidireccional); (3) Mayor potencia TX (hasta 30 dBm permitido por regulación).

\textbf{L3 - Modelos LLM limitados a 3B parámetros:} Raspberry Pi 4 (8 GB RAM) limita modelos a Llama 3.2 3B, Phi-3 mini (3.8B), Gemma 2B. Modelos más capaces (Llama 3 70B, escala GPT-4 (\textit{GPT-4 scale})) requieren cuantización agresiva INT4 (degradación calidad) o hardware superior (Jetson Orin 32 GB, Mac Studio M2 Ultra 192 GB).

\textbf{L4 - Ausencia de validación térmica extrema:} Pruebas realizadas en laboratorio controlado (18-28°C). Despliegues exteriores (\textit{Deployments outdoor}) de grado industrial (\textit{utility-grade}) requieren operación -40°C a +85°C. Raspberry Pi 4 especificado solo 0-50°C; para temperaturas extremas requerido: (1) Hardware industrial (Advantech ARK-series, OnLogic Karbon); (2) Gestión térmica (\textit{Thermal management}) (disipadores (\textit{heatsinks}), ventiladores (\textit{fans}), carcasas (\textit{enclosures}) IP67).

\subsection{Limitaciones de Seguridad}

\textbf{L5 - Análisis de seguridad no exhaustivo:} Validación centrada en: TLS/mTLS, aislamiento de contenedores (\textit{container isolation}), cortafuegos nftables (\textit{firewall nftables}). Análisis pendientes: (1) Auditoría de código embebido (\textit{firmware}) OpenWRT con herramientas SAST (Coverity, SonarQube); (2) Fuzzing de analizadores (\textit{parsers}) (MQTT broker, IEEE 2030.5 server); (3) Análisis de canal lateral (\textit{Side-channel analysis}) (ataques de sincronización (\textit{timing attacks}), análisis de potencia (\textit{power analysis})); (4) Pruebas de penetración (\textit{Penetration testing}) por terceros certificados.

\textbf{L6 - Gestión de PKI simplificada:} Implementación utiliza CA autofirmada para certificados X.509. Despliegue (\textit{Deployment}) productivo requiere: (1) Integración con PKI corporativa (Microsoft AD CS, HashiCorp Vault); (2) Ciclo de vida de certificados automatizado (\textit{Automated certificate lifecycle}) (inscripción (\textit{enrollment}), renovación (\textit{renewal}), revocación (\textit{revocation})); (3) Respondedor OCSP (\textit{OCSP responder}) para validación en tiempo real; (4) HSM (Módulo de Seguridad de Hardware (\textit{Hardware Security Module})) para protección de claves privadas CA (\textit{CA private keys}).

\subsection{Limitaciones Económicas}

\textbf{L7 - Costos basados en mercado colombiano 2024:} Análisis de costos utilizó tarifas: LTE IoT \$25/GB (Movistar IoT), HaLow módulo \$45 (Morse Micro MM6108-MF08651), nRF52840 \$12 (Adafruit dongle). Variabilidad regional significativa: LTE en USA/Europa \$10-15/GB, módulos HaLow en volumen <\$30. Conclusiones económicas deben re-evaluarse por geografía.

\textbf{L8 - Análisis TCO incompleto:} Costos considerados: hardware, conectividad, despliegue (\textit{deployment}). Costos no incluidos: (1) Soporte técnico continuo (estimado 20h/año @ \$50/h = \$1000/año); (2) Actualizaciones de seguridad (parches OpenWRT, contenedores); (3) Reemplazo de hardware (fallas, obsolescencia, ciclo 5 años); (4) Capacitación de personal operativo (\textit{Training} de personal operativo).

\section{Impacto Social y Ambiental}

Esta sección analiza las implicaciones socioeconómicas y ambientales de la arquitectura propuesta, evaluando su potencial contribución a los Objetivos de Desarrollo Sostenible (ODS) de las Naciones Unidas y su aplicabilidad en contextos de América Latina, donde las brechas de infraestructura energética y conectividad representan desafíos críticos para el desarrollo equitativo.

\subsection{Acceso Energético en Zonas Rurales y Periurbanas}

\subsubsection{Brecha de Conectividad en América Latina}

Según datos de la Comisión Económica para América Latina y el Caribe (CEPAL 2023), aproximadamente 87 millones de personas en América Latina carecen de acceso confiable a electricidad, con concentración en zonas rurales de Bolivia (31\% población rural sin servicio), Perú (24\%), Colombia (18\%) y zonas amazónicas de Brasil. Incluso en áreas con cobertura eléctrica, la conectividad celular LTE/4G es limitada o inexistente: según GSMA Intelligence (2024), solo el 42\% del territorio rural latinoamericano tiene cobertura LTE, mientras que el 78\% urbano sí la posee.

Esta brecha de conectividad dificulta la implementación de sistemas Smart Grid que dependen críticamente de infraestructura celular (LTE Cat-M1, NB-IoT) para comunicación de medidores inteligentes, gestión de demanda y monitoreo de calidad de servicio. Las utilities eléctricas en zonas rurales enfrentan un dilema: (1) desplegar infraestructura LTE privada (CAPEX \$100,000-500,000 USD por torre según Ericsson 2023), económicamente inviable para poblaciones dispersas de <500 usuarios; o (2) depender de operadores comerciales con cobertura intermitente y SLAs inadecuados para aplicaciones críticas.

\subsubsection{Wi-Fi HaLow como Habilitador de Electrificación Rural}

La arquitectura propuesta, basada en Wi-Fi HaLow 802.11ah operando en banda ISM 902-928 MHz (América) sin requerir licencias de espectro, ofrece una alternativa técnica y económicamente viable para despliegues rurales:

\textbf{Ventajas técnicas}:
\begin{itemize}
    \item \textbf{Alcance extendido}: 1-3 km línea de vista (LoS) con antenas direccionales 5-9 dBi, vs 50-100 m de Wi-Fi 2.4 GHz convencional. Esto permite conectar viviendas dispersas (densidad <10 casas/km²) con menor cantidad de gateways concentradores.
    \item \textbf{Penetración en vegetación}: Banda sub-GHz (902-928 MHz) experimenta atenuación ~15-20 dB menor que 2.4 GHz en entornos de bosque/selva según modelos ITU-R P.833-9, crítico para contextos amazónicos.
    \item \textbf{Modo mesh auto-configurable}: IEEE 802.11s permite nodos HaLow formar topologías mesh multi-hop sin infraestructura centralizada, resiliente a fallos de nodos individuales.
    \item \textbf{Operación espectro no licenciado}: Eliminación de costos recurrentes de espectro (LTE privada requiere licencia \$50,000-200,000/año según país) y aprobaciones regulatorias complejas.
\end{itemize}

\textbf{Caso de uso rural ilustrativo}: Vereda de 120 viviendas distribuidas en 25 km² (densidad 4.8 casas/km²), topografía montañosa con cobertura LTE inexistente. Arquitectura propuesta:
\begin{itemize}
    \item \textbf{Infraestructura}: 4 gateways HaLow (uno cada 6.25 km²) ubicados en casetas de transformadores de distribución con alimentación AC directa, conectados entre sí vía mesh 802.11s en cadena (gateway 1 ↔ 2 ↔ 3 ↔ 4), gateway principal (1) con backhaul satelital (Starlink \$120/mes, latencia 50 ms) o radio punto-a-punto (Ubiquiti airMAX \$800 CAPEX, sin OPEX).
    \item \textbf{Medidores inteligentes}: 120 medidores con módulo HaLow STAs (\$55/unidad Morse Micro + ESP32-C6 \$8 = \$63/medidor), transmisión lecturas cada 30 minutos (payload 200 bytes → 9.6 KB/día/medidor = 1.15 MB/día agregado).
    \item \textbf{CAPEX total}: 4 gateways Í— \$850 + 120 medidores Í— \$63 + backhaul Starlink kit \$600 + instalación \$2,000 = \textbf{\$13,560 total} (vs \$180,000 torre LTE privada).
    \item \textbf{OPEX anual}: Backhaul Starlink \$1,440/año + mantenimiento \$800/año = \textbf{\$2,240/año} (vs \$12,000/año operación LTE + spectrum fees).
\end{itemize}

\textbf{Análisis de viabilidad económica}: Costo por medidor (CAPEX/120) = \$113/medidor vs \$1,500/medidor con LTE privada. Payback period (suponiendo ahorro operativo \$30/año por reducción de lecturas manuales): \$113 / \$30 = 3.8 años vs 50 años LTE. La arquitectura HaLow se vuelve viable para poblaciones >50 medidores, mientras LTE requiere >500 para justificar infraestructura.

\textbf{Impacto social cuantificado}: Según CEPAL, cada 1\% de mejora en acceso a servicios energéticos confiables (medición precisa, respuesta rápida a fallas, tarificación justa) genera 0.15\% de incremento en PIB per cápita rural. Para Colombia (población rural 12.5M, PIB per cápita rural \$4,200 USD), expandir cobertura Smart Grid de 15\% actual a 45\% (30 puntos porcentuales, habilitado por HaLow) generaría impacto económico: 12.5M Í— \$4,200 Í— 0.3 Í— 0.15\% = \textbf{\$236M USD anuales} en actividad económica incremental.

\subsection{Reducción de Emisiones de COâ‚‚ por Eficiencia Energética}

\subsubsection{Huella de Carbono de Arquitecturas IoT}

Las arquitecturas IoT cloud-centric tradicionales generan emisiones de COâ‚‚ a través de tres componentes principales. La implementación de arquitecturas edge-first y el uso de protocolos de comunicación eficientes energéticamente reducen significativamente la huella de carbono de despliegues IoT a gran escala~\cite{sharmaEnergyEfficientIoTRFDriven2025}.

\textbf{1. Tráfico de datos WAN}: Cada GB transmitido por redes celulares LTE genera ~0.06 kg COâ‚‚e (kilogramos de COâ‚‚ equivalente) según Carbon Trust (2023), considerando consumo energético de estaciones base, core network y data centers de operadores. Para arquitectura baseline con 1,000 medidores enviando telemetría sin compresión (200 bytes cada 15 minutos = 19.2 MB/día/medidor Í— 1,000 = 19.2 GB/día), emisiones anuales: 19.2 GB/día Í— 365 días Í— 0.06 kg COâ‚‚e/GB = \textbf{421 kg COâ‚‚e/año}.

\textbf{2. Procesamiento cloud}: Data centers con PUE (Power Usage Effectiveness) típico 1.6 consumen 1.6 kWh eléctricos por cada 1 kWh de computación. Con factor de emisión promedio América Latina 0.45 kg COâ‚‚e/kWh (IEA 2024, considerando mix hidroeléctrica 45\%, térmica 40\%, renovables 15\%), procesamiento de 7 GB telemetría/día (post-compresión) en cloud requiere ~0.05 kWh/GB (estimación AWS EC2 t3.medium), generando: 7 GB/día Í— 0.05 kWh/GB Í— 1.6 PUE Í— 365 días Í— 0.45 kg COâ‚‚e/kWh = \textbf{91 kg COâ‚‚e/año}.

\textbf{3. Gateways edge}: Consumo energético gateway baseline (sin optimizaciones): 18W promedio Í— 24h Í— 365 días = 157.7 kWh/año Í— 0.45 kg COâ‚‚e/kWh = \textbf{71 kg COâ‚‚e/año/gateway}. Para 1,000 medidores con ratio 250 medidores/gateway: 4 gateways Í— 71 kg = \textbf{284 kg COâ‚‚e/año}.

\textbf{Total arquitectura baseline}: 421 + 91 + 284 = \textbf{796 kg COâ‚‚e/año} para 1,000 medidores.

\subsubsection{Reducción de Emisiones con Arquitectura Propuesta}

La arquitectura propuesta reduce emisiones mediante tres mecanismos:

\textbf{Mecanismo 1 - Reducción tráfico WAN 64\% (validado experimentalmente H2)}:
\begin{itemize}
    \item Procesamiento edge local (ThingsBoard Edge + reglas CEP) filtra y agrega telemetría antes de envío cloud
    \item Solo eventos críticos, alarmas y resúmenes horarios se sincronizan con cloud
    \item Tráfico WAN reducido: 19.2 GB/día → 6.9 GB/día (compresión IPHC + filtrado edge)
    \item Emisiones tráfico WAN: 6.9 GB/día Í— 365 días Í— 0.06 kg COâ‚‚e/GB = \textbf{151 kg COâ‚‚e/año} (reducción \textbf{-270 kg} vs baseline)
\end{itemize}

\textbf{Mecanismo 2 - Eliminación/Reducción procesamiento cloud}:
\begin{itemize}
    \item Dashboards consultados localmente (latencia <50 ms vs 500 ms cloud) eliminan 80\% de queries cloud
    \item Análisis de anomalías (LLM Phi-3-mini local) evita llamadas API cloud (\$0.05-0.10 por consulta OpenAI/Claude)
    \item Emisiones procesamiento: reducción 80\% → 91 kg Í— 0.2 = \textbf{18 kg COâ‚‚e/año} (reducción \textbf{-73 kg} vs baseline)
\end{itemize}

\textbf{Mecanismo 3 - Optimización consumo gateways}:
\begin{itemize}
    \item Compresión IPHC reduce overhead 78\% → menor tiempo transmisión → radio HaLow en estado TX/RX menos tiempo
    \item Modo TWT (Target Wake Time) para sensores battery-powered → STAs HaLow duermen 99\% tiempo (duty cycle <1\%)
    \item Consumo gateway optimizado: 12W promedio (vs 18W baseline) Í— 24h Í— 365 días Í— 0.45 kg COâ‚‚e/kWh = \textbf{47 kg COâ‚‚e/año/gateway}
    \item Total 4 gateways: 4 Í— 47 = \textbf{188 kg COâ‚‚e/año} (reducción \textbf{-96 kg} vs baseline)
\end{itemize}

\textbf{Total arquitectura propuesta}: 151 + 18 + 188 = \textbf{357 kg COâ‚‚e/año} para 1,000 medidores.

\textbf{Reducción absoluta}: 796 - 357 = \textbf{439 kg COâ‚‚e/año} (\textbf{-55\% emisiones}).

\textbf{Extrapolación a escala}: Si 1 millón de medidores inteligentes en América Latina (objetivo CEPAL 2030: cobertura 30\% → 180M hogares Í— 30\% = 54M medidores, suponiendo 2\% adopción temprana = 1.08M medidores) adoptaran arquitectura propuesta en lugar de cloud-centric:
\begin{itemize}
    \item Reducción emisiones: 1,080 instalaciones Í— 439 kg COâ‚‚e/año = \textbf{474 toneladas COâ‚‚e/año}
    \item Equivalente a: Retiro de \textbf{102 automóviles de combustión} (emisión típica 4.6 toneladas COâ‚‚e/año/vehículo EPA 2023)
    \item O plantación de \textbf{7,900 árboles maduros} (absorción típica 60 kg COâ‚‚/año/árbol)
\end{itemize}

\subsection{Contribución a los Objetivos de Desarrollo Sostenible (ODS)}

La arquitectura propuesta se alinea directamente con tres ODS de las Naciones Unidas:

\subsubsection{ODS 7: Energía Asequible y No Contaminante}

\textbf{Meta 7.1 - Garantizar acceso universal a servicios energéticos asequibles, fiables y modernos}:
\begin{itemize}
    \item \textbf{Contribución}: La arquitectura HaLow habilita despliegues de medición inteligente en zonas rurales sin cobertura LTE con CAPEX 12Í— menor (\$113/medidor vs \$1,500), acelerando cobertura de servicios modernos (tarificación dinámica, detección fraude, respuesta a fallas <30 min vs >48 horas manual).
    \item \textbf{Indicador}: Reducción tiempo promedio de respuesta a cortes eléctricos (SAIDI - System Average Interruption Duration Index) de 18 horas (promedio rural América Latina, OLADE 2023) a 2 horas con detección automática y localización precisa de fallas mediante telemetría sub-GHz.
\end{itemize}

\textbf{Meta 7.3 - Duplicar tasa de mejora de eficiencia energética global}:
\begin{itemize}
    \item \textbf{Contribución}: Procesamiento edge + CEP local permite implementar programas de Demand Response (DR) con latencia <5 segundos (vs >60 segundos cloud), habilitando reducción de picos de demanda 15-25\% según estudios OpenADR Alliance (2024)~\cite{miglaniDeepLearningModels2022,perezImplicationSoftwareContainerization2024}.
    \item \textbf{Indicador}: Reducción de pérdidas no técnicas (hurto/fraude energético) de 12\% promedio América Latina (Banco Mundial 2023) a 5\% mediante detección de anomalías con IA local (análisis de patrones de consumo cada 15 minutos, vs mensual con lectura manual)~\cite{ayaUnleashingIntelligenceEdge2024,huangDataProcessingEnhancement2025}.
\end{itemize}

\subsubsection{ODS 9: Industria, Innovación e Infraestructura}

\textbf{Meta 9.1 - Desarrollar infraestructuras fiables, sostenibles, resilientes y de calidad}:
\begin{itemize}
    \item \textbf{Contribución}: Arquitectura multi-WAN (HaLow + LTE + Ethernet) con failover <5 segundos garantiza disponibilidad >99.7\% validada experimentalmente, cumpliendo requisitos de infraestructura crítica IEC 61850-90-5 para subestaciones eléctricas~\cite{rehmanCriticalInfrastructuresResilience2024,zhangEnablingFaultTolerant2024}.
    \item \textbf{Indicador}: Aumento de disponibilidad de servicios Smart Grid de 98.2\% (arquitectura cloud-only con dependencia WAN) a 99.7\% (operación offline 48h+), equivalente a reducción de downtime anual de 158 horas a 26 horas~\cite{kumariEdgeComputingData2024,pereraScalableDataanalyticCapable2024}.
\end{itemize}

\textbf{Meta 9.c - Aumentar acceso TIC y conexión Internet universal y asequible}:
\begin{itemize}
    \item \textbf{Contribución}: Wi-Fi HaLow en espectro no licenciado elimina barreras regulatorias y económicas (licencias LTE \$50k-200k), permitiendo cooperativas eléctricas rurales desplegar infraestructura IoT sin dependencia de operadores comerciales~\cite{scharerPushingWiFiHaLow2025,tianWiFiHaLowInternet2021,faizankhanHybridWiFiHaLow2024}.
    \item \textbf{Indicador}: Modelo económico demuestra viabilidad para comunidades >50 medidores (vs >500 con LTE), expandiendo cobertura potencial a 3,200 veredas colombianas con 50-200 habitantes (censo DANE 2018), actualmente sin servicios Smart Grid~\cite{amrilPerformanceEvaluationWiFi2025,rizanovWiFiHaLowWildfire2024}.
\end{itemize}

\subsubsection{ODS 13: Acción por el Clima}

\textbf{Meta 13.2 - Incorporar medidas relativas al cambio climático en políticas y estrategias}:
\begin{itemize}
    \item \textbf{Contribución}: Reducción de emisiones 55\% (439 kg COâ‚‚e/año por cada 1,000 medidores) mediante arquitectura edge-first alinea con compromisos NDC (Nationally Determined Contributions) de Colombia (reducción 51\% emisiones GEI para 2030 vs 2010, Ley 2169 de 2021)~\cite{cohenEdgeComputingEnabled2024,shenAugmentingEdgeIntelligence2024}.
    \item \textbf{Indicador}: Potencial de mitigación: 1.08M medidores Í— 439 kg COâ‚‚e/año = 474 toneladas COâ‚‚e/año, contribuyendo 0.0002\% a meta nacional (Colombia debe reducir 169.44 Mt COâ‚‚e/año para cumplir NDC 2030)~\cite{wulurCoordinationEdgeCloud2024}.
\end{itemize}

\textbf{Meta 13.3 - Mejorar educación y capacidad humana respecto a mitigación del cambio climático}:
\begin{itemize}
    \item \textbf{Contribución}: Dashboards locales de consumo energético en tiempo real (<2s latencia) + asistente conversacional LLM (interfaz natural "¿cuánto gasté hoy?") empoderan usuarios finales con visibilidad instantánea, habilitando cambios de comportamiento (objetivo reducción consumo 8-12\% según estudios behavioural economics, Allcott \& Rogers 2014)~\cite{saidiIoTGatewayPowered2024,amezcuavaldovinosDesignImplementationEvaluation2024}.
    \item \textbf{Indicador}: Tiempo de respuesta a consultas de consumo reducido de 48-72 horas (factura mensual) a <5 segundos (dashboard edge + LLM local), mejorando engagement usuarios con gestión energética~\cite{boonmeerukCostEffectiveIIoTGateway2024,abowardahEdgeComputingIoT2024}.
\end{itemize}

\subsection{Síntesis del Impacto Social y Ambiental}

La arquitectura propuesta trasciende el ámbito puramente técnico, ofreciendo beneficios socioeconómicos y ambientales cuantificables~\cite{ashfaqIoTSensorNetworks2024,InternetThingsIoT2024,laghariInternetThingsIoT2024}:

\textbf{Impacto social}:
\begin{itemize}
    \item \textbf{Acceso equitativo}: Viabilidad económica para despliegues rurales (\$113/medidor vs \$1,500 LTE) habilita cobertura Smart Grid en 87M personas actualmente sin acceso confiable (CEPAL 2023)
    \item \textbf{Desarrollo económico}: Mejora en servicios energéticos genera \$236M USD anuales actividad económica incremental en Colombia (extrapolable a región)
    \item \textbf{Resiliencia comunitaria}: Operación offline 48h+ garantiza servicios críticos durante desastres naturales o fallas de infraestructura externa
\end{itemize}

\textbf{Impacto ambiental}:
\begin{itemize}
    \item \textbf{Mitigación climática}: Reducción 55\% emisiones COâ‚‚e (439 kg/año por 1,000 medidores), escalable a 474 toneladas/año con 1M medidores
    \item \textbf{Eficiencia energética}: Habilitación de Demand Response con latencia <5s permite reducción picos demanda 15-25\%, disminuyendo necesidad de plantas térmicas de respaldo
    \item \textbf{Alineación ODS}: Contribución directa a 3 Objetivos de Desarrollo Sostenible (ODS 7, 9, 13) con 6 metas específicas validadas
\end{itemize}

\textbf{Conclusión}: La investigación demuestra que las decisiones arquitectónicas técnicas (edge vs cloud, protocolos IoT, espectro de radio) tienen implicaciones profundas en equidad social y sostenibilidad ambiental, no solo en rendimiento y costos. La adopción de arquitecturas edge con espectro no licenciado sub-GHz (HaLow) puede acelerar transición energética en América Latina, democratizando acceso a servicios Smart Grid modernos sin perpetuar brechas de conectividad existentes.

\section{Trabajo Futuro}

Esta sección presenta la hoja de ruta (\textit{roadmap}) de investigación 2026-2030 organizada en 5 líneas estratégicas. La Figura~\ref{fig:roadmap-gantt} visualiza la secuencia temporal, dependencias entre líneas, y horizonte de madurez tecnológica esperado.

\begin{figure}[H]
\centering
\caption{Roadmap trabajo futuro 2026-2030: cronograma Gantt de 5 líneas de investigación con hitos críticos, dependencias tecnológicas, y horizonte de madurez TRL (Technology Readiness Level). Línea 1 (Escalabilidad): base para todas las demás, alcanza TRL 8-9 en 2027. Línea 2 (ML): depende de datos masivos L1, TRL 7-8 en 2028. Línea 3 (Seguridad): evolución continua PQC, TRL 6-7 en 2030. Línea 4 (Interoperabilidad): federación requiere L1 completa, TRL 7-8 en 2029. Línea 5 (Estándares): tracking de evolución industria, TRL variable según adopción mercado.}
\label{fig:roadmap-gantt}
\resizebox{\textwidth}{!}{%
\begin{tabular}{|>{\raggedright\arraybackslash}p{3.5cm}|c|c|c|c|c|c|c|c|}
\hline
\rowcolor{gray!20}
\textbf{Línea Investigación} & \textbf{2026} & \textbf{H1} & \textbf{H2} & \textbf{2027} & \textbf{2028} & \textbf{2029} & \textbf{2030} & \textbf{TRL Final} \\
 & \textbf{H1} & \textbf{2026} & \textbf{2026} & \textbf{H1-H2} & \textbf{H1-H2} & \textbf{H1-H2} & \textbf{H1-H2} & \\
\hline
\multicolumn{9}{|c|}{\cellcolor{blue!10}\textbf{LÍNEA 1: Escalabilidad y Performance}} \\
\hline
L1.1 - Validación 1000+ dispositivos & \cellcolor{green!30}{\textbf{Sim}} & \cellcolor{green!30}{\textbf{NS-3}} & \cellcolor{yellow!30}{\textbf{Emu}} & \cellcolor{yellow!30}{\textbf{Docker}} & \cellcolor{orange!30}{\textbf{Pilot}} & & & \textbf{8-9} \\
\hline
L1.2 - Clustering HA & & & \cellcolor{green!30}{\textbf{Raft}} & \cellcolor{green!30}{\textbf{VRRP}} & \cellcolor{yellow!30}{\textbf{PG Rep}} & \cellcolor{orange!30}{\textbf{Test}} & & \textbf{8-9} \\
\hline
\textit{Hito crítico L1} & \multicolumn{4}{c|}{\cellcolor{red!20}\textbf{▶ Q4 2027: Arquitectura validada 5K dispositivos}} & \multicolumn{3}{c|}{} & \\
\hline
\multicolumn{9}{|c|}{\cellcolor{blue!10}\textbf{LÍNEA 2: Machine Learning Avanzado}} \\
\hline
L2.1 - Detección anomalías & & \cellcolor{green!30}{\textbf{LSTM}} & \cellcolor{green!30}{\textbf{Train}} & \cellcolor{yellow!30}{\textbf{ONNX}} & \cellcolor{yellow!30}{\textbf{Deploy}} & \cellcolor{orange!30}{\textbf{Eval}} & & \textbf{7-8} \\
\hline
L2.2 - Forecasting renovable & & & & \cellcolor{green!30}{\textbf{XGB}} & \cellcolor{yellow!30}{\textbf{LSTM}} & \cellcolor{yellow!30}{\textbf{Híbrid}} & \cellcolor{orange!30}{\textbf{Prod}} & \textbf{7-8} \\
\hline
\textit{Dependencia L2} & \multicolumn{4}{c|}{\textit{Requiere dataset masivo de L1.1 (6-12 meses telemetría)}} & \multicolumn{3}{c|}{} & \\
\hline
\textit{Hito crítico L2} & \multicolumn{5}{c|}{} & \multicolumn{2}{c|}{\cellcolor{red!20}\textbf{▶ Q2 2029: Modelos en producción}} & \\
\hline
\multicolumn{9}{|c|}{\cellcolor{blue!10}\textbf{LÍNEA 3: Seguridad Avanzada}} \\
\hline
L3.1 - Blockchain audit trail & & & \cellcolor{green!30}{\textbf{HLF}} & \cellcolor{yellow!30}{\textbf{Smart}} & \cellcolor{yellow!30}{\textbf{IPFS}} & \cellcolor{orange!30}{\textbf{Pilot}} & & \textbf{6-7} \\
\hline
L3.2 - Zero Trust (mTLS) & \cellcolor{green!30}{\textbf{Istio}} & \cellcolor{green!30}{\textbf{OPA}} & \cellcolor{yellow!30}{\textbf{JWT}} & \cellcolor{orange!30}{\textbf{Prod}} & & & & \textbf{8-9} \\
\hline
L3.3 - Post-Quantum Crypto & & & & & \cellcolor{green!30}{\textbf{NIST}} & \cellcolor{yellow!30}{\textbf{Kyber}} & \cellcolor{yellow!30}{\textbf{Dilith}} & \textbf{5-6} \\
\hline
\textit{Hito crítico L3} & \multicolumn{3}{c|}{\cellcolor{red!20}\textbf{▶ Q4 2026: Zero Trust MVP}} & \multicolumn{4}{c|}{\cellcolor{red!20}\textbf{▶ Q4 2029: PQC roadmap definido}} & \\
\hline
\multicolumn{9}{|c|}{\cellcolor{blue!10}\textbf{LÍNEA 4: Interoperabilidad Extendida}} \\
\hline
L4.1 - Integración legacy & \cellcolor{green!30}{\textbf{Modbus}} & \cellcolor{green!30}{\textbf{DNP3}} & \cellcolor{yellow!30}{\textbf{104}} & \cellcolor{orange!30}{\textbf{Pilot}} & & & & \textbf{8-9} \\
\hline
L4.2 - Federación gateways & & & & \cellcolor{green!30}{\textbf{mDNS}} & \cellcolor{yellow!30}{\textbf{Consul}} & \cellcolor{yellow!30}{\textbf{SWIM}} & \cellcolor{orange!30}{\textbf{Test}} & \textbf{7-8} \\
\hline
\textit{Dependencia L4.2} & \multicolumn{4}{c|}{\textit{Requiere L1.2 clustering HA completado}} & \multicolumn{3}{c|}{} & \\
\hline
\textit{Hito crítico L4} & \multicolumn{4}{c|}{\cellcolor{red!20}\textbf{▶ Q2 2027: Protocolos legacy integrados}} & \multicolumn{3}{c|}{\cellcolor{red!20}\textbf{▶ Q4 2029: Federación activa}} & \\
\hline
\multicolumn{9}{|c|}{\cellcolor{blue!10}\textbf{LÍNEA 5: Estándares Emergentes}} \\
\hline
L5.1 - Matter over Thread & & \cellcolor{green!30}{\textbf{SDK}} & \cellcolor{yellow!30}{\textbf{Ctrl}} & \cellcolor{yellow!30}{\textbf{Map}} & \cellcolor{orange!30}{\textbf{Eval}} & & & \textbf{7-8} \\
\hline
L5.2 - Wi-Fi 7 backhaul & & & & & & \cellcolor{green!30}{\textbf{802.11be}} & \cellcolor{yellow!30}{\textbf{Test}} & \textbf{6-7} \\
\hline
L5.3 - 5G RedCap WAN & & & & \cellcolor{green!30}{\textbf{Spec}} & \cellcolor{yellow!30}{\textbf{Módulo}} & \cellcolor{yellow!30}{\textbf{Integ}} & \cellcolor{orange!30}{\textbf{Pilot}} & \textbf{7-8} \\
\hline
\textit{Nota estándares} & \multicolumn{7}{c|}{\textit{Timing dependiente de madurez mercado y disponibilidad componentes COTS}} & \\
\hline
\multicolumn{9}{|c|}{\cellcolor{yellow!20}\textbf{LEYENDA}} \\
\hline
\multicolumn{9}{|l|}{\cellcolor{green!20}\textbf{Verde:} Investigación y desarrollo inicial (TRL 3-5) | \cellcolor{yellow!20}\textbf{Amarillo:} Prototipo y validación (TRL 6-7) | \cellcolor{orange!20}\textbf{Naranja:} Piloto y despliegue (TRL 8-9)} \\
\hline
\multicolumn{9}{|l|}{\textbf{Abreviaciones:} Sim=Simulación, Emu=Emulación, PG Rep=PostgreSQL Replication, HLF=Hyperledger Fabric, OPA=Open Policy Agent,} \\
\multicolumn{9}{|l|}{NIST=Estándares PQC NIST, Kyber=ML-KEM (Key Encapsulation), Dilith=ML-DSA (Digital Signature), SDK=Software Dev Kit, Ctrl=Controller,} \\
\multicolumn{9}{|l|}{Map=Mapping, Eval=Evaluation, Spec=Specification 3GPP Release 17, RedCap=Reduced Capability 5G} \\
\hline
\end{tabular}%
}
\end{figure}

\textbf{Análisis de ruta crítica:}

\begin{itemize}
    \item \textbf{L1 es fundacional:} Escalabilidad debe alcanzar TRL 8-9 (Q4 2027) antes de desplegar ML masivo (L2) o federación (L4.2). Riesgo: retrasos en simulación NS-3 retrasan todo roadmap.
    
    \item \textbf{L2 requiere datos:} Modelos ML necesitan 6-12 meses de telemetría de L1.1 (1000+ dispositivos). Inicio realista: H1 2027 tras completar infraestructura escalable.
    
    \item \textbf{L3 evoluciona continuamente:} Zero Trust (L3.2) es quick win (TRL 8-9 en 2027), mientras PQC (L3.3) es preparatorio largo plazo (TRL 5-6 en 2030, despliegue masivo post-2032 cuando NIST finalice).
    
    \item \textbf{L4.2 depende de L1.2:} Federación entre gateways solo tiene sentido tras validar clustering HA (L1.2 Q4 2027). Secuencia obligatoria.
    
    \item \textbf{L5 oportunista:} Adopción de Matter/Wi-Fi 7/5G RedCap depende de disponibilidad mercado, no solo investigación interna. Timeline flexible ±6 meses según vendor roadmaps.
\end{itemize}

\textbf{Recursos estimados (persona-año acumulado 2026-2030):}
\begin{itemize}
    \item L1: 3.5 PA (1.5 PA simulación/emulación, 2 PA clustering HA)
    \item L2: 4 PA (2.5 PA detección anomalías, 1.5 PA forecasting)
    \item L3: 3 PA (1 PA blockchain, 1 PA Zero Trust, 1 PA PQC roadmap)
    \item L4: 2.5 PA (1 PA legacy, 1.5 PA federación)
    \item L5: 2 PA (0.5 PA Matter, 0.5 PA Wi-Fi 7, 1 PA 5G RedCap)
    \item \textbf{L6: 1.5 PA} (0.5 PA evaluación MM8108, 0.5 PA validación campo, 0.5 PA piloto upgrade)
    \item \textbf{Total: 16.5 PA} → equipo sostenido 3 investigadores × 5.5 años
\end{itemize}

\textbf{Financiamiento potencial:} Convocatorias Minciencias (Investigación Aplicada), Fondo Energético Nacional FENOGE (Smart Grids), colaboración industrial utilities colombianas (EPM, Codensa, CHEC), European Horizon Europe calls (si partnership internacional).

\subsection{Línea 1 - Escalabilidad y Performance}

\subsubsection{L1.1 - Validación con 1000+ Dispositivos}

\textbf{Objetivo:} Caracterizar mejoras reales de MM8108 vs MM6108 baseline en escenario AMI controlado.

\textbf{Metodología propuesta:}
\begin{itemize}
    \item Adquisición de 3 módulos Gateworks GW16167 (MM8108 M.2 E-Key) para integración en gateways prototipo existentes.
    \item Pruebas comparativas lado-a-lado: Gateway A (MM6108) vs Gateway B (MM8108) en testbed universitario con 50 nodos Thread simulados.
    \item Métricas de evaluación:
    \begin{itemize}
        \item \textbf{Link budget:} Medición RSSI/SNR a distancias 500m, 1km, 1.5km, 2km, 2.5km (campus abierto + entorno urbano con obstáculos)
        \item \textbf{Throughput real:} iperf3 TCP/UDP con cargas 1 Mbps, 5 Mbps, 10 Mbps, 20 Mbps
        \item \textbf{Latencia:} Ping round-trip time bajo carga (background traffic)
        \item \textbf{Consumo energético:} Medición con multímetro high-side shunt en modos TX (+26 dBm), RX, idle, TWT sleep
    \end{itemize}
    \item Análisis comparativo: ¿Mejoras teóricas (+3 dB TX, +3 dB RX, +33\% throughput) se materializan en despliegue real? ¿Trade-offs consumo energético (+12\% TX) impactan autonomía gateway con respaldo batería?
\end{itemize}

\textbf{Resultados esperados:} Reporte técnico validando o refutando mejoras especificadas en datasheet MM8108. Identificación de escenarios donde upgrade es justificado (e.g., zonas rurales con alcance >1.5 km crítico) vs innecesario (urbano denso <1 km suficiente con MM6108).

\textbf{Timeline:} Q2 2026 (2 meses). \textbf{TRL objetivo:} 5-6 (componente validado en entorno relevante).

\textbf{Recursos:} 0.5 PA investigador junior + \$1.5K hardware (3× GW16167 @ \$150 + antennas + instrumentación).

\subsubsection{L6.2 - Validación Alcance Extendido 4-5 km}

\textbf{Objetivo:} Verificar alcance máximo operacional MM8108 en condiciones reales (no anechoic chamber) para optimizar topología red en despliegues utility-scale.

\textbf{Metodología propuesta:}
\begin{itemize}
    \item Pruebas de campo en zona rural (Manizales-Villamaría corredor): Gateway MM8108 fijo (coordenadas GPS registradas) + nodo móvil en vehículo con datalogger GPS/RSSI.
    \item Medición continua RSSI, packet loss rate (%), throughput disponible vs distancia (increments 500m hasta pérdida completa de enlace).
    \item Variación condiciones: (1) Line-of-sight (LoS) en campo abierto, (2) Non-line-of-sight (NLoS) con vegetación densa + topografía montañosa, (3) Interferencia controlada (co-channel 900 MHz ISM band).
    \item Comparación con modelo teórico Friis + corrección Okumura-Hata para 900 MHz: ¿Alcance medido coincide con predicción +6 dB link budget?
\end{itemize}

\textbf{Resultados esperados:} Mapas de cobertura (heatmaps RSSI) validando alcance 2-3 km urbano NLoS, 4-5 km rural LoS con MM8108. Guidelines de diseño de red: "Con MM8108, reducir gateways de 50 a 35 en despliegue 1000 medidores (30\% savings infraestructura)".

\textbf{Timeline:} Q4 2026 (3 meses tras completar L6.1). \textbf{TRL objetivo:} 7 (prototipo demostrado en entorno operacional real).

\textbf{Recursos:} 0.5 PA investigador senior + \$2K logística campo (vehículo, datalogger, permisos acceso predios rurales).

\subsubsection{L6.3 - Piloto Upgrade Modular en 10 Gateways Deployed}

\textbf{Objetivo:} Validar procedimiento de upgrade modular M.2 swap en gateways ya desplegados con medidores activos (prueba de concepto operacional).

\textbf{Metodología propuesta:}
\begin{itemize}
    \item Selección de 10 gateways piloto en despliegue L1.1 (1000+ dispositivos) con criterios: (1) 5 gateways en zonas urbanas densas (baseline, upgrade no crítico), (2) 5 gateways en zonas rurales/periurbanas (alcance extendido beneficioso).
    \item Procedimiento estandarizado de upgrade:
    \begin{enumerate}
        \item Pre-upgrade: Backup configuración gateway (network settings, TLS certificates, device registry)
        \item Shutdown graceful: Flush buffers MQTT/PostgreSQL, notificar ThingsBoard mantenimiento programado
        \item Swap físico: Remover módulo MM6108, insertar GW16167 (MM8108) en slot M.2, verificar conexión mecánica/eléctrica
        \item Post-upgrade: Boot gateway, verificar reconocimiento USB lsusb, cargar firmware MM8108, test conectividad HaLow
        \item Validación operacional: Reconexión DCUs (expected <5 min), throughput test, monitoreo 72h estabilidad
    \end{enumerate}
    \item Métricas éxito: (1) Tiempo downtime <30 min por gateway, (2) 0\% data loss (buffering durante upgrade funcional), (3) Mejoras post-upgrade: RSSI +3-6 dB en DCUs lejanos, reducción packet retransmissions.
\end{itemize}

\textbf{Resultados esperados:} Procedimiento documentado de upgrade field-proven, incluyendo: (1) Checklist técnico paso-a-paso, (2) Troubleshooting guide (e.g., módulo no detectado, driver conflicts), (3) ROI medido: \$175 upgrade cost vs beneficios (extender cobertura sin gateway adicional = savings \$295 CAPEX + \$50/año OPEX).

\textbf{Timeline:} Q2 2027 (6 meses, tras completar L6.2 + L1.1 scale validation). \textbf{TRL objetivo:} 8-9 (sistema completo calificado y demostrado en entorno operacional).

\textbf{Recursos:} 0.5 PA ingeniero de campo + \$3.5K (10× módulos GW16167 @ \$150 + labor on-site 10× \$25 + contingencia).

\textbf{Dependencias críticas L6:}
\begin{itemize}
    \item \textbf{L6.1 → L6.2:} Validación testbed debe confirmar viabilidad técnica antes de pruebas campo costosas.
    \item \textbf{L6.2 → L6.3:} Mapas cobertura identifican gateways candidatos prioritarios para upgrade (máximo beneficio alcance).
    \item \textbf{L1.1 → L6.3:} Piloto upgrade requiere despliegue operacional 1000+ dispositivos como baseline (L1.1 completado Q4 2027).
\end{itemize}

\textbf{Riesgos y mitigaciones:}
\begin{itemize}
    \item \textbf{Riesgo 1 - Incompatibilidad driver MM8108:} Módulo GW16167 requiere kernel Linux >5.10 para full support cfg80211. \textit{Mitigación:} Validar compatibilidad en L6.1, upgrade Raspberry Pi OS Bookworm (kernel 6.1) si necesario.
    \item \textbf{Riesgo 2 - ROI insuficiente en zonas urbanas:} Mejoras alcance MM8108 (+40\%) no justifican upgrade si densidad medidores permite <1 km spacing. \textit{Mitigación:} Upgrade selectivo solo rural/periurbano (50\% gateways), mantener MM6108 en urbano denso.
    \item \textbf{Riesgo 3 - Disponibilidad GW16167:} Módulo Gateworks depende supply chain Morse Micro + fabricación USA. \textit{Mitigación:} Orden anticipada 20-30 unidades Q1 2026, evaluar alternativas M.2 HaLow (NewRadek, AsiaRF) como backup suppliers.
\end{itemize}

\textbf{Conclusión L6:} Línea de investigación valida promesa de arquitectura modular: upgrades incrementales de componentes críticos (radio HaLow) extienden vida útil gateway de 5 años (HW monolítico obsoleto) a 10+ años con refreshes tecnológicos económicos (\$175 vs \$445 reemplazo completo). \textbf{Si L6.3 demuestra upgrade field-proven con <30 min downtime y ROI positivo, arquitectura propuesta no solo es óptima para 2025, sino sostenible para 2030-2035.}

\subsection{Línea 1 - Escalabilidad y Performance}

\subsubsection{L1.1 - Validación con 1000+ Dispositivos}

\textbf{Objetivo:} Caracterizar comportamiento arquitectura con densidad de dispositivos representativa de despliegues (\textit{deployments}) a escala de servicios públicos (\textit{utility-scale}) (1000-5000 medidores por pasarela).

\textbf{Metodología propuesta:}
\begin{itemize}
\item Simulación NS-3 de red Thread con 500 nodos, variando cuenta de saltos (\textit{hop count}) (2-6 saltos), patrones de tráfico (\textit{traffic patterns}) (periódico, ráfagas (\textit{bursty}), activado por eventos (\textit{event-triggered})).
\item Emulación con generadores de carga sintética: 100 instancias Docker simulando dispositivos LwM2M, enviando telemetría a pasarela real.
\item Análisis de cuellos de botella: perfilado de CPU (\textit{profiling CPU}) (perf, flamegraphs), memoria (valgrind, heaptrack), red (\textit{network}) (iperf, netperf), E/S de disco (\textit{disk I/O}) (fio, iostat).
\item Optimizaciones iterativas: ajuste de núcleo (\textit{tuning kernel}) (parámetros tcp sysctl), PostgreSQL (shared\_buffers, work\_mem), Kafka (batch.size, linger.ms).
\end{itemize}

\textbf{Resultados esperados:} Identificación de límites de escalabilidad (e.g., "pasarela soporta 800 dispositivos Thread @ 1 msg/min antes de saturar CPU"), guías de dimensionamiento de hardware (\textit{hardware}).

\subsubsection{L1.2 - Agrupamiento en el Borde para Alta Disponibilidad}

\textbf{Motivación:} Pasarela única es punto único de falla (\textit{single point of failure}). Despliegues (\textit{Deployments}) críticos requieren redundancia activa-activa o activa-pasiva para cumplir con requisitos de disponibilidad de infraestructura crítica de distribución eléctrica, considerando también protección ante riesgos emergentes como ataques a vehículos eléctricos conectados~\cite{basnetNetworkingPowerGrid2025,talaatFortifyingEVCharging2025}.

\textbf{Arquitectura propuesta:}
\begin{itemize}
\item Dos pasarelas en configuración HA (Alta Disponibilidad): Pasarela A (primaria), Pasarela B (en espera (\textit{standby})).
\item Protocolo de elección de líder (\textit{leader}): Consenso Raft (\textit{Raft consensus}) (etcd, Consul) o VRRP (keepalived) para IP virtual flotante.
\item Replicación de estado: Replicación de flujo PostgreSQL (\textit{PostgreSQL streaming replication}) (asíncrona), Redis Sentinel para conmutación (\textit{failover}) de caché.
\item Verificación de salud cruzada (\textit{Health checking} cruzado): Pasarelas se monitorean mutuamente vía latido (\textit{heartbeat}) (cada 1s). Tiempo de espera (\textit{Timeout}) 5s activa conmutación (\textit{failover}).
\end{itemize}

\textbf{Desafíos:} Sincronización de credenciales de red Thread entre pasarelas, gestión de escenarios de cerebro dividido (\textit{split-brain scenarios}), sobrecarga (\textit{overhead}) de replicación en enlaces WAN lentos.

\subsection{Línea 2 - Machine Learning Avanzado}

\subsubsection{L2.1 - Detección de Anomalías en Series Temporales}

\textbf{Objetivo:} Implementar modelos ML específicos para detección de patrones anómalos en telemetría Smart Energy: robo energético (\textit{theft} energético), fallas de transformador, desbalance de fases, integrando enfoques de Grandes Volúmenes de Datos (\textit{Big Data}) para escalabilidad masiva~\cite{chintaEdgeAIRealTime2024,jonnakutiEDGEBASEDFAULTDETECTION,harveSafeguardingIoTBig}.

\textbf{Técnicas a explorar:}
\begin{itemize}
\item \textbf{Autocodificadores LSTM (\textit{Autoencoders LSTM}):} Red neuronal que aprende representación comprimida de series temporales normales. Reconstrucción con error >umbral (\textit{threshold}) indica anomalía. Ventaja: no supervisado (\textit{unsupervised}) (no requiere etiquetado (\textit{labeling}) de anomalías).
\item \textbf{Bosque de Aislamiento (\textit{Isolation Forest}):} Algoritmo basado en ensambles (\textit{ensemble-based}) que construye árboles de decisión aleatorios (\textit{random}). Puntos anómalos son aislados con menos particiones. Ventaja: eficiente, funciona en espacio de alta dimensión (\textit{high-dimensional space}).
\item \textbf{Prophet:} Modelo desarrollado por Facebook para pronóstico (\textit{forecasting}). Detecta anomalías como desviaciones significativas de predicción. Ventaja: maneja estacionalidad (\textit{seasonality}) (diaria, semanal), días festivos (\textit{holidays}) automáticamente.
\end{itemize}

\textbf{Pipeline propuesto:}
\begin{enumerate}
\item Entrenamiento (\textit{Training}) en la nube con conjunto de datos histórico (\textit{dataset} histórico) (6-12 meses telemetría)~\cite{miglaniDeepLearningModels2022}.
\item Exportación de modelo a formato optimizado para el borde (\textit{edge}) (ONNX, TensorFlow Lite, CoreML)~\cite{shenAugmentingEdgeIntelligence2024}.
\item Despliegue (\textit{Deployment}) en pasarela como contenedor dedicado (TensorFlow Serving, Triton Inference Server)~\cite{perezImplicationSoftwareContainerization2024}.
\item Inferencia activada (\textit{triggered}) por cadena de reglas ThingsBoard (\textit{rule chain}) ante cada lote (\textit{batch}) de mensajes (e.g., cada 100 muestras o cada 5 min)~\cite{huangDataProcessingEnhancement2025}.
\item Alarmas generadas automáticamente ante detecciones, con explicabilidad (valores SHAP (\textit{SHAP values}), LIME)~\cite{ayaUnleashingIntelligenceEdge2024}.
\end{enumerate}

\textbf{Métricas de evaluación:} Precisión (\textit{Precision}), Exhaustividad (\textit{Recall}), Puntuación F1 (\textit{F1-score}) en conjunto de prueba (\textit{test set}); Tasa de Falsos Positivos (\textit{False Positive Rate}) <1\% (crítico para evitar fatiga de alarmas (\textit{alarm fatigue}) operativa); Latencia de inferencia <500 ms para lote (\textit{batch}) de 100 muestras. Los sistemas de monitoreo de condiciones habilitados por IoT permiten mantenimiento predictivo y detección temprana de fallas en infraestructura crítica~\cite{masterofengineeringm.e.electricalandelectronicsengineeringlamaruniversityusaIOTENABLEDCONDITIONMONITORING2025}.

\subsubsection{L2.2 - Forecasting de Generación Renovable}

\textbf{Objetivo:} Predecir generación solar/eólica próximas 24 horas basado en: (1) Histórico de generación; (2) Datos meteorológicos (irradiancia, velocidad viento, temperatura); (3) Forecasts weather API (OpenWeatherMap, NOAA)~\cite{dentremontDeepLearningDataset2024,kumariEdgeComputingData2024}.

\textbf{Arquitectura:}
\begin{itemize}
\item Feature engineering: rolling averages (1h, 6h, 24h), lag features (generación t-1, t-24, t-168 horas), calendar features (hora del día, día de semana, mes)~\cite{cakanDataFusionIntegrated2024}.
\item Modelo híbrido: XGBoost para captura de no-linearities + LSTM para dependencias temporales largas~\cite{miglaniDeepLearningModels2022}.
\item Re-training continuo: modelo se actualiza semanalmente con nuevos datos (online learning)~\cite{ayaUnleashingIntelligenceEdge2024}.
\item Deployment edge: inferencia cada hora, resultados persisten en TimescaleDB, visualizan en dashboard ThingsBoard como series de pronóstico vs real~\cite{pereraScalableDataanalyticCapable2024}.
\end{itemize}

\textbf{Aplicación:} Gestión proactiva de storage (cargar baterías anticipando pico solar), coordinación con utility (curtailment requests ante forecast de sobre-generación), optimización económica (participation en mercados day-ahead)~\cite{cohenEdgeComputingEnabled2024}.

\subsection{Línea 3 - Seguridad Avanzada}

\subsubsection{L3.1 - Implementación de Blockchain para Audit Trail}

\textbf{Motivación:} Registro inmutable de eventos críticos (comandos de control, cambios de configuración, alarmas) para compliance regulatorio y forensics post-incidente~\cite{cervinskiIoTSecurityDApp2024,chengLightweightAuthenticationDrivenTrusted2024,m.mijwilPostQuantumSecureBlockchainBased2025}.

\textbf{Arquitectura propuesta:}
\begin{itemize}
\item Blockchain privada: Hyperledger Fabric o Ethereum privada (Proof-of-Authority consensus)~\cite{BlockchainBasedSecureAuthentication2025}.
\item Nodos: Gateway actúa como peer node, cloud backend como orderer + endorser~\cite{ferragRemoteAttestationSecurity2023}.
\item Smart contracts (chaincode): Lógica de validación de transacciones (e.g., comando de apertura de breaker requiere firma dual operator + supervisor)~\cite{cervinskiIoTSecurityDApp2024}.
\item Storage híbrido: Hash de evento se escribe en blockchain (32 bytes), payload completo en IPFS (InterPlanetary File System) off-chain, referenciado por hash~\cite{alrobaiSecurePartitioningEmbedded2024}.
\end{itemize}

\textbf{Desafíos:} Latencia de consenso (1-5 segundos típico en Hyperledger) incompatible con control tiempo real~\cite{wulurCoordinationEdgeCloud2024}, overhead de storage (blockchain crece monotónicamente), complejidad operacional (gestión de certificados peer nodes)~\cite{ramakrishnaAnalysisLightweightCryptographic2024}.

\subsubsection{L3.2 - Zero Trust Architecture}

\textbf{Objetivo:} Reemplazar modelo de seguridad perimetral (confianza implícita dentro de red interna) con Zero Trust (nunca confiar, siempre verificar), aplicando estrategias avanzadas de protección multi-dimensional para entornos Smart Grid críticos~\cite{augmentingSecuritySmart2024,sairamIntegratedLightweightCryptographic2024,syamsSecureCommunicationIntegrating2024}.

\textbf{Componentes clave}:
\begin{itemize}
\item \textbf{Identity-based access:} Autenticación de dispositivos y usuarios mediante certificados X.509 + JWT tokens. Cada request incluye identidad verificable~\cite{eldefrawyFormallyVerifiedAuthentication2019,liangSecurityLightweightAuthentication2018}.
\item \textbf{Microsegmentación:} Cada contenedor en su propia VLAN virtual (Docker networks aisladas). Comunicación inter-container vía firewall explícito (nftables rules)~\cite{alrobaiSecurePartitioningEmbedded2024}.
\item \textbf{Mínimo privilegio (\textit{Least privilege}):} Servicios ejecutan con mínimos permisos necesarios. Ejemplo: Puente MQTT (\textit{MQTT Bridge}) solo puede escribir a tema Kafka (\textit{Kafka topic}) telemetry, no puede leer tema (\textit{topic}) commands~\cite{kandahNavigatingIoTSecurity2025}.
\item \textbf{Verificación continua (\textit{Continuous verification}):} Re-autenticación periódica (actualización JWT (\textit{JWT refresh}) cada 15 min). Analítica de comportamiento (\textit{Behavioral analytics}) detectan actividad anómala (e.g., súbito pico (\textit{spike}) en comandos desde usuario)~\cite{nandalSECURITYRISKSIoT2025,matiasEnhancingEffectivenessSecurity2024}.
\end{itemize}

\textbf{Implementación práctica:} Malla de servicios (\textit{Service mesh}) (Istio, Linkerd) para hacer cumplir (\textit{enforce}) políticas mTLS entre microservicios, Agente de Política Abierta (\textit{Open Policy Agent (OPA)}) para autorización de granularidad fina (\textit{fine-grained}) basada en atributos.

\subsection{Línea 4 - Interoperabilidad Extendida}

\subsubsection{L4.1 - Integración con Protocolos Legacy}

\textbf{Objetivo:} Permitir coexistencia con sistemas SCADA legacy que utilizan protocolos pre-IP: Modbus RTU/TCP, DNP3, IEC 60870-5-104. Esta integración es crítica para modernizar infraestructuras eléctricas existentes sin requerir reemplazo completo de equipamiento legacy, protegiendo inversiones de capital previas~\cite{alsuwaidiSecuringSmartGrid2024}.

\textbf{Estrategia de integración:}
\begin{itemize}
\item Pasarela de modo dual (\textit{Gateway dual-mode}): Interfaz RS-485 para Modbus RTU (PLCs, RTUs antiguos) + Ethernet para Modbus TCP/DNP3.
\item Traductor de protocolo en contenedor (\textit{Protocol translator} containerizado): Servicio que lee registros Modbus (\textit{Modbus registers}) periódicamente, mapea a objetos IEEE 2030.5, publica vía MQTT.
\item Configuración de mapeo (\textit{Mapping configuration}): Archivo YAML (\textit{YAML file}) define correspondencia dirección Modbus (\textit{Modbus address}) ↔ recurso IEEE 2030.5 (\textit{IEEE 2030.5 resource}). Ejemplo: \texttt{40001: {type: voltage, phase: A, unit: V}}.
\item Bidireccional: No solo telemetría sino también comandos. Mensaje MQTT (\textit{MQTT message}) para desconectar interruptor (\textit{trip breaker}) se traduce a código de función Modbus (\textit{Modbus function code}) 05 (Escribir Bobina Única (\textit{Write Single Coil})).
\end{itemize}

\textbf{Caso de uso:} Modernización (\textit{Retrofit}) de subestación heredada (\textit{legacy}) con telemetría moderna sin reemplazar RTUs existentes (costo-prohibitivo).

\subsubsection{L4.2 - Federación de Pasarelas}

\textbf{Motivación:} Despliegues (\textit{Deployments}) a escala de servicios públicos (\textit{Utility-scale}) requieren cientos de pasarelas distribuidas geográficamente. Gestión centralizada desde la nube introduce latencia y punto único de falla (\textit{single point of failure}).

\textbf{Arquitectura entre pares (\textit{peer-to-peer}):}
\begin{itemize}
\item Pasarelas se descubren automáticamente vía mDNS (red local) o descubrimiento de servicios Consul (\textit{Consul service discovery}) (WAN).
\item Cada pasarela publica capacidades (\textit{capabilities}): protocolos soportados, dispositivos conectados, carga actual (CPU/RAM).
\item Solicitudes se enrutan a la pasarela óptima: comando para dispositivo X se enruta a pasarela que gestiona X, balanceo de carga (\textit{load balancing}) para consultas (\textit{queries}) agregadas distribuye entre pasarelas con carga baja.
\item Protocolo de rumores (\textit{Gossip protocol}) (Memberlist, SWIM) mantiene vista consistente de membresía de cluster (\textit{cluster membership}) ante fallas de nodos.
\end{itemize}

\textbf{Aplicación:} Microrredes interconectadas donde pasarelas coordinan comercio local de energía (\textit{local energy trading}), aislamiento coordinado (\textit{islanding coordinated}), procedimientos de arranque en negro (\textit{black start procedures}) sin dependencia de la nube. Arquitecturas distribuidas de pasarelas basadas en agrupamiento (\textit{clustering}) permiten escalabilidad horizontal y tolerancia a fallos mediante replicación de estado y balanceo automático de carga~\cite{effahHardwareEvaluationClusterBased2024}.

\subsection{Línea 5 - Estándares Emergentes}

\subsubsection{L5.1 - Adopción de Matter sobre Thread}

\textbf{Contexto:} Matter (antes Project CHIP) es estándar de interoperabilidad IoT desarrollado por CSA (Connectivity Standards Alliance) con soporte de Apple, Google, Amazon~\cite{shahinzadehSmartHomeConnectivity2024}. Define application layer sobre Thread, Wi-Fi, Ethernet.

\textbf{Oportunidades:}
\begin{itemize}
\item Ecosistema device amplio: 1000+ productos Matter-certified previstos para 2025 (termostatos, switches inteligentes, sensores).
\item Commissioning simplificado: QR code scanning vía smartphone + Matter controller (app iOS/Android).
\item Interoperabilidad vendor-agnostic: Dispositivo Matter de fabricante A controlable por gateway de fabricante B sin custom integration.
\end{itemize}

\textbf{Trabajo futuro:}
\begin{itemize}
\item Implementar Matter controller en gateway (chip-tool open-source de CSA).
\item Mapeo Matter clusters (On/Off, LevelControl, ElectricalMeasurement) a IEEE 2030.5 resources.
\item Validación de latencia extremo-a-extremo Matter device → gateway → ThingsBoard.
\end{itemize}

\subsubsection{L5.2 - Wi-Fi 7 como Evolución de HaLow}

\textbf{Contexto:} Wi-Fi 7 (IEEE 802.11be) introduce mejoras sobre Wi-Fi 6: canales de 320 MHz (\textit{320 MHz channels}), 4096-QAM, Operación Multi-Enlace (\textit{Multi-Link Operation (MLO)}), latencia <5 ms garantizada.

\textbf{Comparativa futura HaLow (802.11ah) vs Wi-Fi 7 (802.11be):}
\begin{itemize}
\item \textbf{HaLow ventajas persistentes:} Alcance largo (penetración sub-1 GHz), consumo ultra-bajo (ciclo de trabajo TWT (\textit{TWT duty cycle}) <0.1\%), costo de módulos menor.
\item \textbf{Wi-Fi 7 ventajas emergentes:} Rendimiento (\textit{Throughput}) masivo (hasta 46 Gbps), latencia determinística (TWT Activado (\textit{Triggered TWT})), compatibilidad hacia atrás (\textit{backward compatibility}) con Wi-Fi 6/5.
\end{itemize}

\textbf{Estrategia híbrida:} HaLow para red de campo (\textit{field network}) (sensores, actuadores alimentados por batería), Wi-Fi 7 para enlace troncal (\textit{backhaul}) (pasarela a pasarela, pasarela a borde en la nube (\textit{gateway-to-cloud edge})) donde rendimiento (\textit{throughput}) crítico.

\subsection{Línea 6 - Evolución Hardware Modular: Upgrade a Chipsets Next-Gen}

\textbf{Motivación:} Como documentado en Cap 2 (análisis productos comerciales) y Cap 4 (TCO upgrade modular), la arquitectura propuesta basada en interfaces estándar (USB 2.0, M.2 E-Key) facilita actualizaciones incrementales de componentes wireless sin reemplazo completo del gateway. Esta línea investiga validación técnica y económica de upgrades a chipsets HaLow de siguiente generación.

\subsubsection{L6.1 - Evaluación Morse Micro MM8108 en Testbed}

\textbf{Objetivo:} Caracterizar mejoras reales de MM8108 vs MM6108 baseline en escenario AMI controlado.

\textbf{Metodología propuesta:}
\begin{itemize}
    \item Adquisición de 3 módulos Gateworks GW16167 (MM8108 M.2 E-Key) para integración en gateways prototipo existentes.
    \item Pruebas comparativas lado-a-lado: Gateway A (MM6108) vs Gateway B (MM8108) en testbed universitario con 50 nodos Thread simulados.
    \item Métricas de evaluación:
    \begin{itemize}
        \item \textbf{Link budget:} Medición RSSI/SNR a distancias 500m, 1km, 1.5km, 2km, 2.5km (campus abierto + entorno urbano con obstáculos)
        \item \textbf{Throughput real:} iperf3 TCP/UDP con cargas 1 Mbps, 5 Mbps, 10 Mbps, 20 Mbps
        \item \textbf{Latencia:} Ping round-trip time bajo carga (background traffic)
        \item \textbf{Consumo energético:} Medición con multímetro high-side shunt en modos TX (+26 dBm), RX, idle, TWT sleep
    \end{itemize}
    \item Análisis comparativo: ¿Mejoras teóricas (+3 dB TX, +3 dB RX, +33\% throughput) se materializan en despliegue real? ¿Trade-offs consumo energético (+12\% TX) impactan autonomía gateway con respaldo batería?
\end{itemize}

\textbf{Resultados esperados:} Reporte técnico validando o refutando mejoras especificadas en datasheet MM8108. Identificación de escenarios donde upgrade es justificado (e.g., zonas rurales con alcance >1.5 km crítico) vs innecesario (urbano denso <1 km suficiente con MM6108).

\textbf{Timeline:} Q2 2026 (2 meses). \textbf{TRL objetivo:} 5-6 (componente validado en entorno relevante).

\textbf{Recursos:} 0.5 PA investigador junior + \$1.5K hardware (3× GW16167 @ \$150 + antennas + instrumentación).

\subsubsection{L6.2 - Validación Alcance Extendido 4-5 km}

\textbf{Objetivo:} Verificar alcance máximo operacional MM8108 en condiciones reales (no anechoic chamber) para optimizar topología red en despliegues utility-scale.

\textbf{Metodología propuesta:}
\begin{itemize}
    \item Pruebas de campo en zona rural (Manizales-Villamaría corredor): Gateway MM8108 fijo (coordenadas GPS registradas) + nodo móvil en vehículo con datalogger GPS/RSSI.
    \item Medición continua RSSI, packet loss rate (%), throughput disponible vs distancia (increments 500m hasta pérdida completa de enlace).
    \item Variación condiciones: (1) Line-of-sight (LoS) en campo abierto, (2) Non-line-of-sight (NLoS) con vegetación densa + topografía montañosa, (3) Interferencia controlada (co-channel 900 MHz ISM band).
    \item Comparación con modelo teórico Friis + corrección Okumura-Hata para 900 MHz: ¿Alcance medido coincide con predicción +6 dB link budget?
\end{itemize}

\textbf{Resultados esperados:} Mapas de cobertura (heatmaps RSSI) validando alcance 2-3 km urbano NLoS, 4-5 km rural LoS con MM8108. Guidelines de diseño de red: "Con MM8108, reducir gateways de 50 a 35 en despliegue 1000 medidores (30\% savings infraestructura)".

\textbf{Timeline:} Q4 2026 (3 meses tras completar L6.1). \textbf{TRL objetivo:} 7 (prototipo demostrado en entorno operacional real).

\textbf{Recursos:} 0.5 PA investigador senior + \$2K logística campo (vehículo, datalogger, permisos acceso predios rurales).

\subsubsection{L6.3 - Piloto Upgrade Modular en 10 Gateways Deployed}

\textbf{Objetivo:} Validar procedimiento de upgrade modular M.2 swap en gateways ya desplegados con medidores activos (prueba de concepto operacional).

\textbf{Metodología propuesta:}
\begin{itemize}
    \item Selección de 10 gateways piloto en despliegue L1.1 (1000+ dispositivos) con criterios: (1) 5 gateways en zonas urbanas densas (baseline, upgrade no crítico), (2) 5 gateways en zonas rurales/periurbanas (alcance extendido beneficioso).
    \item Procedimiento estandarizado de upgrade:
    \begin{enumerate}
        \item Pre-upgrade: Backup configuración gateway (network settings, TLS certificates, device registry)
        \item Shutdown graceful: Flush buffers MQTT/PostgreSQL, notificar ThingsBoard mantenimiento programado
        \item Swap físico: Remover módulo MM6108, insertar GW16167 (MM8108) en slot M.2, verificar conexión mecánica/eléctrica
        \item Post-upgrade: Boot gateway, verificar reconocimiento USB lsusb, cargar firmware MM8108, test conectividad HaLow
        \item Validación operacional: Reconexión DCUs (expected <5 min), throughput test, monitoreo 72h estabilidad
    \end{enumerate}
    \item Métricas éxito: (1) Tiempo downtime <30 min por gateway, (2) 0\% data loss (buffering durante upgrade funcional), (3) Mejoras post-upgrade: RSSI +3-6 dB en DCUs lejanos, reducción packet retransmissions.
\end{itemize}

\textbf{Resultados esperados:} Procedimiento documentado de upgrade field-proven, incluyendo: (1) Checklist técnico paso-a-paso, (2) Troubleshooting guide (e.g., módulo no detectado, driver conflicts), (3) ROI medido: \$175 upgrade cost vs beneficios (extender cobertura sin gateway adicional = savings \$295 CAPEX + \$50/año OPEX).

\textbf{Timeline:} Q2 2027 (6 meses, tras completar L6.2 + L1.1 scale validation). \textbf{TRL objetivo:} 8-9 (sistema completo calificado y demostrado en entorno operacional).

\textbf{Recursos:} 0.5 PA ingeniero de campo + \$3.5K (10× módulos GW16167 @ \$150 + labor on-site 10× \$25 + contingencia).

\textbf{Dependencias críticas L6:}
\begin{itemize}
    \item \textbf{L6.1 → L6.2:} Validación testbed debe confirmar viabilidad técnica antes de pruebas campo costosas.
    \item \textbf{L6.2 → L6.3:} Mapas cobertura identifican gateways candidatos prioritarios para upgrade (máximo beneficio alcance).
    \item \textbf{L1.1 → L6.3:} Piloto upgrade requiere despliegue operacional 1000+ dispositivos como baseline (L1.1 completado Q4 2027).
\end{itemize}

\textbf{Riesgos y mitigaciones:}
\begin{itemize}
    \item \textbf{Riesgo 1 - Incompatibilidad driver MM8108:} Módulo GW16167 requiere kernel Linux >5.10 para full support cfg80211. \textit{Mitigación:} Validar compatibilidad en L6.1, upgrade Raspberry Pi OS Bookworm (kernel 6.1) si necesario.
    \item \textbf{Riesgo 2 - ROI insuficiente en zonas urbanas:} Mejoras alcance MM8108 (+40\%) no justifican upgrade si densidad medidores permite <1 km spacing. \textit{Mitigación:} Upgrade selectivo solo rural/periurbano (50\% gateways), mantener MM6108 en urbano denso.
    \item \textbf{Riesgo 3 - Disponibilidad GW16167:} Módulo Gateworks depende supply chain Morse Micro + fabricación USA. \textit{Mitigación:} Orden anticipada 20-30 unidades Q1 2026, evaluar alternativas M.2 HaLow (NewRadek, AsiaRF) como backup suppliers.
\end{itemize}

\textbf{Conclusión L6:} Línea de investigación valida promesa de arquitectura modular: upgrades incrementales de componentes críticos (radio HaLow) extienden vida útil gateway de 5 años (HW monolítico obsoleto) a 10+ años con refreshes tecnológicos económicos (\$175 vs \$445 reemplazo completo). \textbf{Si L6.3 demuestra upgrade field-proven con <30 min downtime y ROI positivo, arquitectura propuesta no solo es óptima para 2025, sino sostenible para 2030-2035.}

\section{Impacto y Contribuciones}

\subsection{Impacto Académico}

\textbf{Publicaciones derivadas:}
\begin{itemize}
\item Artículo IEEE IoT Journal (\textit{Paper IEEE IoT Journal}): "Multi-Protocol Edge Gateway Architecture for Smart Energy: Integrating Thread, HaLow and LTE" (en preparación).
\item Conferencia IEEE SmartGridComm 2025: "Empirical Evaluation of IEEE 2030.5 Latency in Edge Computing Scenarios" (aceptado).
\item Capítulo de libro Springer: "Edge Computing for Critical Infrastructure: A Smart Grid Perspective" (propuesto).
\end{itemize}

\textbf{Formación de recurso humano:}
\begin{itemize}
\item 2 tesis de pregrado dirigidas: (1) "Implementación de cliente LwM2M en ESP32-C6"; (2) "Análisis de alcance Wi-Fi HaLow en entornos urbanos".
\item 1 pasantía industrial: Integración de pasarela con plataforma SCADA comercial (empresa de servicios públicos (\textit{utility}) regional).
\end{itemize}

\subsection{Impacto Industrial}

\textbf{Transferencia tecnológica:}
\begin{itemize}
\item Repositorio de código abierto (\textit{open-source}) con 450+ estrellas en GitHub (6 meses post-publicación proyectado).
\item Adopción por 2 empresas de servicios públicos (\textit{utilities}) colombianas para pilotos (\textit{pilots}) (300 medidores cada una, Q3 2025 inicio).
\item Interés de proveedores (\textit{vendors}) (Morse Micro, Nordic Semiconductor) para integración en diseños de referencia (\textit{reference designs}) comerciales.
\end{itemize}

\textbf{Impacto económico estimado}:
\begin{itemize}
    \item Reducción CAPEX: Gateway propuesto \$450 vs soluciones comerciales \$1200-2000 (ahorro 62-77\%).
    \item Reducción OPEX: Costos conectividad \$12/mes vs \$85/mes cloud-centric (ahorro 85.9\% por gateway).
    \item Para deployment 500 gateways @ 10 años: ahorro total \$((500Í—(1200-450)) + (500 Í— 10 Í— 12 Í— (85-12))) = \$375k + \$4.38M = \textbf{\$4.76M}.
\end{itemize}

\section{Reflexiones Finales}

La presente investigación demostró que una arquitectura IoT edge bien diseñada, combinando protocolos heterogéneos (Thread, HaLow, LTE), tecnologías de containerización, y conformidad con estándares abiertos (IEEE 2030.5, ISO/IEC 30141), puede satisfacer simultáneamente requerimientos aparentemente contradictorios de sistemas Smart Energy: baja latencia Y alta disponibilidad, procesamiento inteligente Y consumo energético eficiente, interoperabilidad multi-vendor Y seguridad robusta.

El cambio de paradigma de arquitecturas cloud-centric a edge-centric no es mera optimización técnica, sino habilitador de casos de uso transformadores: control volt-VAR en tiempo real, gestión autónoma de microrredes, detección predictiva de fallas, coordinación peer-to-peer de recursos distribuidos. Estos casos de uso, a su vez, son pilares de la transición energética hacia sistemas descarbonizados, resilientes y participativos.

El trabajo futuro propuesto —escalabilidad, ML avanzado, seguridad Zero Trust, federación de gateways— no son meras extensiones incrementales, sino evolución hacia verdaderos "nervous systems" distribuidos para infraestructura eléctrica, donde inteligencia emerge de coordinación local entre nodos autónomos, no de orquestación centralizada.

La convergencia de protocolos 6LoWPAN, plataformas edge open-source, y estándares de interoperabilidad crea, por primera vez, condiciones para ecosistemas Smart Energy genuinamente abiertos y competitivos. El presente trabajo aspira ser contribución modesta pero concreta hacia esa visión.

