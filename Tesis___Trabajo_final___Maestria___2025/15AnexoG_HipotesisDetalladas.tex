\chapter{Hipótesis Específicas Detalladas}
\label{anexo:hipotesis-detalladas}

Este anexo presenta el desglose detallado de las ocho hipótesis específicas que concretan la hipótesis general presentada en §1.2.4. Cada hipótesis específica es validada empíricamente en los Capítulos 4 (Implementación) y 5 (Resultados).

\section{H1 - Optimización mediante 6LoWPAN/CoAP/LwM2M}

\textbf{Enunciado:} La implementación del stack 6LoWPAN (compresión IPHC/NHC) + CoAP (overhead 4-10 bytes) + LwM2M (objetos binarios TLV) sobre IEEE 802.15.4 reduce el overhead de paquetes en >70\% y la latencia por salto en >40\% comparado con MQTT/JSON sobre TCP/IP, logrando tiempos de transmisión <15 ms por hop en topologías mesh de hasta 5 saltos.

\textbf{Métricas de validación:}
\begin{itemize}
\item Overhead de paquetes: (bytes\_header / bytes\_payload) comparado entre stacks
\item Latencia por salto: tiempo\_transmision\_hop medido con timestamps NTP
\item Topología mesh: configuraciones 2, 3, 4, 5 saltos entre nodo end-device y gateway
\end{itemize}

\textbf{Validación experimental:} Sección §5.3.1

\section{H2 - Procesamiento Edge con IA}

\textbf{Enunciado:} El despliegue de servicios containerizados edge (ThingsBoard Edge, TimescaleDB, Kafka) con integración de modelos LLM locales (Ollama + Llama 3.2 3B) permite: (a) reducción de tráfico WAN en >65\% mediante procesamiento local, (b) latencia de inferencia <500 ms para detección de anomalías, (c) disponibilidad de servicios >99\% durante desconexiones WAN >72 horas, y (d) precisión de detección de anomalías >95\% en patrones de consumo energético.

\textbf{Métricas de validación:}
\begin{itemize}
\item Reducción tráfico WAN: (GB\_cloud - GB\_edge) / GB\_cloud × 100\%
\item Latencia inferencia: timestamp\_request → timestamp\_response para queries LLM
\item Disponibilidad offline: uptime\_servicios\_locales durante partición WAN 72h+
\item Precisión anomalías: (true\_positives + true\_negatives) / total\_events × 100\%
\end{itemize}

\textbf{Validación experimental:} Secciones §5.2.2 (Edge Computing), §5.2.4 (IA Local)

\section{H3 - Arquitectura Multi-Banda 802.11ah}

\textbf{Enunciado:} La arquitectura basada en gateways HaLow con selección estratégica de bandwidth según caso de uso maximiza eficiencia operacional:
\begin{itemize}
\item \textbf{2 MHz}: Óptimo para conexiones estables con sensores remotos (>2 km alcance, sensibilidad -96 dBm, tráfico <100 kbps, entornos NLOS con penetración indoor superior), logrando PDR >98\% en condiciones adversas con SNR 8-12 dB.
\item \textbf{4 MHz}: Balance ideal para gestión de red (1-1.5 km alcance, throughput 40 Mbps agregado, latencia <50 ms P95), soportando 50+ nodos con tráfico moderado (lecturas cada 15 min) sin degradación >10\%.
\item \textbf{8 MHz}: Maximiza throughput para alto tráfico con línea de vista (backhaul de concentradores, >80 Mbps, latencia <20 ms P99, alcance 0.5-1 km LOS), permitiendo agregación de datos de 100+ dispositivos por gateway.
\end{itemize}

\textbf{Métricas de validación:}
\begin{itemize}
\item Alcance efectivo: distancia\_max con PDR >95\% medida con site survey
\item Sensibilidad: RSSI\_min (dBm) para establecer enlace estable
\item Throughput agregado: suma de throughput de todos los clientes asociados
\item Latencia P95/P99: percentiles 95 y 99 de distribución de latencias
\item PDR vs SNR: curva experimental de Packet Delivery Ratio en función de Signal-to-Noise Ratio
\end{itemize}

\textbf{Validación experimental:} Sección §5.6 (Validación de Throughput HaLow)

\section{H4 - Compresión 6LoWPAN de Headers}

\textbf{Enunciado:} La compresión IPHC (IPv6 Header Compression) de 6LoWPAN reduce headers IPv6+UDP de 48 bytes a 2-7 bytes (compresión >85\%), y la compresión NHC (Next Header Compression) para CoAP reduce overhead adicional de 10-20 bytes a 2-4 bytes, resultando en payloads efectivos >90\% del MTU IEEE 802.15.4 (127 bytes) para aplicaciones Smart Energy.

\textbf{Métricas de validación:}
\begin{itemize}
\item Compresión IPHC: (48 - bytes\_compressed\_header) / 48 × 100\%
\item Compresión NHC: (bytes\_uncompressed\_coap - bytes\_compressed\_coap) / bytes\_uncompressed\_coap × 100\%
\item Payload efectivo: bytes\_payload / 127 × 100\%
\end{itemize}

\textbf{Validación experimental:} Sección §5.3.1 (captura Wireshark de paquetes 6LoWPAN)

\section{H5 - Eficiencia CoAP vs MQTT}

\textbf{Enunciado:} CoAP sobre UDP con modos Non-Confirmable (NON) para telemetría no crítica y Confirmable (CON) para comandos críticos, combinado con Observe para subscripciones, reduce latencia en >50\% y overhead de red en >60\% comparado con MQTT/TCP, logrando tiempos de respuesta <30 ms para transacciones GET/POST en redes Thread mesh.

\textbf{Métricas de validación:}
\begin{itemize}
\item Latencia CoAP: tiempo\_request → tiempo\_response para GET/POST
\item Latencia MQTT: tiempo\_publish → tiempo\_subscribe\_callback
\item Overhead CoAP: bytes\_header\_coap + bytes\_udp + bytes\_ipv6
\item Overhead MQTT: bytes\_header\_mqtt + bytes\_tcp + bytes\_ipv6
\item Reducción latencia: (latencia\_mqtt - latencia\_coap) / latencia\_mqtt × 100\%
\item Reducción overhead: (overhead\_mqtt - overhead\_coap) / overhead\_mqtt × 100\%
\end{itemize}

\textbf{Validación experimental:} Sección §5.3.1

\section{H6 - LwM2M para Gestión Eficiente}

\textbf{Enunciado:} LwM2M con objetos estándar OMA (Device, Connectivity Monitoring, Firmware Update) y transporte CoAP reduce tráfico de gestión de dispositivos en >75\% comparado con soluciones propietarias HTTP/REST, permitiendo actualizaciones OTA de firmware con transferencia block-wise sobre enlaces de baja velocidad (<250 kbps) sin timeouts.

\textbf{Métricas de validación:}
\begin{itemize}
\item Tráfico gestión LwM2M: bytes\_totales operaciones Register, Update, Read, Write, Execute durante 24h
\item Tráfico gestión HTTP/REST: bytes\_totales operaciones equivalentes API REST durante 24h
\item Reducción tráfico: (bytes\_rest - bytes\_lwm2m) / bytes\_rest × 100\%
\item Tiempo actualización OTA: duración transferencia firmware completo (ej. 512 KB) con block-wise
\item Tasa de éxito OTA: (actualizaciones\_exitosas / actualizaciones\_intentadas) × 100\%
\end{itemize}

\textbf{Validación experimental:} Sección §4.3.2 (Firmware OTA), §5.3.1

\section{H7 - Procesamiento CEP Local}

\textbf{Enunciado:} El motor de reglas Complex Event Processing (CEP) de ThingsBoard Edge desplegado localmente en gateway procesa >10,000 eventos/seg con latencia <10 ms P99, ejecutando rule chains complejas (filtrado, agregación, transformación, alarmas) sin requerir round-trip WAN, reduciendo latencia de respuesta en >80\% comparado con procesamiento cloud.

\textbf{Métricas de validación:}
\begin{itemize}
\item Throughput CEP: eventos\_procesados / segundo medido con load testing
\item Latencia P99 local: percentil 99 de distribución tiempo\_ingestion → tiempo\_alarm
\item Latencia P99 cloud: percentil 99 incluyendo RTT WAN para procesamiento remoto
\item Reducción latencia: (latencia\_cloud\_p99 - latencia\_edge\_p99) / latencia\_cloud\_p99 × 100\%
\end{itemize}

\textbf{Validación experimental:} Sección §5.2.2

\section{H8 - Ventaja Comparativa Integral}

\textbf{Enunciado:} La arquitectura propuesta supera a arquitecturas tradicionales (cloud-centric MQTT/LTE) en al menos 5 de 7 métricas clave: latencia (<30\% baseline), overhead paquetes (<40\% baseline), tráfico WAN (<35\% baseline), disponibilidad offline (>72h vs 0h), precisión IA (>95\% vs N/A), alcance HaLow (>150\% vs WiFi), y eficiencia energética (<60\% baseline).

\textbf{Métricas de validación:}
\begin{enumerate}
\item \textbf{Latencia}: E2E\_latency\_propuesta / E2E\_latency\_baseline
\item \textbf{Overhead}: (header\_bytes\_propuesta / payload\_bytes) / (header\_bytes\_baseline / payload\_bytes)
\item \textbf{Tráfico WAN}: GB\_wan\_propuesta / GB\_wan\_baseline
\item \textbf{Disponibilidad offline}: horas\_operacion\_autonoma\_propuesta vs baseline
\item \textbf{Precisión IA}: accuracy\_deteccion\_anomalias\_propuesta vs baseline\_sin\_ia
\item \textbf{Alcance}: distancia\_km\_halow / distancia\_km\_wifi\_24ghz
\item \textbf{Eficiencia energética}: mJ\_por\_bit\_propuesta / mJ\_por\_bit\_baseline
\end{enumerate}

\textbf{Criterio éxito:} $\geq$5 de 7 métricas cumpliendo los umbrales especificados

\textbf{Validación experimental:} Sección §5.9 (Comparación con Literatura), §5.10 (Discusión de Resultados)

\section{Resumen de Mapeo Hipótesis-Validación}

\begin{table}[h]
\centering
\caption{Mapeo entre Hipótesis Específicas y Secciones de Validación Experimental}
\label{tab:mapeo-hipotesis-validacion}
\begin{tabular}{|l|p{7cm}|l|}
\hline
\textbf{ID} & \textbf{Hipótesis} & \textbf{Validación} \\
\hline
H1 & Stack 6LoWPAN/CoAP/LwM2M optimizado & §5.3.1 \\
\hline
H2 & Procesamiento Edge con IA & §5.2.2, §5.2.4 \\
\hline
H3 & Arquitectura Multi-Banda 802.11ah & §5.6 \\
\hline
H4 & Compresión 6LoWPAN headers & §5.3.1 \\
\hline
H5 & Eficiencia CoAP vs MQTT & §5.3.1 \\
\hline
H6 & LwM2M gestión eficiente & §4.3.2, §5.3.1 \\
\hline
H7 & Procesamiento CEP local & §5.2.2 \\
\hline
H8 & Ventaja comparativa integral & §5.9, §5.10 \\
\hline
\end{tabular}
\end{table}

Este anexo establece el marco de validación experimental que conecta el planteamiento teórico del Capítulo 1 con la implementación del Capítulo 4 y los resultados empíricos del Capítulo 5, garantizando trazabilidad completa entre hipótesis, métricas y evidencia experimental.
