\chapter{Anexo D: Especificaciones IEEE 2030.5 y Configuraciones}
\label{anexo:configs}

Este anexo documenta las especificaciones completas de configuración para los componentes del gateway, incluyendo ejemplos XML IEEE 2030.5, comandos UCI para HaLow, y optimizaciones para TimescaleDB.

\section{Ejemplos XML IEEE 2030.5}

\subsection{Device Capability (DCAP)}

Documento XML completo del endpoint de descubrimiento de capacidades:

\begin{verbatim}
<?xml version="1.0" encoding="UTF-8"?>
<DeviceCapability xmlns="urn:ieee:std:2030.5:ns">
  <href>/dcap</href>
  <pollRate>900</pollRate>
  <TimeLink href="/tm"/>
  <MirrorUsagePointListLink href="/mup" all="0"/>
  <MessagingProgramListLink href="/msg" all="0"/>
  <EndDeviceListLink href="/edev" all="0"/>
  <DERProgramListLink href="/derp" all="0"/>
  <SelfDeviceLink href="/sdev"/>
</DeviceCapability>
\end{verbatim}

\subsection{Time Synchronization (TM)}

Respuesta de sincronización horaria con calidad máxima:

\begin{verbatim}
<?xml version="1.0" encoding="UTF-8"?>
<Time xmlns="urn:ieee:std:2030.5:ns">
  <currentTime>1730000000</currentTime>
  <dstEndTime>1698627600</dstEndTime>
  <dstOffset>3600</dstOffset>
  <dstStartTime>1710046800</dstStartTime>
  <localTime>1730000000</localTime>
  <quality>7</quality>
  <tzOffset>-18000</tzOffset>
</Time>
\end{verbatim}

\textbf{Campos importantes:}
\begin{itemize}
    \item \texttt{currentTime}: Tiempo UNIX en segundos (UTC).
    \item \texttt{quality}: 0-7, donde 7 indica sincronización NTP con precisión < 100 ms.
    \item \texttt{tzOffset}: Offset en segundos desde UTC (Colombia: -18000 = UTC-5).
    \item \texttt{dstOffset}: Offset adicional durante horario de verano (si aplica).
\end{itemize}

\subsection{Mirror Usage Point (MUP)}

Ejemplo de telemetría de medición reflejada:

\begin{verbatim}
<?xml version="1.0" encoding="UTF-8"?>
<MirrorUsagePoint xmlns="urn:ieee:std:2030.5:ns">
  <mRID>0123456789ABCDEF0123456789ABCDEF</mRID>
  <deviceLFDI>0123456789ABCDEF</deviceLFDI>
  <MirrorMeterReading>
    <mRID>mr_energy_001</mRID>
    <description>Active Energy Delivered</description>
    <Reading>
      <consumptionBlock>0</consumptionBlock>
      <qualityFlags>0</qualityFlags>
      <timePeriod>
        <duration>900</duration>
        <start>1730000000</start>
      </timePeriod>
      <touTier>0</touTier>
      <value>123456789</value>
      <localID>1</localID>
    </Reading>
    <ReadingType>
      <accumulationBehaviour>4</accumulationBehaviour>
      <commodity>1</commodity>
      <dataQualifier>0</dataQualifier>
      <flowDirection>1</flowDirection>
      <intervalLength>900</intervalLength>
      <kind>12</kind>
      <phase>0</phase>
      <powerOfTenMultiplier>0</powerOfTenMultiplier>
      <timeAttribute>0</timeAttribute>
      <uom>72</uom>
    </ReadingType>
  </MirrorMeterReading>
  <MirrorMeterReading>
    <mRID>mr_power_001</mRID>
    <description>Instantaneous Active Power</description>
    <Reading>
      <qualityFlags>0</qualityFlags>
      <timePeriod>
        <duration>900</duration>
        <start>1730000000</start>
      </timePeriod>
      <value>1250</value>
      <localID>2</localID>
    </Reading>
    <ReadingType>
      <accumulationBehaviour>0</accumulationBehaviour>
      <commodity>1</commodity>
      <dataQualifier>0</dataQualifier>
      <flowDirection>1</flowDirection>
      <intervalLength>0</intervalLength>
      <kind>12</kind>
      <phase>0</phase>
      <powerOfTenMultiplier>0</powerOfTenMultiplier>
      <timeAttribute>0</timeAttribute>
      <uom>38</uom>
    </ReadingType>
  </MirrorMeterReading>
  <MirrorMeterReading>
    <mRID>mr_voltage_001</mRID>
    <description>RMS Voltage</description>
    <Reading>
      <qualityFlags>0</qualityFlags>
      <timePeriod>
        <duration>900</duration>
        <start>1730000000</start>
      </timePeriod>
      <value>2305</value>
      <localID>3</localID>
    </Reading>
    <ReadingType>
      <accumulationBehaviour>0</accumulationBehaviour>
      <commodity>1</commodity>
      <dataQualifier>0</dataQualifier>
      <flowDirection>1</flowDirection>
      <intervalLength>0</intervalLength>
      <kind>12</kind>
      <phase>0</phase>
      <powerOfTenMultiplier>-1</powerOfTenMultiplier>
      <timeAttribute>0</timeAttribute>
      <uom>29</uom>
    </ReadingType>
  </MirrorMeterReading>
</MirrorUsagePoint>
\end{verbatim}

\textbf{ReadingType - Unidades de Medida (uom):}
\begin{itemize}
    \item \texttt{38}: Watts (W) - Potencia activa
    \item \texttt{72}: Watt-hours (Wh) - Energía activa
    \item \texttt{29}: Voltage (V) - Voltaje RMS
    \item \texttt{5}: Current (A) - Corriente RMS
    \item \texttt{63}: Volt-Ampere Reactive (VAr) - Potencia reactiva
\end{itemize}

\subsection{End Device List}

Lista de dispositivos registrados con identificadores LFDI/SFDI:

\begin{verbatim}
<?xml version="1.0" encoding="UTF-8"?>
<EndDeviceList xmlns="urn:ieee:std:2030.5:ns" all="3">
  <EndDevice href="/edev/001">
    <changedTime>1730000000</changedTime>
    <enabled>true</enabled>
    <lFDI>0123456789ABCDEF</lFDI>
    <sFDI>01234567</sFDI>
    <FunctionSetAssignmentsListLink href="/edev/001/fsa" all="4"/>
    <RegistrationLink href="/edev/001/rg"/>
  </EndDevice>
  <EndDevice href="/edev/002">
    <changedTime>1730001000</changedTime>
    <enabled>true</enabled>
    <lFDI>FEDCBA9876543210</lFDI>
    <sFDI>FEDCBA98</sFDI>
    <FunctionSetAssignmentsListLink href="/edev/002/fsa" all="4"/>
    <RegistrationLink href="/edev/002/rg"/>
  </EndDevice>
  <EndDevice href="/edev/003">
    <changedTime>1730002000</changedTime>
    <enabled>true</enabled>
    <lFDI>1234567890ABCDEF</lFDI>
    <sFDI>12345678</sFDI>
    <FunctionSetAssignmentsListLink href="/edev/003/fsa" all="4"/>
    <RegistrationLink href="/edev/003/rg"/>
  </EndDevice>
</EndDeviceList>
\end{verbatim}

\section{Configuraciones UCI para HaLow 802.11ah}

\subsection{Modo Access Point (AP)}

Configuración completa del gateway como AP HaLow:

\begin{verbatim}
# Interfaz inalámbrica HaLow (wlan2)
uci set wireless.halow=wifi-device
uci set wireless.halow.type='mac80211'
uci set wireless.halow.path='platform/soc/1e140000.pcie/pci0000:00/0000:00:00.0/0000:01:00.0'
uci set wireless.halow.channel='7'        # 917 MHz (S1G)
uci set wireless.halow.bandwidth='8'      # 8 MHz (opciones: 1, 2, 4, 8, 16)
uci set wireless.halow.hwmode='11ah'
uci set wireless.halow.country='US'
uci set wireless.halow.txpower='20'       # 20 dBm = 100 mW
uci set wireless.halow.legacy_rates='0'
uci set wireless.halow.mu_beamformer='0'
uci set wireless.halow.mu_beamformee='0'

# Interfaz virtual AP
uci set wireless.halow_ap=wifi-iface
uci set wireless.halow_ap.device='halow'
uci set wireless.halow_ap.mode='ap'
uci set wireless.halow_ap.network='halow_lan'
uci set wireless.halow_ap.ssid='SmartGrid-HaLow-AP'
uci set wireless.halow_ap.encryption='sae'
uci set wireless.halow_ap.key='<WPA3-PSK-SECURE-KEY>'
uci set wireless.halow_ap.ieee80211w='2'  # PMF obligatorio
uci set wireless.halow_ap.sae_pwe='2'     # Hash-to-Element (H2E)
uci set wireless.halow_ap.wpa_disable_eapol_key_retries='1'
uci set wireless.halow_ap.max_inactivity='600'  # 10 min timeout
uci set wireless.halow_ap.disassoc_low_ack='0'
uci set wireless.halow_ap.skip_inactivity_poll='0'

# Red virtual para HaLow
uci set network.halow_lan=interface
uci set network.halow_lan.proto='static'
uci set network.halow_lan.ipaddr='192.168.100.1'
uci set network.halow_lan.netmask='255.255.255.0'
uci set network.halow_lan.ip6assign='64'
uci set network.halow_lan.ip6hint='100'

# DHCP server para clientes HaLow
uci set dhcp.halow=dhcp
uci set dhcp.halow.interface='halow_lan'
uci set dhcp.halow.start='100'
uci set dhcp.halow.limit='150'
uci set dhcp.halow.leasetime='12h'
uci set dhcp.halow.dhcpv6='server'
uci set dhcp.halow.ra='server'
uci set dhcp.halow.ra_management='1'

# Firewall zone
uci set firewall.halow_zone=zone
uci set firewall.halow_zone.name='halow'
uci set firewall.halow_zone.input='ACCEPT'
uci set firewall.halow_zone.output='ACCEPT'
uci set firewall.halow_zone.forward='ACCEPT'
uci set firewall.halow_zone.network='halow_lan'

uci set firewall.halow_lan_forwarding=forwarding
uci set firewall.halow_lan_forwarding.src='halow'
uci set firewall.halow_lan_forwarding.dest='lan'

uci set firewall.halow_wan_forwarding=forwarding
uci set firewall.halow_wan_forwarding.src='halow'
uci set firewall.halow_wan_forwarding.dest='wan'

# Aplicar configuración
uci commit wireless
uci commit network
uci commit dhcp
uci commit firewall

# Reiniciar servicios
wifi reload
/etc/init.d/network restart
/etc/init.d/firewall restart
\end{verbatim}

\subsection{Modo Station (STA)}

Configuración del gateway para conectarse a AP HaLow remoto:

\begin{verbatim}
# Interfaz HaLow como Station
uci set wireless.halow=wifi-device
uci set wireless.halow.type='mac80211'
uci set wireless.halow.channel='auto'    # Auto-scan
uci set wireless.halow.bandwidth='8'
uci set wireless.halow.hwmode='11ah'
uci set wireless.halow.country='US'
uci set wireless.halow.disabled='0'

uci set wireless.halow_sta=wifi-iface
uci set wireless.halow_sta.device='halow'
uci set wireless.halow_sta.mode='sta'
uci set wireless.halow_sta.network='wan_halow'
uci set wireless.halow_sta.ssid='SmartGrid-HaLow-Backhaul'
uci set wireless.halow_sta.encryption='sae'
uci set wireless.halow_sta.key='<WPA3-PSK-BACKHAUL>'
uci set wireless.halow_sta.ieee80211w='2'

# Red WAN via HaLow
uci set network.wan_halow=interface
uci set network.wan_halow.proto='dhcp'
uci set network.wan_halow.metric='20'  # Métrica menor = mayor prioridad

# Agregar a mwan3 para failover
uci set mwan3.wan_halow=interface
uci set mwan3.wan_halow.enabled='1'
uci set mwan3.wan_halow.family='ipv4'
uci set mwan3.wan_halow.track_ip='8.8.8.8'
uci set mwan3.wan_halow.track_ip='1.1.1.1'
uci set mwan3.wan_halow.track_method='ping'
uci set mwan3.wan_halow.reliability='1'
uci set mwan3.wan_halow.count='1'
uci set mwan3.wan_halow.size='56'
uci set mwan3.wan_halow.max_ttl='60'
uci set mwan3.wan_halow.timeout='2'
uci set mwan3.wan_halow.interval='5'
uci set mwan3.wan_halow.down='3'
uci set mwan3.wan_halow.up='3'

uci commit wireless
uci commit network
uci commit mwan3

wifi reload
/etc/init.d/network restart
/etc/init.d/mwan3 restart
\end{verbatim}

\subsection{Modo Mesh 802.11s}

Configuración para red mesh sin controlador centralizado:

\begin{verbatim}
# Interfaz HaLow Mesh
uci set wireless.halow=wifi-device
uci set wireless.halow.type='mac80211'
uci set wireless.halow.channel='7'
uci set wireless.halow.bandwidth='8'
uci set wireless.halow.hwmode='11ah'
uci set wireless.halow.country='US'
uci set wireless.halow.txpower='20'

uci set wireless.halow_mesh=wifi-iface
uci set wireless.halow_mesh.device='halow'
uci set wireless.halow_mesh.mode='mesh'
uci set wireless.halow_mesh.mesh_id='smartgrid-mesh'
uci set wireless.halow_mesh.mesh_fwding='1'
uci set wireless.halow_mesh.mesh_ttl='31'
uci set wireless.halow_mesh.mesh_rssi_threshold='-80'
uci set wireless.halow_mesh.encryption='sae'
uci set wireless.halow_mesh.key='<MESH-KEY>'
uci set wireless.halow_mesh.network='mesh_lan'

# Red mesh
uci set network.mesh_lan=interface
uci set network.mesh_lan.proto='batadv_hardif'
uci set network.mesh_lan.master='bat0'
uci set network.mesh_lan.mtu='1532'

uci set network.bat0=interface
uci set network.bat0.proto='static'
uci set network.bat0.ipaddr='10.100.0.1'
uci set network.bat0.netmask='255.255.0.0'
uci set network.bat0.ip6assign='64'

# Batman-adv
uci set batman-adv.bat0=mesh
uci set batman-adv.bat0.aggregated_ogms='1'
uci set batman-adv.bat0.ap_isolation='0'
uci set batman-adv.bat0.bonding='0'
uci set batman-adv.bat0.fragmentation='1'
uci set batman-adv.bat0.gw_mode='server'
uci set batman-adv.bat0.log_level='0'
uci set batman-adv.bat0.orig_interval='5000'
uci set batman-adv.bat0.bridge_loop_avoidance='1'
uci set batman-adv.bat0.distributed_arp_table='1'
uci set batman-adv.bat0.multicast_mode='1'

uci commit wireless
uci commit network
uci commit batman-adv

# Cargar módulo kernel
modprobe batman-adv

wifi reload
/etc/init.d/network restart
\end{verbatim}

\subsection{Modo EasyMesh (IEEE 1905.1)}

Configuración para mesh gestionado con controlador y agentes:

\begin{verbatim}
# Controlador EasyMesh (Gateway principal)
uci set easymesh.config=easymesh
uci set easymesh.config.enabled='1'
uci set easymesh.config.role='controller'

# Interfaz backhaul HaLow
uci set wireless.halow_backhaul=wifi-iface
uci set wireless.halow_backhaul.device='halow'
uci set wireless.halow_backhaul.mode='ap'
uci set wireless.halow_backhaul.network='backhaul'
uci set wireless.halow_backhaul.ssid='mesh-backhaul-5g'
uci set wireless.halow_backhaul.encryption='sae'
uci set wireless.halow_backhaul.key='<BACKHAUL-KEY>'
uci set wireless.halow_backhaul.multi_ap='2'  # Backhaul BSS
uci set wireless.halow_backhaul.ieee80211w='2'
uci set wireless.halow_backhaul.hidden='1'

# Interfaz frontal para clientes
uci set wireless.halow_front=wifi-iface
uci set wireless.halow_front.device='halow'
uci set wireless.halow_front.mode='ap'
uci set wireless.halow_front.network='lan'
uci set wireless.halow_front.ssid='SmartGrid-HaLow'
uci set wireless.halow_front.encryption='sae'
uci set wireless.halow_front.key='<CLIENT-KEY>'
uci set wireless.halow_front.multi_ap='1'  # Fronthaul BSS
uci set wireless.halow_front.ieee80211w='2'

# Red backhaul
uci set network.backhaul=interface
uci set network.backhaul.proto='static'
uci set network.backhaul.ipaddr='192.168.200.1'
uci set network.backhaul.netmask='255.255.255.0'

# Servicios EasyMesh
uci set ieee1905.ieee1905=ieee1905
uci set ieee1905.ieee1905.enabled='1'
uci set ieee1905.ieee1905.al_interface='eth0'
uci set ieee1905.ieee1905.management_interface='br-lan'

uci commit easymesh
uci commit wireless
uci commit network
uci commit ieee1905

/etc/init.d/easymesh enable
/etc/init.d/easymesh start
wifi reload
\end{verbatim}

\section{Optimización TimescaleDB}

\subsection{Configuración PostgreSQL + TimescaleDB}

Optimizaciones para almacenamiento de series temporales de alta frecuencia:

\begin{verbatim}
# postgresql.conf (dentro del contenedor)
# Ubicación: /var/lib/postgresql/data/postgresql.conf

# --- Memoria ---
shared_buffers = 2GB              # 25% de RAM (para RPi4 8GB)
effective_cache_size = 6GB        # 75% de RAM
work_mem = 16MB                   # Por operación de sort/hash
maintenance_work_mem = 512MB      # Para VACUUM, CREATE INDEX

# --- Escritura ---
wal_buffers = 16MB
checkpoint_completion_target = 0.9
max_wal_size = 4GB
min_wal_size = 1GB
wal_compression = on

# --- Checkpoints (reducir I/O en SSD) ---
checkpoint_timeout = 30min
checkpoint_warning = 5min

# --- Queries ---
random_page_cost = 1.1            # SSD, no HDD
effective_io_concurrency = 200    # Para NVMe
max_worker_processes = 4          # CPUs disponibles
max_parallel_workers_per_gather = 2
max_parallel_workers = 4

# --- Logging ---
logging_collector = on
log_destination = 'csvlog'
log_directory = 'log'
log_filename = 'postgresql-%Y-%m-%d.log'
log_rotation_age = 1d
log_rotation_size = 100MB
log_min_duration_statement = 1000  # Log queries > 1s

# --- TimescaleDB ---
shared_preload_libraries = 'timescaledb'
timescaledb.max_background_workers = 4
\end{verbatim}

\subsection{Schema y Hypertables}

Creación de tablas optimizadas para telemetría:

\begin{verbatim}
-- Crear extensión TimescaleDB
CREATE EXTENSION IF NOT EXISTS timescaledb;

-- Tabla principal de telemetría
CREATE TABLE telemetry (
  time        TIMESTAMPTZ NOT NULL,
  device_id   TEXT NOT NULL,
  metric      TEXT NOT NULL,
  value       DOUBLE PRECISION,
  unit        TEXT,
  quality     SMALLINT DEFAULT 0
);

-- Convertir a hypertable (particionado automático por tiempo)
SELECT create_hypertable('telemetry', 'time', 
  chunk_time_interval => INTERVAL '1 day');

-- Índices para queries frecuentes
CREATE INDEX idx_telemetry_device_time ON telemetry (device_id, time DESC);
CREATE INDEX idx_telemetry_metric_time ON telemetry (metric, time DESC);

-- Compresión automática (chunks > 7 días)
ALTER TABLE telemetry SET (
  timescaledb.compress,
  timescaledb.compress_segmentby = 'device_id,metric',
  timescaledb.compress_orderby = 'time DESC'
);

SELECT add_compression_policy('telemetry', INTERVAL '7 days');

-- Retención automática (eliminar datos > 1 año)
SELECT add_retention_policy('telemetry', INTERVAL '365 days');

-- Continuous Aggregates (vistas materializadas)
CREATE MATERIALIZED VIEW telemetry_15min
WITH (timescaledb.continuous) AS
SELECT time_bucket('15 minutes', time) AS bucket,
       device_id,
       metric,
       AVG(value) AS avg_value,
       MAX(value) AS max_value,
       MIN(value) AS min_value,
       COUNT(*) AS sample_count
FROM telemetry
GROUP BY bucket, device_id, metric
WITH NO DATA;

-- Refrescar cada 5 minutos
SELECT add_continuous_aggregate_policy('telemetry_15min',
  start_offset => INTERVAL '1 hour',
  end_offset => INTERVAL '5 minutes',
  schedule_interval => INTERVAL '5 minutes');

-- Vista agregada horaria
CREATE MATERIALIZED VIEW telemetry_hourly
WITH (timescaledb.continuous) AS
SELECT time_bucket('1 hour', time) AS bucket,
       device_id,
       metric,
       AVG(value) AS avg_value,
       MAX(value) AS max_value,
       MIN(value) AS min_value,
       STDDEV(value) AS stddev_value,
       COUNT(*) AS sample_count
FROM telemetry
GROUP BY bucket, device_id, metric
WITH NO DATA;

SELECT add_continuous_aggregate_policy('telemetry_hourly',
  start_offset => INTERVAL '1 day',
  end_offset => INTERVAL '1 hour',
  schedule_interval => INTERVAL '1 hour');
\end{verbatim}

\subsection{Queries de Ejemplo}

\begin{verbatim}
-- Telemetría reciente de un dispositivo (últimos 15 min)
SELECT time, metric, value, unit
FROM telemetry
WHERE device_id = 'meter_001' 
  AND time > NOW() - INTERVAL '15 minutes'
ORDER BY time DESC;

-- Consumo energético diario agregado
SELECT time_bucket('1 day', time) AS day,
       device_id,
       MAX(value) - MIN(value) AS daily_energy_kwh
FROM telemetry
WHERE metric = 'energy_kwh'
  AND time > NOW() - INTERVAL '30 days'
GROUP BY day, device_id
ORDER BY day DESC;

-- Potencia promedio por hora (usando continuous aggregate)
SELECT bucket AS hour,
       device_id,
       avg_value AS avg_power_w,
       max_value AS peak_power_w
FROM telemetry_hourly
WHERE metric = 'power_w'
  AND bucket > NOW() - INTERVAL '7 days'
ORDER BY bucket DESC, device_id;

-- Alertas: voltaje fuera de rango (207-242V, RETIE Colombia)
SELECT time, device_id, value AS voltage_v
FROM telemetry
WHERE metric = 'voltage_v'
  AND time > NOW() - INTERVAL '1 hour'
  AND (value < 207.0 OR value > 242.0)
ORDER BY time DESC;

-- Dispositivos con mayor consumo (últimas 24h)
SELECT device_id,
       MAX(value) - MIN(value) AS energy_consumed_kwh
FROM telemetry
WHERE metric = 'energy_kwh'
  AND time > NOW() - INTERVAL '24 hours'
GROUP BY device_id
ORDER BY energy_consumed_kwh DESC
LIMIT 10;
\end{verbatim}

\subsection{Mantenimiento}

\begin{verbatim}
-- Ver tamaño de hypertables y chunks
SELECT hypertable_name, 
       pg_size_pretty(hypertable_size(format('%I.%I', hypertable_schema, hypertable_name))) AS size
FROM timescaledb_information.hypertables
ORDER BY hypertable_size(format('%I.%I', hypertable_schema, hypertable_name)) DESC;

-- Ver chunks comprimidos
SELECT chunk_schema, chunk_name, 
       pg_size_pretty(before_compression_total_bytes) AS before,
       pg_size_pretty(after_compression_total_bytes) AS after,
       round((1 - after_compression_total_bytes::numeric / before_compression_total_bytes::numeric) * 100, 2) AS compression_ratio
FROM timescaledb_information.compressed_chunk_stats
ORDER BY before_compression_total_bytes DESC;

-- Forzar compresión manual de chunks antiguos
SELECT compress_chunk(i) 
FROM show_chunks('telemetry', older_than => INTERVAL '7 days') i;

-- Actualizar estadísticas para optimizador de queries
ANALYZE telemetry;
ANALYZE telemetry_15min;
ANALYZE telemetry_hourly;

-- Vacuuming manual (liberar espacio)
VACUUM ANALYZE telemetry;
\end{verbatim}

\section{Generación de Certificados X.509 para mTLS}

\subsection{Autoridad Certificadora (CA)}

\begin{verbatim}
#!/bin/bash
# Crear CA para IEEE 2030.5 mTLS

# CA privada
openssl ecparam -name prime256v1 -genkey -noout -out ca.key
chmod 600 ca.key

# Certificado CA (válido 10 años)
openssl req -new -x509 -sha256 -key ca.key -out ca.crt -days 3650 \
  -subj "/C=CO/ST=Antioquia/L=Medellin/O=SmartGrid CA/CN=SmartGrid Root CA"

# Verificar CA
openssl x509 -in ca.crt -text -noout
\end{verbatim}

\subsection{Certificado Servidor IEEE 2030.5}

\begin{verbatim}
# Key privada servidor
openssl ecparam -name prime256v1 -genkey -noout -out server.key

# CSR (Certificate Signing Request)
openssl req -new -sha256 -key server.key -out server.csr \
  -subj "/C=CO/ST=Antioquia/L=Medellin/O=SmartGrid/CN=gateway.local"

# Extensiones SAN (Subject Alternative Name)
cat > server_ext.cnf <<EOF
subjectAltName = DNS:gateway.local,DNS:*.gateway.local,IP:192.168.1.1
extendedKeyUsage = serverAuth
EOF

# Firmar con CA (válido 2 años)
openssl x509 -req -sha256 -in server.csr -CA ca.crt -CAkey ca.key \
  -CAcreateserial -out server.crt -days 730 -extfile server_ext.cnf

# Verificar cadena
openssl verify -CAfile ca.crt server.crt
\end{verbatim}

\subsection{Certificado Cliente SEP 2.0}

\begin{verbatim}
# Key privada cliente
openssl ecparam -name prime256v1 -genkey -noout -out client.key

# CSR cliente
openssl req -new -sha256 -key client.key -out client.csr \
  -subj "/C=CO/ST=Antioquia/L=Medellin/O=SmartGrid/CN=meter001"

# Extensiones cliente
cat > client_ext.cnf <<EOF
extendedKeyUsage = clientAuth
EOF

# Firmar con CA
openssl x509 -req -sha256 -in client.csr -CA ca.crt -CAkey ca.key \
  -CAcreateserial -out client.crt -days 730 -extfile client_ext.cnf

# LFDI (Long Form Device Identifier) = SHA256 del certificado
openssl x509 -in client.crt -outform DER | openssl dgst -sha256 -binary | xxd -p -c 32
\end{verbatim}

\subsection{Prueba mTLS}

\begin{verbatim}
# Curl con autenticación mutua
curl -v --cacert ca.crt --cert client.crt --key client.key \
  https://gateway.local:8883/dcap

# OpenSSL s_client test
openssl s_client -connect gateway.local:8883 \
  -CAfile ca.crt -cert client.crt -key client.key \
  -showcerts
\end{verbatim}
