\section{Análisis de Costos}

\subsection{Costos de Hardware}

\begin{table}[h]
\centering
\begin{tabular}{|l|r|r|r|}
\hline
\textbf{Componente} & \textbf{Cantidad} & \textbf{Precio Unit.} & \textbf{Total} \\
\hline
Nodo (ESP32C6 + RS485) & 300 & \$15 & \$4,500 \\
DCU (ESP32C6 + HaLow) & 3 & \$80 & \$240 \\
Gateway (ESP32C6 + HaLow) & 1 & \$100 & \$100 \\
ThingsBoard (cloud) & 1 & \$50/mes & \$600/año \\
\hline
\textbf{Total} & & & \textbf{\$5,440 + \$600/año} \\
\hline
\end{tabular}
\caption{Costos de implementación de la arquitectura propuesta para escenario piloto real de 300 medidores Itron SL7000 (barrio residencial). Período: Q4 2024. Distribución CAPEX: 60\% hardware (gateways Raspberry Pi 4, radios HaLow, ESP32C6), 30\% infraestructura (instalación, cableado), 10\% desarrollo SW. OPEX anual estimado: 15\% del CAPEX (conectividad LTE backup, mantenimiento).}
\label{tab:implementation-costs}
\end{table}

\subsection{Comparación con Alternativas}

\begin{table}[H]
\centering
\caption{Comparación de arquitecturas de edge gateway para Smart Energy AMI: propuesta (Raspberry Pi 4 + OpenWRT + ThingsBoard Edge) vs alternativas comerciales (Cisco IoT Gateway, Dell Edge Gateway, HPE Edgeline). Criterios evaluados: costo por unidad (USD), capacidad de procesamiento (GFLOPS), memoria (GB), flexibilidad de protocolos (número de radios soportadas), vendor lock-in (escala 1-5), y madurez (años en producción).}
\label{tab:edge-comparison}
\resizebox{\textwidth}{!}{%
\begin{tabular}{|>{\centering\arraybackslash}p{2.8cm}|>{\centering\arraybackslash}p{2.5cm}|>{\centering\arraybackslash}p{2.5cm}|>{\centering\arraybackslash}p{2.5cm}|>{\centering\arraybackslash}p{3cm}|}
\hline
\rowcolor{blue!20}
\textbf{Característica} & \textbf{Propuesta Tesis} & \textbf{Celular NB-IoT} & \textbf{PLC G3-PLC/PRIME} & \textbf{LoRaWAN} \\
\hline
\textbf{Costo inicial (300 medidores)} & \textcolor{green}{\textbf{\$5,440}} & \$15,000 & \$12,000-15,000 & \$8,000 \\
\hline
\textbf{Costo operativo anual} & \textcolor{green}{\textbf{\$600}} (\$2/med.) & \textcolor{red}{\$36,000} (\$120/med.) & \$3,600 (\$12/med.) & \$1,800 (\$6/med.) \\
\hline
\textbf{Alcance típico} & \textcolor{blue}{\textbf{1-3 km}} HaLow & \textbf{5-15 km} & 150-500m (PLC) & \textcolor{green}{\textbf{5-15 km}} \\
\hline
\textbf{Latencia E2E} & \textcolor{green}{\textbf{248 ms}} & 10-30 s & 5-15 s & \textcolor{orange}{30-300 s} (Clase A) \\
\hline
\textbf{Throughput por nodo} & \textcolor{blue}{\textbf{150-900 kbps}} & 60-250 kbps & 50-128 kbps & \textcolor{orange}{0.3-50 kbps} \\
\hline
\textbf{Seguridad} & \textcolor{green}{\textbf{E2E TLS + WPA3}} & 3GPP security & AES-128 & AES-128 LoRaWAN \\
\hline
\textbf{Escalabilidad} & \textcolor{blue}{\textbf{8K devices/AP}} & Unlimited & 500-2000/subnet & 10K/gateway \\
\hline
\textbf{Resiliencia offline} & \textcolor{green}{\textbf{7 días buffer}} & No buffer & No buffer & \textcolor{orange}{Limited buffer} \\
\hline
\textbf{Edge computing} & \textcolor{green}{\textbf{Sí (Ollama LLM)}} & \textcolor{red}{No disponible} & \textcolor{red}{No} & \textcolor{red}{No} \\
\hline
\textbf{Dependencias infraestructura} & \textcolor{green}{\textbf{Mínimas}} & \textcolor{orange}{Torres celulares} & \textcolor{red}{Grid eléctrico} & \textcolor{orange}{Gateways LoRaWAN} \\
\hline
\textbf{Flexibilidad protocolo} & \textcolor{green}{\textbf{Multi-protocolo}} & \textcolor{orange}{UDP/TCP} & \textcolor{red}{PLC específico} & \textcolor{orange}{LoRaWAN only} \\
\hline
\rowcolor{yellow!20}
\textbf{Ventaja principal} & \textbf{Costo-eficiencia} + Edge AI & \textbf{Cobertura global} & \textbf{Sin RF} & \textbf{Largo alcance} \\
\hline
\rowcolor{red!20}
\textbf{Limitación principal} & Cobertura local & \textcolor{red}{\textbf{Costo operativo}} & Dependencia grid & \textcolor{red}{\textbf{Latencia alta}} \\
\hline
\end{tabular}%
}
\end{table}

La solución propuesta resulta significativamente más económica que alternativas: Celular NB-IoT requiere \$10/mes/dispositivo (\$36,000/año, inviable), PLC (G3-PLC/PRIME) tiene mayor costo de nodos (\$30-40) sin ventajas claras, y LoRaWAN presenta mayor latencia (clase A) y menor throughput aunque alcance similar.

\textbf{Análisis detallado NB-IoT:} NB-IoT (Narrowband IoT, 3GPP Release 13) representa la alternativa de conectividad directa celular más desplegada globalmente para AMI, con casos de uso documentados en Vodafone (Europa), AT\&T (USA), y Telefónica (Latinoamérica)~\cite{NBIoT-AMI-TCO-2024}. Ventajas: (1) \textbf{Sin infraestructura propia}: elimina CAPEX de gateways/DCUs, simplifica deployment, (2) \textbf{Cobertura celular existente}: penetración urbana 95\%+, (3) \textbf{Escalabilidad carrier-grade}: millones de dispositivos soportados por red existente. Limitaciones: (1) \textbf{OPEX dominante en TCO}: plan datos \$10/mes/medidor × 12 meses = \$120/año representa 67\% del TCO 10 años (\$182/medidor según~\cite{NBIoT-AMI-TCO-2024}), vs arquitectura propuesta 7.5\% OPEX/TCO, (2) \textbf{Latencia 10-30s}: inadecuada para aplicaciones near-realtime (DER control, Demand Response), (3) \textbf{Cobertura rural limitada}: penetración <70\% en zonas rurales Colombia (fuente: Claro coverage map 2024), donde HaLow opera independiente de infraestructura carrier. \textbf{Conclusión}: NB-IoT óptimo para deployments dispersos (<10 medidores/km$^2$) donde amortización de gateway es prohibitiva, pero inviable para densidades urbanas >50 med/km$^2$ donde arquitectura propuesta reduce TCO 10 años en 70\% (\$54 vs \$182/medidor).

\subsection{Análisis de Sensibilidad Económica}

Para evaluar la robustez de la propuesta ante variaciones en costos de mercado, se realiza un análisis de sensibilidad considerando tres escenarios: optimista (-20\% CAPEX, -30\% OPEX), base (valores nominales), y pesimista (+20\% CAPEX, +30\% OPEX). Los drivers de variación incluyen fluctuaciones de precio de componentes (ESP32C6, módulos HaLow), costos de planes de datos LTE, y economías de escala en compras volumétricas.

\begin{table}[H]
\centering
\caption{Análisis de sensibilidad TCO 10 años para despliegue de 300 medidores. Escenarios: Optimista (componentes -20\%, planes datos -30\%), Base (valores nominales sección 4.12), Pesimista (componentes +20\%, planes datos +30\%). Asunciones: amortización lineal 10 años, tasa descuento 8\%, inflación 3\% anual. Comparación vs alternativa cloud comercial ThingsBoard Professional Edition (\$1,161/medidor baseline Tabla 5.3).}
\label{tab:tco-sensitivity-analysis}
\resizebox{\textwidth}{!}{%
\begin{tabular}{|l|r|r|r|r|r|}
\hline
\rowcolor{gray!20}
\textbf{Concepto} & \textbf{Optimista} & \textbf{Base} & \textbf{Pesimista} & \textbf{Cloud} & \textbf{Variación} \\
 & \textbf{(-20\%/-30\%)} & \textbf{(Nominal)} & \textbf{(+20\%/+30\%)} & \textbf{Comercial} & \textbf{(\%)} \\
\hline
\multicolumn{6}{|c|}{\cellcolor{blue!10}\textbf{CAPEX Inicial (Año 0)}} \\
\hline
Nodos (300× ESP32C6) & \$3,600 & \$4,500 & \$5,400 & \$15,000 & -76\% / -64\% \\
DCUs (3× HaLow AP) & \$192 & \$240 & \$288 & \$3,000 & -94\% / -90\% \\
Gateway (Raspberry Pi 4) & \$80 & \$100 & \$120 & \$500 & -84\% / -76\% \\
Instalación (15\% hardware) & \$581 & \$726 & \$871 & \$2,775 & -79\% / -69\% \\
Desarrollo SW inicial & \$2,000 & \$2,500 & \$3,000 & \$10,000 & -80\% / -70\% \\
\hline
\textbf{Subtotal CAPEX} & \textcolor{green}{\textbf{\$6,453}} & \textbf{\$8,066} & \textcolor{orange}{\textbf{\$9,679}} & \textcolor{red}{\textbf{\$31,275}} & \textbf{-79\% / -69\%} \\
\hline
\multicolumn{6}{|c|}{\cellcolor{blue!10}\textbf{OPEX Anual (Años 1-10)}} \\
\hline
Conectividad LTE (1GB/mes) & \$294 & \$420 & \$546 & \$36,000 & -99\% / -98\% \\
Mantenimiento HW (5\% CAPEX) & \$323 & \$403 & \$484 & \$1,564 & -79\% / -69\% \\
Actualizaciones SW & \$140 & \$200 & \$260 & \$5,000 & -97\% / -95\% \\
Energía (0.5W × 303 × \$0.15/kWh) & \$200 & \$200 & \$200 & \$400 & -50\% / -50\% \\
\hline
\textbf{Subtotal OPEX/año} & \textcolor{green}{\textbf{\$957}} & \textbf{\$1,223} & \textcolor{orange}{\textbf{\$1,490}} & \textcolor{red}{\textbf{\$42,964}} & \textbf{-98\% / -97\%} \\
\hline
\multicolumn{6}{|c|}{\cellcolor{yellow!20}\textbf{TCO 10 Años (NPV @ 8\%)}} \\
\hline
Valor presente OPEX & \$6,420 & \$8,206 & \$10,001 & \$288,325 & -98\% / -97\% \\
\textbf{TCO Total 10 años} & \textcolor{green}{\textbf{\$12,873}} & \textbf{\$16,272} & \textcolor{orange}{\textbf{\$19,680}} & \textcolor{red}{\textbf{\$319,600}} & \textbf{-96\% / -94\%} \\
\hline
\textbf{Costo por medidor} & \textcolor{green}{\textbf{\$42.91}} & \textbf{\$54.24} & \textcolor{orange}{\textbf{\$65.60}} & \textcolor{red}{\textbf{\$1,065.33}} & \textbf{-96\% / -94\%} \\
\hline
\multicolumn{6}{|c|}{\cellcolor{green!10}\textbf{Breakeven vs Cloud Comercial}} \\
\hline
Ahorro 10 años & \textcolor{green}{\$306,727} & \$303,328 & \textcolor{green}{\$299,920} & --- & +1\% / -1\% \\
Meses para ROI & \textcolor{green}{\textbf{2.0 meses}} & \textbf{2.3 meses} & \textcolor{orange}{\textbf{2.7 meses}} & --- & -13\% / +17\% \\
\hline
\end{tabular}%
}
\end{table}

\textbf{Análisis de resultados:}

\begin{itemize}
    \item \textbf{Robustez del modelo:} Incluso en escenario pesimista (+20\%/+30\%), TCO propuesto (\$65.60/medidor) es 94\% menor que cloud comercial (\$1,065.33/medidor). Esto demuestra que la propuesta mantiene ventaja económica significativa ante fluctuaciones de mercado.
    
    \item \textbf{Sensibilidad CAPEX vs OPEX:} Variación de ±20\% en CAPEX impacta solo ±\$1,613 en TCO total (10\% variación), mientras que ±30\% en OPEX impacta ±\$1,781 (11\% variación). Sensibilidad similar indica que ambos componentes tienen peso comparable en TCO 10 años, pero OPEX domina en horizontes más largos.
    
    \item \textbf{ROI resiliente:} Punto de equilibrio se alcanza entre 2.0-2.7 meses incluso en escenario pesimista, vs 3-5 años típicos en proyectos IoT enterprise. Esto es posible por OPEX conectividad extremadamente bajo (\$1.40/medidor/mes LTE backup) comparado con NB-IoT (\$10/medidor/mes).
    
    \item \textbf{Drivers de variación:}
    \begin{itemize}
        \item \textbf{CAPEX:} Precio ESP32C6 varía \$12-18 según volumen (Mouser 2024: \$2.50 unit, \$1.80 × 1000). Módulos HaLow \$50-80 según proveedor (Morse Micro ME1000: \$60, NewRadio NR-7000: \$75). Variación ±20\% es conservadora para órdenes >1K unidades.
        \item \textbf{OPEX:} Planes LTE 1GB/mes varían \$5-15/mes según país y contrato multi-año (Colombia 2024: Claro IoT \$7/mes, Movistar M2M \$12/mes). Variación ±30\% cubre incertidumbre tarifaria 10 años.
    \end{itemize}
    
    \item \textbf{Economías de escala:} Para despliegue 10K medidores, CAPEX unitario cae a \$45/medidor (-37\% vs 300 medidores) por precios volumétricos y amortización de desarrollo SW fijo. TCO 10 años baja a \$38/medidor, ahorro \$1.03M vs cloud.
    
    \item \textbf{Punto crítico conectividad:} Si costo LTE excediera \$25/mes/medidor (817\% aumento), TCO igualaría cloud comercial. Esto es improbable dado que planes M2M actuales rondan \$7-12/mes y tendencia es a la baja con LTE-M/NB-IoT masivo.
\end{itemize}

\textbf{Conclusión:} El análisis de sensibilidad valida la robustez económica de la propuesta. Margen de seguridad >900\% en OPEX conectividad y >400\% en CAPEX hardware garantizan viabilidad financiera incluso ante shocks de mercado. Recomendación: contratos multi-año con operadores M2M pueden fijar OPEX y mitigar riesgo inflación.
