\chapter{Introducción}

\section{Contexto y Motivación}

\subsection{El Desafío de las Redes Smart Energy}

La transición energética global hacia sistemas descentralizados, con alta penetración de energías renovables distribuidas (DER) y gestión activa de la demanda (DSM), exige infraestructuras de medición inteligente (AMI - Advanced Metering Infrastructure) capaces de recolectar, transmitir y procesar datos de millones de puntos de consumo en tiempo cuasi-real. Según la Agencia Internacional de Energía (IEA), se proyecta la instalación de más de 1.300 millones de medidores inteligentes a nivel global para 2030, generando aproximadamente 15 petabytes de datos de telemetría diarios.

Las arquitecturas tradicionales basadas en comunicación directa dispositivo-nube enfrentan limitaciones críticas: latencias elevadas (>200 ms), dependencia estricta de conectividad WAN continua, costos operacionales prohibitivos en escenarios de alta densidad de dispositivos, y dificultades para garantizar requisitos de tiempo real exigidos por aplicaciones de respuesta a la demanda (DR) y gestión de microrredes.

\subsection{Estado Actual de las Tecnologías de Comunicación IoT}

El ecosistema IoT para aplicaciones industriales y de infraestructura crítica se caracteriza por una heterogeneidad de tecnologías de comunicación, cada una optimizada para rangos específicos de alcance, throughput, latencia y consumo energético.

\subsubsection{Comparativa Técnica de Protocolos Mesh 2.4 GHz}

\begin{table}[h]
\centering
\caption{Comparación Thread, Zigbee y Bluetooth Mesh}
\label{tab:mesh-comparison}
\begin{tabular}{|p{3.5cm}|p{3.5cm}|p{3.5cm}|p{3.5cm}|}
\hline
\textbf{Característica} & \textbf{Thread 1.3.1} & \textbf{Zigbee 3.0} & \textbf{Bluetooth Mesh} \\
\hline
\textbf{Capa física} & IEEE 802.15.4 & IEEE 802.15.4 & Bluetooth 5.3 LE \\
\hline
\textbf{Frecuencia} & 2.4 GHz & 2.4 GHz / Sub-GHz & 2.4 GHz \\
\hline
\textbf{Topología} & Mesh (MLE routing) & Mesh (AODV-based) & Managed Flooding \\
\hline
\textbf{IPv6 nativo} & Sí (6LoWPAN) & No (propietario) & No (GATT proxy) \\
\hline
\textbf{Número máx. nodos} & >250 & 65,535 (teórico) & 32,767 \\
\hline
\textbf{Latencia (3 hops)} & 40-60 ms & 80-120 ms & 100-200 ms \\
\hline
\textbf{Consumo RX/TX} & 19 mA / 22 mA & 24 mA / 31 mA & 9.2 mA / 10.5 mA \\
\hline
\textbf{Sleep current} & 5 µA (ESP32-C6) & 10 µA típico & 2 µA (nRF52840) \\
\hline
\textbf{Interoperabilidad} & OTBR estándar & Requiere coordinador & Requiere provisioner \\
\hline
\textbf{Security} & TLS/DTLS 1.2 & AES-128 CCM & AES-CCM \\
\hline
\end{tabular}
\end{table}

Thread emerge como protocolo preferencial para redes de campo en Smart Energy debido a su routing IPv6 nativo, que facilita integración con infraestructuras IP existentes, estandarización completa bajo Thread Group (miembro de Connectivity Standards Alliance), y soporte multi-vendor garantizado mediante certificación Thread 1.3.1.

\subsubsection{Plataformas de Edge Computing - Análisis Comparativo}

\begin{table}[h]
\centering
\caption{Comparación Plataformas Edge IoT}
\label{tab:edge-platforms}
\begin{tabular}{|p{3cm}|p{3cm}|p{3cm}|p{3cm}|p{3cm}|}
\hline
\textbf{Plataforma} & \textbf{ThingsBoard Edge} & \textbf{AWS IoT Greengrass} & \textbf{Azure IoT Edge} & \textbf{Node-RED} \\
\hline
\textbf{Arquitectura} & Monolítica Java & Microservices Python & Containerizada .NET & Flow-based JS \\
\hline
\textbf{Sincronización} & Bidireccional & Unidireccional & Bidireccional & Manual \\
\hline
\textbf{Rule Engine local} & Sí (full chain) & Lambda local & Módulos custom & Function nodes \\
\hline
\textbf{Almacenamiento} & PostgreSQL/Cassandra & DynamoDB local & SQLite/Custom & Context store \\
\hline
\textbf{Dashboard local} & Sí (full featured) & No (CloudWatch) & No (portal cloud) & UI integrado \\
\hline
\textbf{Autonomía offline} & Ilimitada & Limitada & Limitada & Ilimitada \\
\hline
\textbf{Footprint RAM} & 1-4 GB & 512 MB - 2 GB & 256 MB - 1 GB & 128-512 MB \\
\hline
\textbf{Licenciamiento} & Apache 2.0 & Propietario & Propietario & Apache 2.0 \\
\hline
\textbf{Curva aprendizaje} & Media & Alta & Alta & Baja \\
\hline
\end{tabular}
\end{table}

ThingsBoard Edge se posiciona como solución robusta para aplicaciones industriales que requieren continuidad operacional durante particiones WAN prolongadas, con capacidades completas de rule engine, dashboards interactivos y sincronización bidireccional de configuraciones y datos históricos.

\subsubsection{HaLow - Posicionamiento frente a Alternativas de Última Milla}

\begin{table}[h]
\centering
\caption{Comparación Tecnologías Última Milla Smart Energy}
\label{tab:lastmile-comparison}
\begin{tabular}{|p{3.5cm}|p{2.5cm}|p{2.5cm}|p{2.5cm}|p{2.5cm}|}
\hline
\textbf{Característica} & \textbf{HaLow 802.11ah} & \textbf{LoRaWAN} & \textbf{LTE Cat-M1} & \textbf{Wi-Fi 6} \\
\hline
\textbf{Frecuencia} & Sub-GHz (900 MHz) & Sub-GHz (868/915 MHz) & LTE Bands & 2.4/5 GHz \\
\hline
\textbf{Alcance típico} & 1-2 km & 5-15 km & 10-35 km & 50-100 m \\
\hline
\textbf{Throughput máx.} & 40 Mbps (4 MHz BW) & 50 kbps & 1 Mbps & 9.6 Gbps \\
\hline
\textbf{Latencia típica} & 10-30 ms & 1-5 seg & 50-100 ms & <10 ms \\
\hline
\textbf{Topología} & Star/Mesh & Star (sin mesh) & Star (celular) & Star \\
\hline
\textbf{Consumo TX (avg)} & 180 mA @ 1 MHz BW & 120 mA & 220 mA & 350 mA \\
\hline
\textbf{Cobertura indoor} & Excelente (penetración) & Media & Buena & Limitada \\
\hline
\textbf{Espectro} & No licenciado ISM & No licenciado ISM & Licenciado (operador) & No licenciado ISM \\
\hline
\textbf{Despliegue} & Privado (CAPEX) & Gateway privado & Suscripción MVNO & Privado (CAPEX) \\
\hline
\textbf{Costo por nodo} & \$25-40 módulo & \$8-15 módulo & \$12-25 módulo & \$5-10 módulo \\
\hline
\end{tabular}
\end{table}

Wi-Fi HaLow combina throughput superior a LoRaWAN (40 Mbps vs 50 kbps), latencia determinística (<30 ms vs 1-5 seg), y ausencia de costos recurrentes de conectividad frente a LTE Cat-M1, posicionándose como tecnología óptima para backhaul de gateways Smart Energy en zonas urbanas y suburbanas densas.

\subsection{Brechas en Arquitecturas IoT Existentes}

El análisis del estado del arte revela limitaciones estructurales en arquitecturas IoT contemporáneas:

\begin{itemize}
\item \textbf{Dependencia cloud-centric}: Las arquitecturas tradicionales dispositivo → cloud presentan Single Points of Failure (SPOF) en enlaces WAN. Estudios empíricos en despliegues urbanos reportan disponibilidades de 94-96\% en conectividad celular LTE (downtimes acumulados 18-25 días/año), insuficientes para aplicaciones críticas.

\item \textbf{Overhead de traducción multi-protocolo}: Los gateways convencionales implementan traductores application-layer (ej. Thread → MQTT → HTTP → Cloud), introduciendo latencias acumuladas de 150-300 ms y complejidad en mantenimiento de mapeos de datos.

\item \textbf{Escalabilidad limitada del cloud ingestion}: Plataformas cloud IoT típicamente cobran por mensaje ingestado (\$5-10 por millón de mensajes), resultando en costos prohibitivos para aplicaciones de telemetría de alta frecuencia (ej. 10,000 medidores reportando cada 5 minutos generan \$2,880/mes solo en ingesta).

\item \textbf{Ausencia de estándares de interoperabilidad}: La mayoría de soluciones comerciales implementan APIs propietarias, dificultando la migración entre vendors y bloqueando clientes en ecosistemas cerrados.
\end{itemize}

\subsubsection{Análisis Cuantitativo de Overhead en Arquitecturas Tradicionales}

\begin{table}[h]
\centering
\caption{Latencia por Arquitectura (device → cloud storage)}
\label{tab:latency-overhead}
\begin{tabular}{|p{4.5cm}|p{3cm}|p{3cm}|p{3cm}|}
\hline
\textbf{Componente} & \textbf{Cloud-Centric} & \textbf{Edge-Lite (Node-RED)} & \textbf{Propuesta (Edge Full)} \\
\hline
\textbf{Device → Gateway} & 40 ms (Thread) & 40 ms (Thread) & 40 ms (Thread) \\
\hline
\textbf{Gateway → WAN} & 80 ms (LTE) & 15 ms (Ethernet) & 15 ms (HaLow/Eth) \\
\hline
\textbf{WAN → Cloud} & 50 ms (RTT) & 50 ms (RTT) & N/A (local) \\
\hline
\textbf{Cloud processing} & 30 ms (ingestion) & 30 ms (ingestion) & N/A \\
\hline
\textbf{Cloud → DB write} & 10 ms (RDS write) & 10 ms (RDS write) & 8 ms (TimescaleDB) \\
\hline
\textbf{\textbf{TOTAL P50}} & \textbf{210 ms} & \textbf{145 ms} & \textbf{63 ms} \\
\hline
\textbf{\textbf{TOTAL P99}} & \textbf{450 ms} & \textbf{310 ms} & \textbf{95 ms} \\
\hline
\end{tabular}
\end{table}

La arquitectura propuesta reduce latencia end-to-end en 70\% (P50) y 79\% (P99) respecto a arquitecturas cloud-centric, eliminando el round-trip WAN mediante procesamiento local completo.

\section{Planteamiento del Problema}

\subsection{Definición del Problema de Investigación}

¿Cómo diseñar e implementar una arquitectura IoT de borde multi-protocolo para aplicaciones Smart Energy que optimice simultáneamente: (1) la eficiencia de comunicación mediante el stack 6LoWPAN/CoAP/LwM2M sobre IEEE 802.15.4 reduciendo latencia y overhead de paquetes; (2) las capacidades de procesamiento en tiempo real mediante edge gateways con IA integrada para gestión inteligente de recursos; y (3) la conectividad de última milla mediante arquitectura basada en IEEE 802.11ah (HaLow) con selección adaptativa de bandwidth (2 MHz para conexiones estables, 4 MHz para gestión balanceada, 8 MHz para alto tráfico con línea de vista), garantizando latencia end-to-end <100 ms, reducción de tráfico WAN >60\%, y disponibilidad >99\% durante desconexiones prolongadas?

El problema aborda tres dimensiones técnicas fundamentales:

\textbf{Dimensión 1 - Optimización del stack de protocolos}: Implementación de 6LoWPAN con compresión de headers IPv6 (de 40 bytes a 2 bytes), CoAP como protocolo de aplicación ligero (overhead 4 bytes vs 100+ bytes HTTP), y LwM2M para gestión eficiente de dispositivos, logrando reducción de latencia >40\% y overhead de paquetes >70\% respecto a stacks tradicionales MQTT/HTTP.

\textbf{Dimensión 2 - Edge Computing con IA}: Despliegue de gateways edge con capacidades de procesamiento local mediante servicios containerizados (ThingsBoard Edge, TimescaleDB, Kafka), integración de modelos de lenguaje (LLM) locales para análisis en tiempo real de telemetría, detección de anomalías sin dependencia cloud, y gestión inteligente de recursos con adaptación dinámica a condiciones de red.

\textbf{Dimensión 3 - Arquitectura multi-banda IEEE 802.11ah}: Diseño de arquitectura de gateways basada en nodos HaLow con selección estratégica de bandwidth según caso de uso: 2 MHz (sensibilidad -99 dBm, alcance máximo >2 km, conexiones estables en NLOS, ideal para sensores remotos), 4 MHz (balance latencia/alcance, 40 Mbps agregado, gestión de red con throughput moderado), y 8 MHz (throughput >80 Mbps, latencia <20 ms, escenarios de alto tráfico con línea de vista como backhaul de concentradores).

\subsection{Delimitación del Problema}

La investigación se delimita a tres pilares técnicos fundamentales:

\textbf{Pilar 1 - Stack de Protocolos 6LoWPAN/CoAP/LwM2M}:
\begin{itemize}
\item \textbf{Capa de adaptación}: 6LoWPAN (RFC 6282) sobre IEEE 802.15.4-2020 con compresión de headers IPHC/NHC.
\item \textbf{Capa de aplicación}: CoAP (RFC 7252) con modos Confirmable (CON) y Non-Confirmable (NON), block-wise transfer (RFC 7959), y Observe (RFC 7641).
\item \textbf{Gestión de dispositivos}: LwM2M 1.2 (OMA SpecWorks) con objetos estándar (Security 0, Server 1, Device 3, Connectivity Monitoring 4, Firmware Update 5).
\item \textbf{Métricas de evaluación}: Overhead de headers (bytes), latencia por salto (ms), eficiencia espectral (bits/Hz), tasa de entrega de paquetes (PDR), consumo energético por transmisión (mJ/bit).
\end{itemize}

\textbf{Pilar 2 - Edge Gateway con IA}:
\begin{itemize}
\item \textbf{Plataforma hardware}: ARMv8 Cortex-A53 quad-core @ 1.8 GHz (Raspberry Pi 4 o Banana Pi BPI-R4), 4-8 GB RAM, NVMe SSD 256 GB, periféricos M.2 para LTE.
\item \textbf{Sistema operativo}: OpenWRT 23.05.x con kernel Linux 5.15 LTS + parches PREEMPT\_RT para latencias determinísticas.
\item \textbf{Stack de servicios}: ThingsBoard Edge 3.6 (procesamiento CEP local), PostgreSQL 15 + TimescaleDB 2.13 (series temporales), Apache Kafka 7.5 (message broker), Ollama + Llama 3.2 3B (inferencia LLM local).
\item \textbf{Capacidades IA}: Detección de anomalías en consumo energético, mantenimiento predictivo basado en patrones de alarmas, optimización de rutas mesh mediante reinforcement learning, compresión adaptativa de datos según ancho de banda disponible.
\item \textbf{Métricas de evaluación}: Latencia de inferencia IA (<500 ms objetivo), precisión de detección de anomalías (>95\%), overhead de CPU/RAM bajo carga, tiempo de failover multi-WAN (<30s).
\end{itemize}

\textbf{Pilar 3 - Arquitectura Multi-Banda IEEE 802.11ah}:
\begin{itemize}
\item \textbf{Bandwidths evaluados}: 1 MHz (150 kbps, sensibilidad -99 dBm, >2 km NLOS), 2 MHz (300-450 kbps, sensibilidad -96 dBm, 1.5-2 km, óptimo para conexiones estables), 4 MHz (600-900 kbps, sensibilidad -91 dBm, 1-1.5 km, balance throughput/alcance), 8 MHz (1.2-1.8 Mbps, sensibilidad -85 dBm, 0.5-1 km LOS, alto tráfico backhaul).
\item \textbf{Topologías}: Star (gateway central + nodos finales), Mesh 802.11s (routing HWMP, auto-healing), EasyMesh (IEEE 1905.1, roaming coordinado).
\item \textbf{Casos de uso específicos}: 2 MHz para sensores remotos baja frecuencia (lecturas horarias, zonas rurales, penetración indoor), 4 MHz para gestión de red balanceada (lecturas 15 min, zonas suburbanas, densidad media), 8 MHz para backhaul de concentradores (agregación de datos, zonas urbanas LOS, latencia crítica <50 ms).
\item \textbf{Métricas de evaluación}: Link budget por bandwidth, throughput agregado vs número de clientes, latencia P50/P95/P99, tasa de handover en movilidad, cobertura efectiva por MCS.
\end{itemize}

\textbf{Dominio de aplicación}: Redes de telemetría Smart Energy en entornos urbanos/suburbanos con densidad 1,000-10,000 puntos de medición por km², con validación en caso de estudio real de 900 medidores residenciales.

\textbf{Estándares implementados}: IEEE 802.15.4-2020, IEEE 802.11ah-2016, IEEE 2030.5-2023, ISO/IEC 30141:2024, OMA LwM2M 1.2, RFC 6282 (6LoWPAN), RFC 7252 (CoAP).

Se excluyen del alcance: implementación de IEC 61850 (subestaciones), PLC (G3-PLC/PRIME), certificación formal Thread/Matter, redes 5G/NR-Light, y blockchain para auditoría (trabajo futuro).

\subsection{Justificación}

\subsubsection{Justificación Técnica}

Las arquitecturas edge-computing para IoT industrial requieren capacidades de procesamiento local, almacenamiento persistente y autonomía operacional que las soluciones cloud-centric tradicionales no pueden garantizar. La integración de Wi-Fi HaLow como tecnología de backhaul representa una innovación técnica respecto al estado del arte (dominado por LTE/LoRaWAN), aprovechando sus ventajas de throughput (40 Mbps vs 1 Mbps LTE Cat-M1), latencia (<30 ms vs >50 ms), y ausencia de costos recurrentes de conectividad.

\subsubsection{Justificación Económica}

Análisis de TCO (Total Cost of Ownership) para despliegue de 1,000 puntos de medición durante 5 años:

\begin{itemize}
\item \textbf{Cloud-centric + LTE:} CAPEX \$150k (hardware) + OPEX \$180k (conectividad \$15/nodo/año) = \$330k
\item \textbf{Propuesta HaLow:} CAPEX \$200k (hardware + APs HaLow) + OPEX \$25k (mantenimiento) = \$225k
\item \textbf{Ahorro proyectado:} 32\% (\$105k en 5 años)
\end{itemize}

\subsubsection{Justificación Académica}

La investigación contribuye al cuerpo de conocimiento en arquitecturas IoT heterogéneas mediante:
\begin{itemize}
\item Diseño de arquitectura de referencia para gateways multi-PHY conformes con ISO/IEC 30141.
\item Caracterización empírica de latencias en integración Thread ↔ HaLow.
\item Metodología de implementación de IEEE 2030.5 Function Sets sobre plataformas embebidas Linux.
\item Evaluación comparativa de estrategias de failover multi-WAN en gateways IoT.
\end{itemize}

\subsection{Metodología de Investigación}

La investigación sigue un enfoque mixto que combina Design Science Research (DSR) para el diseño de artefactos tecnológicos, Investigación Experimental para la validación de hipótesis cuantitativas, y Estudio de Caso para la evaluación en contexto real.

\subsubsection{Fase 1 - Análisis y Diseño (Design Science)}

\textbf{Objetivos:} Especificar requisitos funcionales/no funcionales, diseñar arquitectura de referencia multi-capa, definir interfaces entre componentes.

\textbf{Actividades:}
\begin{enumerate}
\item Revisión sistemática de literatura sobre arquitecturas IoT edge y estándares Smart Energy (IEEE 2030.5, ISO/IEC 30141, IEC 61850).
\item Análisis comparativo de tecnologías de comunicación (Thread, Zigbee, BLE Mesh, HaLow, LoRaWAN, LTE Cat-M1).
\item Diseño de arquitectura de 4 capas: Conectividad, Orquestación, Procesamiento, Aplicación.
\item Especificación de interfaces: OTBR APIs, MQTT topics, IEEE 2030.5 REST endpoints.
\item Modelado de latencias mediante teoría de colas (M/M/1 para gateway, M/G/∞ para cloud).
\end{enumerate}

\textbf{Entregables:} Diagrama de arquitectura (Capítulo 3), especificación de requisitos (Capítulo 3.3), diseño de base de datos TimescaleDB (Anexo B).

\subsubsection{Fase 2 - Implementación (Engineering)}

\textbf{Objetivos:} Implementar gateway prototipo funcional, integrar componentes hardware/software, desarrollar servicios containerizados.

\textbf{Actividades:}
\begin{enumerate}
\item Configuración plataforma hardware: Banana Pi BPI-R4 (4x Cortex-A53 @ 1.8 GHz, 4 GB RAM) + nRF52840 RCP (Thread) + Morse Micro MM6108 (HaLow) + Quectel EG25-G (LTE).
\item Instalación y configuración OpenWRT 23.05.x con kernel real-time patches (PREEMPT\_RT).
\item Despliegue stack Docker Compose: ThingsBoard Edge 3.6.0, PostgreSQL 15 + TimescaleDB 2.13, Apache Kafka 7.5.0, IEEE 2030.5 Server (Python/Flask), Ollama LLM (Llama 3.2 3B).
\item Implementación IEEE 2030.5 Function Sets: DCAP, Time, EndDevice, MirrorUsagePoint, MirrorMeterReading, Messaging (XML schemas según estándar).
\item Configuración mwan3 para failover multi-WAN (Ethernet métrica 10, HaLow STA métrica 15, LTE métrica 20).
\item Desarrollo nodos IoT: ESP32-C6 Thread LwM2M + sensor BME280 (temperatura/humedad/presión).
\end{enumerate}

\textbf{Entregables:} Documentación de instalación (Anexo A), archivos docker-compose.yml (Anexo B), scripts de integración (Anexo C), código fuente nodos IoT (Anexo E).

\subsubsection{Fase 3 - Validación Experimental}

\textbf{Objetivos:} Validar hipótesis mediante mediciones empíricas, caracterizar rendimiento del sistema, evaluar resiliencia ante fallos.

\textbf{Experimentos:}

\begin{enumerate}
\item \textbf{Exp. 1 - Latencia end-to-end:} Medir latencia desde generación de telemetría en nodo IoT hasta persistencia en TimescaleDB. Variables independientes: número de nodos (N=5,10,25), frecuencia de muestreo (5s, 30s, 60s). Variables dependientes: latencia P50/P95/P99, jitter. Duración: 72 horas por configuración.

\item \textbf{Exp. 2 - Disponibilidad durante desconexión WAN:} Simular partición WAN de 48 horas desconectando Ethernet y deshabilitando LTE. Métricas: porcentaje de mensajes bufferizados exitosamente, tiempo de sincronización post-reconexión, disponibilidad de servicios locales (dashboards, alarmas).

\item \textbf{Exp. 3 - Throughput agregado HaLow:} Saturar enlace HaLow con tráfico concurrente de múltiples nodos. Medir throughput agregado vs número de clientes (N=1,5,10,20). Configuraciones: 1 MHz/2 MHz bandwidth, MCS 0-10.

\item \textbf{Exp. 4 - Failover multi-WAN:} Provocar fallas en interfaces Ethernet → HaLow → LTE. Medir tiempo de detección de falla, tiempo de conmutación, pérdida de paquetes durante transición.

\item \textbf{Exp. 5 - Overhead de procesamiento:} Caracterizar CPU/RAM/storage bajo cargas de 10/50/100 dispositivos. Identificar cuellos de botella mediante profiling (perf, flamegraphs).
\end{enumerate}

\textbf{Herramientas de medición:} Wireshark/tshark para captura de paquetes, Grafana + Prometheus para métricas de sistema, scripts Python para análisis estadístico (pandas, scipy).

\textbf{Entregables:} Datasets de mediciones (repositorio GitHub), gráficas de resultados (Capítulo 4), análisis estadístico (ANOVA, t-tests).

\subsubsection{Fase 4 - Evaluación Comparativa}

\textbf{Objetivos:} Comparar arquitectura propuesta vs soluciones baseline (cloud-centric, edge-lite).

\textbf{Baseline 1 - Cloud-Centric:} Nodos Thread → OTBR → Gateway LTE → AWS IoT Core → Lambda → DynamoDB.

\textbf{Baseline 2 - Edge-Lite:} Nodos Thread → OTBR → Node-RED (local) → AWS IoT Core (sync).

\textbf{Criterios de comparación:}
\begin{itemize}
\item Latencia P50/P99 device → storage
\item Disponibilidad durante partición WAN 48h
\item Throughput máximo (mensajes/seg)
\item Consumo energético gateway (Watts)
\item Costos OPEX (USD/mes para 100 dispositivos)
\item Complejidad de deployment (horas-persona)
\end{itemize}

\textbf{Entregables:} Tabla comparativa (Capítulo 4), análisis de trade-offs, recomendaciones de uso.

\section{Hipótesis}

\subsection{Hipótesis General}

Una arquitectura IoT para Smart Energy basada en: (1) stack de protocolos optimizado 6LoWPAN/CoAP/LwM2M sobre IEEE 802.15.4, (2) edge gateways con capacidades de procesamiento local e IA integrada, y (3) conectividad de última milla mediante IEEE 802.11ah con selección adaptativa de bandwidth (2/4/8 MHz), permite reducir la latencia end-to-end en >70\%, el overhead de paquetes en >60\%, el tráfico WAN en >65\%, garantizando disponibilidad >99\% durante desconexiones prolongadas y procesamiento inteligente en tiempo real, comparado con arquitecturas tradicionales basadas en MQTT/HTTP sobre conectividad celular.

\subsection{Hipótesis Específicas}

\textbf{H1 - Optimización mediante 6LoWPAN/CoAP/LwM2M:} La implementación del stack 6LoWPAN (compresión IPHC/NHC) + CoAP (overhead 4-10 bytes) + LwM2M (objetos binarios TLV) sobre IEEE 802.15.4 reduce el overhead de paquetes en >70\% y la latencia por salto en >40\% comparado con MQTT/JSON sobre TCP/IP, logrando tiempos de transmisión <15 ms por hop en topologías mesh de hasta 5 saltos.

\textbf{H2 - Procesamiento Edge con IA:} El despliegue de servicios containerizados edge (ThingsBoard Edge, TimescaleDB, Kafka) con integración de modelos LLM locales (Ollama + Llama 3.2 3B) permite: (a) reducción de tráfico WAN en >65\% mediante procesamiento local, (b) latencia de inferencia <500 ms para detección de anomalías, (c) disponibilidad de servicios >99\% durante desconexiones WAN >72 horas, y (d) precisión de detección de anomalías >95\% en patrones de consumo energético.

\textbf{H3 - Arquitectura Multi-Banda 802.11ah:} La arquitectura basada en gateways HaLow con selección estratégica de bandwidth según caso de uso maximiza eficiencia operacional:
\begin{itemize}
\item \textbf{2 MHz}: Óptimo para conexiones estables con sensores remotos (>2 km alcance, sensibilidad -96 dBm, tráfico <100 kbps, entornos NLOS con penetración indoor superior), logrando PDR >98\% en condiciones adversas con SNR 8-12 dB.
\item \textbf{4 MHz}: Balance ideal para gestión de red (1-1.5 km alcance, throughput 40 Mbps agregado, latencia <50 ms P95), soportando 50+ nodos con tráfico moderado (lecturas cada 15 min) sin degradación >10\%.
\item \textbf{8 MHz}: Maximiza throughput para alto tráfico con línea de vista (backhaul de concentradores, >80 Mbps, latencia <20 ms P99, alcance 0.5-1 km LOS), permitiendo agregación de datos de 100+ dispositivos por gateway.
\end{itemize}

\textbf{H4 - Compresión 6LoWPAN de Headers:} La compresión IPHC (IPv6 Header Compression) de 6LoWPAN reduce headers IPv6+UDP de 48 bytes a 2-7 bytes (compresión >85\%), y la compresión NHC (Next Header Compression) para CoAP reduce overhead adicional de 10-20 bytes a 2-4 bytes, resultando en payloads efectivos >90\% del MTU IEEE 802.15.4 (127 bytes) para aplicaciones Smart Energy.

\textbf{H5 - Eficiencia CoAP vs MQTT:} CoAP sobre UDP con modos Non-Confirmable (NON) para telemetría no crítica y Confirmable (CON) para comandos críticos, combinado con Observe para subscripciones, reduce latencia en >50\% y overhead de red en >60\% comparado con MQTT/TCP, logrando tiempos de respuesta <30 ms para transacciones GET/POST en redes Thread mesh.

\textbf{H6 - LwM2M para Gestión Eficiente:} LwM2M con objetos estándar OMA (Device, Connectivity Monitoring, Firmware Update) y transporte CoAP reduce tráfico de gestión de dispositivos en >75\% comparado con soluciones propietarias HTTP/REST, permitiendo actualizaciones OTA de firmware con transferencia block-wise sobre enlaces de baja velocidad (<250 kbps) sin timeouts.

\textbf{H7 - Procesamiento CEP Local:} El motor de reglas Complex Event Processing (CEP) de ThingsBoard Edge desplegado localmente en gateway procesa >10,000 eventos/seg con latencia <10 ms P99, ejecutando rule chains complejas (filtrado, agregación, transformación, alarmas) sin requerir round-trip WAN, reduciendo latencia de respuesta en >80\% comparado con procesamiento cloud.

\textbf{H8 - Ventaja Comparativa Integral:} La arquitectura propuesta supera a arquitecturas tradicionales (cloud-centric MQTT/LTE) en al menos 5 de 7 métricas clave: latencia (<30\% baseline), overhead paquetes (<40\% baseline), tráfico WAN (<35\% baseline), disponibilidad offline (>72h vs 0h), precisión IA (>95\% vs N/A), alcance HaLow (>150\% vs WiFi), y eficiencia energética (<60\% baseline).

\section{Objetivos}

\subsection{Objetivo General}

Diseñar, implementar y validar una arquitectura IoT centrada en edge gateways para aplicaciones Smart Energy que integre: (1) stack de protocolos optimizado 6LoWPAN/CoAP/LwM2M sobre IEEE 802.15.4 para reducción de latencia y overhead, (2) capacidades de procesamiento edge con IA local para gestión inteligente de recursos en tiempo real, y (3) conectividad de última milla mediante IEEE 802.11ah con estrategia multi-banda (2/4/8 MHz) adaptada a casos de uso específicos, garantizando latencia end-to-end <100 ms, reducción de tráfico WAN >65\%, y disponibilidad >99\% con conformidad a estándares IEEE 2030.5-2023 e ISO/IEC 30141:2024.

\subsection{Objetivos Específicos}

\textbf{OE1 - Stack de Protocolos Optimizado 6LoWPAN/CoAP/LwM2M:}
\begin{itemize}
\item Implementar capa de adaptación 6LoWPAN (RFC 6282) con compresión IPHC/NHC sobre IEEE 802.15.4, validando reducción de overhead de headers >85\% (de 48 bytes a <7 bytes) en tráfico de telemetría Smart Energy.
\item Desplegar protocolo CoAP (RFC 7252) con modos CON/NON, Observe (RFC 7641) para subscripciones, y block-wise transfer (RFC 7959), midiendo latencia <30 ms para transacciones request/response en topologías mesh 3-5 saltos.
\item Integrar LwM2M 1.2 (OMA SpecWorks) con objetos estándar (Security, Server, Device, Connectivity Monitoring, Firmware Update) para gestión unificada de dispositivos, validando reducción de tráfico de gestión >75\% vs soluciones HTTP/REST propietarias.
\item Caracterizar empíricamente PDR (Packet Delivery Ratio), latencia por hop, y consumo energético por bit transmitido en función de topología mesh (star, tree, mesh completo) y carga de red (5/10/25/50 nodos).
\end{itemize}

\textbf{OE2 - Edge Gateway con Procesamiento en Tiempo Real e IA:}
\begin{itemize}
\item Desplegar stack de servicios containerizados (ThingsBoard Edge, PostgreSQL + TimescaleDB, Apache Kafka, IEEE 2030.5 Server) sobre OpenWRT 23.05 con kernel PREEMPT\_RT, garantizando latencias de procesamiento <10 ms P99 para pipeline MQTT ingestion → rule engine → TimescaleDB persistence.
\item Integrar motor de inferencia LLM local (Ollama + Llama 3.2 3B) con latencia <500 ms para análisis de telemetría en tiempo real, implementando casos de uso: (a) detección de anomalías en consumo con precisión >95\%, (b) mantenimiento predictivo basado en patrones de alarmas, (c) compresión adaptativa de datos según bandwidth disponible.
\item Implementar gestión inteligente de recursos con adaptación dinámica: priorización de tráfico crítico (alarmas) vs no crítico (históricos), ajuste automático de frecuencia de muestreo según condiciones de red, y compactación de datos mediante CBOR/Protocol Buffers reduciendo payload >40\%.
\item Validar resiliencia mediante buffering persistente local con capacidad >100,000 mensajes (~500 MB), sincronización bidireccional post-desconexión WAN >72h con catch-up <30 minutos, y disponibilidad de servicios locales (dashboards, rule engine) >99\% durante particiones WAN.
\end{itemize}

\textbf{OE3 - Arquitectura Multi-Banda IEEE 802.11ah con Nodos HaLow:}
\begin{itemize}
\item Diseñar arquitectura de red basada en gateways edge con nodos HaLow (Morse Micro MM6108) soportando topologías Star (simple), Mesh 802.11s (auto-healing HWMP), y EasyMesh (IEEE 1905.1 roaming coordinado), validando escalabilidad a 50+ nodos por gateway sin degradación >10\% de latencia.
\item Caracterizar empíricamente desempeño por bandwidth:
  \begin{itemize}
  \item \textbf{2 MHz}: Sensibilidad -96 dBm, alcance >2 km NLOS, throughput 300-450 kbps, MCS 1-2, latencia <100 ms P95, PDR >98\% con SNR 8-12 dB. Caso de uso: sensores remotos rurales, lecturas horarias, penetración indoor.
  \item \textbf{4 MHz}: Sensibilidad -91 dBm, alcance 1-1.5 km, throughput 40 Mbps agregado, MCS 3-4, latencia <50 ms P95, soporte 50+ nodos concurrentes. Caso de uso: gestión balanceada zonas suburbanas, lecturas cada 15 min.
  \item \textbf{8 MHz}: Sensibilidad -85 dBm, alcance 0.5-1 km LOS, throughput >80 Mbps, MCS 5-7, latencia <20 ms P99. Caso de uso: backhaul de concentradores en zonas urbanas con línea de vista, agregación de 100+ dispositivos.
  \end{itemize}
\item Implementar algoritmo de selección adaptativa de bandwidth basado en: (a) condiciones de propagación (RSSI, SNR, PDR histórico), (b) requisitos de aplicación (latencia, throughput, prioridad), y (c) densidad de red (número de nodos activos, carga agregada).
\item Evaluar escalabilidad arquitectónica: topología Star (2,500 endpoints, 3 km), Mesh 802.11s (7,500 endpoints, 9 km, auto-healing <10s), EasyMesh (12,500 endpoints, roaming transparente, band steering 2/4/8 MHz).
\end{itemize}

\textbf{OE4 - Validación Experimental Comparativa:}
\begin{itemize}
\item Realizar benchmarking cuantitativo vs 2 baselines: (a) Cloud-centric (MQTT/JSON/TCP sobre LTE Cat-M1), (b) Edge-lite (Node-RED local + MQTT cloud).
\item Métricas comparadas: latencia end-to-end P50/P95/P99, overhead de paquetes (bytes header/payload), tráfico WAN (GB/mes), disponibilidad offline (horas), precisión IA (% detección correcta), alcance (km), consumo energético (mJ/bit).
\item Generar datasets públicos de mediciones (latencias, throughput, PDR) con 10+ nodos IoT ESP32-C6 Thread LwM2M en despliegue piloto de 72 horas continuas bajo condiciones variables de carga y propagación.
\end{itemize}

\textbf{OE5 - Caso de Estudio Smart Energy Real:}
\begin{itemize}
\item Desplegar prototipo funcional para 900 medidores residenciales con topología: 300 nodos ESP32-C6 Thread por gateway × 3 gateways Raspberry Pi 4 + OpenWRT + HaLow, validando arquitectura en condiciones reales urbanas/suburbanas.
\item Implementar conformidad IEEE 2030.5-2023 (Function Sets: DCAP, Time, EndDevice, MirrorUsagePoint, MirrorMeterReading, Messaging) con validación de interoperabilidad funcional vía test suite OpenADR VTN.
\item Documentar lecciones aprendidas, patrones de diseño arquitectónicos, y guías de implementación técnica (instalación OpenWRT, configuración HaLow 4 modos, despliegue stack Docker, tuning kernel PREEMPT\_RT) en anexos técnicos completos.
\end{itemize}

\section{Alcances y Limitaciones}

\subsection{Alcances}

\begin{enumerate}
\item \textbf{Diseño arquitectónico}: Especificación completa de arquitectura multi-capa con definición de componentes, interfaces y flujos de datos, mapeo a vistas ISO/IEC 30141 (funcional, información, despliegue, operacional).

\item \textbf{Implementación prototipo}: Gateway funcional basado en Banana Pi BPI-R4 con integración Thread (nRF52840 RCP), HaLow (Morse Micro MM6108), LTE (Quectel EG25-G), OpenWRT 23.05.x y stack Docker Compose con 7 servicios.

\item \textbf{Conformidad estándares}: Implementación de IEEE 2030.5-2023 Function Sets (DCAP, Time, EndDevice, MirrorUsagePoint, MirrorMeterReading, Messaging) y mapeo ISO/IEC 30141:2024.

\item \textbf{Nodos IoT}: Desarrollo de nodos ESP32-C6 Thread con cliente LwM2M, sensores BME280 y firmware actualizable OTA.

\item \textbf{Validación experimental}: Medición empírica de latencia, throughput, disponibilidad, failover y overhead en condiciones controladas de laboratorio y despliegue piloto urbano.

\item \textbf{Documentación técnica}: Anexos con guías de instalación (OpenWRT, docker-compose), configuraciones UCI completas, schemas IEEE 2030.5 XML, código fuente completo (GitHub).

\item \textbf{Evaluación comparativa}: Benchmarking cuantitativo vs 2 baselines (AWS IoT Core cloud-centric, Node-RED edge-lite) con métricas de latencia, disponibilidad, costos, complejidad.
\end{enumerate}

\subsection{Limitaciones}

\begin{enumerate}
\item \textbf{Escala de despliegue}: Validación con 10 nodos IoT y 2 gateways en área de 300 metros. No se valida escalabilidad a miles de dispositivos en despliegue real.

\item \textbf{Hardware específico}: Implementación dependiente de Morse Micro MM6108 (único chipset HaLow comercialmente disponible en 2024). Resultados pueden no generalizar a futuros chipsets.

\item \textbf{Certificación formal}: No se realiza certificación formal Thread 1.3.1 ni IEEE 2030.5. Conformidad validada mediante interoperabilidad funcional, no certificación oficial.

\item \textbf{Seguridad}: Implementación de TLS 1.2/1.3 y certificados X.509, pero sin auditoría de seguridad formal ni penetration testing exhaustivo.

\item \textbf{Estándares excluidos}: No se implementa IEC 61850 (comunicación en subestaciones) ni interoperabilidad PLC (Power Line Communication).

\item \textbf{Cobertura geográfica}: Validación en entorno urbano/suburbano. No se valida en zonas rurales remotas con cobertura celular limitada.

\item \textbf{Condiciones ambientales}: Pruebas en condiciones de laboratorio (20-25°C, humedad controlada). No se valida operación en extremos de rango industrial (-40°C a +85°C).

\item \textbf{Regulaciones RF}: Operación en banda ISM 902-928 MHz (EE.UU./América). Requiere adaptación para bandas 863-868 MHz (Europa) o 755-787 MHz (China).
\end{enumerate}

\section{Contribuciones Esperadas}

\subsection{Contribuciones Académicas}

\begin{enumerate}
\item \textbf{Arquitectura de referencia IoT heterogénea}: Especificación de arquitectura multi-capa para gateways edge que integra múltiples PHYs (802.15.4, 802.11ah, LTE), conforme con ISO/IEC 30141:2024, documentando patrones de diseño, trade-offs arquitectónicos y decisiones de ingeniería.

\item \textbf{Caracterización empírica Thread ↔ HaLow}: Primera caracterización publicada de latencias, throughput y reliability en integración Thread-HaLow mediante bridge Ethernet transparente, incluyendo análisis de overhead de OTBR y impacto de topologías mesh.

\item \textbf{Metodología IEEE 2030.5 sobre Linux embebido}: Documentación de estrategias de implementación de Function Sets IEEE 2030.5 sobre plataformas resource-constrained (ARMv8, 4 GB RAM), incluyendo optimizaciones de XML parsing, caching y gestión de certificados.

\item \textbf{Benchmarking arquitecturas edge IoT}: Dataset público de mediciones comparativas (latencia, throughput, overhead) entre arquitecturas cloud-centric, edge-lite y edge-full, proporcionando guías de selección arquitectónica basadas en requisitos de aplicación.
\end{enumerate}

\subsection{Contribuciones Técnicas}

\begin{enumerate}
\item \textbf{Implementación open-source IEEE 2030.5}: Servidor Python/Flask que implementa 6 Function Sets con schemas XML validados, autenticación TLS mutua y RBAC, disponible bajo licencia Apache 2.0 en repositorio GitHub.

\item \textbf{Configuraciones OpenWRT para HaLow}: Documentación completa de configuración UCI para driver Morse Micro MM6108 (SPI), incluyendo scripts de inicialización, configuración hostapd y troubleshooting.

\item \textbf{Stack Docker Compose optimizado}: Composición de servicios edge (ThingsBoard, TimescaleDB, Kafka, IEEE 2030.5, Ollama) con resource management, health checks y restart policies, optimizado para hardware Cortex-A53.

\item \textbf{Firmware nodos IoT Thread LwM2M}: Implementación ESP-IDF para ESP32-C6 con cliente LwM2M (Wakaama), driver BME280, Deep Sleep scheduling y OTA segura.
\end{enumerate}

\subsection{Contribuciones a la Industria}

\begin{enumerate}
\item \textbf{Reducción de costos operacionales}: Demostración de viabilidad económica de arquitectura HaLow-based vs LTE, con TCO 32\% inferior en despliegues de 1,000+ puntos durante 5 años.

\item \textbf{Guía de implementación práctica}: Documentación técnica completa (instalación, configuración, troubleshooting) que permite replicación de arquitectura por integradores de sistemas y utilities.

\item \textbf{Caso de negocio para HaLow}: Evaluación cuantitativa de beneficios (throughput, latencia, costos) de Wi-Fi HaLow vs LoRaWAN/LTE Cat-M1 en aplicaciones Smart Energy, acelerando adopción de estándar IEEE 802.11ah.

\item \textbf{Interoperabilidad multi-vendor}: Validación de conformidad IEEE 2030.5 que facilita integración con dispositivos certificados de múltiples fabricantes, reduciendo lock-in tecnológico.
\end{enumerate}

\section{Organización del Documento}

El presente documento se estructura en los siguientes capítulos:

\textbf{Capítulo 1 - Introducción}: Contextualización del problema, estado actual de tecnologías IoT, brechas identificadas, planteamiento del problema, hipótesis, objetivos, metodología, alcances y contribuciones esperadas.

\textbf{Capítulo 2 - Marco Teórico}: Fundamentos de redes Smart Energy, protocolos de comunicación IoT (Thread, HaLow, LTE Cat-M1), estándares de interoperabilidad (IEEE 2030.5, ISO/IEC 30141, IEC 61850), tecnologías de edge computing (Docker, TimescaleDB, Kafka), plataformas IoT (ThingsBoard), seguridad en sistemas IoT, y estado del arte de arquitecturas edge heterogéneas.

\textbf{Capítulo 3 - Gateway de Telemetría}: Arquitectura del gateway multi-protocolo, conformidad con estándares internacionales, requisitos funcionales/no funcionales, arquitectura jerárquica de 3 niveles IoT, diseño de hardware y software, y Stack de Servicios Containerizados.

\textbf{Capítulo 4 - Arquitectura de Telemetría}: Visión general de arquitectura end-to-end, capa de dispositivos (medidores inteligentes), capa de campo (nodos Thread, DCUs), capa de agregación (gateway HaLow), capa de aplicación (ThingsBoard cloud), análisis de seguridad end-to-end, y modelado de latencias mediante teoría de colas.

\textbf{Capítulo 5 - Conclusiones y Trabajo Futuro}: Síntesis de la investigación, cumplimiento de objetivos, validación de hipótesis, contribuciones académicas y técnicas, lecciones aprendidas, limitaciones del trabajo, y recomendaciones para trabajo futuro.

\textbf{Anexos}: Instalación OpenWRT y configuración HaLow (Anexo A), Docker Compose y servicios (Anexo B), Scripts de integración (Anexo C), Especificaciones IEEE 2030.5 (Anexo D), Implementación nodo IoT ESP32-C6 (Anexo E), Configuraciones OpenWRT UCI completas (Anexo F).
