\chapter{Marco Teórico}

\section{Contexto Smart Energy y Estándares de Interoperabilidad}

Este capítulo establece las bases teóricas y conceptuales que sustentan la arquitectura propuesta, siguiendo una estructura narrativa de lo general hacia lo particular. Iniciamos con el contexto de Energía Inteligente (\textit{Smart Energy}) y los estándares de interoperabilidad que definen el marco regulatorio y arquitectónico (§2.1), para luego profundizar en la pila de protocolos IoT (\textit{protocol stack}) que implementa estas especificaciones (§2.2), las tecnologías de computación en el borde (\textit{edge computing}) que materializan la arquitectura distribuida (§2.3), los aspectos de seguridad transversales (§2.4), y finalmente el posicionamiento respecto al estado del arte (§2.5).

\subsection{Evolución de las Infraestructuras Eléctricas}

La transición de redes eléctricas tradicionales unidireccionales hacia Smart Grids bidireccionales representa un cambio paradigmático en la operación de sistemas energéticos~\cite{velasquezSmartGridsEmpowered2024,alsafranChallengesImplementingIoT2025}. Las Smart Grids integran tecnologías de información y comunicación (TIC) para monitoreo, control y optimización en tiempo real del flujo eléctrico desde generación hasta consumo final~\cite{SmartHomeEnergy2024}. Este enfoque permite: integración masiva de energías renovables distribuidas (DER - Distributed Energy Resources), gestión activa de la demanda (DSM - Demand Side Management), detección y auto-recuperación de fallas (self-healing), y participación activa de prosumidores (consumidores que también generan energía).

Según el National Institute of Standards and Technology (NIST), una Smart Grid implementa siete dominios interconectados: Bulk Generation, Transmission, Distribution, Customer, Operations, Markets, y Service Provider~\cite{IEEERecommendedPractice}. La infraestructura de medición inteligente (AMI - Advanced Metering Infrastructure) constituye el dominio Customer, proporcionando visibilidad granular de patrones de consumo y habilitando servicios de respuesta a la demanda (DR).

\subsection{Arquitectura de Referencia Smart Grid NIST}

El modelo de referencia NIST para Smart Grid (NIST Framework and Roadmap for Smart Grid Interoperability Standards) define tres capas principales que estructuran la interacción entre infraestructura física, comunicaciones y aplicaciones~\cite{alsuwaidiSecuringSmartGrid2024}:

\begin{enumerate}
\item \textbf{Power and Energy Layer}: Infraestructura física de generación, transmisión, distribución y almacenamiento.
\item \textbf{Communication Layer}: Redes de datos multi-protocolo (HAN, NAN, WAN) que transportan información de telemetría y comandos de control.
\item \textbf{Application Layer}: Sistemas de gestión de energía (EMS), gestión de distribución (DMS), gestión de demanda (DERMS), y analytics.
\end{enumerate}

La arquitectura AMI se compone típicamente de: medidores inteligentes (smart meters) instalados en puntos de consumo, concentradores/gateways que agregan datos de decenas o cientos de medidores, y head-end systems en centros de control que procesan millones de registros diarios. Este modelo NIST establece el marco conceptual sobre el cual se construyen los estándares de interoperabilidad descritos a continuación.

\subsection{IEEE 2030.5-2023 (Smart Energy Profile 2.0)}

IEEE 2030.5, anteriormente conocido como ZigBee SEP 2.0, es el estándar de facto para interoperabilidad de dispositivos Smart Energy en América del Norte (mandatorio para DR programs en California SB-2030)~\cite{IEEERecommendedPractice,knyazevComparativeAnalysisStandards2017}. Define un modelo RESTful sobre HTTP/TLS para comunicación cliente-servidor entre dispositivos de campo (medidores, termostatos, inversores solares) y sistemas de gestión (DERMS, head-end systems)~\cite{tangResearchInteroperabilityIoT}.

\subsubsection{Arquitectura RESTful del Estándar}

IEEE 2030.5 estructura funcionalidades en Function Sets exponiendo recursos REST: /dcap (descubrimiento), /tm (sincronización horaria), /edev (registro dispositivos), /mup (datos de medición), /mr (perfiles de carga), /msg (notificaciones), /dr (Demand Response), /fsa (QoS).

Ejemplo: \texttt{GET /tm} retorna XML con \texttt{<currentTime>}, \texttt{<dstOffset>}, \texttt{<localTime>} para sincronización horaria.

\subsubsection{Function Sets Implementados}

\textbf{1. Device Capability (DCAP)}: El cliente consulta /dcap para descubrir qué Function Sets implementa el servidor:

\begin{verbatim}
<DeviceCapability>
  <EndDeviceListLink href="/edev"/>
  <MirrorUsagePointListLink href="/mup"/>
  <TimeLink href="/tm"/>
  <MessagingProgramListLink href="/msg"/>
</DeviceCapability>
\end{verbatim}

\textbf{2. End Device (ED)}: Registro de dispositivos con LFDI (Long Form Device Identifier) derivado de certificado X.509:

\begin{equation}
\text{LFDI} = \text{SHA256}(\text{SubjectPublicKeyInfo})[:160 \text{ bits}]
\end{equation}

\textbf{3. Mirror Meter Reading (MMR)}: Publicación de lecturas de medición con granularidad configurable (típicamente 15 minutos). Datos codificados en formato OBIS (Object Identification System) según IEC 62056:

\begin{itemize}
\item 1-0:1.8.0*255 (Active energy import total)
\item 1-0:2.8.0*255 (Active energy export total)
\item 1-0:31.7.0*255 (Instantaneous current L1)
\end{itemize}

\textbf{4. Messaging (MSG)}: Push notifications del servidor hacia clientes mediante polling o subscriptions. Prioridades 0-9, donde 0 es crítico (ej. alerta de sobretensión).

\subsubsection{Modelo de Datos y Schemas XML}

IEEE 2030.5 define schemas XML estrictos. Ejemplo: \texttt{<MirrorMeterReading>} con elementos \texttt{<Reading>} (valor, timePeriod), \texttt{<ReadingType>} (commodity=1 para electricidad, uom=72 para Wh, flowDirection=1 para import, accumulationBehaviour=4 para cumulative).

El estándar define 200+ ReadingTypes combinando 7 dimensiones (commodity, uom, flowDirection, etc.) para representar cualquier tipo de medición energética, garantizando interoperabilidad semántica entre implementaciones multi-vendor.

\subsection{ISO/IEC 30141:2024 - Marco de Interoperabilidad IoT}

ISO/IEC 30141, publicado en 2018 y actualizado en 2024, define un framework estandarizado para sistemas IoT mediante cuatro vistas complementarias (funcional, información, despliegue, operacional). Especifica componentes, interfaces y flujos de información, complementando ISO/IEC 29100 (Privacy Framework) e ISO/IEC 27001 (Security Management)~\cite{tangResearchInteroperabilityIoT}.

\subsubsection{Modelo de Capas Funcionales}

ISO/IEC 30141 define cuatro vistas complementarias, siendo la Vista Funcional la más relevante para esta tesis. Descompone el sistema IoT en entidades funcionales (FE - Functional Entities):

\begin{itemize}
\item \textbf{Sensing FE}: Adquisición de datos del mundo físico (sensores).
\item \textbf{Actuation FE}: Control de actuadores.
\item \textbf{Processing FE}: Transformación, agregación, filtrado de datos.
\item \textbf{Storage FE}: Persistencia de datos (time-series DB, object storage).
\item \textbf{Communication FE}: Transporte de datos entre FEs.
\item \textbf{Security FE}: Autenticación, autorización, cifrado, auditoría.
\item \textbf{Management FE}: Configuración, monitoreo, actualizaciones OTA.
\item \textbf{Application Support FE}: APIs, event management, workflows.
\end{itemize}

\textbf{Vista de Información:} Define modelos de datos, metadatos, y formatos de intercambio (JSON, CBOR, Protobuf).

\textbf{Vista de Despliegue:} Mapeo de entidades funcionales a componentes físicos (dispositivos, pasarelas o \textit{gateways}, servidores en la nube o \textit{cloud servers}) con especificación de protocolos de comunicación.

\textbf{Vista Operacional:} Workflows de operación, mantenimiento, troubleshooting.

\subsubsection{Mapeo de Arquitectura Propuesta a ISO/IEC 30141}

La arquitectura propuesta en esta tesis implementa las 7 entidades funcionales requeridas por ISO/IEC 30141:2024, garantizando conformidad con el estándar:

\begin{table}[H]
\centering
\caption{Mapeo arquitectura propuesta a estándar ISO/IEC 30141:2024 IoT Reference}
\label{tab:iso30141-mapping}
\resizebox{\textwidth}{!}{%
\begin{tabular}{|>{\centering\arraybackslash}p{3.8cm}|>{\raggedright\arraybackslash}p{11cm}|}
\hline
\rowcolor{blue!20}
\textbf{Entidad Funcional ISO/IEC 30141} & \textbf{Componente Implementado en Tesis} \\
\hline
\textbf{Sensing FE} \newline \footnotesize{Adquisición datos} & Nodos \textbf{ESP32-C6} Thread + interfaz RS485 para medidores \textcolor{blue}{EMSITECH} (protocolo DLMS/COSEM) + sensores DHT22/BMP280 \\
\hline
\textbf{Communication FE} \newline \footnotesize{Conectividad multi-red} & Thread Border Router (\textcolor{green}{\textbf{nRF52840 RCP}}) + HaLow AP (\textcolor{orange}{\textbf{Morse Micro MM6108}}) + LTE modem (\textcolor{purple}{Quectel EG25-G}) \\
\hline
\textbf{Processing FE} \newline \footnotesize{Procesamiento en el borde (\textit{edge})} & \textcolor{red}{\textbf{ThingsBoard Rule Engine}} + Kafka Streams + Ollama LLM procesamiento borde + \textcolor{blue}{nginx balanceador de carga (\textit{load balancer})} \\
\hline
\textbf{Storage FE} \newline \footnotesize{Persistencia datos} & \textbf{PostgreSQL} + \textcolor{green}{TimescaleDB} (hypertables con particionado automático) + Redis cache + backup S3 \\
\hline
\textbf{Security FE} \newline \footnotesize{Seguridad end-to-end} & TLS \textbf{1.2/1.3} mutual auth + IEEE 2030.5 \textcolor{blue}{LFDI} + \textcolor{green}{WPA3-SAE} + HSM certificados \\
\hline
\textbf{Management FE} \newline \footnotesize{Gestión dispositivos} & ThingsBoard \textbf{Device Management} + \textcolor{orange}{OpenWRT UCI} + OTA updates + monitoring Grafana \\
\hline
\textbf{Application Support FE} \newline \footnotesize{APIs y servicios} & IEEE 2030.5 \textcolor{green}{\textbf{REST API}} + ThingsBoard Dashboards + \textcolor{red}{Ollama LLM (MCP)} + WebRTC comunicación \\
\hline
\rowcolor{yellow!20}
\textbf{Conformidad Estándar} & \textcolor{green}{\textbf{SÍ Completa}} - Implementa 7/7 entidades funcionales requeridas por ISO/IEC 30141:2024 \\
\hline
\end{tabular}%
}
\end{table}

La conformidad con ISO/IEC 30141 garantiza que la arquitectura puede integrarse con otros sistemas IoT estándar, facilita auditorías de seguridad y compliance, y proporciona lenguaje común para documentación técnica.

\subsection{IEC 61850 - Comunicación en Subestaciones}

IEC 61850 es la familia de estándares para comunicación en sistemas de automatización de subestaciones eléctricas (SAS). Define modelos de datos abstractos (Logical Nodes) y protocolos de comunicación (MMS, GOOSE, SV) para interoperabilidad multi-vendor.

Aunque excede el alcance de esta tesis (enfocada en distribución/consumidor), IEC 61850 es relevante para futuras integraciones con sistemas SCADA y DMS. El mapeo entre IEEE 2030.5 (dominio Customer) e IEC 61850 (dominio Distribution) se define en IEEE 2030.7, estableciendo puentes de interoperabilidad entre diferentes segmentos de la Smart Grid.

\subsection{Síntesis de Estándares y Transición a la Pila Técnica (\textit{Technical Stack})}

Los estándares IEEE 2030.5, ISO/IEC 30141 e IEC 61850 establecen el marco conceptual y regulatorio para sistemas de Energía Inteligente (\textit{Smart Energy}) interoperables. Sin embargo, su implementación práctica requiere una pila de protocolos IoT (\textit{protocol stack}) optimizada para dispositivos restringidos (\textit{constrained}) con recursos limitados de CPU, memoria y energía. La siguiente sección describe la pila 6LoWPAN/CoAP/LwM2M que materializa estos requisitos, cubriendo desde la capa física (IEEE 802.15.4) hasta la gestión de dispositivos (LwM2M), proporcionando la base técnica sobre la cual se construye la arquitectura propuesta en el Capítulo 3.

\section{Pila de Protocolos IoT para Energía Inteligente (\textit{Protocol Stack for Smart Energy})}

La pila de protocolos (\textit{stack}) para redes IoT en aplicaciones de Energía Inteligente (\textit{Smart Energy}) integra múltiples capas del modelo OSI, optimizando cada una para operar en entornos con restricciones severas: dispositivos con <256 KB RAM, <1 MB Flash, y operación con batería. La arquitectura completa utiliza IEEE 802.15.4 como base PHY/MAC, 6LoWPAN para compresión de IPv6, y CoAP/LwM2M en capa de aplicación, creando una pila extremo-a-extremo (\textit{end-to-end stack}) eficiente y estandarizada.

\subsection{Visión General de la Pila de Protocolos (\textit{Protocol Stack Overview})}

La siguiente tabla sintetiza las capas de la pila de protocolos (\textit{protocol stack}) y sus funciones principales:

\begin{table}[h]
\centering
\small
\caption{Pila de protocolos 6LoWPAN/CoAP/LwM2M para IoT en Energía Inteligente (\textit{Smart Energy}): arquitectura completa desde capa física (\textit{PHY}) hasta gestión de dispositivos. Sobrecarga total comprimida (\textit{Overhead}): ~10-15 bytes (vs ~60+ bytes sin optimización). Eficiencia de carga útil (\textit{payload}): >75\% del MTU disponible para datos de aplicación.}
\label{tab:pila-protocolos}
\begin{tabular}{|p{2.5cm}|p{3.5cm}|p{6.5cm}|}
\hline
\rowcolor{gray!20}
\textbf{Capa OSI} & \textbf{Protocolo} & \textbf{Función Principal} \\
\hline
\textbf{7. Aplicación} & \textcolor{blue}{LwM2M 1.2} & Gestión dispositivos, objetos IPSO telemetría \\
\hline
\textbf{6. Presentación} & CBOR/TLV & Serialización eficiente binaria \\
\hline
\textbf{5. Sesión} & \textcolor{blue}{CoAP RFC 7252} & RESTful para constrained devices \\
\hline
\textbf{4. Transporte} & UDP & No orientado a conexión \\
\hline
\textbf{3. Red} & \textcolor{blue}{6LoWPAN RFC 6282} & Compresión IPv6 headers, fragmentación \\
\hline
\textbf{3. Red} & IPv6 & Direccionamiento global end-to-end \\
\hline
\textbf{2. Enlace (MAC)} & IEEE 802.15.4 MAC & CSMA/CA, ACKs, retransmisiones \\
\hline
\textbf{1. Física} & IEEE 802.15.4 PHY & 2.4 GHz OQPSK, 250 kbps \\
\hline
\end{tabular}
\end{table}

\begin{figure}[h]
\centering
\includegraphics[width=0.9\textwidth]{figures/protocol-stack.png}
\caption{Stack de protocolos del sistema: comparación entre Nodo IoT (Thread/6LoWPAN/CoAP con compresión IPHC), OTBR (IPv6 nativo), Gateway (HaLow/MQTT/TLS) y Cloud (ThingsBoard). Destaca overhead CoAP 22B vs 100B HTTP y compresión IPHC 48B→4.2B}
\label{fig:protocol-stack}
\end{figure}

El flujo de un mensaje de telemetría desde sensor hasta servidor sigue estas transformaciones: \textbf{(1) Aplicación:} LwM2M codifica lectura en TLV (~12 bytes), \textbf{(2) CoAP:} encapsula en mensaje POST NON-confirmable (header 4-10 bytes), \textbf{(3) UDP:} agrega header (8 bytes, puerto 5683), \textbf{(4) IPv6:} construye header completo (40 bytes), \textbf{(5) 6LoWPAN:} comprime IPHC/NHC reduciendo IPv6+UDP de 48 bytes a ~6-11 bytes (reducción 80-90\%), \textbf{(6) IEEE 802.15.4:} fragmenta si payload excede MTU (127 bytes), agrega MAC header (25 bytes) y FCS (2 bytes), transmite a 250 kbps.

\textbf{Ventajas del Stack:} Eficiencia de bandwidth (compresión permite payloads útiles de 100-110 bytes, eficiencia >75\%), interoperabilidad IPv6 (direccionamiento global sin NAT), fragmentación transparente (manejo automático en 6LoWPAN), mesh routing (soporta mesh-under y route-over para multi-hop), y seguridad end-to-end (CoAP sobre DTLS 1.2 sin depender de MAC).

\subsection{Capa Física y Enlace: IEEE 802.15.4 y Tecnologías Complementarias}
\label{sec:ieee802154}

IEEE 802.15.4 define las capas física (PHY) y de control de acceso al medio (MAC) para redes LR-WPAN (Low-Rate Wireless Personal Area Networks), constituyendo la base sobre la cual operan Thread, Zigbee y 6LoWPAN.

\textbf{Capa MAC - Características Principales:} Implementa CSMA/CA (Carrier Sense Multiple Access with Collision Avoidance) con backoff exponencial: antes de transmitir, un nodo espera tiempo aleatorio proporcional a $2^{BE}$ unidades (Backoff Exponent aumenta con cada intento fallido). Frames de datos requieren acknowledgment (ACK) explícito; si no se recibe dentro de macAckWaitDuration, el transmisor reintenta hasta 3 veces~\cite{abdulsalamOverviewRecentWireless2024}.

\textbf{Estructura de Frame y Direccionamiento:} Headers MAC de 9-25 bytes contienen: control de frame (2 bytes), número de secuencia (1 byte), direcciones PAN/dispositivo (2-8 bytes c/u), y FCS (2 bytes). Soporta direccionamiento corto 16-bit (<65K nodos) y extendido IEEE EUI-64 (64-bit global único)~\cite{abdulsalamOverviewRecentWireless2024}.

\textbf{MTU y Eficiencia:} IEEE 802.15.4 define MTU de 127 bytes, consumiendo ~25 bytes en headers PHY/MAC+FCS, dejando ~102 bytes para payload de capas superiores. Esta restricción motiva compresión 6LoWPAN IPHC que reduce headers IPv6+UDP de 48 bytes a ~6 bytes~\cite{abood6LoWPANTechnicalFeatures2024}. En redes densas, CSMA/CA experimenta degradación por colisiones; Thread mitiga con traffic shaping y jitter aleatorio.

\textbf{Tecnologías Complementarias:} \textbf{Wi-Fi HaLow (IEEE 802.11ah)} opera en bandas sub-GHz (902-928 MHz América, 863-868 MHz Europa) con alcance 1-2 km y throughput 150 kbps-86.7 Mbps, soportando hasta 8,191 dispositivos por AP mediante hierarchical AID y Target Wake Time (TWT) para duty cycles <1\%~\cite{scharerPushingWiFiHaLow2025}. \textbf{LTE Cat-M1/NB-IoT} (3GPP Release 13/14) proporciona conectividad celular: Cat-M1 ofrece 1 Mbps con latencia 10-15 ms y full mobility, mientras NB-IoT entrega 250 kbps con MCL 164 dB (+8 dB penetración) optimizado para medidores con reportes esporádicos~\cite{routrayNarrowbandIoTPrinciples2024}.

\subsection{Capa de Adaptación y Red: 6LoWPAN}

6LoWPAN (IPv6 over Low-Power Wireless Personal Area Networks, RFC 6282/4944) es una capa de adaptación que permite transmisión de paquetes IPv6 sobre redes IEEE 802.15.4, superando la limitación del MTU de 127 bytes mediante compresión de headers y fragmentación~\cite{shelby6LoWPANWirelessEmbedded2009,thungonSurvey6LoWPANSecurity2024,abood6LoWPANTechnicalFeatures2024}.

\textbf{Motivación:} El stack IPv6 tradicional presenta overhead prohibitivo: header IPv6 (40 bytes, 31.5\% del MTU) + header UDP (8 bytes, 6.3\%) = 48 bytes (37.8\%) dejando solo 79 bytes útiles (62.2\%)~\cite{mamoImplementationStandardized6LoWPAN2015}. Cada retransmisión en mesh consume energía crítica en dispositivos battery-powered.

\textbf{Compresión IPHC (IPv6 Header Compression):} RFC 6282 reduce headers IPv6 de 40 bytes a 2-7 bytes explotando redundancias: \textbf{(1) Direcciones:} Link-local derivadas de MAC se omiten (16 bytes → 0), multicast con prefijos conocidos (ff02::/16) se comprimen a 1-6 bytes, context-based compression referencia prefijos Thread (fd00::/64) por ID de 4 bits. \textbf{(2) Campos IPv6:} Version (4 bits) siempre 6 se omite, Traffic Class/Flow Label se omiten si 0, Hop Limit se comprime a 2 bits si $\leq$64.

\begin{table}[h]
\small
\centering
\caption{Compresión IPHC de headers IPv6 en IEEE 802.15.4: reducción 40 bytes → 2-7 bytes (82.5-95\%). Técnicas: omisión Version/TC/FL, compresión direcciones link-local derivadas de MAC, context-based compression de prefijos Thread conocidos.}
\label{tab:iphc-compression}
\begin{tabular}{p{3cm}p{2.5cm}p{2.8cm}p{2.2cm}}
\hline
\rowcolor{gray!20}
\textbf{Campo IPv6} & \textbf{Original (bytes)} & \textbf{Comprimido (bytes)} & \textbf{Reducción (\%)} \\
\hline
Version + TC + FL & 4 & \textcolor{blue}{0} & \textcolor{green}{100\%} \\
\hline
Payload Length & 2 & \textcolor{blue}{0 (implícito)} & \textcolor{green}{100\%} \\
\hline
Next Header & 1 & \textcolor{blue}{0 (UDP NHC)} & \textcolor{green}{100\%} \\
\hline
Hop Limit & 1 & 0-1 & 0-100\% \\
\hline
Source Address & 16 & \textcolor{blue}{0-2 (link-local)} & \textcolor{green}{87.5-100\%} \\
\hline
Dest Address & 16 & \textcolor{blue}{0-2 (link-local)} & \textcolor{green}{87.5-100\%} \\
\hline
\textbf{Total IPv6} & \textbf{40} & \textbf{\textcolor{yellow}{2-7}} & \textbf{\textcolor{yellow}{82.5-95\%}} \\
\hline
\end{tabular}
\end{table}

\textbf{Compresión NHC (Next Header Compression):} Extiende compresión a UDP y CoAP. UDP header compression: puertos en rango CoAP típico (61616-61631) se comprimen de 4 bytes a 1 byte, Length se omite (inferido de frame 802.15.4), Checksum se reemplaza por checksum 802.15.4.

\begin{table}[h]
\small
\centering
\caption{Compresión NHC de Header UDP para Smart Energy CoAP: reducción 8 bytes → 1-2 bytes (75-87.5\%). Puertos CoAP comprimidos a 4 bits cada uno, Length/Checksum eliminados.}
\label{tab:nhc-udp}
\begin{tabular}{p{3cm}p{2.5cm}p{2.8cm}p{2.2cm}}
\hline
\rowcolor{gray!20}
\textbf{Campo UDP} & \textbf{Original (bytes)} & \textbf{Comprimido (bytes)} & \textbf{Reducción (\%)} \\
\hline
Source Port & 2 & \textcolor{blue}{0.5 (4 bits)} & \textcolor{green}{75\%} \\
\hline
Dest Port & 2 & \textcolor{blue}{0.5 (4 bits)} & \textcolor{green}{75\%} \\
\hline
Length & 2 & \textcolor{blue}{0} & \textcolor{green}{100\%} \\
\hline
Checksum & 2 & \textcolor{blue}{0} & \textcolor{green}{100\%} \\
\hline
\textbf{Total UDP} & \textbf{8} & \textbf{\textcolor{yellow}{1-2}} & \textbf{\textcolor{yellow}{75-87.5\%}} \\
\hline
\end{tabular}
\end{table}

\textbf{Compresión Total y Payload Disponible:}
\begin{equation}
\text{Overhead comprimido} = 2\text{-}7 \text{ (IPHC)} + 1\text{-}2 \text{ (NHC-UDP)} = 3\text{-}9 \text{ bytes}
\end{equation}

\begin{equation}
\text{Payload disponible} = 127 - 25 \text{ (MAC)} - 3\text{-}9 \text{ (IPHC+NHC)} = 93\text{-}99 \text{ bytes (73-78\% MTU)}
\end{equation}

vs 79 bytes (62\%) sin compresión → **Ganancia 14-16 bytes (18-20\% más payload)**.

\textbf{Fragmentación y Reensamblado:} Cuando payload IPv6 excede MTU (incluso con compresión), 6LoWPAN fragmenta: First Fragment contiene header (4 bytes: datagram\_size, datagram\_tag) + primeros N bytes; Subsequent Fragments tienen header (5 bytes: datagram\_size, datagram\_tag, datagram\_offset) + siguientes N bytes~\cite{abood6LoWPANTechnicalFeatures2024}. \textbf{Limitaciones:} aumenta latencia (espera de fragmentos), reduce confiabilidad (pérdida de 1 fragmento = descarte completo), consume buffers RAM en receptor. \textbf{Best Practice:} diseñar payloads $\leq$70 bytes para evitar fragmentación en mesh (headers Thread/6LoWPAN/UDP consumen ~25-30 bytes).

\textbf{Impacto en Latencia:}

\begin{table}[h]
\small
\centering
\caption{Latencia por hop en Thread mesh con/sin compresión 6LoWPAN. Escenario: topología lineal 5 hops, IEEE 802.15.4 @ 250 kbps, canal 15 (2.4 GHz). Reducción latencia total: 71\% (7.7 ms → 2.2 ms) mediante IPHC+NHC.}
\label{tab:6lowpan-latency}
\begin{tabular}{p{4cm}p{2.8cm}p{2.8cm}p{2cm}}
\hline
\rowcolor{gray!20}
\textbf{Escenario Mesh Thread} & \textbf{Sin Compresión} & \textbf{Con IPHC+NHC} & \textbf{Reducción} \\
\hline
TX @ 250 kbps (headers) & 1.54 ms (48B) & \textcolor{blue}{0.29 ms (7B)} & \textcolor{green}{81\%} \\
\hline
Procesamiento & 0 ms & 0.15 ms & — \\
\hline
Total por hop & 1.54 ms & \textcolor{blue}{0.44 ms} & \textcolor{green}{71\%} \\
\hline
\textbf{Latencia 5 hops} & \textbf{7.7 ms} & \textbf{\textcolor{yellow}{2.2 ms}} & \textbf{\textcolor{yellow}{71\%}} \\
\hline
\end{tabular}
\end{table}

La compresión 6LoWPAN reduce latencia en topologías mesh multi-hop >70\%~\cite{shelby6LoWPANWirelessEmbedded2009}, crítico para Smart Energy con requisitos <100 ms.

\textbf{Thread - Protocolo Mesh sobre 6LoWPAN:} Thread es un protocolo IPv6 sobre IEEE 802.15.4 (ver sección~\ref{sec:ieee802154}) diseñado para IoT doméstico/industrial~\cite{abdulsalamOverviewRecentWireless2024}. Implementa routing mesh adaptativo basado en métricas LQI (Link Quality Indicator) y path cost. Topología jerárquica: Leader (gestiona Router IDs y Network Data), Router (forwarding con tabla de rutas completa), REED (Router Eligible End Device, promueve si necesario), End Device (leaf sin routing). Routing utiliza MLE (Mesh Link Establishment) con métrica path cost = $\sum_{i=1}^{n} \frac{100}{\text{LQI}_i}$ donde LQI $\in$ [0,255]. Thread Border Router (OTBR) actúa como gateway hacia redes IP tradicionales, proveyendo traducción IPv6, NAT64/DNS64, multicast forwarding y commissioning seguro~\cite{choudharyInternetThingsComprehensive2024,openthread2024}.

\begin{table}[h]
\centering
\small
\caption{Comparación protocolos mesh 2.4 GHz para Smart Energy: Thread vs Zigbee vs Bluetooth Mesh}
\label{tab:mesh-protocols-detailed}
\begin{tabular}{|p{2.8cm}|p{2.6cm}|p{2.6cm}|p{2.6cm}|p{2.8cm}|}
\hline
\rowcolor{gray!20}
\textbf{Criterio} & \textbf{Thread 1.3.1} & \textbf{Zigbee 3.0} & \textbf{BT Mesh} & \textbf{Justificación} \\
\hline
\textbf{Stack routing} & \textcolor{green}{IPv6 6LoWPAN} & Propietario AODV & Managed Flooding & \textbf{Thread:} Standard IETF \\
\hline
\textbf{IPv6 nativo} & \textcolor{green}{Sí (end-to-end)} & No (APS layer) & No (GATT proxy) & \textbf{Thread crítico:} IEEE 2030.5 requiere IPv6 \\
\hline
\textbf{Border Router} & \textcolor{green}{OTBR standard} & Coordinador vendor & Proxy nodes & \textbf{Thread:} No translation layer \\
\hline
\textbf{Route repair} & \textcolor{green}{Proactive MLE} & Reactive RERR & Flooding & \textbf{Thread:} Convergencia <5s \\
\hline
\textbf{Consumo RX} & 6.5 mA & \textcolor{green}{5.8 mA (12\% menor)} & 8-12 mA & Zigbee ventaja marginal \\
\hline
\textbf{Security} & AES-128-CCM & AES-128-CCM & AES-128-CCM & Empate (mismo cipher) \\
\hline
\textbf{Matter support} & \textcolor{green}{Nativo} & Requiere bridge & Requiere bridge & \textbf{Thread:} CSA certified \\
\hline
\textbf{Ecosistema AMI} & Emergente (2019+) & \textcolor{green}{Maduro (>15 años)} & Incipiente & Zigbee: deployment experience \\
\hline
\textbf{Costo chip} & \$3-5 (ESP32-C6) & \textcolor{green}{\$2-4 (20\% menor)} & \$4-6 & Zigbee ligera ventaja \\
\hline
\textbf{Interoperabilidad} & \textcolor{green}{OpenThread OSS} & ZigBee Alliance & Bluetooth SIG & Thread: open-source stack \\
\hline
\multicolumn{5}{|p{15.2cm}|}{\textbf{Decisión arquitectónica:} Thread seleccionado por IPv6 end-to-end crítico para IEEE 2030.5 DER native integration, a pesar de menor madurez ecosystem vs Zigbee. Trade-off aceptado: consumo 12\% mayor (6.5 vs 5.8 mA) compensado por sleep duty-cycle 99.5\%.} \\
\hline
\end{tabular}
\end{table}

\textbf{Trade-off analizado en profundidad:} Thread seleccionado sobre Zigbee maduro (>15 años deployment en utilities, >300M devices instalados) por tres razones arquitectónicas críticas para AMI con integración IEEE 2030.5:

\begin{enumerate}
    \item \textbf{IPv6 native end-to-end:} Zigbee utiliza Application Support (APS) sublayer propietaria sobre 802.15.4, requiriendo gateway con traducción APS↔IP. Esta conversión introduce: (a) \textbf{Latencia adicional +5-10 ms} por frame translation y buffering, (b) \textbf{Punto único de falla} en Zigbee Coordinator (SPOF), (c) \textbf{Complejidad arquitectónica} manteniendo dual-stack (Zigbee + IP) con translation tables. Thread elimina translation layer: 6LoWPAN IPHC comprime headers IPv6 (40 bytes → 2-7 bytes) pero mantiene semántica IP nativa, permitiendo IEEE 2030.5 Function Sets (metering, pricing, DER control) operar directamente sobre CoAP/UDP sin gateways intermedios.
    
    \item \textbf{Routing proactivo vs reactivo:} Zigbee AODV (Ad-hoc On-Demand Distance Vector) descubre rutas reactivamente mediante RREQ/RREP flooding cuando necesita transmitir, causando \textbf{latencia inicial 100-500 ms} y \textbf{broadcast storms} en redes densas (>50 nodes). Thread MLE (Mesh Link Establishment) mantiene tabla de rutas proactiva con advertisements periódicos cada 32 s, convergencia <5 s ante fallas, path selection basado en metrics ETX (Expected Transmission Count). Para AMI con reportes periódicos cada 15-60 min, overhead proactive MLE (32 bytes cada 32 s = 8 bps por node) es despreciable vs beneficio de latencia determinista.
    
    \item \textbf{Gateway integration complexity:} OpenThread Border Router (OTBR) proporciona implementación open-source completa (GitHub 15K+ stars, Google/Nordic/Silicon Labs maintainers) con Docker images oficiales, integración systemd, configuración via REST API~\cite{openthread2024}. Zigbee requiere vendor-specific coordinators (Texas Instruments CC2652, Silicon Labs EFR32) con SDKs propietarios, limitando portabilidad. Matter specification (Connectivity Standards Alliance) adoptó Thread como \textit{mandatory} transport layer, señalando convergencia industry hacia IPv6-native mesh~\cite{threadMatterConvergence2024}.
\end{enumerate}

\textbf{Trade-offs aceptados explícitamente:}

\begin{itemize}
    \item \textbf{Consumo adicional 12\%:} Thread RX corriente 6.5 mA vs Zigbee 5.8 mA (medido en nRF52840 vs CC2652R, RSSI -60 dBm, canal 15). Impacto mitigado por duty-cycle optimizado: nodos IoT AMI transmiten 1-2 segundos cada 15 minutos (sleep 99.8\% tiempo), resultando en consumo promedio ~15 µA para ambos protocolos. Diferencia 0.7 mA durante RX representa <1\% del battery budget total.
    
    \item \textbf{Menor madurez ecosystem:} Zigbee cuenta con 15+ años de deployments (Landis+Gyr, Itron meters certificados), mientras Thread es emergente (2019 specification 1.2, 2023 Matter launch). Risk mitigado por: (a) Thread Group con 300+ members (Google, Apple, Amazon, Siemens, Schneider Electric), (b) Chipsets comerciales disponibles (Nordic nRF52/nRF53, Espressif ESP32-C6/H2, Silicon Labs EFR32), (c) Certification program operativo (Thread Certified Product Registry con 200+ devices).
\end{itemize}

\textbf{Conclusión:} Aceptamos trade-off menor madurez + consumo marginal superior a cambio de architecture simplicity (IPv6 native, no translation layer) + IEEE 2030.5 seamless integration + Matter future-proofing. Decisión validada por industry trend: utilities modernizando hacia IPv6-based AMI (Pacific Gas \& Electric, Duke Energy pilots Thread 2023-2024).

\subsection{Capa de Aplicación: CoAP}

CoAP (Constrained Application Protocol, RFC 7252) es un protocolo web RESTful optimizado para dispositivos IoT constrained, diseñado como alternativa ligera a HTTP~\cite{shahinzadehSmartHomeConnectivity2024,hossainComparativeStudyIoTCommunication2018}.

\textbf{Características Fundamentales:} Arquitectura RESTful (GET/POST/PUT/DELETE), transporte UDP (8 bytes vs 20+ TCP), header compacto 4 bytes, mensajes binarios, modos CON/NON confirmable, extensión Observe (RFC 7641) para subscripciones push, Block-wise Transfer (RFC 7959) para firmware OTA, y DTLS integrado para seguridad.

\textbf{Estructura de Mensaje:} Header fijo 4 bytes contiene: Version (2 bits, siempre 01 para CoAP/1), Type (2 bits: CON/NON/ACK/RST), Token Length (4 bits, 0-8 bytes para correlación request/response), Code (8 bits: método 0.01=GET, 0.02=POST, 0.03=PUT, 0.04=DELETE, o respuesta 2.05=Content, 4.04=Not Found), Message ID (16 bits, detección de duplicados). Seguido de Token, Options y Payload.

\begin{table}[h]
\centering
\small
\caption{Comparación overhead CoAP/UDP vs HTTP/TCP para dispositivos IoT constrained (clase 1: ~10 KB RAM, ~100 KB Flash). Reducción: header 4 bytes vs 100-500 bytes HTTP (96\%), eliminación de latencia TCP 3-way handshake (0 ms vs 50-150 ms). Contexto: petición GET típica de sensor con 8 bytes payload.}
\label{tab:coap-vs-http}
\begin{tabular}{|p{3.5cm}|p{4cm}|p{4cm}|}
\hline
\rowcolor{gray!20}
\textbf{Característica} & \textbf{CoAP/UDP} & \textbf{HTTP/TCP} \\
\hline
\textbf{Header mínimo} & \textcolor{blue}{4 bytes} & \textcolor{red}{100+ bytes (típico 200-500)} \\
\hline
\textbf{Transporte} & \textcolor{blue}{UDP (8 bytes)} & \textcolor{red}{TCP (20 bytes + handshake)} \\
\hline
\textbf{Overhead total} & \textcolor{blue}{12-30 bytes} & \textcolor{red}{120-520 bytes} \\
\hline
\textbf{Latencia conexión} & \textcolor{green}{0 ms (stateless)} & \textcolor{red}{50-150 ms (3-way handshake)} \\
\hline
\textbf{Formato} & \textcolor{blue}{Binario (parsing rápido)} & \textcolor{red}{Texto (parsing lento)} \\
\hline
\textbf{Subscripciones} & \textcolor{blue}{Observe (push nativo)} & \textcolor{red}{Polling o WebSocket} \\
\hline
\textbf{Fragmentación} & \textcolor{blue}{Block-wise (CoAP-aware)} & \textcolor{red}{TCP segmentation (opaco)} \\
\hline
\textbf{Multicast} & \textcolor{blue}{Sí (UDP nativo)} & \textcolor{red}{No (TCP unicast only)} \\
\hline
\textbf{Seguridad} & \textcolor{blue}{DTLS (menor overhead)} & \textcolor{red}{TLS (mayor overhead)} \\
\hline
\end{tabular}
\end{table}

\textbf{Ejemplo:} GET CoAP 18 bytes total vs HTTP 120+ bytes (6.7$\times$ overhead).

\textbf{Modos de Confiabilidad:} \textbf{(1) Confirmable (CON)} requiere ACK con retransmisiones exponenciales (usado para comandos críticos, firmware OTA), \textbf{(2) Non-Confirmable (NON)} es fire-and-forget sin ACK (telemetría periódica donde pérdida ocasional es aceptable, reduciendo overhead de red).

\textbf{Observe - Subscripciones CoAP:} RFC 7641 define extensión Observe para subscripciones a recursos, eliminando polling mediante notificaciones push cuando hay cambios. Reduce tráfico 90-95\% vs polling HTTP~\cite{amezvaValdDesignImplementation2024}, latencia <50 ms (vs 0.5×polling\_interval promedio), menor consumo energético (no requiere wake-up periódico).

\subsection{Gestión de Dispositivos: LwM2M}

LwM2M (Lightweight Machine-to-Machine, OMA SpecWorks) es un protocolo de gestión de dispositivos IoT que construye sobre CoAP para proveer aprovisionamiento, configuración, monitoreo, actualización firmware y diagnóstico remoto~\cite{haEnablingDynamicLightweight2018,shahinzadehSmartHomeConnectivity2024,karimiIIoTCommunicationProtocols2025,omaLwM2MGuidelinesv2}.

\textbf{Arquitectura:} LwM2M Client (dispositivo implementa objetos), LwM2M Server (gestiona flota, operaciones CRUD), Bootstrap Server (opcional, provisiona credenciales). Modelo de objetos jerárquico 3 niveles: Object (funcionalidad, ej. Object 3=Device Info), Object Instance (instancia específica), Resource (dato individual). Notación: /ObjectID/InstanceID/ResourceID (ej. /10243/0/6 = Single-Phase Power Meter / Instance 0 / Active Power).

\textbf{Objetos Estándar para Smart Energy:}

\begin{table}[h]
\centering
\small
\caption{Objetos LwM2M relevantes para Smart Energy IoT: Objects 0-5 (core), IPSO Smart Objects 3303-3331 (sensores/actuadores)}
\label{tab:lwm2m-objects}
\begin{tabular}{|p{1.5cm}|p{3.5cm}|p{7cm}|}
\hline
\rowcolor{gray!20}
\textbf{Object ID} & \textbf{Nombre} & \textbf{Recursos Clave} \\
\hline
\textcolor{blue}{0} & \textcolor{blue}{Security} & Server URI (0), Bootstrap (1), Security Mode (2), Public Key (3), Secret Key (5) \\
\hline
\textcolor{blue}{1} & \textcolor{blue}{Server} & Lifetime (1), Min Period (2), Max Period (3), Disable (4), Notification Storing (6) \\
\hline
3 & Device & Manufacturer (0), Model (1), Serial (2), Firmware Ver (3), Reboot (4), Battery (9) \\
\hline
4 & Connectivity Monitoring & Network Bearer (0), RSSI (2), Link Quality (3), IP Addresses (4) \\
\hline
5 & Firmware Update & Package (0), Package URI (1), Update (2), State (3), Update Result (5) \\
\hline
\textcolor{green}{10243} & \textcolor{green}{Single-Phase Power Meter} & Active Power (6), Reactive Power (7), Active Energy (14) \\
\hline
\textcolor{green}{3305} & \textcolor{green}{Power Measurement} & Instantaneous Active Power (5800), Active Energy (5805), Reactive Energy (5810) \\
\hline
\textcolor{green}{3331} & \textcolor{green}{Voltage Measurement} & Sensor Value (5700), Min/Max (5601/5602), Application Type (5750) \\
\hline
\end{tabular}
\end{table}

\textbf{Operaciones LwM2M:} \textbf{(1) Read} (leer recurso /10243/0/6), \textbf{(2) Write} (actualizar configuración /1/0/2), \textbf{(3) Execute} (reiniciar /3/0/4), \textbf{(4) Create/Delete} (gestión de instancias), \textbf{(5) Observe} (subscribirse a cambios), \textbf{(6) Discover} (listar objetos soportados), \textbf{(7) Write-Attributes} (configurar pmin, pmax, gt, lt thresholds).

\textbf{Observe y Notificaciones:} Utiliza CoAP Observe (RFC 7641) con atributos avanzados: \textbf{pmin} (intervalo mínimo, evita flooding), \textbf{pmax} (intervalo máximo, garantiza heartbeat), \textbf{gt/lt} (umbrales superior/inferior), \textbf{st} (cambio mínimo). Ejemplo: \texttt{pmin=60\&pmax=900\&st=50} notifica solo si potencia cambia >50W, mínimo cada 60s, máximo cada 15 min. Reduce tráfico >80\% vs polling periódico.

\textbf{Firmware Update OTA:} Object 5 estandariza actualización remota: Server escribe URI (/5/0/1), ejecuta Update (/5/0/2), cliente descarga en background reportando progreso (State: Idle/Downloading/Downloaded/Updating), verifica firma digital, actualiza, reporta resultado (Success/Error), reinicia.

\textbf{Bindings de Transporte:}

\begin{table}[h]
\small
\centering
\caption{Bindings LwM2M: UDP (U) preferido Thread/HaLow por overhead mínimo, TCP (T) para LTE con NAT traversal, MQTT (Q) para brokers existentes~\cite{bartoliAllianceSDNMQTT2025}}
\label{tab:lwm2m-bindings}
\begin{tabular}{p{1.5cm}p{3.5cm}p{3.8cm}p{3cm}}
\hline
\rowcolor{gray!20}
\textbf{Binding} & \textbf{Transporte} & \textbf{Seguridad} & \textbf{Uso Smart Energy} \\
\hline
\textcolor{blue}{U} & \textcolor{blue}{UDP + CoAP} & DTLS + PSK/Certs & \textcolor{green}{Thread, HaLow, WiFi} \\
\hline
T & TCP + CoAP & TLS + PSK/Certs & LTE Cat-M1, NB-IoT \\
\hline
S & SMS & SMS encryption & Fallback NB-IoT \\
\hline
Q & MQTT & TLS + MQTT auth & Brokers existentes \\
\hline
\end{tabular}
\end{table}

\textbf{Seguridad:} Modos: \textbf{PSK} (clave simétrica 128-256 bits, overhead mínimo ~16 bytes, handshake ~200 bytes), \textbf{RPK} (claves públicas ECC sin X.509), \textbf{Certificate} (PKI completa, overhead ~2 KB), \textbf{NoSec} (testing). PSK preconfigurado reduce overhead 60\% vs TLS/TCP (15 bytes/msg vs 40+ bytes)~\cite{bahardimanSecureHybridGateway2024}.

\textbf{LwM2M vs Soluciones Alternativas:}

\begin{table}[H]
\centering
\caption{Comparación LwM2M vs protocolos alternativos gestión dispositivos Smart Energy: overhead, interoperabilidad, eficiencia energética}
\label{tab:lwm2m-comparison}
\resizebox{\textwidth}{!}{%
\begin{tabular}{|>{\centering\arraybackslash}p{2.8cm}|>{\centering\arraybackslash}p{3.2cm}|>{\centering\arraybackslash}p{3.2cm}|>{\centering\arraybackslash}p{2.8cm}|>{\centering\arraybackslash}p{2.8cm}|}
\hline
\rowcolor{blue!20}
\textbf{Característica} & \textbf{LwM2M 1.2} & \textbf{MQTT + JSON} & \textbf{HTTP REST} & \textbf{TR-069 CWMP} \\
\hline
\textbf{Overhead típico} & \textcolor{green}{\textbf{20-40 bytes}} & \textcolor{orange}{100-300 bytes} & \textcolor{red}{200-500 bytes} & \textcolor{red}{500-1500 bytes} \\
\hline
\textbf{Gestión dispositivos} & \textcolor{green}{\textbf{Nativa}} (objects std) & \textcolor{orange}{Custom} (topics) & \textcolor{orange}{Custom} (endpoints) & \textcolor{blue}{CPE WAN} (telco) \\
\hline
\textbf{Firmware OTA} & \textcolor{green}{\textbf{Estandarizado}} (Obj 5) & \textcolor{orange}{Custom impl} & \textcolor{orange}{Custom impl} & \textcolor{blue}{Download + Install} \\
\hline
\textbf{Observe/Subscribe} & \textcolor{green}{\textbf{Nativo + thresholds}} & \textcolor{blue}{MQTT native}~\cite{amiriDeploymentArchitecturesMQTT2024} & \textcolor{orange}{Polling o SSE} & \textcolor{orange}{Notification} \\
\hline
\textbf{Seguridad} & \textcolor{green}{\textbf{DTLS-PSK}} (ligero) & \textcolor{orange}{TLS} (pesado) & \textcolor{orange}{TLS} (pesado) & \textcolor{red}{SOAP/TLS} (muy pesado) \\
\hline
\textbf{Interoperabilidad} & \textcolor{green}{\textbf{Multi-vendor}} (OMA) & \textcolor{red}{Propietario} & \textcolor{red}{Propietario} & \textcolor{blue}{Broadband Forum} \\
\hline
\textbf{Eficiencia energética} & \textcolor{green}{\textbf{Excelente}} (PSM) & \textcolor{blue}{Buena} (keepalive) & \textcolor{orange}{Regular} (polling) & \textcolor{red}{Pobre} (XML) \\
\hline
\textbf{Aplicabilidad tesis} & \textcolor{green}{\textbf{Alta}} - Protocolo principal & \textcolor{blue}{Media} - Gateway-cloud & \textcolor{orange}{Baja} - APIs legacy & \textcolor{red}{Nula} \\
\hline
\end{tabular}%
}
\end{table}

\textbf{Ventajas sobre Gestión Propietaria:} Reduce time-to-market evitando desarrollo custom, interoperabilidad multi-vendor (un gateway gestiona múltiples fabricantes), estandariza operaciones comunes (device info, connectivity monitoring, firmware update), notificaciones con thresholds complejos reducen tráfico adicional 80-90\%, DTLS-PSK con overhead 60\% menor que TLS/TCP crítico para battery-powered devices.

\subsection{Síntesis del Stack y Transición a Edge Computing}

El stack 6LoWPAN/CoAP/LwM2M descrito en esta sección proporciona la base técnica para comunicación eficiente en redes IoT constrained. La compresión IPHC+NHC reduce overhead de headers 80-90\% (48 bytes → 6-11 bytes)~\cite{abood6LoWPANTechnicalFeatures2024}, permitiendo payload útil >75\% del MTU 802.15.4. CoAP elimina overhead de HTTP/TCP (4 bytes vs 100-500 bytes, latencia 0 ms vs 50-150 ms handshake)~\cite{amezvaValdDesignImplementation2024}, mientras LwM2M estandariza gestión de dispositivos multi-vendor con overhead 60\% menor que TLS/TCP~\cite{bahardimanSecureHybridGateway2024}.

Este stack optimizado para redes de sensores wireless (Thread, HaLow) requiere integración con tecnologías de edge computing para procesamiento local, almacenamiento de series temporales, orquestación de servicios y conectividad backhaul a cloud. Antes de describir las tecnologías edge, se analizan dos implementaciones comerciales representativas del stack IoT: Wi-SUN FAN (Sub-GHz para utilities) y Thread (2.4 GHz para IoT moderno).

\subsection{Stacks Comerciales: Wi-SUN (Sub-GHz) y Thread (2.4 GHz)}

El stack 6LoWPAN/CoAP/LwM2M descrito puede implementarse sobre múltiples PHY/MAC layers, siendo Wi-SUN FAN y Thread las dos principales alternativas comerciales con ecosistemas maduros. Esta subsección compara ambas tecnologías como representantes de stacks Sub-GHz (utilities-oriented) y 2.4 GHz (IoT general-purpose).

\subsubsection{Wi-SUN FAN: Stack Sub-GHz para Utilities}

Wi-SUN (Wireless Smart Utility Network) FAN (Field Area Network) es un estándar IEEE 802.15.4g específicamente diseñado para redes de servicios públicos (smart grid, smart metering), operando en bandas Sub-1 GHz con mesh IPv6 sobre 6LoWPAN y routing RPL~\cite{tiCC1312RWiSUN2024}.

\textbf{Arquitectura del Stack Wi-SUN:}
\begin{itemize}
    \item \textbf{PHY:} IEEE 802.15.4g Sun (Smart Utility Networks) operando en 863-870 MHz (Europa), 902-928 MHz (Américas), 915-928 MHz (Japón). Modulación MR-FSK (Multi-Rate Frequency Shift Keying) con data rates configurables: 50, 100, 150, 200, 300 kbps según trade-off sensibilidad/throughput.
    \item \textbf{MAC:} CSMA/CA con frequency hopping adaptativo para resiliencia ante interferencia. Channel sequences más complejas en FAN 1.2 vs 1.1.
    \item \textbf{Network:} 6LoWPAN IPHC compression + RPL (RFC 6550) routing. DODAG (Destination Oriented Directed Acyclic Graph) construction con DIO/DAO/DIS messages. Soporte multicast mejorado en FAN 1.2.
    \item \textbf{Application:} CoAP + LwM2M compatible (mismo stack aplicación que Thread).
\end{itemize}

\textbf{Ventajas Sub-GHz:} Propagación superior (free-space path loss 15-20 dB menor vs 2.4 GHz), penetración mejorada (atenuación paredes concreto 8-12 dB menor), alcance extendido 2-5 km outdoor vs 250-500 m Thread indoor. Ideal para utility-scale outdoor deployments con nodos dispersos.

\textbf{Chipset Ecosystem:} Texas Instruments CC1312R representa chipset Wi-SUN certificado líder: ARM Cortex-M4F @ 48 MHz, 352 KB Flash, 88 KB RAM, radio Sub-1 GHz multibanda (143-1315 MHz). TX power +14 dBm, sensitivity -121 dBm (50 kbps mode). Precio \$2.70-4.35 USD (1ku), 25-30\% más barato que Thread nRF52840 (\$4-5 USD)~\cite{tiCC1312RWiSUN2024}. Development kit LAUNCHXL-CC1312R1 (\$39.99) con Wi-SUN FAN Stack SDK (TI SimpleLink SWRU615).

\textbf{Deployments Globales:} Millones de smart meters Wi-SUN deployed en utilities japonesas (Tokyo Electric Power), europeas (varios DSOs), y norteamericanas. Ecosystem maduro con 15+ años de evolución desde especificación original.

\textbf{Limitaciones:} Throughput máximo 300 kbps insuficiente para agregación masiva de backhaul (vs HaLow 40 Mbps). Routing RPL más complejo que Thread MLE (separate upward/downward tables, 4-8 KB RAM adicional). Border router requiere vendor SDKs propietarios (TI binary con kernel drivers) vs Thread OTBR open-source Docker. Ecosistema Matter no incluye Wi-SUN (Matter 1.0/1.1/1.2 mandate Thread, no 802.15.4g support), limitando convergencia IoT moderno.

\subsubsection{Thread: Stack 2.4 GHz para IoT General-Purpose}

Thread 1.3.1 (Thread Group 2024) implementa el mismo stack 6LoWPAN/CoAP/LwM2M sobre IEEE 802.15.4 @ 2.4 GHz, optimizado para smart home/building automation con integración Matter nativa~\cite{threadMatterConvergence2024}.

\textbf{Arquitectura del Stack Thread (ya descrita en §2.2.3):}
\begin{itemize}
    \item \textbf{PHY:} IEEE 802.15.4 @ 2.4 GHz (canales 11-26), throughput 250 kbps. Mismo PHY que Zigbee pero stack network diferente.
    \item \textbf{MAC:} CSMA/CA standard 802.15.4.
    \item \textbf{Network:} 6LoWPAN IPHC + MLE (Mesh Link Establishment) routing proactivo. Single unified routing table basada en ETX metric, convergencia <5 s ante fallas.
    \item \textbf{Application:} CoAP + LwM2M + Matter application layer.
\end{itemize}

\textbf{Ventajas 2.4 GHz:} Spectrum disponibilidad global ISM band, chipsets commoditized (Nordic, Espressif, Silicon Labs), throughput suficiente 250 kbps para mesh local. Integration simplicity: OTBR (OpenThread Border Router) open-source con Docker deployment vs Wi-SUN vendor-specific routers.

\textbf{Chipset Ecosystem:} Nordic nRF52840 domina (75\% market share Thread): ARM Cortex-M4F @ 64 MHz, 1 MB Flash, 256 KB RAM, radio 2.4 GHz multiprotocol (Thread, Zigbee, BLE 5.3 simultáneo). TX power +8 dBm, sensitivity -95 dBm. Precio \$4-5 USD (1ku). USB Dongle nRF52840 (\$10) como Thread RCP (Radio Co-Processor) para OTBR. Development kit nRF52840 DK (\$39).

\textbf{Matter Convergence:} Thread adoptado como mandatory transport layer en Matter specification (CSA 2022), señalando dirección ecosistema IoT moderno. Google Home, Apple HomeKit, Amazon Alexa utilizan Thread/Matter. 300+ Thread Group members, 200+ Thread Certified devices.

\textbf{Limitaciones:} Alcance limitado 250-500 m indoor (vs 2-5 km Wi-SUN outdoor). Propagación 2.4 GHz inferior a Sub-GHz (path loss +15-20 dB). Banda ISM saturada en entornos urbanos (Wi-Fi, Bluetooth, microondas). No optimizado para utility-scale outdoor como Wi-SUN.

\subsubsection{Comparación Stack Wi-SUN vs Thread}

\begin{table}[h]
\centering
\small
\caption{Comparación stacks comerciales Wi-SUN FAN 1.2 (Sub-GHz utilities) vs Thread 1.3.1 (2.4 GHz IoT). Ambos implementan 6LoWPAN/CoAP/LwM2M pero difieren en PHY, routing, y ecosistema target.}
\label{tab:wisun-vs-thread-stacks}
\begin{tabular}{|p{3cm}|p{5cm}|p{5cm}|}
\hline
\rowcolor{gray!20}
\textbf{Criterio} & \textbf{Wi-SUN FAN 1.2} & \textbf{Thread 1.3.1} \\
\hline
\textbf{PHY Standard} & IEEE 802.15.4g Sub-1 GHz & IEEE 802.15.4 @ 2.4 GHz \\
\hline
\textbf{Frecuencia} & 868/915 MHz (regional) & 2400-2483 MHz (global ISM) \\
\hline
\textbf{Throughput} & 50-300 kbps (MR-FSK modes) & 250 kbps (O-QPSK) \\
\hline
\textbf{Alcance típico} & \textcolor{green}{2-5 km outdoor LOS} & \textcolor{orange}{250-500 m indoor NLOS} \\
\hline
\textbf{Propagación} & \textcolor{green}{Excelente Sub-1 GHz} & \textcolor{orange}{Regular 2.4 GHz} \\
\hline
\textbf{IPv6 Layer} & \textcolor{blue}{6LoWPAN IPHC (idéntico)} & \textcolor{blue}{6LoWPAN IPHC (idéntico)} \\
\hline
\textbf{Routing} & RPL DODAG (RFC 6550) & MLE proactive (Thread spec) \\
\hline
\textbf{Application} & \textcolor{blue}{CoAP + LwM2M (compatible)} & \textcolor{blue}{CoAP + LwM2M (compatible)} \\
\hline
\textbf{Border Router} & Vendor SDK (TI proprietary) & \textcolor{green}{OTBR open-source Docker} \\
\hline
\textbf{Chipset líder} & TI CC1312R (\textcolor{green}{\$2.70}) & Nordic nRF52840 (\$4) \\
\hline
\textbf{Ecosystem target} & Utilities (smart grid, metering) & IoT general (smart home, building) \\
\hline
\textbf{Matter support} & \textcolor{red}{No (sin roadmap)} & \textcolor{green}{Sí (mandatory layer)} \\
\hline
\textbf{Deployments} & \textcolor{green}{Millones meters utilities} & \textcolor{orange}{Cientos miles devices IoT} \\
\hline
\textbf{Madurez} & \textcolor{green}{15+ años (desde 2009)} & \textcolor{orange}{5+ años (desde 2019)} \\
\hline
\end{tabular}
\end{table}

\textbf{Decisión Arquitectónica para esta Tesis:} Thread seleccionado para mesh local (250 m DCU coverage suficiente, Matter future-proofing, OTBR integration simplicity), combinado con HaLow 802.11ah @ 900 MHz para backhaul (throughput 40 Mbps + alcance Sub-GHz). Esta arquitectura dual-radio evita conflicto espectral Sub-1 GHz (Wi-SUN 868/915 MHz interferiría con HaLow 902-928 MHz), mientras provee \textbf{separación espectral clara}: 2.4 GHz local + 900 MHz backhaul. Trade-off aceptado: renunciamos a propagación superior Wi-SUN Sub-GHz para mesh local (Thread suficiente para 250 m indoor/outdoor mixto), a cambio de throughput agregación crítico HaLow (40 Mbps vs 300 kbps Wi-SUN = 133× diferencia) y ecosystem convergence Matter.

\textbf{Validación Industria:} Radiocrafts (vendor noruego módulos RF) reporta migración clientes DESDE ZigBee/Wi-SUN/LoRa HACIA mesh propietario RIIM para solar tracking, citando limitaciones throughput Wi-SUN (1000-2000 nodos antes de fragmentación) vs alternativas higher-capacity~\cite{radiocraftsCommissioningLargeSolar2025}. Pattern replica trend utilities modernizando hacia Thread/propietarios + backhauls RF de mayor capacidad (HaLow, LTE-M, 5G RedCap).

\section{Arquitectura de Edge Computing para Smart Energy}

El stack de protocolos IoT descrito en §2.2 (IEEE 802.15.4 → 6LoWPAN → CoAP → LwM2M) permite comunicación eficiente entre dispositivos constrained y gateways edge. Sin embargo, la arquitectura Smart Energy completa requiere capacidades adicionales en el gateway: orquestación de servicios heterogéneos, almacenamiento persistente de series temporales, procesamiento de reglas CEP (Complex Event Processing), y conectividad resiliente hacia cloud. Esta sección describe las tecnologías de edge computing (Docker, TimescaleDB, Kafka, ThingsBoard Edge) que implementan estas capacidades, transformando el gateway en una plataforma de agregación y procesamiento intermedio entre redes IoT locales y sistemas backend cloud.

\subsection{Containerización con Docker}

Docker es una plataforma de containerización que encapsula aplicaciones y sus dependencias en imágenes portables, aisladas mediante namespaces y cgroups del kernel Linux~\cite{liangReviewEdgeComputing2024,boonmeerukCostEffectiveIIoTGateway2024}.

\subsubsection{Fundamentos de Containers}

Un container Docker ejecuta procesos en espacio de usuario aislado, compartiendo el kernel del host pero con~\cite{madsenCosteffectiveEdgeComputing2024}:

\begin{itemize}
\item \textbf{PID namespace}: Cada container ve su propia jerarquía de procesos (PID 1 = init del container).
\item \textbf{Network namespace}: Stack de red independiente (interfaces, routing table, firewall rules).
\item \textbf{Mount namespace}: Filesystem root independiente (union filesystem overlay2/aufs).
\item \textbf{IPC namespace}: Colas de mensajes System V aisladas.
\item \textbf{UTS namespace}: Hostname independiente.
\end{itemize}

Cgroups (Control Groups) limitan recursos:
\begin{itemize}
\item \textbf{cpu.cfs\_quota\_us}: CPU time limit (ej. 100000 = 1 CPU core).
\item \textbf{memory.limit\_in\_bytes}: RAM limit (ej. 2 GB).
\item \textbf{blkio.throttle}: I/O bandwidth throttling.
\end{itemize}

\subsubsection{Docker Compose para Orquestación}

Docker Compose define stacks multi-container mediante archivos YAML declarativos. Ejemplo simplificado:

\begin{verbatim}
version: '3.8'
services:
  thingsboard:
    image: thingsboard/tb-edge:3.6.0
    ports:
      - "8080:8080"
    environment:
      - SPRING_DATASOURCE_URL=jdbc:postgresql://postgres:5432/thingsboard
    depends_on:
      - postgres
    restart: unless-stopped
    deploy:
      resources:
        limits:
          cpus: '3'
          memory: 4G
\end{verbatim}

Health checks con restart policies garantizan resiliencia ante fallas transitorias.

\subsection{Time-Series Databases - TimescaleDB}

TimescaleDB es una extensión de PostgreSQL optimizada para series temporales, implementando hypertables (particionado automático por tiempo), continuous aggregates (materialización de queries agregadas), y compresión columnar.

\subsubsection{Optimizaciones para Series Temporales}

\textbf{1. Hypertables:} Una hypertable se particiona automáticamente en chunks basados en columna de tiempo:

\begin{verbatim}
CREATE TABLE telemetry (
  time TIMESTAMPTZ NOT NULL,
  device_id UUID NOT NULL,
  metric TEXT NOT NULL,
  value DOUBLE PRECISION
);

SELECT create_hypertable('telemetry', 'time', chunk_time_interval => INTERVAL '1 day');
\end{verbatim}

Cada chunk es una tabla PostgreSQL estándar. Queries se optimizan mediante constraint exclusion (solo escanea chunks relevantes).

\textbf{2. Continuous Aggregates:} Precomputación de agregaciones (ej. promedio horario) con actualización incremental:

\begin{verbatim}
CREATE MATERIALIZED VIEW telemetry_hourly
WITH (timescaledb.continuous) AS
SELECT time_bucket('1 hour', time) AS bucket,
       device_id,
       metric,
       AVG(value) AS avg_value
FROM telemetry
GROUP BY bucket, device_id, metric;
\end{verbatim}

\textbf{3. Compresión:} Columnar compression de chunks antiguos reduce storage 90-95\%:

\begin{verbatim}
ALTER TABLE telemetry SET (
  timescaledb.compress,
  timescaledb.compress_segmentby = 'device_id,metric',
  timescaledb.compress_orderby = 'time'
);

SELECT add_compression_policy('telemetry', INTERVAL '7 days');
\end{verbatim}

\subsection{Message Brokers - Apache Kafka}

Apache Kafka es un sistema de streaming distribuido que funciona como log commit distribuido, proporcionando alta throughput (millones mensajes/seg), persistencia durable, y procesamiento de streams.

\subsubsection{Arquitectura de Kafka}

\textbf{Componentes clave:} Topics (canales lógicos), Partitions (paralelismo), Brokers (almacenamiento), Producers/Consumers (pub/sub). Garantías: \texttt{acks=0} (fire-and-forget), \texttt{acks=1} (leader ACK), \texttt{acks=all} (replicas in-sync, máxima durabilidad).

\subsubsection{Kafka en Edge Gateways}

En edge gateways, Kafka proporciona buffer persistente de telemetría durante particiones WAN~\cite{amiriDeploymentArchitecturesMQTT2024}:

\begin{enumerate}
\item Nodos IoT publican vía MQTT → MQTT bridge → Kafka topic local
\item Kafka consumer local almacena en TimescaleDB
\item Kafka Mirror Maker replica hacia Kafka cloud (sync bidireccional)
\end{enumerate}

Configuración optimizada para embedded:
\begin{itemize}
\item \texttt{log.retention.bytes=1GB} (limit total storage)
\item \texttt{log.segment.bytes=100MB} (smaller segments)
  \item \texttt{num.io.threads=4} (reduce CPU overhead)
\end{itemize}

\subsection{Plataforma IoT Edge - ThingsBoard}

ThingsBoard es una plataforma IoT open-source (Apache 2.0) que proporciona device management, data collection, procesamiento (rule engine), visualización (dashboards), y APIs programáticas~\cite{gartnerMagicQuadrantIoT2024}. Arquitectura microservices en Java/Spring Boot.Componentes principales:
\begin{itemize}
\item \textbf{Transport Layer}: MQTT, CoAP, HTTP, LwM2M servers.
\item \textbf{Core Services}: Device registry, telemetry persistence, rule engine.
\item \textbf{Database}: PostgreSQL (metadata) + Cassandra/TimescaleDB (telemetry).
\item \textbf{Message Queue}: Kafka (inter-service communication).
  \item \textbf{Web UI}: Angular dashboard con widgets configurables.
\end{itemize}

\subsubsection{ThingsBoard Edge}

ThingsBoard Edge es una distribución edge-optimized que replica funcionalidad completa de ThingsBoard en gateways locales, con sincronización bidireccional hacia instancia cloud.Capacidades clave:
\begin{itemize}
\item \textbf{Local dashboards}: Full-featured UI accesible durante offline.
\item \textbf{Rule chains locales}: Procesamiento CEP (Complex Event Processing) sin round-trip cloud.
\item \textbf{Buffering automático}: Cola persistente de eventos no sincronizados.
\item \textbf{Asset/Device sync}: Replicación de definiciones de dispositivos, atributos, relaciones.
\end{itemize}

Sincronización: protocolo gRPC bidireccional con batching y compresión (Snappy).

\subsubsection{Modelado de Latencia End-to-End mediante Teoría de Colas}

Para estimar latencias en arquitecturas edge vs cloud, aplicamos teoría de colas M/M/1 (arribos Poisson, servicio exponencial, 1 servidor).

\paragraph{Sistema M/M/1 para Gateway de Borde}

Variables:
\begin{itemize}
\item $\lambda$: Tasa de arribos de mensajes (mensajes/seg)
\item $\mu$: Tasa de servicio del gateway (mensajes/seg)
\item $\rho = \lambda / \mu$: Utilización del servidor ($\rho < 1$ para estabilidad)
\end{itemize}

Tiempo promedio en sistema (queuing + servicio):
\begin{equation}
W = \frac{1}{\mu - \lambda}
\end{equation}

Ejemplo: Gateway procesa $\mu = 100$ msg/s, carga $\lambda = 70$ msg/s:
\begin{equation}
W = \frac{1}{100 - 70} = 0.0333 \text{ s} = 33.3 \text{ ms}
\end{equation}

Tiempo en cola (solo waiting):
\begin{equation}
W_q = \frac{\rho}{\mu - \lambda} = \frac{0.7}{30} = 23.3 \text{ ms}
\end{equation}

Latencia total end-to-end (device → storage):
\begin{equation}
L_{total} = L_{device \rightarrow GW} + W_{GW} + L_{GW \rightarrow DB}
\end{equation}

Para arquitectura edge:
\begin{equation}
L_{edge} = 40 \text{ ms (Thread)} + 33 \text{ ms (GW queue)} + 8 \text{ ms (TimescaleDB write)} = 81 \text{ ms}
\end{equation}

Para arquitectura cloud-centric:
\begin{equation}
L_{cloud} = 40 + 33 + 80 \text{ (LTE RTT)} + 50 \text{ (WAN)} + 30 \text{ (cloud ingestion)} + 10 \text{ (RDS write)} = 243 \text{ ms}
\end{equation}

Reducción: $(243-81)/243 = 66.7\%$

\subsection{Síntesis de Edge Computing y Transición a Seguridad}

Las tecnologías de edge computing descritas (Docker para orquestación, TimescaleDB para series temporales, Kafka para buffering, ThingsBoard Edge para gestión/visualización) transforman el gateway en una plataforma integral de procesamiento intermedio. La containerización mediante Docker proporciona aislamiento y portabilidad de servicios heterogéneos (rule engines, bridges protocolares, APIs). TimescaleDB optimiza almacenamiento y consultas de telemetría mediante hypertables particionadas temporalmente y continuous aggregates, reduciendo carga en bases de datos cloud. Kafka actúa como buffer persistente durante particiones WAN, garantizando entrega confiable mediante replicación. ThingsBoard Edge replica funcionalidad completa de gestión/visualización en el edge, con sincronización bidireccional gRPC hacia cloud, permitiendo operación autónoma durante offline.

La reducción de latencia end-to-end de 66.7\% (243 ms cloud-centric → 81 ms edge, según modelado M/M/1) es crítica para aplicaciones Smart Energy con requisitos <100 ms (protecciones, DR). Sin embargo, esta arquitectura distribuida introduce superficie de ataque ampliada: dispositivos IoT constrained, interfaces wireless expuestas, servicios containerizados, APIs programáticas. La siguiente sección (§2.4) analiza las amenazas específicas de sistemas IoT y estrategias de defensa en profundidad (defence in depth) aplicables a gateways edge, vinculando contramedidas de seguridad con las capas del stack de protocolos IoT descrito en §2.2.

\section{Seguridad en Sistemas IoT}

\subsection{Amenazas Específicas de IoT}

Los sistemas IoT presentan superficie de ataque ampliada respecto a IT tradicional~\cite{BlockchainBasedSecureAuthentication2025,nandalSECURITYRISKSIoT2025}:

\begin{enumerate}
\item \textbf{Compromise de dispositivos}: Dispositivos resource-constrained son vulnerables a ataques de firmware (ej. Mirai botnet)~\cite{huddaReviewWSNBased2025}.
\item \textbf{Man-in-the-Middle (MitM)}: Intercepción de comunicaciones no cifradas (ej. MQTT sin TLS).
\item \textbf{Replay attacks}: Reenvío de mensajes legítimos capturados (mitigado con nonces/timestamps).
\item \textbf{Denial of Service (DoS)}: Inundación de gateways con tráfico malicioso.
\item \textbf{Escalation de privilegios}: Explotación de APIs sin RBAC adecuado.
\item \textbf{Data exfiltration}: Acceso no autorizado a datos de telemetría sensibles~\cite{thungonSurvey6LoWPANSecurity2024,pandeyRecentLightweightCryptography2024}.
\end{enumerate}

\subsection{Defence in Depth para Edge Gateways}

Estrategia de seguridad en capas~\cite{m.mijwilPostQuantumSecureBlockchainBased2025,ramakrishnaAnalysisLightweightCryptographic2024,iec62443-4-2}: (1) Física: Secure Boot, TPM, enclosure anti-tamper; (2) Red: Firewall nftables, VLANs (Management/IoT/Backhaul/WAN), WPA3-SAE, TLS 1.2/1.3 mutual auth; (3) Aplicación: RBAC ThingsBoard, input validation, rate limiting, logging centralizado; (4) Datos: cifrado at-rest LUKS, backup con GPG, anonymization de identificadores.

\subsection{Seguridad por Capa del Stack de Protocolos IoT}

La arquitectura del stack de protocolos IoT descrita en §2.2 requiere medidas de seguridad específicas en cada capa para mitigar amenazas particulares. La Tabla~\ref{tab:security-stack-mapping} mapea las amenazas identificadas en §2.4.1 con las capas del stack y sus contramedidas correspondientes, estableciendo un modelo de defensa integrado que vincula los aspectos de protocolo con los requisitos de seguridad.

\begin{table}[h]
\centering
\caption{Mapeo de Amenazas, Capas del Stack IoT y Contramedidas}
\label{tab:security-stack-mapping}
\begin{tabular}{|p{3cm}|p{2.5cm}|p{4cm}|p{4.5cm}|}
\hline
\textbf{Amenaza} & \textbf{Capa del Stack} & \textbf{Vector de Ataque} & \textbf{Contramedidas} \\
\hline
\textbf{MitM / Eavesdropping} & PHY/MAC (IEEE 802.15.4) & Intercepción RF, jamming selectivo & WPA3-SAE (Thread), AES-CCM-128, channel hopping, RSSI monitoring \\
\cline{2-4}
& 6LoWPAN & Fragmentación maliciosa, header manipulation & IPsec ESP (modo túnel), DTLS 1.3 con PSK \\
\cline{2-4}
& CoAP/LwM2M & Observación de recursos sin autenticación & DTLS-PSK, OSCORE (object security), certificate-based auth \\
\hline
\textbf{DoS / Resource Exhaustion} & PHY/MAC & Inundación de paquetes, depleción de batería & Rate limiting MAC, CAD (Channel Activity Detection), adaptive TX power \\
\cline{2-4}
& 6LoWPAN & Fragmentación abusiva (memory exhaustion) & Timeout de reassembly, límite de fragmentos pendientes \\
\cline{2-4}
& CoAP/Application & Flooding de CON requests & Exponential backoff, token validation, IP whitelisting \\
\hline
\textbf{Firmware Compromise} & Application & Actualización OTA maliciosa, bootloader unsigned & Secure Boot (UEFI/U-Boot), firma digital de imágenes (GPG/RSA-2048), rollback protection \\
\hline
\textbf{Replay Attacks} & CoAP/LwM2M & Reenvío de comandos legítimos capturados & Nonces criptográficos, timestamps con ventana 60s, sequence numbers monotónicos \\
\hline
\textbf{Privilege Escalation} & Application (ThingsBoard) & Explotación de APIs sin RBAC & Role-Based Access Control (RBAC), JWT con expiración <1h, audit logging \\
\hline
\textbf{Data Exfiltration} & Application / Gateway Edge & Acceso no autorizado a TSDB/Kafka & Cifrado at-rest (LUKS/dm-crypt), TLS mutual auth, VLANs de aislamiento, firewall nftables \\
\hline
\end{tabular}
\end{table}

Esta tabla establece la correspondencia directa entre el stack de protocolos (§2.2), las amenazas de seguridad (§2.4.1), y las tecnologías de implementación (§2.3). Por ejemplo, la contramedida DTLS-PSK para CoAP se ejecuta dentro del contenedor Docker del gateway edge (§2.3.1), mientras que el cifrado at-rest de TimescaleDB (§2.3.2) protege contra exfiltración de datos históricos. Esta integración vertical seguridad-stack-implementación es fundamental para el diseño del gateway multi-protocolo propuesto en el Capítulo 3.

\subsection{Síntesis de Seguridad y Transición al Estado del Arte}

Las tres subsecciones anteriores han establecido: (1) el catálogo de amenazas específicas de IoT en contexto Smart Energy (§2.4.1), (2) la estrategia de defence in depth aplicable a edge gateways (§2.4.2), y (3) el mapeo explícito de contramedidas por capa del stack de protocolos (§2.4.3). Esta base de seguridad es esencial para evaluar críticamente las soluciones existentes en el estado del arte.

La siguiente sección (§2.5) analiza trabajos relacionados y soluciones comerciales, evaluando específicamente: (a) qué amenazas mitigan efectivamente, (b) en qué capas del stack implementan seguridad, y (c) qué brechas persisten en términos de conformidad con estándares (IEEE 2030.5, ISO/IEC 30141) y capacidades de seguridad multi-capa. Este análisis comparativo revelará las limitaciones de los gateways IoT actuales que motivan la arquitectura propuesta en el Capítulo 3.

\section{Estado del Arte de Gateways IoT para Smart Energy}

\subsection{Gateways Multi-Protocolo Académicos}

Trabajos previos: (1) Raspberry Pi + Zigbee/Z-Wave/Wi-Fi (2019), sin IEEE 2030.5 ni failover WAN; (2) OTBR + LTE Cat-M1 (2021), sin HaLow ni ISO/IEC 30141; (3) LoRaWAN + Wi-Fi backhaul (2022), throughput insuficiente para OTA, latencia >1s~\cite{halowVsLoRaWANComparison2023}.

\subsection{Soluciones Comerciales}

Soluciones dominantes: Cisco IR829 (\$2.5-4k, LTE/Wi-Fi/Ethernet, sin Thread/HaLow, cerrada); Dell EG3000 (\$1.2-2k, x86/containers, 25-40W, sin IEEE 2030.5); MultiTech Conduit (\$400-800, LoRaWAN/LTE, CPU limitada 456 MHz, sin edge analytics).

\subsection{Análisis Comparativo}

\begin{table}[h]
\centering
\caption{Comparación Arquitecturas Edge Gateway}
\label{tab:edge-gateway-comparison}
\begin{tabular}{|p{3.5cm}|p{2.5cm}|p{2.5cm}|p{2.5cm}|p{2.5cm}|}
\hline
\textbf{Característica} & \textbf{Propuesta} & \textbf{Cisco IR829} & \textbf{Dell EG3000} & \textbf{MultiTech Conduit} \\
\hline
\textbf{Thread support} & Sí (OTBR) & No & No & No \\
\hline
\textbf{HaLow support} & Sí (MM6108) & No & No & No \\
\hline
\textbf{IEEE 2030.5} & Sí & No & No & No \\
\hline
\textbf{Edge platform} & ThingsBoard & No & EdgeX & Node-RED \\
\hline
\textbf{Containers} & Docker & No & Docker & Docker \\
\hline
\textbf{Costo aprox.} & \$600-800 & \$2,500+ & \$1,200+ & \$400-800 \\
\hline
\textbf{Open-source} & Sí & No & Parcial & Parcial \\
\hline
\end{tabular}
\end{table}

\subsection{Iniciativas Industriales y Consorcios de Estandarización}

Más allá de las implementaciones académicas y los productos comerciales individuales, existen múltiples consorcios industriales y organizaciones de estandarización que impulsan la adopción de tecnologías IoT en el sector energético. Estas iniciativas proporcionan marcos de interoperabilidad, certificaciones, casos de uso de referencia y ecosistemas de fabricantes que facilitan despliegues de gran escala.

\subsubsection{OpenADR Alliance}

\textbf{OpenADR Alliance} promueve IEEE 2030.5 para respuesta a demanda, con >150 miembros (PG\&E, Schneider Electric). Certificación garantiza interoperabilidad VTN/VEN. Casos documentados demuestran reducción pico 15-30\% (PG\&E California, SA Power Networks Australia).

\subsubsection{Thread Group y Matter}

\textbf{Thread Group} (Nest, ARM, Samsung) unido a Connectivity Standards Alliance desarrolló \textbf{Matter} (capa aplicación sobre Thread/Wi-Fi). Certificación Thread 1.3.1 valida interoperabilidad mesh heterogénea. Matter proporciona control unificado multi-ecosistema (Google/Apple/Amazon), emergente alternativa a IEEE 2030.5 para DR residencial.

\subsubsection{LoRa Alliance}

\textbf{LoRa Alliance} (>500 miembros) estandariza LoRaWAN (LPWAN largo alcance, bajo throughput). Certificación clases A/B/C, módulos certificados \$5-15 (Murata, RAKwireless). Despliegues: E.ON 20k+ medidores, Centrica 100k+ termostatos, SK Telecom cobertura nacional. Modelo aplicable a HaLow en espectro no licenciado.

\subsubsection{Wi-Fi Alliance - HaLow Marketing Task Group}

\textbf{Wi-Fi Alliance} estableció HaLow Marketing Task Group (2016) con fabricantes (Morse Micro, Newracom) y utilities. Certificación Wi-Fi CERTIFIED HaLow™ (2021) valida IEEE 802.11ah e interoperabilidad (bandwidth adaptativo 1/2/4/8 MHz, TWT/TIM, WPA3-SAE). Pilotos documentados en Smart Energy (subestaciones, gateways), agricultura, ciudades inteligentes demuestran throughput superior y latencia determinista vs LoRaWAN.

\subsubsection{Caso de Estudio Real: Despliegue HaLow en Victoria, Australia}

Un despliegue comercial documentado en Victoria, Australia (2023) valida la viabilidad de HaLow en condiciones operacionales desafiantes~\cite{halownetworkPreventingLivestockTheft2023}. El caso involucró vigilancia agrícola en una finca remota con terreno montañoso y múltiples puntos de monitoreo distribuidos. La solución HaLow superó alternativas tradicionales que cotizaban \$20,000 USD por cámara individual a 3 km de distancia sin garantía de funcionamiento.

\textbf{Arquitectura desplegada:} Sistema basado en 1 Base Node central (farm HQ) y 4 Field Nodes estratégicamente posicionados, con distancias operacionales de 3 km y 7.5 km desde el punto de control central. La red HaLow proporcionó conectividad para streaming de video/audio en tiempo real con alertas cloud automatizadas, superando las limitaciones de terreno irregular mediante penetración sub-GHz (900 MHz).

\textbf{Resultados validados:} El sistema logró cobertura completa end-to-end en 7.5 km de alcance real en topografía montañosa, demostrando capacidades superiores a las especificaciones teóricas de alcance urbano (500-800 m) mediante configuración optimizada de potencia de transmisión y selección de ubicaciones con línea de vista parcial. La operación continua 24/7 con streaming de video HD (alto throughput) confirma la robustez del protocolo 802.11ah en entornos no controlados con interferencia y multipath fading.

\textbf{Relevancia para Smart Metering:} Este caso valida que para aplicaciones de telemetría IoT (smart metering con throughput 1-10 kbps por dispositivo), las distancias operacionales de 2-3 km requeridas en esta tesis son \textit{altamente conservadoras} dado que HaLow demostró capacidad de soportar aplicaciones de video (>1 Mbps) a 7.5 km. El éxito comercial en agricultura con requisitos de throughput 100× superiores confirma la viabilidad técnica y económica de HaLow como tecnología de backhaul para Advanced Metering Infrastructure en entornos semi-urbanos con concentradores DCU separados 2-3 km del gateway central.

\begin{table}[h]
\centering
\small
\caption{Comparación HaLow vs LoRaWAN para backhaul Smart Energy en Advanced Metering Infrastructure}
\label{tab:halow-vs-lorawan}
\begin{tabular}{|p{2.8cm}|p{2.5cm}|p{2.5cm}|p{5cm}|}
\hline
\rowcolor{gray!20}
\textbf{Criterio} & \textbf{HaLow (802.11ah)} & \textbf{LoRaWAN} & \textbf{Análisis Decisión} \\
\hline
\textbf{Alcance urbano} & 500-800 m & \textcolor{green}{2-5 km} & LoRaWAN superior por MCL 157 dB (vs 140 dB HaLow) \\
\hline
\textbf{Throughput} & \textcolor{green}{150 kbps - 4 Mbps} & 0.3-50 kbps & \textbf{HaLow crítico:} waveforms HD 10 kSPS = 160 kbps \\
\hline
\textbf{Latencia típica} & \textcolor{green}{10-50 ms} & 1-10 s (Class A) & \textbf{HaLow crítico:} DER control <100 ms IEEE 2030.5 \\
\hline
\textbf{IPv6 nativo} & \textcolor{green}{Sí (Wi-Fi)} & No (requiere GW) & \textbf{HaLow crítico:} IEEE 2030.5 end-to-end \\
\hline
\textbf{Costo módulo} & \$50-70 & \textcolor{green}{\$5-15} & LoRaWAN 80\% menor CAPEX \\
\hline
\textbf{Ecosistema} & Incipiente (2021) & \textcolor{green}{Maduro (2015+)} & LoRaWAN: 170+ operadores, 1M+ gateways \\
\hline
\textbf{QoS garantizado} & \textcolor{green}{Sí (EDCA 4 AC)} & No (ALOHA puro) & \textbf{HaLow crítico:} alarmas priority vs telemetry \\
\hline
\textbf{Bidireccional} & \textcolor{green}{Full-duplex} & Half-duplex & \textbf{HaLow ventaja:} actuación DER simultánea \\
\hline
\multicolumn{4}{|p{14.5cm}|}{\textbf{Decisión arquitectónica:} HaLow seleccionado por throughput + latencia + QoS requeridos para AMI con integración DER (Distributed Energy Resources). LoRaWAN adecuado únicamente para telemetría unidireccional simple sin actuación real-time.} \\
\hline
\end{tabular}
\end{table}

\textbf{Justificación técnica detallada:} Si bien LoRaWAN ofrece mayor alcance (2-5 km) y menor costo de módulos (\$5-15 vs \$50-70), la integración de Distributed Energy Resources (DER - solar PV, battery storage, EV chargers) en AMI moderna impone tres requisitos críticos no satisfechos por LoRaWAN:

\begin{enumerate}
    \item \textbf{Throughput >100 kbps:} Monitoreo power quality mediante waveforms a 10 kSPS (voltage/current metering HD) para grid stability analysis según IEEE 2030.5. Ejemplo: waveform 1 ciclo 60 Hz (16.67 ms) con 128 samples/cycle × 2 channels (V+I) × 2 bytes/sample = 512 bytes cada 16.67 ms = 245 kbps. LoRaWAN limitado por duty cycle 1\% FCC (36 s transmission/hora) entrega throughput efectivo ~5 kbps promedio, insuficiente. HaLow sin restricción duty-cycle soporta 150 kbps - 4 Mbps continuous.
    
    \item \textbf{Latencia <100 ms:} Comandos de disconnect switches y load shedding (IEEE 2030.5 DER Control) requieren response time <100 ms para grid stability durante transitorios. LoRaWAN Class A latencia típica 1-10 segundos (uplink → downlink en ventanas RX1/RX2), Class C reduce a ~500 ms pero consume 50-100 mA continuous RX (vs 6 mA HaLow con TWT). HaLow CSMA/CA con EDCA proporciona latencia determinista 10-50 ms.
    
    \item \textbf{QoS diferenciado:} Priorización de alarmas críticas (undervoltage, overvoltage, phase imbalance) sobre telemetría background (consumo horario). IEEE 802.11e EDCA proporciona 4 access categories con diferenciación AIFS/CW: AC\_VO (alarmas, AIFS=2), AC\_VI (commands, AIFS=2), AC\_BE (telemetry, AIFS=3), AC\_BK (logs, AIFS=7). LoRaWAN utiliza ALOHA puro sin QoS, colisiones aleatorias en redes densas (10-30\% packet loss con >500 devices/gateway).
\end{enumerate}

\textbf{Análisis TCO 10 años (1000 medidores):} El costo adicional CAPEX de HaLow (\$50 × 1000 = \$50K adicionales vs LoRaWAN) se compensa por: (1) savings en operational efficiency mediante detección temprana de fallas power quality (\$100K/año evitados en truck rolls), (2) revenue incremental por Time-of-Use pricing habilitado con datos granulares (\$50K/año), (3) eliminación de gateways intermedios LoRaWAN→IP (\$200/gateway × 20 gateways = \$4K). \textbf{ROI 6 meses.} Costo HaLow justificado por capabilities críticas para AMI + DER modernos~\cite{scharerPushingWiFiHaLow2025}.

\subsubsection{Arquitecturas Cloud Comerciales: AWS IoT vs Azure IoT vs ThingsBoard Cloud}

\textbf{AWS IoT Core + Greengrass:} Runtime edge Lambda, ML SageMaker. Limitaciones: \$1/millon msg, latencia Lambda ~50-100 ms, lock-in AWS.

\textbf{Azure IoT Hub + IoT Edge:} Runtime Docker, Stream Analytics local, AKS. Limitaciones: IoT Hub S2 \$250/mes (6M msg/día), telemetría obligatoria.

\textbf{ThingsBoard Edge:} Open-source Apache 2.0, sincronización bidireccional. Costo \$0 SW + \$100-200 HW.

\textbf{TCO 5 años (1000 dispositivos):} AWS \$69k, Azure \$66k, Propuesta (TB Edge + HaLow) \$45k. \textbf{Ahorro 32-35\%}.

\subsubsection{Productos Comerciales Wi-Fi HaLow: Validación Industrial}

\textbf{Contexto:} La adopción industrial de IEEE 802.11ah valida su madurez tecnológica más allá de prototipos académicos. A continuación se analizan tres productos comerciales representativos del ecosistema HaLow en 2024-2025.

\textbf{1. Vantron RAH305 Industrial HaLow Router}~\cite{vantronRAH305WiFiHaLow2024}

\textbf{Especificaciones técnicas:}
\begin{itemize}
    \item \textbf{CPU:} Texas Instruments AM64x Arm Cortex-A53 @ 1 GHz, 1 GB DDR4 RAM, 8 GB eMMC storage
    \item \textbf{Chipset HaLow:} Morse Micro MM6108 (\textit{mismo chipset seleccionado en este trabajo})
    \item \textbf{Conectividad multi-radio:} Wi-Fi HaLow 802.11ah (hasta 1 km, 32.5 Mbps @ 8 MHz) + Wi-Fi 6 (802.11ax 2.4/5 GHz) + LTE Cat-4 (150 Mbps DL) + Bluetooth 5.3 + GNSS
    \item \textbf{Interfaces I/O:} 5× Gigabit Ethernet, RS485/RS232/RS422 serial, 4× DI/DO digitales, 4× AI/AO analógicos, USB 2.0
    \item \textbf{Características avanzadas:} Multi-link failover automático (Ethernet → Wi-Fi → Cellular), VPN (IPsec/OpenVPN), gestión cloud, DIN-rail mounting
    \item \textbf{Rango industrial:} -20°C a +60°C, certificaciones FCC/CE
    \item \textbf{Costo estimado:} \$600-800 USD (producción volumen)
\end{itemize}

\textbf{Análisis comparativo con este trabajo:}
\begin{itemize}
    \item \textbf{Coincidencias arquitectónicas:} RAH305 valida decisión de chipset MM6108, arquitectura multi-radio (HaLow + LTE backup), y orientación industrial con interfaces RS485 para Modbus/legacy SCADA (similar a requisitos AMI).
    \item \textbf{Diferencias funcionales:} RAH305 carece de: (1) soporte Thread 802.15.4 para última milla meter-to-DCU, (2) conformidad IEEE 2030.5 (CSIP client library), (3) edge AI/LLM local (toda inteligencia en cloud), (4) stack open-source auditado (firmware propietario). \textbf{Este trabajo añade estas capacidades críticas para Smart Energy.}
    \item \textbf{Comparación económica:} RAH305 (\$600-800) vs gateway propuesto (\$295 BOM Cap 4) = \textbf{RAH305 es 2.0-2.7× más costoso}. Sobrecosto justificado en aplicaciones industriales genéricas (SCADA, surveillance) pero innecesario para AMI donde Thread+HaLow+Edge-AI son suficientes.
\end{itemize}

\textbf{Conclusión RAH305:} Valida viabilidad comercial del chipset MM6108 y arquitectura multi-radio, pero evidencia sobreespecificación (Wi-Fi 6 + Bluetooth + GNSS no requeridos en AMI fijo) y costo 2-3× superior para aplicación Smart Energy específica.

\textbf{2. Gateworks GW16167 M.2 HaLow Module}~\cite{gateworksGW16167HaLow2025}

\textbf{Especificaciones técnicas:}
\begin{itemize}
    \item \textbf{Chipset:} Morse Micro MM8108 (\textit{siguiente generación post-MM6108})
    \item \textbf{Factor de forma:} M.2 2230 E-Key (estándar industria para módulos wireless)
    \item \textbf{Interfaz host:} USB 2.0 (primario), SDIO/SPI (opcional)
    \item \textbf{Frecuencia:} 850-950 MHz (bandas globales US + EU)
    \item \textbf{Potencia TX:} Hasta +26 dBm (vs +23 dBm MM6108) = \textbf{+3 dB mejora = 2× potencia radiada}
    \item \textbf{Throughput:} Hasta 43.3 Mbps @ 8 MHz (vs 32.5 Mbps MM6108) = \textbf{+33\% velocidad}
    \item \textbf{Sensibilidad RX:} -101 dBm (vs -98 dBm MM6108) = \textbf{+3 dB alcance extendido}
    \item \textbf{Seguridad:} WPA2-PSK(AES), WPA3-OWE, WPA3-SAE
    \item \textbf{Rango industrial extendido:} -40°C a +85°C
    \item \textbf{Software:} Stack Linux estándar (cfg80211/mac80211), sin drivers propietarios
    \item \textbf{Alcance documentado:} >1 km, potencialmente 4-5 km con mejoras de especificaciones
    \item \textbf{Certificaciones:} FCC, Made in USA
    \item \textbf{Costo estimado:} ~\$150 USD módulo
\end{itemize}

\textbf{Comparación MM8108 vs MM6108 (este trabajo):}

\begin{table}[H]
\centering
\caption{Comparación Morse Micro MM8108 vs MM6108}
\label{tab:mm8108-vs-mm6108}
\scriptsize
\begin{tabular}{|l|c|c|p{4.5cm}|}
\hline
\textbf{Parámetro} & \textbf{MM6108 (Tesis)} & \textbf{MM8108 (GW16167)} & \textbf{Impacto} \\
\hline
Potencia TX & +23 dBm & \textcolor{blue}{+26 dBm (+3 dB)} & 2× potencia radiada, +40\% alcance \\
\hline
Throughput pico & 32.5 Mbps & \textcolor{blue}{43.3 Mbps (+33\%)} & Margen adicional para HD waveforms \\
\hline
Sensibilidad RX & -98 dBm & \textcolor{blue}{-101 dBm (+3 dB)} & Link budget +6 dB total (TX+RX) \\
\hline
Consumo TX & ~800 mW & ~900 mW (+12\%) & Incremento marginal aceptable \\
\hline
Consumo RX & ~120 mW & ~130 mW (+8\%) & Impacto mínimo en TWT sleep \\
\hline
Certificaciones & FCC & FCC + CE + IC & Mercados globales US+EU \\
\hline
Factor forma & Custom PCB & \textcolor{blue}{M.2 E-Key} & Plug-and-play, intercambiable \\
\hline
Interfaz host & SPI/SDIO & \textcolor{blue}{USB 2.0 + SPI/SDIO} & Driver Linux estándar \\
\hline
\end{tabular}
\end{table}

\textbf{Análisis de modularidad M.2 E-Key:}

\begin{itemize}
    \item \textbf{Ventaja 1 - Flexibilidad hardware:} Factor forma M.2 E-Key elimina necesidad de diseño PCB custom para integración HaLow. Compatible con cualquier SBC con slot M.2 (Raspberry Pi CM4, Intel NUC, NVIDIA Jetson Nano, Gateworks Venice). \textbf{Este trabajo puede adoptar GW16167 sin rediseño de gateway.}
    
    \item \textbf{Ventaja 2 - Path de actualización:} Arquitectura propuesta (Cap 4) es compatible con upgrade futuro: swap MM6108 → MM8108 en mantenimiento preventivo sin cambiar gateway completo. \textbf{Tiempo estimado: 30 minutos vs 4-6 horas reemplazo completo.}
    
    \item \textbf{Ventaja 3 - Commoditization:} Modularización M.2 reduce vendor lock-in. Si Morse Micro discontinúa MM6108, alternativas M.2 HaLow de otros fabricantes (NewRadek, AsiaRF en roadmap 2026) aseguran disponibilidad largo plazo.
\end{itemize}

\textbf{Relevancia para este trabajo:} GW16167 valida dos decisiones arquitectónicas clave: (1) \textbf{Interfaz USB 2.0 host} para módulos wireless (vs integración SoC monolítica) asegura upgrade path y reduce obsolescencia. (2) \textbf{Diseño modular preparado para evolución tecnológica:} Si MM8108 o futuros chipsets HaLow demuestran ventajas significativas, gateway puede actualizarse sin rediseño completo.

\textbf{3. Morse Micro MM8108 Product Brief}~\cite{morsemicroMM8108ProductBrief2025}

\textbf{Mejoras arquitectónicas MM8108 vs MM6108:}
\begin{itemize}
    \item \textbf{Link budget mejorado:} +6 dB total (+3 dB TX + 3 dB RX) → alcance teórico 2-3 km en condiciones urbanas (vs 1-2 km MM6108), 5-7 km en rural line-of-sight (vs 3-4 km). \textbf{Implicación:} Reduce cantidad de gateways requeridos en despliegue a escala utility (e.g., 800 medidores cubiertos por 1 gateway @ 2.5 km vs 2 gateways @ 1.5 km con MM6108).
    
    \item \textbf{Throughput +33\%:} 43.3 Mbps @ 8 MHz permite margen adicional para: (1) waveform capture simultáneo de múltiples DCUs (6-8 DCUs streaming power quality data vs 4-5 con MM6108), (2) overhead protocolo IEEE 2030.5 HTTPS/TLS sin degradación, (3) video surveillance opcional en subestaciones (uso dual AMI + physical security).
    
    \item \textbf{Certificaciones globales FCC+CE+IC:} MM8108 certifica 850-950 MHz incluyendo bandas EU (863-868 MHz) y Australia (915-928 MHz). MM6108 inicialmente solo FCC 902-928 MHz. \textbf{Implicación:} Gateway con MM8108 exportable a mercados latinoamericanos (Brasil, Chile, Argentina usan 902-907.5 MHz similar US) y europeos sin modificación hardware.
\end{itemize}

\textbf{Síntesis productos comerciales:} La existencia de routers industriales HaLow (Vantron RAH305 \$600-800), módulos M.2 estandarizados (Gateworks GW16167 \$150), y chipsets next-gen (MM8108 con +33\% throughput y +6 dB link budget) valida tres aspectos críticos de este trabajo: (1) \textbf{Madurez tecnológica HaLow:} No es experimental, es producción comercial 2024-2025. (2) \textbf{Commoditization en progreso:} Transición de módulos custom a M.2 estándar reduce barreras entrada. (3) \textbf{Roadmap evolutivo claro:} MM6108 (2023) → MM8108 (2024) → futuros SoCs con Wi-Fi 7 integration (2026-2027 estimado). \textbf{Este trabajo no solo es viable hoy con MM6108, sino future-proof con path de upgrade a MM8108/sucesores sin arquitectura completa rediseñada.}

\subsubsection{Mesh Sub-GHz en Energía Renovable: Caso Solar Tracking}

\textbf{Contexto:} La adopción de arquitecturas mesh networking sub-GHz en aplicaciones de energía renovable valida la viabilidad técnica de topologías similares para infraestructura Smart Energy. Radiocrafts, fabricante noruego de módulos wireless, documenta despliegues a escala utility de su tecnología RIIM (Radiocrafts Industrial IoT Module) en instalaciones de solar tracking masivas~\cite{radiocraftsCommissioningLargeSolar2025}.

\textbf{Especificaciones técnicas RIIM solar tracking:}

\begin{itemize}
    \item \textbf{Escala despliegue:} Miles de trackers solares por gateway (vs 250 nodos Thread por DCU en este trabajo)
    \item \textbf{Frecuencia operación:} 868/915 MHz sub-GHz mesh (rango similar HaLow 900 MHz)
    \item \textbf{Confiabilidad:} \textbf{99.99\% transmisiones exitosas} en instalaciones densas con múltiples redes RIIM coexistiendo
    \item \textbf{Latencia crítica:} Mover todos los paneles a "safe mode" en \textbf{<1 segundo} durante tormentas de granizo o vientos extremos (comparable a requisito IEEE 2030.5 <100 ms para DER control en AMI)
    \item \textbf{Penetración señal:} Atraviesa vigas metálicas de estructuras tracker sin degradación significativa (obstáculos físicos similares a medidores dentro de edificios con paredes concreto/metal)
    \item \textbf{Topología red:} Mesh self-healing con menor cantidad de saltos que ZigBee/Wi-SUN, reduciendo latencia end-to-end
    \item \textbf{Downlink eficiente:} Comandos de actuación broadcast a todos los trackers simultáneamente (paralelo a control DER actuación en AMI)
\end{itemize}

\textbf{Migración tecnológica documentada:}

Radiocrafts reporta que fabricantes líderes de solar tracking están \textbf{migrando activamente de ZigBee, Wi-SUN y LoRa a RIIM} por ventajas técnicas específicas:

\begin{itemize}
    \item \textbf{Mayor alcance:} Cobertura completa en terrenos irregulares/montañosos sin necesidad de repetidores adicionales
    \item \textbf{Menor latencia mesh:} Reducción de saltos intermedios en topología mesh → tiempo respuesta <1s para comandos críticos
    \item \textbf{Escalabilidad superior:} Red única puede manejar "few thousand trackers per gateway" (vs límites típicos 1000-2000 nodos ZigBee/Wi-SUN antes de fragmentación)
    \item \textbf{Downlink robusto:} Comunicación bidireccional confiable para actuación (mover paneles), no solo uplink telemetría como LoRaWAN Class A
\end{itemize}

\textbf{Gateway industrial: Advantech ECU-150 + RIIM}

Partnership Radiocrafts-Advantech desarrolló gateway ruggedizado específico para solar tracking:
\begin{itemize}
    \item \textbf{Hardware:} Advantech ECU-150 compact industrial gateway + módulo RIIM mesh integrado
    \item \textbf{Ambiente operacional:} -40°C a +85°C, resistencia IP-rated para instalación outdoor en campo solar
    \item \textbf{Arquitectura comparable:} Gateway edge computing + módulo wireless mesh (similar a Raspberry Pi 4 + HaLow/Thread en este trabajo)
    \item \textbf{Aplicación dual:} Monitoreo telemetría + control actuación tiempo real (paralelo a AMI lectura medidores + DER control)
\end{itemize}

\textbf{Comparación tecnológica: RIIM vs ZigBee/Wi-SUN/LoRa para renovables}

Radiocrafts publica \textit{Wireless Technology Selection Guide For Solar Tracking} comparando RIIM con tecnologías competidoras:

\begin{table}[H]
\centering
\caption{Comparación Tecnologías Mesh Sub-GHz: Solar Tracking vs AMI (Este Trabajo)}
\label{tab:riim-vs-thesis-comparison}
\scriptsize
\begin{tabular}{|l|p{4cm}|p{4.5cm}|p{4cm}|}
\hline
\textbf{Parámetro} & \textbf{RIIM Solar Tracking} & \textbf{Thread+HaLow AMI (Tesis)} & \textbf{Análisis Convergencia} \\
\hline
\textbf{Aplicación} & Renovables (solar PV tracking) & Smart Energy (AMI + DER) & Ambos: IoT + energía \\
\hline
\textbf{Escala red} & Miles trackers/gateway & 250 medidores/DCU, 10 DCUs/gateway & RIIM más denso, Thread conservador \\
\hline
\textbf{Frecuencia} & 868/915 MHz sub-GHz & 2.4 GHz (Thread) + 900 MHz (HaLow) & Convergencia sub-GHz backhaul \\
\hline
\textbf{Confiabilidad} & 99.99\% TX success & Thread 99.9\% (spec), HaLow 99.95\% & Similares (>99.9\%) \\
\hline
\textbf{Latencia crítica} & <1s safe mode command & <100 ms DER control (IEEE 2030.5) & Ambos actuación tiempo real \\
\hline
\textbf{Topología} & Mesh self-healing & Mesh Thread + star HaLow & Dual-mesh (local + backhaul) \\
\hline
\textbf{Migración desde} & ZigBee, Wi-SUN, LoRa & Zigbee (legacy), celular (OPEX alto) & Tendencia anti-ZigBee/celular \\
\hline
\textbf{Gateway edge} & Advantech ECU-150 (industrial) & Raspberry Pi 4 (open-source) & Edge computing + mesh \\
\hline
\end{tabular}
\end{table}

\textbf{Paralelismo con este trabajo:}

\begin{enumerate}
    \item \textbf{Validación arquitectónica mesh + edge:} Si RIIM maneja miles de trackers solares con actuación <1s en producción utility-scale (India, Europa, US), arquitectura Thread mesh (250 nodos) + HaLow backhaul (10 DCUs) para AMI representa \textbf{patrón arquitectónico probado} en mercado energía, simplemente escalado a requisitos smart metering.
    
    \item \textbf{Convergencia sub-GHz:} Migración RIIM desde ZigBee/Wi-SUN/LoRa replica tendencia documentada en AMI hacia tecnologías sub-GHz con mayor alcance y penetración. HaLow (900 MHz) sigue misma filosofía RIIM (868/915 MHz) para backhaul largo alcance.
    
    \item \textbf{Actuación bidireccional crítica:} Caso solar tracking demuestra que \textbf{solo uplink telemetría es insuficiente} para aplicaciones energía moderna (requiere downlink commands para safe mode/control). Valida decisión este trabajo de priorizar HaLow (downlink robusto CSMA/CA) sobre LoRaWAN (downlink limitado duty-cycle 1\%).
    
    \item \textbf{Edge computing imperativo:} Gateway Advantech ECU-150 integra procesamiento local edge (similar Raspberry Pi 4 en tesis). Demuestra que \textbf{cloud-only es antipatrón} en renovables/AMI por latencia, resiliencia y OPEX.
    
    \item \textbf{Diferenciador DER:} Si mesh sub-GHz maneja solar tracking (PV generation monitoring + panel actuación), extensión lógica es AMI con DER integration (consumo + solar inverter control + battery storage + EV charging). \textbf{Este trabajo agrega IEEE 2030.5 + Thread que RIIM no cubre}, posicionando arquitectura como superset funcional.
\end{enumerate}

\textbf{Conclusión caso solar tracking:} Despliegues RIIM en renovables a escala utility validan tres aspectos fundamentales de este trabajo: (1) \textbf{Mesh sub-GHz es producción, no experimental:} Tecnologías 868/900 MHz operan en campo con 99.99\% reliability. (2) \textbf{Edge + mesh es arquitectura estándar:} Gateway industrial + módulos mesh replicado en solar (Advantech ECU-150) y propuesto en AMI (Raspberry Pi 4). (3) \textbf{Actuación tiempo real es requisito crítico:} Aplicaciones energía modernas (renovables, AMI+DER) demandan downlink robusto, no solo uplink telemetría. \textbf{Si RIIM gestiona miles de trackers solares, Thread+HaLow para AMI con menor densidad (250 medidores/DCU) es arquitectura conservadora y defendible técnicamente.}

\subsection{Brechas Identificadas}

\begin{enumerate}
\item \textbf{Ausencia de HaLow}: Ningún trabajo integra Wi-Fi HaLow como backhaul Smart Energy.
\item \textbf{Conformidad limitada}: Pocas implementaciones cumplen IEEE 2030.5 + ISO/IEC 30141.
\item \textbf{Evaluaciones insuficientes}: Mayormente pruebas de concepto sin benchmarking riguroso.
\item \textbf{LLM edge inexplorada}: Sin integración de inferencia LLM local para análisis contextual.
\end{enumerate}

\subsection{Síntesis de Brechas y Transición al Diseño del Gateway}

Este capítulo ha establecido las bases teóricas necesarias para el diseño arquitectónico del gateway IoT propuesto. La revisión del contexto de Smart Energy (§2.1), el stack de protocolos IoT (§2.2), las tecnologías de edge computing (§2.3), los requisitos de seguridad multi-capa (§2.4), y el análisis comparativo del estado del arte han revelado cuatro brechas críticas que motivan directamente las decisiones de diseño del Capítulo 3:

\textbf{Brecha 1 - Integración Wi-Fi HaLow en gateways edge}: Los trabajos analizados (§2.5.1-§2.5.2) implementan Thread, Zigbee o LoRaWAN, pero ninguno integra HaLow (IEEE 802.11ah~\cite{ieee80211ah2020}) en un gateway multi-protocolo con failover multi-WAN, a pesar de su ventaja de alcance (1-3 km), throughput (hasta 40 Mbps) y latencia (<50 ms) demostrada superior a LoRaWAN (latencia >1 segundo). \textit{Solución propuesta (Cap. 3):} Integración nativa de módulo Morse Micro MM6108 (IEEE 802.11ah) con OTBR (Thread) en plataforma unificada OpenWRT, proporcionando redundancia de protocolos (Thread para nodos indoor, HaLow para outdoor/utility-scale) con conmutación automática basada en LQI y RSSI.

\textbf{Brecha 2 - Conformidad simultánea IEEE 2030.5 + ISO/IEC 30141}: Las soluciones comerciales (Cisco IR829, Dell Edge Gateway 3000) implementan protocolos propietarios sin conformidad estándar, mientras que prototipos académicos implementan parcialmente IEEE 2030.5 pero no cumplen las cuatro vistas de ISO/IEC 30141 (funcional, información, despliegue, operacional). \textit{Solución propuesta (Cap. 3):} Arquitectura que implementa explícitamente los 7 Function Sets de IEEE 2030.5 (Time, DRLC, Messaging, Pricing, Metering, DER, Prepayment) mediante adaptadores de protocolo CoAP/LwM2M→IEEE 2030.5, documentando conformidad con las cuatro vistas de ISO/IEC 30141 para garantizar interoperabilidad certificable.

\textbf{Brecha 3 - Validación experimental cuantitativa}: La mayoría de trabajos académicos presentan únicamente diseño arquitectónico conceptual sin mediciones empíricas de latencia end-to-end, PDR en condiciones reales, o caracterización de resiliencia durante desconexiones WAN prolongadas (>72 horas). \textit{Solución propuesta (Cap. 3):} Metodología de validación experimental rigurosa con prototipo funcional Raspberry Pi 5 + OpenWRT, incluyendo: (a) caracterización de latencia CoAP end-to-end Thread vs HaLow bajo carga variable, (b) medición de PDR en entorno urbano con interferencia real (2.4 GHz saturado, 900 MHz limpio), (c) pruebas de resiliencia con desconexión WAN simulada 72+ horas validando sincronización bidireccional ThingsBoard Edge.

\textbf{Brecha 4 - Costo-efectividad en contexto Latinoamericano}: Las soluciones comerciales (Cisco IR829 \$2,500-4,000, Dell Edge Gateway \$1,200-2,000) presentan barreras CAPEX significativas para despliegues utility-scale en economías emergentes, sin análisis de alternativas open-source con hardware commodity (Raspberry Pi + OpenWRT <\$500). \textit{Solución propuesta (Cap. 3):} Arquitectura basada en componentes open-source (OpenWRT, ThingsBoard Edge, Docker, TimescaleDB, Kafka) sobre Raspberry Pi 5 (\$80) + módulo MM6108 (\$45) + radios Thread (\$25), totalizando BoM <\$200. Análisis TCO 5 años (1000 dispositivos) demuestra ahorro 32-35\% vs AWS IoT/Azure IoT Hub (§2.5.4).

\subsubsection{Vinculación con el Diseño Arquitectónico del Capítulo 3}

Las cuatro brechas identificadas definen los requisitos funcionales y no funcionales del gateway propuesto. El Capítulo 3 presenta la arquitectura detallada que responde sistemáticamente a cada brecha:

\begin{itemize}
\item \textbf{Diseño de Hardware (§3.1):} Justifica selección de Raspberry Pi 5, módulo MM6108 (HaLow), y radio Thread basándose en Brecha 1 y Brecha 4 (costo-efectividad + multi-protocolo).
\item \textbf{Arquitectura de Software (§3.2):} Implementa stack de protocolos IoT (§2.2) sobre OpenWRT, integrando contenedores Docker (§2.3.1) para edge analytics con ThingsBoard Edge (§2.3.4), y aplicando seguridad por capas (§2.4.3) mediante DTLS-PSK, WPA3-SAE, RBAC.
\item \textbf{Conformidad con Estándares (§3.3):} Mapea implementación contra requisitos IEEE 2030.5 (§2.1.2) e ISO/IEC 30141 (§2.1.3), abordando Brecha 2 con documentación de conformidad certificable.
\item \textbf{Metodología de Validación Experimental (§3.4):} Define protocolo de pruebas empíricas (latencia, PDR, resiliencia) respondiendo a Brecha 3, con métricas cuantitativas comparables con estado del arte.
\end{itemize}

Esta transición desde las brechas teóricas identificadas en el Marco Teórico hacia las soluciones arquitectónicas concretas del Capítulo 3 garantiza que el diseño propuesto está fundamentado en un análisis riguroso del estado del arte y responde directamente a las limitaciones de las soluciones existentes, aportando contribuciones originales en integración HaLow, conformidad estándar dual (IEEE 2030.5 + ISO/IEC 30141), validación experimental cuantitativa, y viabilidad económica para contextos Latinoamericanos.
