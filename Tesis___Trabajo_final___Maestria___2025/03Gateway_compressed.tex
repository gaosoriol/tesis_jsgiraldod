\chapter{Gateway de Telemetría para Smart Energy}

\section{Introducción}

El gateway constituye el componente central de la arquitectura propuesta, actuando como puente entre las redes de campo (Thread 802.15.4) y las redes de área amplia (HaLow 802.11ah), consolidando datos de múltiples medidores inteligentes y transmitiéndolos de manera segura hacia la plataforma cloud.

\subsection{Funciones Principales}

El gateway cumple cinco funciones críticas: (1) agregación de datos de múltiples DCUs, (2) traducción de protocolos entre Thread/CoAP y MQTT/HTTP, (3) seguridad mediante TLS/mTLS, (4) resiliencia con buffering local, y (5) edge computing para preprocesamiento.

\section{Conformidad con Estándares}

\subsection{IEEE 2030.5-2023 (Smart Energy Profile 2.0)}

El gateway implementa cinco Function Sets de IEEE 2030.5: Device Capability (DCAP), Time (TM), Metering Mirror (MM), Messaging (MSG) y End Device (ED). La API REST expone endpoints estándar (\texttt{/dcap}, \texttt{/tm}, \texttt{/mup/<deviceID>}, \texttt{/msg}, \texttt{/edev}) con autenticación TLS 1.3 y certificados ECC P-256.

Los ejemplos completos de respuestas XML IEEE 2030.5 para todos los Function Sets se presentan en el Anexo D.

\subsection{ISO/IEC 30141:2024 (IoT Reference Architecture)}

El gateway implementa entidades funcionales IoT: (1) Aplicación (ThingsBoard Edge), (2) Administración de Servicios (Docker orchestration), (3) IoT Service (Thread OTBR, HaLow AP, IEEE 2030.5 Server), (4) Comunicación (MQTT, CoAP, HTTP/TLS), y (5) Seguridad (PKI, mTLS, RBAC).

\section{Plataforma de Hardware}

\subsection{Raspberry Pi 4 Model B}

Seleccionado por su balance costo/rendimiento: SoC Broadcom BCM2711 (Quad-core Cortex-A72 @ 1.8 GHz), 4 GB RAM, 16 GB eMMC + 256 GB NVMe SSD (vía USB 3.0 adapter), Ethernet Gigabit, 4× USB 3.0/2.0, GPIO 40-pin para expansión.

\subsection{Interfaces de Conectividad}

\begin{itemize}
    \item \textbf{Thread 802.15.4}: nRF52840 USB Dongle con firmware OpenThread RCP
    \item \textbf{HaLow 802.11ah}: Newracom MM6108 módulo SPI (915-928 MHz, hasta 1 km)
    \item \textbf{LTE Cat-M1/NB-IoT}: Quectel BG95-M3 USB modem (backup WAN)
    \item \textbf{Ethernet}: WAN primaria 1000BASE-T
\end{itemize}

La justificación comparativa detallada de hardware (Raspberry Pi vs Orange Pi vs Jetson Nano) se presenta en la Tabla~\ref{tab:hardware_comparison} del documento original.

\section{Sistema Operativo y Software Base}

\subsection{OpenWRT 23.05 como Base}

OpenWRT proporciona: (1) kernel Linux 5.15 LTS con drivers ath11k para HaLow, (2) sistema de configuración UCI para networking, (3) gestión de paquetes opkg (>4000 paquetes), (4) firewall nftables con zonas configurables, y (5) soporte Docker 20.10.24 para servicios containerizados.

El proceso completo de instalación de OpenWRT en Raspberry Pi 4, incluyendo particionamiento, boot config y configuración de interfaces, se documenta en el Anexo A.

\subsection{Servicios Containerizados (Docker)}

La plataforma se despliega con seis servicios Docker: OpenThread Border Router (OTBR), ThingsBoard Edge 3.6.0, PostgreSQL 15 + TimescaleDB, IEEE 2030.5 Server (Python/Flask), Bridge Thread↔ThingsBoard (Python/MQTT), y Apache Kafka 7.5.0. Los archivos \texttt{docker-compose.yml} completos se presentan en el Anexo B.

\section{Conectividad y Protocolos}

\subsection{Thread 802.15.4 (Acceso a Nodos)}

Thread es un protocolo mesh IPv6 optimizado para IoT: baja latencia (<50 ms típico), bajo consumo (nodos sleep <10 µA), auto-healing con rerouting automático, y escalabilidad hasta 250 nodos por red. El gateway ejecuta OTBR para formar red Thread en canal 15 (2.435 GHz) con prefix \texttt{fd00::/64}.

\subsection{HaLow 802.11ah (Backhaul)}

HaLow opera en sub-1 GHz (915-928 MHz en Región 2) con cuatro ventajas sobre WiFi tradicional: (1) penetración superior (3-5 dB mejor que 2.4 GHz), (2) alcance extendido (hasta 1 km LoS), (3) bajo consumo (RAW/TIM para sleep), y (4) escalabilidad (hasta 8191 STAs por AP).

El gateway soporta cuatro modos HaLow configurables vía UCI: AP Router (NAT), STA Client, Mesh 802.11s, y EasyMesh 1905.1. Las configuraciones UCI completas se presentan en el Anexo D.

\subsection{Protocolos de Aplicación}

\subsubsection{MQTT}

MQTT (Message Queuing Telemetry Transport) implementa patrón Pub/Sub con tres niveles QoS. El gateway usa: (1) QoS 1 para telemetría (at least once), (2) QoS 2 para alarmas críticas (exactly once), (3) retained messages para último valor, y (4) Last Will Testament (LWT) para detección de desconexión.

Topics utilizados: \texttt{v1/devices/me/telemetry} (TB Edge), \texttt{thread/telemetry/<node\_id>} (desde OTBR), \texttt{smartgrid/meter/<id>/energy} (formato custom).

\subsubsection{CoAP}

CoAP (Constrained Application Protocol) se usa en Thread mesh: protocolo UDP con 4-byte header, métodos RESTful (GET/POST/PUT/DELETE), Observe para suscripciones, y DTLS+PSK para seguridad. Latencia típica Thread: 30-80 ms end-to-end.

\subsubsection{LwM2M}

LwM2M (Lightweight M2M) proporciona gestión de dispositivos sobre CoAP con modelo de objetos estándar (OMA SpecWorks). Objetos implementados: Security (0), Server (1), Device (3), Connectivity (4), Firmware Update (5), Location (6), Temperature (3303), Humidity (3304), On/Off (3312).

La implementación completa de referencia de un nodo IoT ESP32-C6 con cliente LwM2M AVSystems Anjay, incluyendo todos los archivos fuente (\texttt{main.c}, \texttt{lwm2m\_client.c}, objetos IPSO 3303/3304), \texttt{CMakeLists.txt} e instrucciones de compilación ESP-IDF, se documenta en el Anexo E.

\subsubsection{HTTP/REST}

HTTP/REST se utiliza para: (1) IEEE 2030.5 Server (puerto 8883/HTTPS), (2) ThingsBoard Edge API (puerto 8080/HTTP), (3) LuCI web UI (puerto 80/HTTP), y (4) Ollama LLM API local (puerto 11434/HTTP).

\section{Arquitectura de Datos}

\subsection{TimescaleDB (Series Temporales)}

TimescaleDB extiende PostgreSQL con optimizaciones para series temporales: (1) particionamiento automático (hypertables), (2) compresión 10-20× después de 7 días, (3) continuous aggregates para vistas materializadas (15-min, 1-hora, 1-día), (4) retención automática (90 días), y (5) queries eficientes con \texttt{time\_bucket()}.

El esquema completo de TimescaleDB, incluyendo definición de hypertables, políticas de compresión, continuous aggregates y cinco queries SQL de ejemplo, se presenta en el Anexo D.

\subsection{Apache Kafka (Message Bus)}

Kafka proporciona: (1) desacoplamiento productor/consumidor, (2) replay de mensajes históricos, (3) escalabilidad horizontal, (4) durabilidad con replicación, y (5) integración empresarial. Topics utilizados: \texttt{telemetry}, \texttt{alarms}, \texttt{commands}.

Los scripts Python para Kafka Producer (MQTT→Kafka) y Kafka Consumer (ThingsBoard Edge) se documentan en el Anexo C.

\section{Seguridad}

\subsection{Public Key Infrastructure (PKI)}

Infraestructura de tres niveles: CA raíz (RSA 4096), certificados servidor (ECC P-256 con SAN), certificados cliente (ECC P-256 con CN=deviceID). LFDI (Long Form Device Identifier) derivado de certificado: \texttt{SHA-256(SubjectPublicKeyInfo)[0:20]}.

Los comandos OpenSSL completos para generación de CA, certificados servidor y certificados cliente se presentan en el Anexo D.

\subsection{TLS/mTLS}

Todas las conexiones externas usan TLS 1.3: (1) IEEE 2030.5 con mTLS obligatorio, (2) ThingsBoard Edge con JWT/TLS, (3) MQTT con TLS+ACL, y (4) Kafka con SASL/SSL. Cipher suites: \texttt{TLS\_AES\_256\_GCM\_SHA384} (AES-GCM), \texttt{TLS\_CHACHA20\_POLY1305\_SHA256} (ChaCha20).

\subsection{Firewall y Zonas}

OpenWRT nftables con cuatro zonas: \texttt{wan} (Ethernet/LTE, default REJECT), \texttt{lan} (br-lan, default ACCEPT), \texttt{thread} (wpan0, default DROP excepto MQTT), \texttt{halow} (wlan2, default DROP excepto CoAP/LwM2M).

\section{Resiliencia y Alta Disponibilidad}

\subsection{Almacenamiento Persistente (SSD NVMe)}

Todos los datos críticos en SSD: PostgreSQL/TimescaleDB (80 GB), ThingsBoard Edge data (20 GB), Docker volumes (30 GB), backups diarios (50 GB). Backup automático vía \texttt{cron}: PostgreSQL dump diario, TB Edge config export, certificados y claves.

\subsection{Failover WAN (Ethernet ↔ LTE)}

Configuración mwan3 con dos interfaces: \texttt{wan} (Ethernet, peso 10, check \texttt{ping 8.8.8.8}), \texttt{wwan} (LTE, peso 5, backup). Conmutación automática en <10 segundos.

\subsection{Buffering Local}

En caso de desconexión cloud, ThingsBoard Edge bufferiza: (1) telemetría hasta 7 días (límite disco), (2) comandos downlink encolados, (3) sincronización automática al reconectar. PostgreSQL como buffer persistente.

\section{Edge Computing y AI Local}

\subsection{Preprocesamiento de Datos}

Operaciones locales: (1) filtrado de outliers (límites ±3σ), (2) agregación temporal (promedios 15-min), (3) compresión gzip antes de transmisión cloud, y (4) detección de anomalías con reglas (voltaje <200V o >250V).

\subsection{Ollama LLM Local (Opcional)}

Modelo Llama 3.2:3B (2 GB RAM) para: (1) clasificación de alarmas, (2) generación de resúmenes de consumo, (3) respuestas automáticas a consultas simples, y (4) análisis de patrones. Inferencia: 50-80 tokens/s en Cortex-A72.

\section{Monitoreo y Gestión}

\subsection{Prometheus + Grafana}

Métricas recolectadas: CPU load, memoria libre, temperatura SoC, uptime, tráfico de red (bytes TX/RX por interfaz), latencia MQTT, mensajes Kafka/s, queries PostgreSQL/s. Dashboards Grafana con alertas configurables (correo/Telegram).

\subsection{Logging Centralizado}

Syslog-ng recolecta logs de: Docker containers, kernel, OpenWRT services, iptables firewall. Rotación automática (logrotate): 10 MB max por archivo, 7 días retención. Niveles: \texttt{error}, \texttt{warning}, \texttt{info}, \texttt{debug}.

\section{Pruebas y Validación}

\subsection{Pruebas de Conectividad}

Validación de cada interfaz: (1) Thread: 20 nodos ESP32-C6 transmitiendo cada 60s, (2) HaLow: 10 DCUs a 500m-1km con RSSI -70 a -85 dBm, (3) LTE: failover con desconexión simulada de Ethernet, (4) Cloud: transmisión continua 24h con 0 pérdidas.

\subsection{Pruebas de Latencia}

Latencias end-to-end medidas: Nodo Thread → Gateway MQTT → TB Edge Cloud: mediana 380 ms, p95 720 ms, p99 1.2 s. Componentes: Thread mesh 60 ms, OTBR bridge 20 ms, MQTT local 10 ms, WAN uplink 250-600 ms, TB Cloud processing 40 ms.

\subsection{Pruebas de Carga}

Capacidad del gateway: 200 nodos Thread (80% carga CPU), 1000 mensajes MQTT/s (40% RAM), 50 MB/h tráfico uplink (3% ancho de banda 4G LTE). Limitante: CPU Cortex-A72 para bridge MQTT.

\section{Comparación con Alternativas Comerciales}

Comparación gateway propuesto vs. soluciones comerciales (Cisco IXM-LPWA-800, Multitech Conduit, Kerlink iBTS):

\begin{itemize}
    \item \textbf{Costo}: USD \$180 (propuesto) vs. USD \$800-2000 (comercial)
    \item \textbf{Flexibilidad}: Plataforma abierta (OpenWRT/Docker) vs. firmware propietario
    \item \textbf{Protocolos}: Thread + HaLow + LTE vs. típicamente solo LoRaWAN/LTE
    \item \textbf{Edge computing}: Ollama LLM local vs. no disponible
    \item \textbf{Escalabilidad}: Hasta 200 nodos (suficiente para ~500-1000 medidores vía DCUs)
\end{itemize}

\section{Limitaciones y Consideraciones}

\begin{itemize}
    \item \textbf{CPU}: Cortex-A72 limitado para >500 nodos Thread directos (solución: DCUs intermedias)
    \item \textbf{Temperatura}: Operación -10°C a +50°C (requiere cooling en gabinete exterior)
    \item \textbf{Regulación espectro}: HaLow 915-928 MHz Región 2, verificar disponibilidad local
    \item \textbf{Soporte HaLow}: Driver ath11k en desarrollo, posibles bugs en kernel 5.15
    \item \textbf{Consumo}: 15W típico (Raspberry Pi + módems), requiere UPS/batería para blackout
\end{itemize}

\section{Conclusiones del Capítulo}

El gateway propuesto combina hardware asequible (Raspberry Pi 4), sistema operativo flexible (OpenWRT 23.05), y stack de software open-source (Docker, OTBR, ThingsBoard Edge, TimescaleDB) para implementar una solución completa de telemetría Smart Energy conforme a estándares IEEE 2030.5 e ISO/IEC 30141.

Las principales contribuciones son: (1) integración Thread + HaLow en plataforma única, (2) edge computing con AI local (Ollama), (3) arquitectura containerizada para mantenimiento simplificado, (4) conformidad estándar IEEE 2030.5 con API REST, y (5) costo reducido (80-90\% vs. soluciones comerciales).

La implementación detallada, incluyendo configuraciones UCI, scripts Python, esquemas de base de datos y código fuente de nodos IoT, se documenta completamente en los Anexos B, C, D y E.
