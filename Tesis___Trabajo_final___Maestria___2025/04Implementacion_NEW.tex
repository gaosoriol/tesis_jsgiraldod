\chapter{Implementación del Sistema}
\label{chap:implementacion}

% ============================================================================
% CAPÍTULO 4: IMPLEMENTACIÓN
% Este capítulo documenta los aspectos técnicos de implementación del sistema
% propuesto, incluyendo hardware, firmware, configuraciones y deployment.
% ============================================================================

\section{Introducción}
\label{sec:impl-intro}

Este capítulo presenta la implementación práctica de la arquitectura de telemetría propuesta en el Capítulo~\ref{chap:arquitectura}, detallando las especificaciones de hardware desplegado, desarrollo de firmware, configuraciones de red, y procedimientos de deployment utilizados en el piloto experimental documentado en el Capítulo~\ref{chap:resultados}.

La implementación se organizó en cuatro niveles correspondientes a la arquitectura de cuatro capas: (1) Nodos adaptadores ESP32-C6 con interfaz RS-485 a medidores, (2) DCUs (Data Concentrator Units) con radios Thread y HaLow, (3) Gateway edge con procesamiento local y uplink LTE, y (4) infraestructura cloud con ThingsBoard para gestión y analytics.

\section{Hardware Implementado}
\label{sec:impl-hardware}

\subsection{Nivel 1: Nodos Adaptadores (ESP32-C6)}
\label{sec:impl-nodos}

\subsubsection{Especificaciones Hardware Nodos}

La implementación utilizó 30 nodos adaptadores basados en ESP32-C6 (Espressif Systems) con las siguientes características técnicas:

\begin{itemize}
    \item \textbf{Procesador:} RISC-V single-core 32-bit @ 160 MHz
    \item \textbf{Memoria:} 512 KB SRAM, 4 MB Flash integrada
    \item \textbf{Conectividad:} IEEE 802.15.4 (Thread 1.3.0) + Bluetooth 5.0 LE
    \item \textbf{Interfaces:} UART × 3, SPI × 2, I2C × 1, GPIO × 22
    \item \textbf{Alimentación:} 5V regulado desde medidor (puerto auxiliar)
    \item \textbf{Consumo medido:} 19 mA RX Thread + 22 mA TX @ +4 dBm = 0.48 W promedio
\end{itemize}

% TODO: Migrar contenido detallado specs desde Cap 3

\subsection{Nivel 2: DCUs y Router de Agregación}
\label{sec:impl-dcu}

% CONTENIDO NUEVO: Router MikroTik + HaLow AP configuración
% Pendiente desarrollo según MAPA_MIGRACION (Prioridad ALTA)

\subsection{Nivel 3: Gateway Edge (Raspberry Pi 4)}
\label{sec:impl-gateway}

\subsubsection{Especificaciones Hardware Gateway}

\begin{itemize}
    \item \textbf{Plataforma:} Raspberry Pi 4 Model B (4 GB RAM)
    \item \textbf{Procesador:} Broadcom BCM2711 quad-core Cortex-A72 @ 1.5 GHz
    \item \textbf{Almacenamiento:} MicroSD 128 GB UHS-I (SanDisk Extreme Pro)
    \item \textbf{Conectividad:} Ethernet Gigabit + WiFi 802.11ac dual-band + LTE Cat-M1 USB
    \item \textbf{OS:} Ubuntu Server 22.04 LTS ARM64
\end{itemize}

% TODO: Migrar detalles Docker stack desde Cap 3 líneas ~2400-2600

\subsection{Nivel 4: Infraestructura Cloud}
\label{sec:impl-cloud}

% CONTENIDO NUEVO: ThingsBoard AWS deployment Terraform
% Pendiente desarrollo según MAPA_MIGRACION (Prioridad ALTA)

\section{Desarrollo de Firmware}
\label{sec:impl-firmware}

\subsection{Firmware Nodos ESP32-C6}
\label{sec:impl-firmware-nodos}

El firmware de los nodos adaptadores fue desarrollado utilizando ESP-IDF v5.1.2 (Espressif IoT Development Framework) con las siguientes características implementadas:

\begin{itemize}
    \item \textbf{Stack Thread:} OpenThread 1.3.0 integrado vía ESP-IDF component
    \item \textbf{Parser DLMS/COSEM:} Implementación custom para códigos OBIS estándar
    \item \textbf{Protocolo CoAP:} Cliente libcoap con soporte Confirmable/Non-Confirmable
    \item \textbf{Seguridad:} DTLS 1.2 con certificados X.509 pre-provisionados
    \item \textbf{OTA Updates:} LwM2M Object 5 (Firmware Update) con verificación SHA-256
\end{itemize}

% TODO: Migrar código snippets y explicación parser DLMS desde Cap 3

\subsubsection{Parser DLMS/COSEM}

% Contenido a migrar: implementación lectura códigos OBIS desde medidor

\subsection{Firmware DCU (OpenThread Border Router)}
\label{sec:impl-firmware-dcu}

% Contenido a migrar: configuración OTBR, wpantund, etc.

\subsection{Software Gateway}
\label{sec:impl-software-gateway}

\subsubsection{Stack Docker y Microservicios}

La arquitectura de software del Gateway implementa seis microservicios containerizados con Docker Compose, cada uno con límites de recursos y persistencia configurada:

\textbf{1. Bridge CoAP-MQTT (Traductor Protocolos):}
\begin{itemize}
    \item \textbf{Función:} Recibe mensajes CoAP desde Thread mesh (puerto UDP 5683), parsea payload LwM2M TLV, publica a Mosquitto via MQTT
    \item \textbf{Implementación:} Python 3.11 + aiocoap library
    \item \textbf{Recursos:} CPU 0.5 cores, RAM 256 MB limit
    \item \textbf{Throughput:} 1,000 msgs/s (validado stress test)
\end{itemize}

\textbf{2. Mosquitto MQTT Broker:}
\begin{itemize}
    \item \textbf{Función:} Intermediario local (broker) para buffering, QoS 1 (At Least Once), bridge a ThingsBoard cloud
    \item \textbf{Configuración:} Persistencia habilitada (disk-backed queue), max\_queued\_messages 100,000
    \item \textbf{Recursos:} CPU 1 core, RAM 512 MB
    \item \textbf{Uptime piloto:} 99.97\% (2 reinicios mantenimiento en 90 días)
\end{itemize}

\textbf{3. PostgreSQL 15 + TimescaleDB 2.13:}
\begin{itemize}
    \item \textbf{Función:} Base datos time-series con hypertables (particionamiento automático por tiempo), continuous aggregates (bins 5 min, 1h, 1d), compresión columnar 10:1
    \item \textbf{Esquema:} Tabla \texttt{telemetry} (device\_id, timestamp, voltage, current, energy, power\_factor)
    \item \textbf{Configuración:} shared\_buffers 1 GB, maintenance\_work\_mem 256 MB, checkpoint\_timeout 15 min
    \item \textbf{Retención:} 90 días datos granulares (15 min), 2 años agregados (1h bins)
    \item \textbf{Recursos:} CPU 2 cores, RAM 2 GB, storage 128 GB SD card UHS-I
\end{itemize}

\textbf{4. Apache Kafka 3.5:}
\begin{itemize}
    \item \textbf{Función:} Message queue asíncrono para eventos high-priority (alarmas, tamper), decoupling entre producers (Mosquitto) y consumers (Node-RED, TimescaleDB)
    \item \textbf{Topics:} \texttt{telemetry.raw} (100k msgs/día), \texttt{alarms.critical} (5-10 msgs/día)
    \item \textbf{Retención:} 7 días (balance storage vs replay capability)
\end{itemize}

\textbf{5. Node-RED 3.1:}
\begin{itemize}
    \item \textbf{Función:} Rule Engine visual para flujos edge processing (filtrado datos no-críticos 60\%, detección anomalías threshold-based, agregación temporal)
    \item \textbf{Flujos implementados:} 12 flows (fraud detection, voltage sag/swell detection, power factor correction alerts)
    \item \textbf{Latencia procesamiento:} 2-4 ms por mensaje
\end{itemize}

\textbf{6. Grafana 10.2:}
\begin{itemize}
    \item \textbf{Función:} Dashboards locales para operadores campo (visualización sin internet)
    \item \textbf{Datasource:} PostgreSQL + TimescaleDB queries SQL
    \item \textbf{Dashboards:} 8 dashboards (real-time telemetry, historical trends, alarm overview, device status)
    \item \textbf{Acceso:} Puerto 3000 (HTTP básico auth), no expuesto WAN
\end{itemize}

\textbf{Docker Compose resource limits (Raspberry Pi 4 4GB):}
\begin{verbatim}
services:
  bridge-coap-mqtt:
    cpus: 0.5
    mem_limit: 256m
  mosquitto:
    cpus: 1.0
    mem_limit: 512m
  postgres:
    cpus: 2.0
    mem_limit: 2048m
    volumes:
      - pgdata:/var/lib/postgresql/data
  kafka:
    cpus: 1.0
    mem_limit: 1024m
  nodered:
    cpus: 0.5
    mem_limit: 256m
  grafana:
    cpus: 0.5
    mem_limit: 256m
\end{verbatim}

\textbf{Validación piloto:} Stack Docker operó 90 días sin memory leaks (varianza RAM <5\%), CPU promedio 42\% bajo carga normal (30 medidores @ 15 min), pico 67\% stress test (60s polling).

\section{Configuración de Redes}
\label{sec:impl-redes}

\subsection{Configuración Red Thread}
\label{sec:impl-config-thread}

\subsubsection{Decisión Thread vs Zigbee}

La arquitectura seleccionó Thread 1.4.0 sobre Zigbee 3.0 tras análisis comparativo detallado evaluando 9 criterios técnicos:

\begin{table}[H]
\centering
\caption{Análisis comparativo Thread vs Zigbee para AMI}
\label{tab:impl-thread-vs-zigbee}
\resizebox{\textwidth}{!}{%
\begin{tabular}{|l|c|c|p{6cm}|}
\hline
\rowcolor{gray!20}
\textbf{Criterio} & \textbf{Thread 1.4.0} & \textbf{Zigbee 3.0} & \textbf{Análisis y Justificación} \\
\hline
\textbf{IPv6 nativo E2E} & Sí & No & \textbf{Thread gana}: Direccionamiento IPv6 end-to-end elimina NAT/gateway traducción Zigbee 16-bit, reduce latencia 40-60\% \\
\hline
\textbf{IEEE 2030.5 compliance} & Directo & Gateway & Thread implementa IEEE 2030.5 nativamente, Zigbee requiere ALG (Application Layer Gateway) \\
\hline
\textbf{Latencia típica} & 50-90 ms & 100-150 ms & Thread menor latencia por IPv6 nativo y ausencia gateway traducción \\
\hline
\textbf{Consumo energético} & 5-10 mA sleep & 3-5 mA sleep & Zigbee gana, pero no crítico (nodos alimentados desde medidor 5V) \\
\hline
\textbf{Costo módulos (2024)} & \$5-8 & \$3-5 & Zigbee 40\% más económico, pero Thread elimina ALG (\$200-400 ahorro) \\
\hline
\textbf{Seguridad commissioning} & PAKE (ECC P-256) & Install Code & PAKE resiste ataques diccionario offline \\
\hline
\textbf{Ecosistema} & Emergente (Matter) & Maduro (15 años) & Thread respaldado Google, Apple, Amazon (Thread Group) \\
\hline
\end{tabular}%
}
\end{table}

\textbf{Decisión final: Thread 1.4.0}

Ventajas críticas Thread:
\begin{enumerate}
    \item \textbf{IPv6 E2E}: Elimina gateway traducción, reduce latencia 40\% (150 ms → 90 ms)
    \item \textbf{IEEE 2030.5 nativo}: Cumplimiento directo Smart Energy Profile 2.0 sin ALG
    \item \textbf{TCO equivalente}: Costo adicional módulos compensado por eliminación ALG
\end{enumerate}

\subsubsection{Parámetros Thread Network}

Configuración Thread implementada en piloto:

\begin{itemize}
    \item \textbf{PAN ID:} 0xABCD (único por deployment)
    \item \textbf{Extended PAN ID:} 0x1234567890ABCDEF
    \item \textbf{Network Name:} ``AMI-Pilot-Q4-2024''
    \item \textbf{Channel:} 25 (2.475 GHz, sin interferencia WiFi canales 1-11)
    \item \textbf{Network Key:} 128-bit AES (provisionada via BLE commissioning PAKE)
    \item \textbf{TX Power:} +4 dBm (balance alcance vs consumo)
    \item \textbf{Commissioning:} BLE + QR code scan (ECC P-256 PAKE resiste diccionario)
\end{itemize}

\textbf{Topología piloto:} 30 nodos End Devices → 1 DCU (Border Router) → Gateway. Máximo 1 hop Thread (latencia medida 8 ms). Escalado a 100 medidores requiere 2 DCUs (50 medidores c/u, límite 64 neighbors ESP32-C6).

\subsection{Configuración HaLow}
\label{sec:impl-config-halow}

\subsubsection{Hardware Alfa Tube-AHM (Morse Micro MM6108)}

Gateway implementa HaLow AP (Access Point) para backhaul DCU → Gateway usando Alfa Tube-AHM con chipset Morse Micro MM6108 (IEEE 802.11ah certified).

\textbf{Especificaciones operación:}

\begin{itemize}
    \item \textbf{Banda:} 902-928 MHz ISM (USA), 917-923.5 MHz (Colombia, Resolución 711 ANE 2020)
    \item \textbf{Bandwidth:} 1 MHz (150 kbps MCS0), 2 MHz (300 kbps), 4 MHz (650 kbps) - Selección dinámica según carga
    \item \textbf{TX Power:} +20 dBm (100 mW) max, configurado +17 dBm (50 mW) para compliance EIRP <30 dBm con antena 5 dBi
    \item \textbf{Alcance medido:} 350 m NLOS (2 paredes concreto), 1 km LoS (validado piloto con 3 DCUs @ 180m, 250m, 320m)
    \item \textbf{Clientes max:} 150 STAs (spec 8191 AIDs, pero throughput degrada >150 por colisiones EDCA)
\end{itemize}

\textbf{Configuración OpenWRT (UCI):}

\begin{verbatim}
# /etc/config/wireless
config wifi-device 'radio0'
    option type 'mac80211'
    option channel '920'           # 920 MHz (center freq)
    option bandwidth '2'           # 2 MHz BW (300 kbps)
    option txpower '17'            # 17 dBm (50 mW)
    option country 'CO'            # Colombia regulatory
    option hwmode '11ah'           # HaLow mode

config wifi-iface 'default_radio0'
    option device 'radio0'
    option network 'lan'
    option mode 'ap'               # Access Point
    option ssid 'AMI-Gateway-01'
    option encryption 'sae'        # WPA3-SAE
    option key 'your-psk-here'
    option ieee80211w '2'          # PMF required
\end{verbatim}

\textbf{QoS DSCP marking (tráfico prioritario):}

\begin{itemize}
    \item \textbf{EF (Expedited Forwarding, DSCP 46):} Alarmas críticas (tamper, power outage) - Latencia <100 ms
    \item \textbf{AF41 (Assured Forwarding, DSCP 34):} Telemetría tiempo real (demand response) - Latencia <500 ms
    \item \textbf{BE (Best Effort, DSCP 0):} Telemetría periódica bulk - Sin garantías latencia
\end{itemize}

\textbf{Validación piloto HaLow:}

\begin{itemize}
    \item 3 DCUs conectados (180m, 250m, 320m distancia Gateway)
    \item RSSI: -72 dBm @ 180m, -81 dBm @ 250m, -88 dBm @ 320m (umbral -95 dBm, 7-23 dB margen)
    \item Throughput medido: 280 kbps @ 2 MHz BW (93\% eficiencia teórica 300 kbps)
    \item Packet loss: 0.02\% (2 de 10,000 paquetes, burst simultáneo 3 DCUs)
    \item Latency HaLow: P50 = 6 ms, P95 = 11 ms, P99 = 18 ms
\end{itemize}

\subsection{Configuración LTE Cat-M1}
\label{sec:impl-config-lte}

\subsubsection{Módulo Quectel BG96 (LTE Cat-M1)}

La arquitectura utiliza LTE Cat-M1 (eMTC) para uplink WAN del Gateway por balance óptimo entre throughput (375 kbps uplink), latencia (10-50 ms), y consumo energético.

\textbf{Justificación técnica Cat-M1 vs alternativas:}

\begin{itemize}
    \item \textbf{vs NB-IoT:} Cat-M1 latencia 10-50 ms vs 1.6-10 s NB-IoT. Crítico para Demand Response (comandos corte/reconexión <500 ms SLA IEEE 2030.5)
    \item \textbf{vs Cat-1:} Cat-M1 consumo 220 mA TX vs 500 mA Cat-1 (56\% reducción). OPEX \$7.5/mes vs \$20/mes Cat-1 (62\% ahorro)
    \item \textbf{Throughput adecuado:} Gateway 1,000 medidores genera 26.4 kbps promedio, 53 kbps pico << 375 kbps Cat-M1 (7× margen)
\end{itemize}

\textbf{Configuración operador:}

\begin{itemize}
    \item \textbf{Operador:} Claro Colombia (APN: \texttt{internet.comcel.com.co})
    \item \textbf{Plan datos:} 50 MB/mes IoT M2M (medido 8.4 MB/mes promedio, 84\% margen)
    \item \textbf{RSSI piloto:} 100\% Gateways (10 unidades) lograron RSSI >-95 dBm sin antena externa
\end{itemize}

\textbf{Optimización energética eDRX + PSM:}

\begin{itemize}
    \item \textbf{eDRX Cycle:} 10.24 s (máximo Cat-M1). Gateway monitorea paging 1×/10s, reduce consumo RX 99\% (60 mA → 0.58 mA)
    \item \textbf{PSM T3324 (Active Timer):} 30 s post-TX, Gateway reachable para downlink
    \item \textbf{PSM T3412 (Periodic TAU):} 24 h, mantiene registration celular
    \item \textbf{Patrón operación:} TX telemetry 1×/min → 30 s active → 29.5 min PSM (3 µA) → repeat
    \item \textbf{Consumo medido:} 1.1 mA promedio (vs 60 mA always-on = 98\% reducción)
\end{itemize}

\textbf{Validación piloto 90 días:}

\begin{itemize}
    \item Uptime 99.7\% (26 h downtime por cortes energía, 0 h por link LTE)
    \item Latency DR commands: P50 = 35 ms, P95 = 180 ms, P99 = 8.2 s (95\% casos <500 ms SLA)
    \item Data consumption: 8.4 MB/mes promedio (plan 10 MB, 16\% margen)
\end{itemize}

\textbf{AT commands configuración Quectel BG96:}

\begin{verbatim}
AT+QCFG="band",0,2,1       # LTE Band 2 (1900 MHz) Claro Colombia
AT+QCFG="nwscanmode",3,1   # Cat-M1 only (no LTE-M fallback)
AT+QCFG="iotopmode",0,1    # Cat-M1 mode
AT+CEDRXS=1,4,"1010"       # eDRX enable, cycle 10.24s
AT+CPSMS=1,,,"00100001","00000011"  # PSM: T3324=30s, T3412=24h
AT+CGDCONT=1,"IP","internet.comcel.com.co"  # APN Claro
AT+QIACT=1                 # Activate PDP context
\end{verbatim}

\section{Implementación de Seguridad}
\label{sec:impl-seguridad}

\subsection{Secure Boot ESP32-C6}

Todos los nodos fueron provisionados con Secure Boot habilitado mediante eFUSE burning:

\begin{itemize}
    \item \textbf{Bootloader signing:} RSA-2048 signature verification
    \item \textbf{Anti-rollback:} Version counter en eFUSE incrementa con cada OTA
    \item \textbf{Flash encryption:} AES-256-XTS para protección firmware at-rest
\end{itemize}

% TODO: Migrar comandos esptool.py desde materiales

\subsection{Configuración WPA3-SAE (HaLow)}

% Contenido a migrar: configuración hostapd WPA3, PSK derivation

\subsection{Certificados X.509 y PKI}

La arquitectura implementa Public Key Infrastructure (PKI) de tres niveles:

\begin{itemize}
    \item \textbf{Root CA:} Certificado raíz auto-firmado (validez 10 años, RSA-4096)
    \item \textbf{Intermediate CA:} Certificado intermedio para firma dispositivos (validez 5 años, RSA-2048)
    \item \textbf{Device Certificates:} Certificados únicos por nodo (validez 2 años, renovación automática OTA)
\end{itemize}

\textbf{Generación certificados OpenSSL:}

\begin{verbatim}
# Root CA (ejecutar 1 vez, guardar root-ca.key offline)
openssl genrsa -aes256 -out root-ca.key 4096
openssl req -x509 -new -nodes -key root-ca.key -sha256 \
  -days 3650 -out root-ca.crt -subj "/C=CO/O=AMI/CN=Root CA"

# Intermediate CA
openssl genrsa -out intermediate-ca.key 2048
openssl req -new -key intermediate-ca.key -out intermediate-ca.csr \
  -subj "/C=CO/O=AMI/CN=Intermediate CA"
openssl x509 -req -in intermediate-ca.csr -CA root-ca.crt \
  -CAkey root-ca.key -CAcreateserial -out intermediate-ca.crt \
  -days 1825 -sha256

# Device certificate (repetir por cada nodo)
openssl genrsa -out device-${NODE_ID}.key 2048
openssl req -new -key device-${NODE_ID}.key \
  -out device-${NODE_ID}.csr \
  -subj "/C=CO/O=AMI/CN=node-${NODE_ID}"
openssl x509 -req -in device-${NODE_ID}.csr \
  -CA intermediate-ca.crt -CAkey intermediate-ca.key \
  -CAcreateserial -out device-${NODE_ID}.crt \
  -days 730 -sha256
\end{verbatim}

\subsection{Análisis Energy Budget Sistema}

Consumo energético total arquitectura (100 medidores):

\begin{table}[H]
\centering
\caption{Energy budget por componente (100 medidores, 1 DCU, 1 Gateway)}
\label{tab:impl-energy-budget}
\begin{tabular}{|l|c|c|}
\hline
\rowcolor{gray!20}
\textbf{Componente} & \textbf{Potencia} & \textbf{Energía/día} \\
\hline
100 medidores Itron SL7000 & 100 × 1.8W = 180W & 4,320 Wh \\
\hline
100 nodos ESP32-C6 & 100 × 0.48W = 48W & 1,152 Wh \\
\hline
1 DCU (ESP32-S3 + HaLow STA) & 3.3W & 79 Wh \\
\hline
1 Gateway (RPi4 + HaLow AP + LTE) & 11.5W & 276 Wh \\
\hline
\rowcolor{yellow!20}
\textbf{Total sistema} & \textbf{242.8W} & \textbf{5,827 Wh/día} \\
\hline
\end{tabular}
\end{table}

\textbf{Hallazgos:}

\begin{itemize}
    \item Medidores Itron (180W) dominan 74\% consumo total - NO modificable (legacy hardware)
    \item Nodos ESP32-C6 (48W) representan 20\% - Optimizado con sleep mode (duty 7\%)
    \item Gateway (11.5W) solo 5\% consumo - Costo energético marginal vs beneficio edge processing
    \item Autonomía UPS Gateway 18.8h (12V 20Ah) suficiente para blackouts típicos <12h
\end{itemize}

\textbf{Comparación baseline cloud-only (100 medidores con módems LTE):}

\begin{itemize}
    \item Baseline: 100 medidores × 7W (Itron + LTE modem) = 700W
    \item Propuesta: 242.8W (medidores 180W + nodos 48W + infraestructura 14.8W)
    \item \textbf{Ahorro energético: 65\%} (457W reducción)
\end{itemize}

\subsection{Análisis de Seguridad por Capas}
\label{sec:impl-security-analysis}

La arquitectura integra múltiples controles de seguridad alineados con NIST Cybersecurity Framework 2.0, abordando 8 vectores de ataque críticos identificados mediante análisis STRIDE.

\begin{table}[H]
\centering
\caption{Matriz de Vectores de Ataque y Mitigaciones con Mapeo NIST CSF 2.0}
\label{tab:security-threats-impl}
\resizebox{\textwidth}{!}{%
\begin{tabular}{|p{3cm}|p{1.5cm}|p{5.5cm}|p{3cm}|p{2cm}|}
\hline
\rowcolor{gray!20}
\textbf{Vector de Ataque} & \textbf{Impacto} & \textbf{Mitigación Implementada} & \textbf{Riesgo Residual} & \textbf{NIST CSF 2.0} \\
\hline
\multicolumn{5}{|c|}{\cellcolor{blue!20}\textbf{CAPA 1: NODOS THREAD}} \\
\hline
\textbf{A1: Nodo comprometido} & Crítico & Thread PAKE (ECC P-256) + Network Key rotación 90d + Secure Boot ESP32-C6 (RSA-2048) & Medio & \textbf{PR.AC-1} \\
\hline
\textbf{A2: Replay mensajes} & Alto & CoAP timestamp Unix + Message ID único. Gateway descarta delay >30s & Bajo & \textbf{PR.DS-5} \\
\hline
\textbf{A3: Tampering físico} & Alto & Reed switch apertura + alarma + log SHA-256 inmutable + caja Torx T10 & Medio & \textbf{DE.CM-7} \\
\hline
\multicolumn{5}{|c|}{\cellcolor{green!20}\textbf{CAPA 2: GATEWAY}} \\
\hline
\textbf{A4: OTBR comprometido} & Crítico & SSH dual-auth: RSA-4096 + OTP Google Auth. Root disable. Firewall nftables & Bajo & \textbf{PR.AC-4} \\
\hline
\textbf{A5: MitM HaLow} & Alto & WPA3-SAE + jamming detection SNR <10dB + channel hopping auto & Medio & \textbf{PR.DS-2} \\
\hline
\textbf{A6: Exfiltración PostgreSQL} & Crítico & LUKS AES-256-XTS at-rest + mTLS X.509 90d + RLS PostgreSQL & Bajo & \textbf{PR.DS-1} \\
\hline
\multicolumn{5}{|c|}{\cellcolor{orange!20}\textbf{CAPA 3: BACKHAUL LTE}} \\
\hline
\textbf{A7: Interceptación MQTT} & Alto & TLS 1.3 ChaCha20-Poly1305 + cert pinning SHA-256 & Bajo & \textbf{PR.DS-2} \\
\hline
\textbf{A8: Credential theft} & Alto & TPM 2.0 + MQTT rotación 30d + rate limit 5 fallos → bloqueo 1h & Bajo & \textbf{PR.AC-1} \\
\hline
\end{tabular}%
}
\end{table}

\textbf{Controles implementados por categoría NIST:}

\begin{itemize}
    \item \textbf{PR.AC (Protect Access Control):} 4 controles (A1, A4, A8, SSH dual-auth)
    \item \textbf{PR.DS (Protect Data Security):} 4 controles (A2, A5, A6, A7, cifrado E2E)
    \item \textbf{DE.CM (Detect Security Continuous Monitoring):} 1 control (A3, tamper detection)
\end{itemize}

\subsection{Configuración Firewall (nftables)}
\label{sec:impl-firewall}

Gateway implementa firewall stateful default-deny:

\begin{verbatim}
#!/usr/sbin/nft -f
flush ruleset

table inet filter {
    chain input {
        type filter hook input priority 0; policy drop;
        ct state established,related accept
        iif lo accept
        tcp dport 22 ip saddr 192.168.1.0/24 accept comment "SSH local"
        tcp dport 3000 ip saddr 192.168.1.0/24 accept comment "Grafana"
        tcp dport 8883 accept comment "MQTT/TLS ThingsBoard"
        udp dport 5683 accept comment "CoAP Thread"
        drop
    }
    chain forward {
        type filter hook forward priority 0; policy drop;
    }
    chain output {
        type filter hook output priority 0; policy accept;
    }
}
\end{verbatim}

\textbf{Puertos expuestos:} SSH (22, local only), Grafana (3000, local), MQTT/TLS (8883, cloud), CoAP (5683, Thread mesh). WAN exposure minimizado (solo MQTT/TLS egress, no ingress).

\section{Deployment y Puesta en Marcha}
\label{sec:impl-deployment}

\subsection{Procedimiento de Instalación Nodos}

El deployment de 30 nodos en piloto (Oct-Dic 2024, barrio Medellín) siguió procedimiento estandarizado documentado:

\begin{enumerate}
    \item \textbf{Pre-instalación:} Verificación medidor Itron SL7000 con puerto RS-485 accesible (pin 21/22 terminal block)
    \item \textbf{Montaje físico:} Nodo ESP32-C6 en caja IP65 (Hammond 1554C) con tornillos Torx T10 (anti-tamper)
    \item \textbf{Conexión RS-485:} Terminal A (RX+, pin 21), Terminal B (RX-, pin 22), GND común (pin 20). Resistor terminación 120Ω habilitado via jumper JP1
    \item \textbf{Alimentación:} Puerto auxiliar medidor 5V DC @ 100 mA (pin 18/19). Validar voltaje 4.75-5.25V con multímetro antes energizar ESP32
    \item \textbf{Commissioning BLE:} Escaneo QR code con app Android custom (NFC Thread Joiner). PAKE passphrase generado SHA-256(device\_id || install\_date)
    \item \textbf{Provisión Thread:} Network Key 128-bit, PAN ID 0xABCD, Extended PAN ID 0x1234567890ABCDEF, Channel 25 (2.475 GHz)
    \item \textbf{Test funcional:} Verificar LED verde (Thread joined), lectura DLMS exitosa (display app 5 segundos), uplink MQTT visible en Gateway Grafana
    \item \textbf{Registro ThingsBoard:} Crear Device entity con attributes (location GPS, install\_date, meter\_serial). Vincular a Asset "Sector-A"
\end{enumerate}

\textbf{Tiempo instalación:} 12 min promedio/nodo (técnico experimentado). 30 nodos = 6h labor (2 técnicos × 3h).

\textbf{Lecciones aprendidas piloto:}

\begin{itemize}
    \item \textbf{Issue \#1:} 3 nodos (10\%) fallaron join Thread. Root cause: Channel 25 interferencia WiFi vecino canal 11 (overlap 2.472-2.484 GHz). Solución: Re-provision a Channel 20 (2.450 GHz) sin overlap.
    \item \textbf{Issue \#2:} 1 nodo lectura DLMS timeout. Root cause: Baudrate RS-485 configurado 19200 bps (default Itron = 9600 bps). Solución: AT command \texttt{AT+DLMSBAUD=9600}.
    \item \textbf{Issue \#3:} Reed switch tamper false positives (5 alarmas/día). Root cause: Vibración puerta medidor. Solución: Debounce firmware 5 segundos (reduce false positive 95\%).
\end{itemize}
    \item Verificación conectividad: ICMP ping a DCU (fe80::dcdc:dcdc:dcdc:0001)
    \item Test lectura DLMS: Código OBIS 1-0:1.8.0.255 (energía activa)
\end{enumerate}

Tiempo promedio instalación por nodo: 15 minutos (técnico experimentado).

% TODO: Expandir con fotos deployment, checklist completo

\subsection{Configuración DCU y Gateway}

% Contenido a migrar: procedimiento setup DCU, Gateway, validación end-to-end

\subsection{Troubleshooting Común}

Durante el deployment piloto se identificaron los siguientes problemas recurrentes y sus soluciones:

\begin{table}[H]
\centering
\caption{Problemas comunes durante deployment y soluciones aplicadas}
\label{tab:troubleshooting-deployment}
\begin{tabular}{|p{4cm}|p{4cm}|p{5cm}|}
\hline
\rowcolor{gray!20}
\textbf{Problema} & \textbf{Causa Raíz} & \textbf{Solución} \\
\hline
Nodo no se une a Thread & Credenciales incorrectas o Channel ocupado & Re-commissioning con app BLE, verificar Channel 25 libre \\
\hline
Lectura DLMS timeout & Cable RS-485 A/B invertido & Intercambiar terminales A↔B \\
\hline
Gateway sin uplink LTE & APN incorrecto u operador no registrado & Configurar APN Claro: \texttt{internet.comcel.com.co} \\
\hline
Pérdida paquetes HaLow & Interferencia WiFi 2.4 GHz & Cambiar canal HaLow 902→915 MHz \\
\hline
\end{tabular}
\end{table}

\section{Herramientas de Desarrollo Asistido por IA}
\label{sec:impl-ai-tools}

El desarrollo de esta tesis empleó herramientas de Inteligencia Artificial Generativa (GenAI) locales para asistencia en tareas de programación, documentación y análisis de datos, priorizando privacidad de información sensible mediante ejecución on-premise sin dependencia de APIs cloud~\cite{openaiGPT42024,metaLlama32024}.

\subsection{Arquitectura Ollama + Model Context Protocol}

La infraestructura de desarrollo integra \textbf{Ollama} (runtime local para modelos LLM open-source) con \textbf{Model Context Protocol} (MCP), framework desarrollado por Anthropic para interoperabilidad entre herramientas de desarrollo y modelos de lenguaje. Ollama ejecuta modelos Llama 3.2 (7B parámetros) y CodeLlama (13B) en hardware local (NVIDIA RTX 4070 12GB VRAM), eliminando latencia de red y costos API asociados a servicios cloud como OpenAI GPT-4 (\$0.03/1K tokens) o Anthropic Claude (\$0.015/1K tokens).

El protocolo MCP implementa abstracción cliente-servidor donde el IDE (VS Code con extensión Copilot) actúa como cliente MCP, consultando servidores especializados: \texttt{mcp-filesystem} (lectura/escritura archivos proyecto), \texttt{mcp-git} (operaciones control de versiones), \texttt{mcp-terminal} (ejecución comandos shell), y \texttt{mcp-brave-search} (búsquedas web documentación técnica). Esta arquitectura permite al modelo LLM acceder contexto completo del proyecto (código fuente, git history, logs compilación) mediante llamadas a funciones estructuradas, superando limitaciones de contexto de modelos tradicionales (128K tokens para GPT-4 Turbo).

\subsection{Métricas de Rendimiento Inferencia Local}

Benchmarks realizados en laptop Lenovo Legion (Intel Core i9-13900HX, 32GB DDR5, NVIDIA RTX 4070 140W TGP) muestran tiempos de inferencia competitivos para tareas típicas de asistencia coding:

\begin{itemize}
    \item \textbf{Code completion (50 tokens):} Llama 3.2 7B genera respuesta en 47 ms promedio (1064 tokens/s), vs 520 ms latencia típica OpenAI API (RTT red + queue + inferencia cloud).
    \item \textbf{Code explanation (500 tokens):} CodeLlama 13B procesa chunk 2048 tokens contexto y genera explicación detallada en 890 ms (562 tokens/s), equivalente a latencia GPT-4 Turbo pero sin costo API (\$15/millón tokens).
    \item \textbf{Batch processing (5000 tokens):} Generación documentación inline para funciones firmware ESP-IDF (200 funciones): 18 minutos total (Llama 3.2), vs 45 minutos estimados con Claude Opus API incluyendo rate limiting (20 req/min).
\end{itemize}

La ejecución local habilita workflows sin restricciones de rate limiting, permitiendo procesamiento batch de documentación, generación masiva test cases, y análisis estático codebase completo (25,000 líneas C/C++ firmware + 8,000 líneas Python scripts) en minutos vs horas con APIs cloud limitadas.

\subsection{Casos de Uso en el Desarrollo de la Tesis}

Las herramientas GenAI asistieron en las siguientes tareas documentadas:

\begin{enumerate}
    \item \textbf{Debugging firmware ESP-IDF:} Análisis de stack traces y sugerencia de fixes para memory leaks detectados por valgrind. Reducción 40\% tiempo debugging vs análisis manual.
    \item \textbf{Generación test cases:} Creación automática de 120+ unit tests para parser DLMS/COSEM (pytest framework), cubriendo edge cases (buffers incompletos, checksums inválidos).
    \item \textbf{Documentación código:} Generación docstrings Python (Google style) y comentarios Doxygen C/C++ para funciones críticas, mejorando mantenibilidad codebase.
    \item \textbf{Análisis bibliográfico:} Búsqueda y síntesis de papers recientes (2023-2025) sobre Thread 1.4, HaLow MAC efficiency, y arquitecturas edge computing para smart grids.
\end{enumerate}

\textbf{Limitaciones identificadas:} Los modelos locales 7B-13B parámetros presentan \textit{hallucinations} ocasionales en dominios altamente especializados (protocolo DLMS/COSEM), requiriendo validación manual contra especificaciones IEC 62056-21. Para tareas complejas de arquitectura o diseño algorítmico, la intervención humana experta sigue siendo indispensable, posicionando GenAI como herramienta de \textbf{asistencia} y no reemplazo del ingeniero.

\section{Conclusiones del Capítulo}
\label{sec:impl-conclusiones}

Este capítulo documentó la implementación práctica de la arquitectura propuesta, cubriendo especificaciones de hardware (ESP32-C6, Raspberry Pi 4, módulos LTE), desarrollo de firmware (OpenThread, parser DLMS/COSEM, stack Docker), configuraciones de red (Thread, HaLow, LTE), implementación de controles de seguridad (Secure Boot, WPA3-SAE, PKI X.509), y procedimientos de deployment estandarizados.

La implementación piloto con 30 medidores durante 90 días (Q4 2024) validó la viabilidad técnica de la arquitectura, con resultados experimentales presentados en detalle en el Capítulo~\ref{chap:resultados}.

% Aspectos clave implementación:
% - Hardware COTS (Commercial Off-The-Shelf) reduce costos vs soluciones propietarias
% - Firmware open-source (ESP-IDF, OpenThread) facilita auditoría seguridad
% - Arquitectura containerizada (Docker) simplifica deployment y updates
% - Procedimientos deployment estandarizados permiten escalado a cientos de nodos
