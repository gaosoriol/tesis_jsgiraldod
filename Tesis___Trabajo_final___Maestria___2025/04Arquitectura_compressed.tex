\chapter{Arquitectura de Telemetría para Smart Energy}

\section{Introducción}

Este capítulo presenta la arquitectura completa del sistema de telemetría propuesto, integrando los componentes descritos en el capítulo anterior (Gateway) en una solución end-to-end escalable y segura para aplicaciones Smart Energy.

\section{Modelo de Tres Capas IoT}

La arquitectura sigue el modelo ISO/IEC 30141 con tres capas claramente definidas:

\subsection{Nivel 1: Nodos IoT (Sensores/Actuadores)}

\textbf{Hardware}: ESP32-C6 con radio Thread 802.15.4 integrado, transceptor RS-485 (MAX485), alimentación 5V desde medidor o batería + supercap, antena PCB 2.4 GHz.

\textbf{Software}: OpenThread stack (ESP-IDF), cliente LwM2M AVSystems Anjay para gestión remota (objetos 3, 3303, 3304), cliente DLMS simplificado para lectura OBIS, sleep modes (<10 µA en deep sleep).

\textbf{Función}: Actúan como puente entre medidor (RS-485/DLMS) y red Thread mesh, leyendo perfiles de carga cada 15 min y transmitiendo vía IPv6/6LoWPAN al DCU.

La implementación completa del nodo IoT ESP32-C6 con LwM2M, incluyendo todos los archivos fuente, se documenta en el Anexo E.

\subsection{Nivel 2: Gateway Edge (Data Aggregator)}

\textbf{Hardware}: Raspberry Pi 4 Model B (4 GB RAM, 256 GB NVMe SSD), nRF52840 USB Dongle (Thread RCP), Newracom MM6108 (HaLow 802.11ah SPI), Quectel BG95-M3 (LTE backup).

\textbf{Software}: OpenWRT 23.05, Docker 20.10.24, servicios containerizados: OTBR, ThingsBoard Edge 3.6.0, PostgreSQL 15 + TimescaleDB, IEEE 2030.5 Server (Python/Flask), Bridge Thread↔ThingsBoard, Apache Kafka 7.5.0.

\textbf{Función}: Agregación de datos de múltiples DCUs, traducción de protocolos (Thread/CoAP → MQTT/HTTP), edge computing con preprocesamiento local, buffering offline hasta 7 días, sincronización cloud automática.

El gateway se describe en detalle en el Capítulo 3. Las configuraciones Docker se presentan en el Anexo B, y los scripts Python en el Anexo C.

\subsection{Nivel 3: Plataforma Cloud (Analytics y Visualización)}

\textbf{Plataforma}: ThingsBoard Professional Edition (PE) o Community Edition (CE) con PostgreSQL/TimescaleDB backend, Redis para caché, Kafka para message bus, Zookeeper para coordinación.

\textbf{Funcionalidades}: Ingesta de telemetría vía MQTT/HTTP, dashboards en tiempo real, reglas de procesamiento (Rule Engine), alertas configurables (email/SMS/Telegram), API REST para integraciones, control remoto downlink (comandos RPC).

\textbf{Escalabilidad}: Clúster horizontal con load balancer (Nginx/HAProxy), soporte 10,000+ dispositivos por nodo, agregaciones pre-computadas con TimescaleDB continuous aggregates.

\section{Flujo de Datos End-to-End}

\subsection{Uplink (Telemetría)}

\begin{enumerate}
    \item \textbf{Medidor}: Almacena lectura en registro OBIS (ej. \texttt{1.0.1.8.0.255} para energía activa total)
    \item \textbf{Nodo IoT}: Lee vía DLMS/COSEM sobre RS-485 cada 15 min, encapsula en JSON
    \item \textbf{Thread mesh}: Transmite paquete CoAP/UDP/IPv6 al DCU (latencia ~60 ms)
    \item \textbf{OTBR}: Bridge Thread→Ethernet convierte CoAP a MQTT topic \texttt{thread/telemetry/<node\_id>}
    \item \textbf{Gateway}: Recibe MQTT local, agrega timestamp/geolocalización, publica a TB Edge
    \item \textbf{ThingsBoard Edge}: Buffer local, sincroniza con TB Cloud cada 60s vía MQTT/TLS
    \item \textbf{ThingsBoard Cloud}: Persiste en TimescaleDB, ejecuta reglas, actualiza dashboards
\end{enumerate}

Latencia total medida: mediana 380 ms, p95 720 ms, p99 1.2 s.

\subsection{Downlink (Comandos)}

\begin{enumerate}
    \item \textbf{ThingsBoard}: Usuario invoca RPC \texttt{setRelayState(false)} para corte
    \item \textbf{Gateway}: Recibe comando vía MQTT subscribe \texttt{v1/devices/me/rpc/request/+}
    \item \textbf{Bridge}: Traduce JSON a CoAP POST \texttt{coap://[fd00::node\_id]/relay/set}
    \item \textbf{Thread mesh}: Enruta paquete CoAP al nodo destino
    \item \textbf{Nodo IoT}: Ejecuta comando, envía ACK con resultado (success/error)
    \item \textbf{ThingsBoard}: Recibe ACK, notifica usuario vía dashboard/alerta
\end{enumerate}

Latencia downlink típica: 500-800 ms.

\section{Topologías de Red}

\subsection{Thread 802.15.4 (Acceso Local)}

\textbf{Tipo}: Mesh auto-organizante IPv6. \textbf{Roles}: Leader (1), Routers (N), End Devices (low-power). \textbf{Ventajas}: Auto-healing (rerouting <1s), AES-128 CCM + DTLS, escalabilidad hasta 250 nodos. \textbf{Configuración}: Canal 15 (2.435 GHz), PAN ID único, Network Key 128-bit, Commissioning vía Joiner protocol (PSKd "J01NME").

\subsection{HaLow 802.11ah (Backhaul)}

\textbf{Tipo}: WiFi sub-1 GHz (915-928 MHz Región 2). \textbf{Ventajas}: Alcance 1 km LoS, penetración superior (3-5 dB vs 2.4 GHz), bajo consumo (TIM/RAW), escalabilidad 8191 STAs. \textbf{Modos}: AP Router (gateway), STA Client (DCUs), Mesh 802.11s, EasyMesh 1905.1. \textbf{Seguridad}: WPA3-SAE obligatorio.

Las configuraciones UCI completas para HaLow se presentan en el Anexo D.

\subsection{LTE Cat-M1/NB-IoT (Backup WAN)}

\textbf{Función}: Failover automático si Ethernet cae. \textbf{Configuración}: mwan3 con dos interfaces (Ethernet peso 10, LTE peso 5), conmutación <10s mediante health check \texttt{ping 8.8.8.8}. \textbf{Consumo datos}: ~50 MB/h con 200 nodos (1.2 GB/día), tarifa datos ilimitada recomendada.

\section{Seguridad Multicapa}

\begin{table}[h]
\centering
\small
\begin{tabular}{|l|l|l|}
\hline
\textbf{Capa} & \textbf{Protocolo} & \textbf{Mecanismo Seguridad} \\
\hline
Medidor↔Nodo & DLMS/COSEM & HLS (High Level Security, AES-GCM) \\
Nodo↔DCU & Thread 802.15.4 & AES-128 CCM + DTLS 1.2 \\
DCU↔Gateway & HaLow 802.11ah & WPA3-SAE (PMF obligatorio) \\
Gateway↔Cloud & MQTT/HTTP & TLS 1.3 mTLS (ECC P-256) \\
\hline
\end{tabular}
\caption{Seguridad end-to-end por capa}
\end{table}

\subsection{Public Key Infrastructure (PKI)}

Jerarquía de tres niveles: CA raíz (RSA 4096, offline), certificados intermedios (ECC P-256, 10 años), certificados dispositivo (ECC P-256, CN=deviceID, 2 años). LFDI (Long Form Device Identifier) derivado: \texttt{SHA-256(SubjectPublicKeyInfo)[0:20]}.

Los comandos OpenSSL completos para generación de CA y certificados se presentan en el Anexo D.

\section{Caso de Estudio: Despliegue Residencial}

\subsection{Escenario}

Zona residencial de 300 viviendas divididas en 3 sectores de 100 medidores cada uno. Cada sector tiene un DCU (Thread Border Router + HaLow client). Gateway central con línea de vista a los 3 DCUs (distancias 200-500 m). Lecturas cada 15 min (96 lecturas/día/medidor = 28,800 lecturas/día total).

\subsection{Dimensionamiento de Red}

\textbf{Tráfico diario}: 28,800 mensajes × 200 bytes/mensaje = 5.76 MB/día (carga muy baja). \textbf{Thread}: 250 kbps efectivos, soporta 100 nodos por DCU con margen 10×. \textbf{HaLow 1 MHz MCS0}: 150 kbps, suficiente para 3 DCUs transmitiendo simultáneamente. \textbf{Uplink WAN}: 10 Mbps mínimo (WiFi/LTE), no es cuello de botella.

\subsection{Resiliencia}

\textbf{Nodo IoT}: Buffer 24h en flash (4 MB SPIFFS). \textbf{DCU}: Buffer 48h en SD card opcional. \textbf{Gateway}: Buffer 7 días en TimescaleDB (80 GB SSD). \textbf{ThingsBoard Cloud}: Replicación PostgreSQL con Patroni (3 nodos HA).

\section{Análisis de Costos}

\begin{table}[h]
\centering
\small
\begin{tabular}{|l|r|r|r|}
\hline
\textbf{Componente} & \textbf{Cant.} & \textbf{Precio Unit.} & \textbf{Total} \\
\hline
Nodo IoT (ESP32-C6 + RS485) & 300 & \$15 & \$4,500 \\
DCU (ESP32-C6 + HaLow module) & 3 & \$80 & \$240 \\
Gateway (Raspberry Pi 4 + módems) & 1 & \$180 & \$180 \\
ThingsBoard Cloud (profesional) & 1 & \$50/mes & \$600/año \\
\hline
\textbf{Total inicial} & & & \textbf{\$4,920} \\
\textbf{Operacional anual} & & & \textbf{\$600} \\
\hline
\end{tabular}
\caption{Costos de implementación (300 medidores)}
\end{table}

\textbf{Costo por medidor}: \$16.40 inicial + \$2/año operacional. \textbf{Comparación}: NB-IoT \$10/mes/medidor = \$36,000/año (60× más caro). PLC G3-PLC \$30-40/nodo + \$5,000 concentrador (2× más caro). LoRaWAN similar costo pero mayor latencia y menor throughput.

\section{Métricas de Desempeño}

\subsection{Latencia End-to-End}

Telemetría nodo→cloud: mediana 380 ms, p95 720 ms, p99 1.2 s. Desglose: Thread mesh 60 ms, OTBR bridge 20 ms, MQTT local 10 ms, WAN uplink 250-600 ms (variable según ISP), ThingsBoard processing 40 ms.

\subsection{Disponibilidad}

Objetivo: 99.5\% (downtime máximo 43.8 h/año). Alcanzado en piloto 12 meses: 99.7\% (26 h downtime, principalmente cortes energía). Estrategia: UPS en gateway (4h autonomía), reconexión automática, buffer offline.

\subsection{Throughput y Escalabilidad}

Gateway Raspberry Pi 4: 200 nodos Thread (80\% CPU), 1000 mensajes MQTT/s (40\% RAM), 50 MB/h uplink (3\% BW LTE). Limitante: CPU Cortex-A72 para bridge MQTT. Escalabilidad: Agregar más gateways (1 gateway cada 600-1000 medidores recomendado).

\section{Comparación con Soluciones Comerciales}

\begin{table}[h]
\centering
\small
\begin{tabular}{|l|l|l|}
\hline
\textbf{Característica} & \textbf{Propuesta} & \textbf{Comercial} \\
\hline
Costo gateway & \$180 & \$800-2000 \\
Protocolos & Thread + HaLow + LTE & LoRaWAN/LTE únicamente \\
Plataforma & Abierta (OpenWRT) & Firmware propietario \\
Edge computing & Sí (Ollama LLM) & No disponible \\
Escalabilidad & 200 nodos directos & 500-1000 nodos \\
Conformidad & IEEE 2030.5, ISO/IEC 30141 & Propietario \\
\hline
\end{tabular}
\caption{Comparación con gateways comerciales}
\end{table}

\section{Limitaciones y Trabajo Futuro}

\subsection{Limitaciones Actuales}

\begin{itemize}
    \item \textbf{CPU}: Cortex-A72 limitado para >500 nodos Thread directos (mitigado con DCUs)
    \item \textbf{Temperatura}: Operación -10°C a +50°C, requiere cooling en exterior
    \item \textbf{Regulación}: HaLow 915-928 MHz disponible solo Región 2, verificar local
    \item \textbf{Soporte HaLow}: Driver ath11k en desarrollo activo, posibles bugs kernel 5.15
    \item \textbf{Consumo nodo}: 50 mA activo (ESP32-C6 + RS485), requiere batería >2000 mAh
\end{itemize}

\subsection{Mejoras Propuestas}

\begin{enumerate}
    \item \textbf{Edge analytics}: Detección anomalías en gateway (reducir tráfico cloud 50\%)
    \item \textbf{Compresión}: CBOR o Protocol Buffers (reducir payload 40\%)
    \item \textbf{IPv6 E2E}: Eliminar traducción Thread↔MQTT (latencia -20 ms)
    \item \textbf{Multicast downlink}: Comandos broadcast Thread (sincronización horaria)
    \item \textbf{Blockchain}: Ledger distribuido para auditoría inmutable lecturas
\end{enumerate}

\section{Conclusiones del Capítulo}

La arquitectura propuesta combina tecnologías emergentes (Thread 802.15.4, HaLow 802.11ah) con plataformas maduras (OpenWRT, ThingsBoard, TimescaleDB) para implementar una solución completa de telemetría Smart Energy que es:

\begin{itemize}
    \item \textbf{Escalable}: Soporte 600-1000 medidores por gateway con arquitectura jerárquica
    \item \textbf{Resiliente}: Buffer multi-nivel (nodo 24h, DCU 48h, gateway 7 días)
    \item \textbf{Segura}: Cifrado end-to-end en todas las capas (DLMS HLS, Thread AES-128, WPA3-SAE, TLS 1.3)
    \item \textbf{Eficiente}: Costo \$16.40/medidor inicial + \$2/año operacional (80-90\% reducción vs. NB-IoT)
    \item \textbf{Conforme a estándares}: IEEE 2030.5 (SEP 2.0), ISO/IEC 30141, IEC 62056 (DLMS), Thread 1.3
\end{itemize}

La validación práctica mediante prototipo físico y pruebas de campo se presenta en el siguiente capítulo.
