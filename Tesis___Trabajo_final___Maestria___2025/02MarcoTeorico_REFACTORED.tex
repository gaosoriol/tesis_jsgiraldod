\chapter{Marco Teórico}

Este capítulo establece las bases teóricas que sustentan la arquitectura propuesta, estructuradas según la jerarquía de 4 niveles. Iniciamos con el ecosistema de estandarización IoT y Smart Energy que define el marco regulatorio (§2.1). Luego progresamos a través de las tecnologías específicas de cada nivel arquitectónico: tecnologías de red de campo para Nivel 1 - sensores (§2.2), infraestructura de distribución híbrida para Nivel 2 - conectividad de barrio (§2.3), y finalmente las plataformas de edge computing y cloud que implementan Niveles 3-4 (§2.4).

\section{Ecosistema de Estandarización IoT y Smart Energy}

Esta sección presenta el marco de estándares internacionales que guían el diseño de infraestructuras AMI interoperables, desde el modelo de referencia NIST para Smart Grid hasta los protocolos de comunicación mandatorios (IEEE 2030.5) y los frameworks de arquitectura IoT (ISO/IEC 30141).

\subsection{Arquitectura de Referencia Smart Grid NIST}

Las Smart Grids integran tecnologías de información y comunicación (TIC) para monitoreo, control y optimización en tiempo real del flujo eléctrico desde generación hasta consumo final~\cite{SmartHomeEnergy2024,velasquezSmartGridsEmpowered2024}. Este enfoque permite: integración masiva de energías renovables distribuidas (DER), gestión activa de la demanda (DSM), detección y auto-recuperación de fallas (self-healing), y participación activa de prosumidores.

El modelo de referencia NIST para Smart Grid (NIST Framework Release 4.0~\cite{NISTFramework2022}) define tres capas principales~\cite{alsuwaidiSecuringSmartGrid2024}:

\begin{enumerate}
\item \textbf{Power and Energy Layer}: Infraestructura física de generación, transmisión, distribución y almacenamiento.
\item \textbf{Communication Layer}: Redes de datos multi-protocolo (HAN, NAN, WAN) que transportan información de telemetría y comandos de control.
\item \textbf{Application Layer}: Sistemas de gestión de energía (EMS), gestión de distribución (DMS), gestión de demanda (DERMS), y analytics.
\end{enumerate}

La arquitectura AMI se compone típicamente de: medidores inteligentes instalados en puntos de consumo, concentradores/gateways que agregan datos de decenas o cientos de medidores, y head-end systems en centros de control que procesan típicamente 1-10 millones de registros diarios en redes de 100K-1M medidores~\cite{alsafranChallengesImplementingIoT2025}.

\subsection{IEEE 2030.5-2018 (Smart Energy Profile 2.0)}

IEEE 2030.5, anteriormente conocido como ZigBee SEP 2.0, es el estándar ampliamente adoptado para interoperabilidad de dispositivos Smart Energy en América del Norte, mandatorio para programas de respuesta a la demanda (Demand Response) en California según Senate Bill 2030 y Hawaii Rule 14H~\cite{IEEERecommendedPractice,CaliforniaCPUC2023}. Define un modelo RESTful sobre HTTP/TLS para comunicación cliente-servidor entre dispositivos de campo (medidores, termostatos, inversores solares) y sistemas de gestión (DERMS, head-end systems)~\cite{tangResearchInteroperabilityIoT}.

\subsubsection{Arquitectura RESTful del Estándar}

IEEE 2030.5 estructura funcionalidades en Function Sets exponiendo recursos REST~\cite{IEEERecommendedPractice}: /dcap (descubrimiento), /tm (sincronización horaria), /edev (registro dispositivos), /mup (datos de medición), /mr (perfiles de carga), /msg (notificaciones), /dr (Demand Response), /fsa (QoS).

\subsubsection{Function Sets Implementados}

\textbf{Device Capability (DCAP)}: El cliente consulta /dcap para descubrir qué Function Sets implementa el servidor.

\textbf{End Device (ED)}: Registro de dispositivos con LFDI (Long Form Device Identifier) derivado de certificado X.509, proporcionando $2^{160}$ identificadores únicos, garantizando ausencia de colisiones incluso en despliegues globales~\cite{IEEERecommendedPractice}.

\textbf{Mirror Meter Reading (MMR)}: Publicación de lecturas de medición con granularidad configurable (típicamente 15 minutos). Datos codificados en formato OBIS (Object Identification System) según IEC 62056 DLMS/COSEM~\cite{IEC62056-2021}.

\subsection{ISO/IEC 30141:2024 - Marco de Interoperabilidad IoT}

ISO/IEC 30141, publicado originalmente en 2018 y actualizado en 2024~\cite{ISOIEC30141v2024}, define un framework estandarizado para sistemas IoT mediante cuatro vistas complementarias (funcional, información, despliegue, operacional). Especifica componentes, interfaces y flujos de información, complementando ISO/IEC 29100 (Privacy Framework) e ISO/IEC 27001 (Security Management)~\cite{ISOIEC30141v2024,tangResearchInteroperabilityIoT}.

\subsubsection{Modelo de Capas Funcionales}

ISO/IEC 30141 define cuatro vistas complementarias, siendo la Vista Funcional la más relevante para esta tesis. Descompone el sistema IoT en 7 entidades funcionales (FE)~\cite{tangResearchInteroperabilityIoT}: Sensing/Actuation (adquisición de datos y control), Processing (transformación y agregación), Storage (persistencia de datos), Communication (transporte entre FEs), Security (autenticación y cifrado), Management (configuración y actualizaciones OTA), Application Support (APIs y workflows).

\subsubsection{Mapeo de Arquitectura Propuesta a ISO/IEC 30141}

La arquitectura propuesta implementa las 7 entidades funcionales requeridas por ISO/IEC 30141:2024, garantizando conformidad con el estándar: (1) Sensing FE mediante medidores DLMS/COSEM y nodos Thread ESP32-C6, (2) Communication FE con gateway multi-radio Thread/HaLow, (3) Processing FE edge en Raspberry Pi 4 con ThingsBoard Edge Rule Engine, (4) Processing FE cloud con microservicios distribuidos y Kafka streams, (5) Storage FE local con TimescaleDB 7 días retention, (6) Storage FE cloud con ThingsBoard PostgreSQL 5 años, (7) Security FE con Thread AES-128-CCM-8, WPA3-SAE y mTLS 1.3, (8) Management FE con OTA firmware updates, (9) Application Support FE con REST API y IEEE 2030.5 adapter.

\section{Tecnologías de Red de Campo (Nivel 1: Sensores y Medidores)}

El Nivel 1 de la arquitectura comprende dispositivos de campo con restricciones severas: <256 KB RAM, <1 MB Flash, operación con batería. Esta sección presenta las tecnologías de red de área personal (WPAN) optimizadas para estos entornos: 6LoWPAN para compresión IPv6, CoAP para aplicaciones RESTful ligeras, LwM2M para gestión de dispositivos, y Thread como stack comercial~\cite{shelby6LoWPANWirelessEmbedded2009,abdulsalamOverviewRecentWireless2024}.

\subsection{Stack de Protocolos 6LoWPAN/CoAP/LwM2M}

El stack IoT combina eficiencia de transmisión con interoperabilidad IPv6. 6LoWPAN (IPv6 over Low-Power Wireless Personal Area Networks) reduce headers IPv6 de 40 bytes a 2-7 bytes mediante compresión IPHC (IPv6 Header Compression)~\cite{shelby6LoWPANWirelessEmbedded2009}, permitiendo payload útil >75\% del MTU 802.15.4 (127 bytes).

\textbf{Compresión IPHC:} Explota redundancias en headers IPv6: direcciones link-local derivadas de MAC se omiten (16 bytes → 0), prefijos Thread conocidos se comprimen por ID de 4 bits, campos Version/Traffic Class/Flow Label se omiten si son valores por defecto~\cite{shelby6LoWPANWirelessEmbedded2009}.

\textbf{NHC (Next Header Compression):} Extiende compresión a UDP, reduciendo 8 bytes a 1-2 bytes. Puertos CoAP típicos (61616-61631) se comprimen a 4 bits cada uno~\cite{shelby6LoWPANWirelessEmbedded2009}.

\begin{equation}
\text{Payload disponible} = 127 - 25 \text{ (MAC)} - 3{-}9 \text{ (IPHC+NHC)} = 93{-}99 \text{ bytes (73-78\% MTU)}
\end{equation}

\subsection{CoAP: Protocolo de Aplicación Ligero}

CoAP (Constrained Application Protocol, RFC 7252) es un protocolo web RESTful optimizado para dispositivos IoT, diseñado como alternativa ligera a HTTP~\cite{shahinzadehSmartHomeConnectivity2024}. Header compacto de 4 bytes vs 100-500 bytes HTTP, transporte UDP (8 bytes vs 20+ TCP), latencia 0 ms conexión (stateless) vs 50-150 ms TCP handshake~\cite{karimiIIoTCommunicationProtocols2025}.

\textbf{Observe - Subscripciones push:} RFC 7641 define extensión para subscripciones asíncronas a recursos, eliminando polling ineficiente. Reduce tráfico 90-95\% vs polling HTTP, latencia notification <50 ms~\cite{shelby6LoWPANWirelessEmbedded2009}.

\subsection{LwM2M: Gestión de Dispositivos IoT}

LwM2M (Lightweight Machine-to-Machine, OMA SpecWorks) construye sobre CoAP para proveer aprovisionamiento, configuración, monitoreo, actualización firmware y diagnóstico remoto~\cite{haEnablingDynamicLightweight2018,karimiIIoTCommunicationProtocols2025}.

\textbf{Modelo de objetos:} Jerárquico 3 niveles - Object/Instance/Resource (ej. /10243/0/6 = Single-Phase Power Meter / Instance 0 / Active Power).

\textbf{Operaciones:} Read, Write, Execute, Create/Delete, Observe, Discover, Write-Attributes (configurar umbrales pmin/pmax/gt/lt).

\textbf{Firmware OTA:} Object 5 estandariza actualización remota con descarga en background, verificación de firma digital, y reporte de resultado.

\subsection{Thread: Stack IPv6 sobre IEEE 802.15.4}

Thread es un protocolo IPv6 sobre IEEE 802.15.4 @ 2.4 GHz diseñado para IoT doméstico e industrial~\cite{abdulsalamOverviewRecentWireless2024}. Implementa routing mesh adaptativo basado en métricas LQI y path cost. Topología jerárquica incluye Leader, Router, REED, y End Device. Thread Border Router (OTBR) actúa como gateway hacia redes IP tradicionales~\cite{choudharyInternetThingsComprehensive2024,openthread2024}.

Thread ofrece IPv6 end-to-end nativo (crítico para IEEE 2030.5), routing proactivo MLE con convergencia menor a 5 segundos, OpenThread Border Router open-source, y Matter specification que adoptó Thread como mandatory transport layer~\cite{threadMatterConvergence2024}.

\section{Tecnologías de Distribución e Interconexión (Nivel 2: Backhaul Híbrido)}

El Nivel 2 interconecta múltiples redes de campo hacia gateways de procesamiento edge, requiriendo tecnologías con alcance >1 km y throughput >1 Mbps para agregar cientos de medidores.

\subsection{Wi-Fi HaLow (IEEE 802.11ah)}

IEEE 802.11ah opera en bandas sub-1 GHz (902-928 MHz USA, 863-868 MHz Europa) optimizando alcance y penetración~\cite{IEEE802.11ah-2016,adameTimeVariantChannelModeling2019}. Ofrece alcance superior a 1 km outdoor line-of-sight (200-400 m indoor), throughput de 150 kbps hasta 86.7 Mbps (típicamente 4 Mbps @ MCS7), latencia de 8-15 ms, eficiencia energética mediante Target Wake Time (TWT) que permite dormir 99.9\% del tiempo, y chipsets comerciales disponibles desde 2023 (Morse Micro MM6108, Newracom NRC7292)~\cite{IEEE802.11ah-2016,khorovSurveyIEEE802112016,morsemicroMM61082024,liEstimatingPropagationCharacteristics2024}.

\textbf{Aplicabilidad AMI:} Un Access Point HaLow puede servir 100-500 medidores en radio 1 km, agregando 2.4 Mbps efectivo. 500 medidores generan 120 MB/día = 11 kbps promedio, bien dentro de capacidad HaLow~\cite{alsafranChallengesImplementingIoT2025}.

\textbf{Validación comercial - Caso Victoria, Australia:} Despliegue documentado (2023) validó HaLow en condiciones operacionales: streaming video HD en 7.5 km alcance real en topografía montañosa~\cite{halownetworkPreventingLivestockTheft2023}. Confirma que para telemetría IoT (1-10 kbps/dispositivo), distancias 2-3 km requeridas en esta tesis son altamente conservadoras.

\section{Plataformas IoT y Edge Computing (Niveles 3-4)}

\subsection{Computación en el Borde (Edge Computing)}

Docker permite encapsular aplicaciones en containers aislados mediante namespaces y cgroups del kernel Linux~\cite{liangReviewEdgeComputing2024}. TimescaleDB optimiza PostgreSQL para series temporales con hypertables (particionado automático), continuous aggregates y compresión columnar 10-20×~\cite{boonmeerukCostEffectiveIIoTGateway2024}.

ThingsBoard Edge replica funcionalidad completa en gateways locales con sincronización bidireccional hacia instancia cloud: dashboards locales, rule chains CEP sin round-trip, buffering automático durante offline, protocolo gRPC con batching y compresión~\cite{gartnerMagicQuadrantIoT2024}.

\subsection{Agentes de IA y Modelos de Lenguaje en el Borde}

La computación en el borde permite desplegar modelos de lenguaje (LLM) locales para análisis contextual de anomalías, respuestas automáticas a consultas, y procesamiento de lenguaje natural sin dependencia de conectividad cloud. Frameworks como Ollama permiten inferencia de modelos compactos (7B parámetros) en hardware edge (Raspberry Pi 5, 8 GB RAM) con latencia <200 ms, habilitando asistentes virtuales para gestión de AMI con privacidad de datos garantizada.

\subsection{Seguridad Transversal}

Los sistemas IoT presentan superficie de ataque ampliada: compromise de dispositivos, Man-in-the-Middle, replay attacks, DoS, escalation de privilegios, data exfiltration~\cite{BlockchainBasedSecureAuthentication2025,nandalSECURITYRISKSIoT2025}.

\textbf{Defence in Depth:} Estrategia en capas: (1) Física - Secure Boot, TPM; (2) Red - Firewall nftables, VLANs, WPA3-SAE~\cite{IEEE80211ah2016}, TLS 1.2/1.3 mutual auth; (3) Aplicación - RBAC ThingsBoard, rate limiting; (4) Datos - cifrado at-rest LUKS, backup GPG.

\subsection{Brechas Identificadas en Estado del Arte}

\textbf{Ausencia de HaLow en gateways edge:} Ningún trabajo integra Wi-Fi HaLow como backhaul Smart Energy con failover multi-WAN, a pesar de ventaja de alcance (1-3 km) y throughput (40 Mbps) sobre LoRaWAN (latencia >1s).

\textbf{Conformidad limitada IEEE 2030.5 + ISO/IEC 30141:} Soluciones comerciales implementan protocolos propietarios, prototipos académicos cumplen parcialmente estándares sin documentar las cuatro vistas de ISO/IEC 30141.

\textbf{Validación experimental insuficiente:} Mayoría de trabajos presentan diseño conceptual sin mediciones empíricas de latencia end-to-end, PDR en condiciones reales, o caracterización de resiliencia durante desconexiones WAN prolongadas.

\textbf{Costo-efectividad en contexto Latinoamericano:} Soluciones comerciales (Cisco IR829 \$2,500-4,000, Dell EG \$1,200-2,000) presentan barreras CAPEX significativas para economías emergentes, sin análisis de alternativas open-source con hardware commodity.

\subsection{Transición al Capítulo 3}

Este capítulo estableció las bases teóricas del ecosistema de estandarización (§2.1), tecnologías de red de campo (§2.2), infraestructura de distribución (§2.3), y plataformas edge/cloud (§2.4). Las cuatro brechas identificadas motivan directamente el diseño arquitectónico del Capítulo 3: integración HaLow nativa, conformidad estándar dual, validación experimental cuantitativa, y viabilidad económica para contextos Latinoamericanos.
